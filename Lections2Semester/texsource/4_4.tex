\section{Изометрии. Ортогональные  и унитарные преобразования}

\subsection{Изометрия (изоморфизм евклидовых (унитарных) пространств)}

Пусть даны два евклидовых (унитарных) пространства $\mathcal{E}_1$ и $\mathcal{E}_2$.

\defin{Отображение $\varphi: \mathcal{E}_1\to \mathcal{E}_2$ называется {\it изометрией}, или {\it изоморфизмом евклидовых (унитарных) пространств},
если $\varphi$ --- изоморфизм векторных пространств с дополнительным условием: 
$\forall \, \vek{a}, \vek{b} \in \mathcal{E}_1$ выполнено $(\varphi (\vek{a}), \varphi (\vek{b}))=(\vek{a},\vek{b})$.
}

Итак, изоморфизм еквлидовых пространств  --- это обычный изоморфизм векторных пространств, который вдобавок сохраняет скалярное произведение.

\defin{Два евклидовых (унитарных) пространства $\mathcal{E}_1$ и $\mathcal{E}_2$
называются {\it изоморфными}, если существует изоморфизм $\varphi: \mathcal{E}_1\to \mathcal{E}_2$.
}

Обозначение $\cong$ (как и в случае обычного изоморфизма).
Как и для обычного изоморфизма показывается, что отношение $\cong$ --- отношение эквивалентности.


\begin{predl}\label{p10_2_3} 
Пусть $\dim \mathcal{E}_1 = \dim \mathcal{E}_2 = n<\infty $ и 
$\varphi \in L(\mathcal{E}_1\to \mathcal{E}_2)$, $\bazis{e} = (\vek{e}_1, \ldots, \vek{e}_n)$ --- ОНБ в $\mathcal{E}_1$.
Тогда $\varphi$ является изоморфизмом евклидовых (унитарных) пространств $\Leftrightarrow$ $(\varphi(\vek{e}_1), \ldots, \varphi(\vek{e}_n))$ --- ОНБ в $\mathcal{E}_2$. 
\end{predl}
\dok 
\dokright Сразу следует из определения изоморфизма евклидовых (унитарных) пространств:
$\varphi(\vek{e}_i), \varphi(\vek{e}_j)) = (\vek{e}_i, \vek{e}_j)) = \delta_{ij}$.\\
\dokleft Так как $\varphi$ переводит базис в базис, то $\varphi$ --- (обычный) изоморфизм векторных пространств (см.....).
Возьмем произвольные векторы $\vek{a}$ и  $\vek{b}$ из $V$ и разложим их по базису $\bazis{e}$:
$\vek{a} = \sum\limits_{i=1}^n x_i \vek{e}_i$, $\vek{b} = \sum\limits_{i=1}^n y_i \vek{e}_i$. Тогда
$\varphi(\vek{a}) = \sum\limits_{i=1}^n x_i \varphi(\vek{e}_i)$, $\varphi(\vek{b}) = \sum\limits_{i=1}^n y_i \varphi(\vek{e}_i)$.
Тогда по следствию 4 из предложения \ref{p10_2_2}:
$(\vek{a}, \vek{b}) = \sum\limits_{i=1}^n x_i\overline{y_i} = (\varphi(\vek{a}), \varphi(\vek{b}))$.
\edok

\begin{theor}\label{t10_2_1} 
Два конечномерных евклидовых (унитарных) пространства $\mathcal{E}_1$ и $\mathcal{E}_2$
изоморфны $\Leftrightarrow$ $\dim \mathcal{E}_1 = \dim \mathcal{E}_2$.
\end{theor}
\dok
\dokright 
Если $\dim \mathcal{E}_1 \neq \dim \mathcal{E}_2$, то $\mathcal{E}_1$ и $\mathcal{E}_2$ не изоморфны даже как обычные векторные пространства.
\dokleft
Если $\dim \mathcal{E}_1 = \dim \mathcal{E}_2$, то достаточно взять ОНБ в каждом из пространств и взять линейное отображение $\varphi: \mathcal{E}_1 \to \mathcal{E}_2$, 
переводящее ОНБ в ОНБ. % (такое $\varphi$ существует по ....).
Согласно предложению \ref{p10_2_3}, $\varphi$ будет изоморфизмом.
\edok

\otstup

{\bf Упражнение.} Докажите, что в определении изометрии условие 
$(\varphi (\vek{a}), \varphi (\vek{b}))=(\vek{a},\vek{b})$
можно заменить на более слабое $(\varphi (\vek{a}), \varphi (\vek{a}))=(\vek{a},\vek{a})$.


\subsection{Ортогональные  и унитарные преобразования}


\defin{
Линейное преобразование $\varphi: \mathcal{E}\to \mathcal{E}$ называется ортогональным (унитарным), если
 $\forall$ $\vek{a}, \vek{b}\in V$ выполнено
$$(\varphi(\vek{a}), \varphi(\vek{b}))= (\vek{a}, \vek{b}).$$
}

Ортогональное преобразование сохраняется скалярное произведение. Геометрически это означает сохранение длин и (в случае евклидова пространства) углов,
т.е. если мыслить себе векторы как радиус-векторы с началом в $O$, то происходит <<движение>> с неподвижной точкой $O$.
Для конечномерных евклидовых (унитарных) пространств ортогональные (унитарные) преобразование --- это в точности изоморфизмы на себя:

\begin{predl}\label{p10_5_1} 
Пусть $\dim \mathcal{E}=n<\infty $, $\varphi \in L(\mathcal{E}\to \mathcal{E})$. 
Тогда $\varphi$ --- ортогональное (унитарное) $\Leftrightarrow$ $\varphi$ --- изоморфизм евклидовых (унитарных) пространств.
\end{predl}
\dok \dokright Достаточно доказать биективность $\varphi$. Но из определелния ортогональности преобразования следует, что 
$\varphi$ переводит ОНБ в некоторую ортонормированную систему из $n$ векторов, т.е. в ОНБ. %\label{p10_2_3}\\
\\
\dokleft Очевидно по определению изоморфизма.
\edok

\begin{sled1}
Пусть $\dim \mathcal{E}=n<\infty $, $\varphi \in L(\mathcal{E}\to \mathcal{E})$, 
$(\vek{e}_1, \ldots, \vek{e}_n)$ --- ОНБ в $\mathcal{E}$.
Тогда $\varphi$ является ортогональным  (унитарным) преобразованием $\Leftrightarrow$ $(\varphi(\vek{e}_1), \ldots, \varphi(\vek{e}_n))$ --- ОНБ в $\mathcal{E}$. 
\end{sled1}


\begin{sled2}
Пусть $\dim \mathcal{E}=n<\infty $, $\bazis{e}=(\vek{e}_1, \ldots, \vek{e}_n)$ --- ОНБ в $\mathcal{E}$.
Пусть $\varphi \in L(\mathcal{E}\to \mathcal{E})$ так, что 
$\varphi \rsootv{\bazis{e}, \bazis{e}} A$.
Тогда $\varphi$ является ортогональным  (унитарным) преобразованием $\Leftrightarrow$ $A$ --- ортогональная (унитарная) матрица.
\end{sled2}
%МАТРИЦА!!

\begin{sled3}
Пусть $\dim \mathcal{E}=n<\infty $, $\varphi \in L(\mathcal{E}\to \mathcal{E})$.
Тогда $\varphi$ является ортогональным  (унитарным) преобразованием $\Leftrightarrow$ $\varphi$ обратимо, причем $\varphi ^{*} = \varphi ^{-1}$. 
\end{sled3}
\dok
Докажем в терминах матриц (хотя можно вывести непосредственно из определения).
Пусть $\bazis{e}$ --- ОНБ в $\mathcal{E}$ так, что 
$\varphi \rsootv{\bazis{e}, \bazis{e}} A$. Тогда $\varphi^{*} \rsootv{\bazis{e}, \bazis{e}} A^{*}$, 
$\varphi^{-1} \rsootv{\bazis{e}, \bazis{e}} A^{-1}$. Поскольку $A^{*}=A^{-1}$, получаем $\varphi^{*}=\varphi^{-1}$.
\edok


\begin{predl}[Групповые свойства]\label{p10_5_2} 
Пусть $\dim \mathcal{E}=n<\infty $, $\varphi, \psi \in L(\mathcal{E}, \mathcal{E})$ --- ортогональные (унитарные) преобразования.
Тогда преобразования $\varphi \psi$ и $\varphi ^{-1}$ --- также ортогональные (унитарные).
\end{predl}
\dok Можно вывести из определения или из предложения... (ортог. матрицы).
\edok


\begin{predl}\label{p10_5_3} 
Пусть $\varphi \in L(\mathcal{E}, \mathcal{E})$ --- ортогональное (унитарное) и $\lambda_0$ --- его характеристическое число.
Тогда $|\lambda_0|=1$.
\end{predl}
\dok 1) Докажем вначале утверждение для унитарного пространства (над $\mathbb{C}$). Пусть $\vek{a}$ --- собственный вектор, 
соответствующий собственному значению $\lambda_0$.
Равенство $(\varphi(\vek{a}), \varphi(\vek{a})) = (\vek{a}, \vek{a})$
принимает вид $(\lambda_0 \vek{a}, \lambda_0 \vek{a}) = (\vek{a}, \vek{a})$, откуда 
$\lambda_0 \overline{\lambda_0}  (\vek{a}, \vek{a}) = (\vek{a},  \vek{a})$ $\Rightarrow$ $\lambda_0 \overline{\lambda_0}=1$ $\Rightarrow$ $|\lambda_0|=1$.\\
2) Покажем, как случай евклидова пространства (над $\mathbb{R}$) свести к разобранному. В п.1) доказано, что 
для каждой унитарной матрицы $A$ уравнение $|A-\lambda E|=0$ имеет лишь корни по модулю равные 1. Этого достаточно, так как  $\varphi$ в любом ОНБ имеет 
ортогональную матрицу (частный случай унитарной матрицы).
\edok

В унитарном (над $\mathbb{C}$) пространстве имеется результат, а аналогичный теореме \ref{t10_4_3}.

\begin{theor}[канонический вид унитарного преобразовния]\label{t10_5_3} 
Пусть $\mathcal{E}$ --- унитарное пространство, $\dim \mathcal{E}<\infty$. Для унитарного преобразования $\varphi\in L(\mathcal{E}, \mathcal{E})$
существует ОНБ из собственных векторов.
\end{theor}
\dok Доказательство повторяет доказательство теоремы \ref{t10_4_3} с следующим небольшим отличием
в обосновании инвариантности $U=\lin{\vek{e}_n}^{\bot}$ относительно $\varphi$:
по теореме \ref{t10_3_1} $U$ инвариантно относительно $\varphi^{*}=\varphi^{-1}$; но тогда по предложению .... главы ... ,
$U$ инвариантно и относительно $\varphi$.
\edok


{\footnotesize 
Аналогом последней теоремы для евклидова пространства (над $\mathbb{R}$) является следующее утверждение о каноническом виде ортогонального преобразования.
Пусть $\varphi\in L(\mathcal{E}, \mathcal{E})$ --- ортогональное. Тогда $\mathcal{E}$ --- 
прямая сумма попарно ортогональных инвариантных подпространств размерности 1 или 2.\\
Доказать это по той же схеме, что и теорему \ref{t10_4_3} используя, что у любого преобразования вещественного пространства
найдется одномерное либо двумерное инвариантное подпространство.
}


\subsection{Полярное разложение}


\begin{theor}\label{t10_6_105} 
Пусть $\dim \mathcal{E}=n<\infty $, $\varphi, \in L(\mathcal{E}, \mathcal{E})$.
Тогда существуют самосопряженное преобразование  $\psi$ и ортогональное (унитарное) преобразование $\theta$ такие, что
$$\varphi = \psi \theta.$$ 
\end{theor}
\dok 
\edok

Мы получили следующую информацию о геометрии произвольного преобразования евклидова пространства:
это композиция некоторого <<движения>> и <<обощенных растяжений>> к ортогональным осям.


{\bf Упражнение.}
Выведите из теоремы существование разложения
$\varphi = \theta_1 \psi _1$, где $\psi _1$ --- самосопряженное, $\theta _1$ --- ортогональное.


