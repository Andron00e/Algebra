\section{Собственные векторы. Диагонализируемость.}

В этом параграфе полагаем, что поле $\mathbb{F}$ --- либо $\mathbb{R}$, либо $\mathbb{C}$.
{\footnotesize Хотя теорию несложно обобщить на случай произвольного поля $\mathbb{F}$, используя
наряду с $\mathbb{F}$ {\it алгебраическое замыкание}  --- расширение
$\mathbb{K}$ поля $\mathbb{F}$, для которого каждый многочлен из $\mathbb{K}[X]$ имеет корень.}


\subsection{Собственные значения. Собственные векторы и собственные подпространства.}


%Следующие определения связаны с изучением одномерных инвариантных подпространств.

\defin{
Константа $\lambda_0 \in \mathbb{F}$ называется {\it собственным значением}
преобразования~$\varphi$, если $\exists $ $\vek{a} \in V$ такой, что $ \vek{a} \neq \vek{0}$ и
\begin{equation}\label{eigen}
\varphi ( \vek{a}) = \lambda _0 \vek{a}. 
\end{equation}
}

\defin{
Если для $ \vek{a} \neq \vek{0}$ выполнено (\ref{eigen}), 
то вектор $\vek{a}$ называется {\it собственным вектором}
преобразования $\varphi$, отвечающим собственному значению $\lambda _0$.
}

\begin{predl}\label{p8_5_6}
Ненулевой вектор $\vek{a}$ является собственным вектором преобразования $\varphi$
$\Leftrightarrow$
одномерное подпространство
$\langle \vek{a} \rangle$ инвариантно относительно $\varphi$.
\end{predl}
\dok  В силу предложения \ref{p8_5_0}, условие 
инвариантности $\langle \vek{a} \rangle$ эквивалентно выполнению (\ref{eigen})
для некоторого $\lambda_0 \in \mathbb{F}$.
\edok


\begin{predl}\label{p8_5_7}
Ненулевой вектор $\vek{a}$ является собственным вектором преобразования $\varphi$,
отвечающим собственному значению $\lambda_0$
$\Leftrightarrow$
$\vek{a} \in \Ker (\varphi -\lambda_0)$.
\end{predl}
\dok  Достаточно заметить, что (\ref{eigen}) эквивалентно условию $(\varphi - \lambda _0)(\vek{a})=\vek{0}$.
\edok

\otstup
Отметим, что пара предложений выше согласуется с предложением \ref{p8_5_66}: наличие одномерных 
инвариантных подпространств для $\varphi$ связано с вырожденностью оператора 
$\varphi - \lambda _0$ (многочлена от $\varphi$ первой степени).

\defin{
Пусть $\lambda_0$ --- собственное значение преобразования $\varphi \in L(V, V)$.
Подпространство $\Ker (\varphi -\lambda_0)$ называется {\it собственным подпространством}
для преобразования $\varphi$ (отвечающим собственному значению $\lambda_0$).
}

Собственное подпространство условимся обозначать $V_{\lambda_0}$.
Таким образом, по определению $V_{\lambda_0} = \Ker (\varphi -\lambda_0)$ ---
содержит все собственные векторы,  отвечающие собственному значению
$\lambda_0$, и $\vek{0}$.

{\footnotesize Иногда, в других курсах и книгах,  собственными подпространствами 
называют также и подпространства $\Ker (\varphi -\lambda_0)$.
Отметим, что введенные формально новые понятия, такие как собственное подпространство,
по сути лишь новая терминология для важных частных случаев определенных ранее понятий.}


\begin{predl}\label{p8_5_8}
Пусть $\lambda _1, \lambda _2, \ldots , \lambda _k$ --- различные собственные значения для
$\varphi \in L(V, V)$. Тогда $\sum\limits_{i=1}^{k} V_{\lambda _i} $ --- прямая сумма.
\end{predl}
\dok Это предложение сразу следует из предложения \ref{Ker_polynom} главы
\ref{lin_otobr} --- достаточно рассмотреть линейные многочлены $f_i(x) = x- \lambda _i$.

 Приведем еще одно, независимое, доказательство. 
Предположим, противное, и пусть, скажем, $\vek{a}_1\in V_{\lambda _1}\cap \sum\limits_{i=2}^{k} V_{\lambda _i}$, 
$\vek{a}_1\neq \vek{0}$ (см. теорему \ref{t7_4_1}, глава \ref{lin_prostr}).
Тогда $\vek{a}_1 = \sum\limits_{i=2}^{k} \vek{a}_i$ для некоторых $ \vek{a}_i\in V_{\lambda _i}$.
Применим к этому равенству преобразование $\psi = \prod\limits_{i=2}^{k}(\varphi-\lambda _i)$.
Так как $(\varphi-\lambda _i)(\vek{a}_i)=\vek{0}$ и в произведении $\prod\limits_{i=2}^{k}(\varphi-\lambda _i)$ преобразования перестановочны, то
$\psi (\vek{a}_i)=\vek{0}$ для $i=2, \ldots, k$. С другой стороны, $\psi (\vek{a}_1 ) = \prod\limits_{i=2}^{k}(\lambda_1-\lambda _i)\vek{a}_1\neq \vek{0}$.
Противоречие.
\edok

\subsection{Характеристический многочлен}

Далее в этом параграфе полагаем $\dim V = n<\infty$.

\defin{
Пусть  $\varphi$ имеет в некотором базисе матрицу $A$.
Функция $|A-\lambda E|$ переменной $\lambda$ называется
{\it характеристическим многочленом} оператора $\varphi$.
}

Для характеристического многочлена примем обозначение $\chi_{\varphi}(\lambda)$.

\defin{Уравнение
$\chi_{\varphi}(\lambda) = 0$ называется
{\it характеристическим уравнением}, а его (комплексные) корни
--- {\it характеристическими числами}  преобразования $\varphi$.
}

Характеристическое уравнение 
дает условие вырожденности оператора $\varphi - \lambda$; это важное 
соображение еще будет зафиксировано ниже (см. теорему \ref{t8_5_1}).
А сперва сделаем некоторые наблюдения про $\chi_{\varphi}(\lambda)$.

\begin{predl}\label{p8_5_0000}
$\chi_{\varphi}(\lambda)$ является многочленом степени $n$:
\begin{equation}\label{chi1}
\chi_{\varphi}(\lambda) = (-1)^n\lambda ^n +a_1 \lambda ^{n-1} + a_2 \lambda ^{n-2} + \ldots +
a_{n-1} \lambda + a_n,
\end{equation}
где $a_i\in \mathbb{F}$.
При этом $$a_1 = (-1)^{n-1} \tr A, \,\,\,\,\,\,   a_n = |A|.$$
\end{predl}
\dok 
Формулу вида (\ref{chi1}) можно получить, пользуясь явным разложением определителя $|A-\lambda E|$.
Отсюда сразу можно получить  $a_n = \chi_{\varphi}(0) = |A|$.\\
Далее, диагональные элементы матрицы $|A-\lambda E|$ --- линейные функции от $\lambda $
(равные $a_{ii}-\lambda$), а внедиагональные элементы --- константы, 
поэтому в явном разложении определителя все слагаемые, кроме произведения диагональных элементов 
будут содержать $\lambda $  в степени не превышающей $n-2$:\\ 
$$\chi_{\varphi}(\lambda) = \prod\limits_{i=1}^n (a_{ii}-\lambda) + g(\lambda), $$
где $\deg g\leq n-2$. Отсюда в $\chi_{\varphi}(\lambda)$ коэффициент  при $\lambda ^n$ 
равен $(-1)^n$, а коэффициент при $\lambda ^{n-1}$ равен $a_1 = (-1)^{n-1} \sum\limits_{i=1}^n a_{ii} =
(-1)^{n-1} \tr A$. 
\edok

\otstup

Теперь разложим (над полем комплексных чисел) многочлен $\chi_{\varphi}(\lambda)$
на линейные множители:
\begin{equation}\label{chi2}
\chi_{\varphi}(\lambda) = (-1)^n \prod\limits_{i=1}^n (\lambda - \mu _i),
\end{equation}
где $\mu_i\in \mathbb{C}$ --- характеристические числа.

Среди $\mu_i$ могут быть равные числа. Группируя равные характеристические числа, примем далее
такие обозначения: пусть $\lambda _1, \lambda _2, \ldots , \lambda _k $
--- попарно различные характеристические числа, а $s_1, s_2, \ldots , s_k$ ---
их (алгебраические) {\it кратности}, так что $\sum\limits_{i=1}^k s_i = n$. Тем самым, формула 
(\ref{chi2}) принимает вид
\begin{equation}\label{chi3}
\chi_{\varphi}(\lambda) = (-1)^n \prod\limits_{i=1}^k (\lambda - \lambda _i)^{s_i}.
\end{equation}

\begin{predl}\label{p8_5_00}
1). Сумма (с учетом кратности) всех характеристических чисел равна $\tr A$.\\
2). Произведение (с учетом кратности) всех характеристических чисел равно $|A|$.
\end{predl}
\dok Достаточно приравнять  
соответствующие коэффициенты в (\ref{chi1}) и (\ref{chi2}) (или же воспользоваться теоремой Виета).
\edok

\otstup

Отметим следующий важный частный случай.

\begin{predl}\label{p8_5_99}
Пусть  матрица $A$ оператора $\varphi$ (в некотором базисе) --- верхнетреугольная. Тогда
характеристические числа совпадают с числами, расположенными на диагонали матрицы~$A$. 
\end{predl}
\dok Это ясно, поскольку  определитель верхнетреугольной матрицы $A-\lambda E$
равен произведению диагональных элементов.
\edok


\begin{sled}%\label{p8_5_99}
Пусть в случае $\mathbb{F}=\mathbb{R}$ оператор $\varphi$ в некотором базисе 
имеет верхнетреугольную матрицу. Тогда все характеристические числа $\varphi$
вещественные. 
\end{sled}


\otstup

Следующее предложение обосновывает корректность обозначения
$\chi_{\varphi}(\lambda)$. % (а не $\chi_{A}(\lambda)$).

\begin{predl}\label{p8_5_9}
Характеристический многочлен 
преобразования $\varphi$ не зависит от выбора базиса в $V$.
\end{predl}
\dok Нужно доказать, что если $\varphi \rsootv{\bazis{e}, \bazis{e}} A$ 
и $\varphi \rsootv{\bazis{e'}, \bazis{e'}} A'$, 
то $|A-\lambda E|= |A'-\lambda E|$. 
По теореме \ref{t8_3_2} 
$A'=S^{-1}AS$, поэтому \\ $|A'-\lambda E| = |S^{-1}AS-\lambda E| = |S^{-1}AS-\lambda S^{-1}ES| =
|S^{-1}(A-\lambda E) S| = 
|S^{-1}|\cdot |A-\lambda E|\cdot | S| = $ \\
$=|S^{-1}|\cdot | S| \cdot |A-\lambda E| = |S^{-1}  S| \cdot |A-\lambda E| = |E| \cdot |A-\lambda E|=|A-\lambda E|.$
\edok

\begin{sled}
Определитель, след, набор характеристических чисел (с учетом кратностей) 
матрицы оператора  $\varphi$ не зависят от выбора базиса.
\end{sled}

%Следующие теоремы проясняют связь между характеристическим многочленом и предыдущим разделом.

\otstup

\begin{theor}\label{t8_5_1}
%$V$ --- векторное пространство над $\mathbb{R} (\mathbb{C})$, %$\dim V = n$,
Для константы $\lambda_0 \in \mathbb{F}$ следующие условия эквивалентны:
$\lambda_0$ --- собственное значение для $\varphi$
$\Leftrightarrow$ $\lambda_0$ --- характеристическое число (т.е. корень $\chi_{\varphi}$).
%корень характеристического многочлена $\chi_{\varphi}(\lambda)$.
\end{theor}
\dok $\lambda_0$ --- собственное значение $\Leftrightarrow$
$\Ker (\varphi - \lambda_0) \neq O$ $\Leftrightarrow$ (переходя к координатам)
СЛУ $(A - \lambda_0 E)X=O$ имеем ненулевое решение $\Leftrightarrow$
квадратная матрица $(A - \lambda_0 E)$ вырожденная 
$\Leftrightarrow$ $|A - \lambda_0 E|=0$.
\edok

\otstup

Таким образом, в случае $\mathbb{F} = \mathbb{C}$ <<собственное значение>>
и <<характеристическое число>> --- это одно и то же. 
В случае $\mathbb{F} = \mathbb{R}$ разница только в том, что характеристическое число может не принадлежать
$\mathbb{R}$; в этом случае собственные значения --- это в точности
вещестенные характеристические числа.

\otstup

Теперь мы готовы зафиксировать результат об инвариантных подпространствах малой размерности.

\begin{theor}\label{t8_5_222}
1). Пусть $\mathbb{F} = \mathbb{C}$. Тогда для $\varphi$ существует одномерное инвариантное подпространство
(или эквивалентно, существует собственный вектор).\\
2). Пусть $\mathbb{F} = \mathbb{R}$ и $n$ нечетно. Тогда для $\varphi$ существует одномерное инвариантное подпространство.\\
3). Пусть $\mathbb{F} = \mathbb{R}$. Тогда для $\varphi$ существует 
ненулевое инвариантное подпространство размерности не выше 2.
\end{theor}
\dok 
1) Следует из теоремы \ref{t8_5_1},  поскольку $\chi_{\varphi}$ имеет (комплексный) корень.\\
2) Следует из теоремы \ref{t8_5_1},  поскольку при нечетном $n$ 
многочлен $\chi_{\varphi}\in \mathbb{R}[X]$ имеет вещественный корень.\\
3) Если $\chi_{\varphi}\in \mathbb{R}[X]$ имеет вещественный корень, найдем, как и в 
пункте 2), одномерное инвариантное подпространство.\\ 
 Иначе, пусть $\lambda_0\in \mathbb{C} \setminus \mathbb{R}$ --- корень $\chi_{\varphi}$.
Многочлен $f (x) = x^2 - (\lambda_0 + \overline{\lambda_0})x + \lambda_0 \overline{\lambda_0}$ 
принадлежит $\mathbb{R}[X]$ (отметим, что  $\overline{\lambda_0}$ --- тоже корень $\chi_{\varphi}$). При этом оператор $f(\varphi)$ вырожден, так как
его матрица $f(A) = (A-\lambda _0 E) (A-\overline{\lambda_0} E)$ вырождена (здесь, как обычно, 
$A$ --- матрица оператора $\varphi$). Действительно, $\det f(A) = 
|A-\lambda _0 E|\cdot |A-\overline{\lambda_0} E| = 0$. Теперь нужное утверждение следует из предложения \ref{p8_5_66}.
\edok

%(В СОГЛАСИИ С предложением \ref{p8_5_66} --- ПОДПРОСТРАНСТВА МАЛЫХ РАЗМЕРНОСТеЙ


\otstup


\begin{theor}\label{t8_5_2}
%Пусть $V$ --- векторное пространство над $\mathbb{R} (\mathbb{C})$, %$\dim V = n$, и
Пусть $\lambda_i \in \mathbb{F}$.
 является
корнем кратности $s_i$ характеристического многочлена $\chi_{\varphi}(\lambda)$ преобразования
$\varphi$.
Тогда $$1\leq \dim V_{\lambda _i} \leq s_i.$$
\end{theor}
\dok Пусть $\dim V_{\lambda _i}=s$. Выберем в $\lambda _i$ базис $(\vek{e}_1, \ldots, \vek{e}_s)$ 
и дополним его до базиса $\bazis{e}= (\vek{e}_1, \ldots, \vek{e}_n)$ в пространстве $V$.
Тогда $\varphi \rsootv{\bazis{e}, \bazis{e}} A$, где $A$ имеет блочный вид $\begin{pmatrix}
\lambda _i E_s & C \\
O & D
\end{pmatrix}.$ Пользуясь формулой детерминанта с углом нулей, имеем\\ $\chi_{\varphi}(\lambda) = |A-\lambda E| = |\lambda _i E_s- \lambda E_s|\cdot 
|D- \lambda E_{n-s}| = 
(\lambda_i-\lambda)^s p(\lambda)$, где $p$ --- многочлен. \\ Тем самым, $s_i\geq s$.
\edok

\otstup

 Величину $\dim V_{\lambda _i}$ иногда называют {\it геометрической кратностью}
собственного значения $\lambda _i$. 
Таким образом, в теореме \ref{t8_5_2} утверждается, что
геометрическая кратность корня не превышает алгебраической кратности.
%дает 



\subsection{Диагонализируемость}

\defin{
Преобразование $\varphi$ называется {\it диагонализируемым}, если в  $V$ существует базис,
в котором матрица $\varphi$ имеет диагональный вид.
}

Ранее мы видели (см. следствие из предложения \ref{p8_5_99}), что 
в случае  наличие характеристических 
чисел, лежащих  вне поля $\mathbb{F}$, 
препятствует не только диагонализируемости, но даже треугольному виду матрицы преобразования.
Поэтому в критерии ниже полагаем, что все характеристические числа оператора $\varphi$
принадлежат $\mathbb{F}$ (в случае $\mathbb{F}=\mathbb{C}$ это условие не несет в себе никаких ограничений,
т.е. выполнено всегда).


\begin{theor}[критерий-1 диагонализируемости]\label{t8_5_3}
% $V$ --- векторное пространство над $\mathbb{R} (\mathbb{C})$. %, $\dim V = n$.
Пусть оператор $\varphi$
имеет попарно различные характеристические числа
$\lambda_1, \lambda_2, \ldots , \lambda_k\in \mathbb{F}$  кратностей
$s_1, s_2, \ldots , s_k$ соответственно. Тогда следующие условия эквивалентны:\\
1) $\varphi$ диагонализируем;\\
2) В $V$ существует базис из собственных векторов; \\ % $\varphi$;
3) $\dim V_{\lambda _i} = s_i$ для $i=1, 2, \ldots, k$;\\
%(в частности, в случае $V$ над $\mathbb{R}$ необходимо все $\lambda_1, \lambda_2, \ldots , \lambda_k$
%вещественные).\\
4) $V = \bigoplus \limits_{i=1}^k V_{\lambda _i}$. % является прямой суммой собственных подпространств.
\end{theor}
\dok 1) $\Leftrightarrow$ 2). Очевидно следует из определения матрицы линейного преобразования.\\
2) $\Rightarrow$ 3). %Докажем например, что $\dim V_{\lambda _1} = s_1$. 
В базисе из собственных векторов пусть $t_i$ векторов соответствуют $\lambda _i$ ($i=1, \ldots, k$),
так что $t_1+\ldots +t_k = n$.
%Обозначим для удобства $\vek{e}_1, \ldots,  \vek{e}_{t_1}$ собственные векторы из данного базиса, 
%которые отвечают $ 
Тогда $V_{\lambda _i}$ содержит линейно независимую систему из $t_i$ векторов,
откуда $\dim V_{\lambda _i} \geq  t_i$. 
С учетом теоремы \ref{t8_5_2}, имеем
$$t_i\leq \dim V_{\lambda _i} \leq  s_i, \,\,\,\,\,\,\,i=1, \ldots, k.$$
Суммируем эти неравенства. Поскольку $s_1+\ldots +s_k =  t_1+\ldots +t_k = n$, 
все неравенства должны обращаться в равенства, т.е. $\dim V_{\lambda _i} =  s_i$ для всех $i=1, \ldots, k.$\\
3) $\Rightarrow$ 4). Следует из теоремы  \ref{p8_5_8} с учетом $s_1+\ldots +s_k =  n$.\\
4) $\Rightarrow$ 2). В каждом $V_{\lambda_i}$ выберем базис. Объединение 
этих базисов будет нужным базисом в $V$ (см. критерий-2 прямой суммы --- теорема \ref{t7_4_2}
главы \ref{lin_prostr}), и этот базис составлен из собственных векторов. 
\edok

\otstup



\begin{sled}
%Пусть $V$ --- векторное пространство над $\mathbb{R}$, $\varphi \in L(V, V)$.
Если $\chi_{\varphi} (\lambda )$ имеет $n$ различных вещественных корней, принадлежащих 
$\mathbb{F}$, то $\varphi$ диагонализируемо.
\end{sled}

\otstup

Это следствие 
очерчивает достаточно большой класс диагонализируемых операторов.

\otstup

Вообще, определять диагонализируем оператор или нет, часто бывает удобно, 
используя условие 3) из теоремы \ref{t8_5_3}.

\otstup

Посмотрим, что означает условие диагонализируемости в переводе <<на матричный язык>>.
%Пусть $\varphi \in L(V, V)$ имеет матрицу $A$ в некотором базисе.
В силу формулы изменения матрицы при замене базиса (см. теорему 
\ref{t8_3_2} главы \ref{lin_otobr}),
для диагонализируемости матрицы $A$ (матрицы оператора $\varphi$ в некотором базисе)
необходимо и достаточно существование такой невырожденной
матрицы $S$, что  матрица $S^{-1}AS$ (подобная матрице $A$) диагональна.

%Поэтому иногда говорят о диагонализируемости квадратной матрицы $A$.


\subsection{Примеры}

%Не все преобразования диагонализируемы. 

\example{I.
Преобразование $\mathbb{R}^2\to \mathbb{R}^2$
поворота на угол, не кратный $\pi$, не диагонализиремо  (нет собственных векторов, 
или поскольку характеристические числа не вещественны).\\
Сжатие к прямой (проходящей через $O$), проектирование на прямую, отражение относительно прямой, 
напротив, диагонализируемые.
}

\example{II.
Матрица $\begin{pmatrix} \lambda_0 & 1 \\ 0 & \lambda_0 \end{pmatrix}$ 
не диагонализируема (как для $\mathbb{F}=\mathbb{R}$, так и для $\mathbb{F}=\mathbb{C}$), 
так как $\lambda _0$ --- корень кратности $2$, но $\dim V_{\lambda _0} = 1$.\\
(Объяснение без использования теоремы \ref{t8_5_3} такое: если бы это преобразование было диагонализируемым,
то диагональный вид был бы нулевой матрицей, и значит, преобразование было бы нулевым, что неверно.)
Этот пример --- частный случай жордановой клетки, общая теория на этот счет --- в следующем параграфе.
}

\example{III.
Пусть $V= \mathbf{P}_n = \lin{1, x, x^2, \ldots, x^n}$ и 
Оператор дифференцирования $d: V\to V$ имеет единственное характеристическое число $0$ кратности $n+1$ 
(см. матрицу $d$ в стандартном базисе --- пример в конце параграфа  \ref{matr_lin_otobr}
\,\,\,\,\,\,\,\,\,\,).
При $n\geq 1$ преобразование $d$ не является диагонализируемым, как и любой линейный 
дифференциальный оператор вида $d-\lambda I_V$.
}



