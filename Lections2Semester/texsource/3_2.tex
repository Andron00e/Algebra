\section{Симметричные билинейные и квадратичные формы}


\subsection{Симметричные билинейные формы}


\defin{
Билинейная (полуторалинейная) форма $\beta$ на пространстве $V$ называется {\it симметричной} ({\it эрмитовой} или {\it эрмитово симметричной}),
если $\forall$ $\vek{a}, \vek{b} \in V$
$$\beta (\vek{a}, \vek{b}) = \overline{\beta (\vek{b}, \vek{a})}.$$
}

Множество всех симметричных (эрмитовых) билинейных (полуторалинейных) форм на пространстве $V$ обозначаем $\mathcal{B}_{sym}(V)$.

Отметим, сразу, что $\forall$ $\vek{a} \in V$ значение $\beta (\vek{a}, \vek{a})$ вещественно
(это утверждение содержательно для комплексного пространства).

\begin{theor}\label{t9_2_1}
Пусть $\dim V<\infty$, $\bazis{e}$ --- базис в $V$. Пусть $\beta \in  \mathcal{B}(V)$, $\beta \rsootv{\bazis{e}} B$.
Тогда $\beta\in \mathcal{B}_{sym}(V)$ $\Leftrightarrow$ $B^{*}=B$ (где, как обычно,  $B^{*} = \overline{B^T}$).
\end{theor}
\dok 
\dokright Надо доказать, что $\overline{b_{ji}}=b_{ij}$ или что $\overline{\beta(\vek{e}_j, \vek{e}_i)}=\beta(\vek{e}_i, \vek{e}_j)$ для всевозможных пар индексов.
Но это сразу следует из определения.

\dokleft Пусть $\vek{a}, \vek{b} \in V$ --- произвольные векторы, $\vek{a}=\bazis{e}X$, $\vek{b}=\bazis{e}Y$.
Тогда $\beta (\vek{a}, \vek{b}) = X^T B \overline{Y}$.
Также $\beta (\vek{b}, \vek{a}) = Y^T B \overline{X}$ или (транспонируем матрицу $1\times 1$)
$\beta (\vek{b}, \vek{a}) = \overline{X}^T B^T Y = \overline{ X^T B^{*} \overline{Y} } = \overline{X^T B \overline{Y}}$. 
Отсюда $\overline{\beta (\vek{b}, \vek{a})}=\beta (\vek{a}, \vek{b})$, что и требовалось.
\edok


%{\bf Упражнение.}
%Определите {\it кососимметричные билинейные формы}, найдите условия на матрицу, эквивалентные кососимметричности формы.


\subsection{Квадратичные формы}


\defin{Пусть $\beta\in \mathcal{B}_{sym}(V)$. Отображение $k: V\to \mathbb{R}$ ($k: V\to \mathbb{C}$), заданное правилом
$k(\vek{a})=\beta (\vek{a}, \vek{a})$ называется {\it квадратичной (эрмитовой) формой} или {\it квадратичной функцией},
порожденной билинейной формой $\beta$.
}

Множество всех (эрмитовых) квадратичных форм обозначим $\mathcal{K}(V)$.

Определение дает возможность говорить о (эрмитово) симметричной матрице (эрмитовой) квадратичной форме.
Координатная запись квадратичной формы имеет вид $k(\vek{a}) = X^TB\overline{X}$ или 
$k(\vek{a}) = \sum \limits_{i=1}^n \sum \limits_{j=1}^n b_{ij} x_i  \overline{x_j} $, 
где $X = \stolbec{x_1\\ x_2\\ \vdots \\ x_n}$ --- координатный столбец вектора $\vek{a}$.
Заметим, что диагональные элементы  матрицы $B$ --- это значения квадратичной формы на базисных векторах:
$b_{ii}=k(\vek{e}_i)$.


\begin{zamech}
В вещественном случае породить квадратичную форму можно было бы и произвольной (не обязательно симметричной) билинейной формой,
однако это не изменило бы запас квадратичных форм: билинейная форма с матрицей $B$ порождает ту же форму, что и симметричная билинейная форма с матрицей
$\dfrac{B+B^T}{2}$.
\end{zamech}


\begin{theor}\label{t9_2_2}
%(Восстановление симметричной билинейной формы по соответствующей квадратичной.)
Данная (эрмитово) квадратичная форма $k$ порождается ровно одной билинейной формой $\beta\in \mathcal{B}_{sym}(V)$.
\end{theor}
\dok Достаточно показать, как значение $\beta (\vek{a}, \vek{b})$ на заданных векторах $\vek{a}$, $\vek{b}$ определяется только по значениям квадратичной формы.

1) Доказательство для $\mathbb{R}$. 
Заметим, что \\ $k(\vek{a}+\vek{b}) = \beta (\vek{a}+\vek{b}, \vek{a}+\vek{b}) = \beta (\vek{a}, \vek{a}) + \beta (\vek{a}, \vek{b})
+ \beta (\vek{b}, \vek{a})+\beta (\vek{b}, \vek{b}) = k (\vek{a}) + 2\beta (\vek{a}, \vek{b})+k (\vek{b})$, поэтому
$\beta (\vek{a}, \vek{b})= \dfrac{k(\vek{a}+\vek{b})-k(\vek{a})-k(\vek{b})}{2}$.

2) Доказательство для $\mathbb{C}$. 
Имеем \\
$k(\vek{a}+\vek{b}) = \beta (\vek{a}+\vek{b}, \vek{a}+\vek{b}) = \beta (\vek{a}, \vek{a}) + \beta (\vek{a}, \vek{b}) + \beta (\vek{b}, \vek{a})+\beta (\vek{b}, \vek{b}) 
= k (\vek{a}) + \beta (\vek{a}, \vek{b})+\beta (\vek{b}, \vek{a})+k (\vek{b})$, 
отсюда $$\beta (\vek{a}, \vek{b})+\beta (\vek{b}, \vek{a}) = k(\vek{a}+\vek{b})-k (\vek{a})-k (\vek{b}).$$
Далее \\
$k(\vek{a}+i\vek{b}) = \beta (\vek{a}+i\vek{b}, \vek{a}+i\vek{b}) = \beta (\vek{a}, \vek{a}) -i \beta (\vek{a}, \vek{b}) + i \beta (\vek{b}, \vek{a})+\beta (\vek{b}, \vek{b}) 
= k (\vek{a}) -i (\beta (\vek{a}, \vek{b})-\beta (\vek{b}, \vek{a}))+k (\vek{b})$, 
отсюда $$\beta (\vek{a}, \vek{b})-\beta (\vek{b}, \vek{a}) = ik(\vek{a}+i\vek{b})+ik (\vek{a})+ik (\vek{b}).$$
Складывая получаенные равенства, получаем выражение $\beta (\vek{a}, \vek{b})$ только через значения 
$k(\vek{a}+\vek{b})$, $k(\vek{a}+i\vek{b})$, $k(\vek{a})$, $k(\vek{b})$.
\edok

\begin{sled}
Тождественно нулевая (эрмитова) квадратичная форма порождается лишь тождественно нулевой (эрмитово) симметричной формой.
\end{sled}

\otstup

Теорема \ref{t9_2_2} показывает, что в определении фактически устанавливается взаимно-однозначное соответствие между множествами $\mathcal{B}_{sym}(V)$ и $\mathcal{K}(V)$. 
Ниже отождествляем эти множества.




