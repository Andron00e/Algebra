

\section{Понятие аффинного пространства}

Идея конструкции абстрактного точечного (аффинного) пространства
---  в сопоставлении \\
<<точка $\leftrightarrow$ радиус-вектор>>.


\subsection{Определение и свойства}



{\it Аффинным пространством}, ассоциированным с векторным пространством $V$, называется множество $S$
(элементы которого будем называть \{точками\})  с отображением $S\times V\to S$ (откладываение от точки вектора, 
обозначать будем <<+>>), такие что\\
A1)  $p+(\vek{a}+\vek{b}) = (p+\vek{a})+\vek{b} $ \,\,\,\,\,\, ($\forall p\in S$, $\forall \vek{a}, \vek{b} \in V$ );\\
A2)  $p+\vek{0} = p$; \,\,\,\,\,\, ($\forall p\in S$);\\
A3)  $\forall p, q\in S$  существует единстивенный $\vek{a}\in V$ такой, что $p+ \vek{a} = q$.\\

\otstup

Вектор $\vek{a}$ из А3) называем {\it вектором, соединяющим точки} $p$ и $q$, и обозначаем $\overline{pq}$.

\otstup

Видим <<правило треугольника>>: $\overline{pq}+\overline{qr}=\overline{pr}$.

\otstup

$V$ естественно является аффинным пространством, ассоциированным с $V$.

\otstup

Наоборот, если в афинном пространстве зафиксировать точку $o$ (<<начало отсчета>>), 
то возникает биекция ({\it векторизация})  $S\to V$:
$$p \mapsto \overline{op}.$$ 

(зависит от выбора начала отсчета)

\otstup

$(o, \bazis{e}) $ --- {\it аффинная система координат} в $S$ (или {\it репер})

(в аналитической геометрии было: ДСК)

\otstup

Координаты точки $p\in S$ в ДСК
$(o, \bazis{e}) $ --- \\
это координаты вектора $\overline{op}$ в базисе $\bazis{e}$.


\otstup

координаты точки $p+\vek{a}$ ???
\\
\\
координаты вектора  $\overline{pq}$ 

\otstup

Замена координат: $$X=SX'+\gamma$$

\otstup
%барицентрические линейные комбинации, барики...

\subsection{Примеры подмножеств, конструкций}


{\it Плоскость}, или {\it линейное подмногообразие} ---
$$p+U, $$
где $U\leq V$.

\\
\\

Размерность 
$k$-мерная плоскость ($k=0$ --- точка, $k=1$ --- прямая, $k=n-1$ --- гиперплоскость). 

\otstup


Любые $k+1$ точек из $S$ принадлежат некоторой плоскости размерности $\leq k$.

$p_0 + \lin{\overline{p_0p_1}, \overline{p_0p_2}, \ldots, \overline{p_0p_k}}$.

%Критерий плоскости --- принадлежность целиком прямой

\otstup

{\footnotesize

Аффинная зависимость, независимость системы из $k+1$ точек.

Связь с линейной зависимостью векторов.

\otstup

Линейные комбинации точек 
(чтобы была корректность, сумма коэффициентов должна равняться 1, пример --- центр масс.)


\otstup
паралелелепипед, симмлекс.

выпуклые линейные комбинации, выпуклая оболочка.
}
