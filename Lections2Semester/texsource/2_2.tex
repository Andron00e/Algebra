\section{Матрица линейного отображения}\label{matr_lin_otobr}

В этом параграфе занимаемся только конечномерными векторными пространствами. Полагаем
$\dim V = n$, $\dim \widetilde{V} = m$, $\dim \widehat{V} = p$.

\subsection{Определение. Координатная запись линейного отображения}

\defin{
Пусть в пространствах $V$ и $\widetilde{V}$ зафиксированы базисы
$\bazis{e}=(\vek{e}_1, \vek{e}_2, \ldots , \vek{e}_n)$ и
$\bazis{f}=(\vek{f}_1, \vek{f}_2, \ldots , \vek{f}_m)$ соответственно.
{\it Матрицей линейного отображения} $\varphi \in L(V, \widetilde{V})$
в паре базисов $\bazis{e}$ и $\bazis{f}$
называется матрица $A\in \mathbf{M}_{m\times n}$, столбцы $a_{\bullet 1}, a_{\bullet 2}, \ldots , a_{\bullet n}$
которой --- координатные столбцы соответственно
векторов $\varphi (\vek{e}_1), \varphi (\vek{e}_2), \ldots , \varphi (\vek{e}_n)$
в базисе $\bazis{f}$.
}

Матрицу линейного преобразования $\varphi \in L(V, V)$ в паре совпадающих базисов $\bazis{e}$ и $\bazis{e}$
будем также называть короче: матрица $\varphi$ в базисе $\bazis{e}$.

Компактно определение можно записать как $(\varphi (\vek{e}_1)\, \varphi (\vek{e}_2)\, \ldots \, \varphi (\vek{e}_n)) = \bazis{f} A$.
Тот факт, что $A$ --- матрица линейного отображения $\varphi$
в паре базисов $\bazis{e}$ и $\bazis{f}$, будем обозначать
$\varphi \rsootv{\bazis{e}, \bazis{f}} A$.


\begin{predl}\label{p8_3_111}
Пусть в пространствах $V$ и $\widetilde{V}$ зафиксированы базисы
$\bazis{e}$ и $\bazis{f}$ соответственно.
Тогда отображение $\varphi \rsootv{\bazis{e}, \bazis{f}} A$ --- биекция между $L(V, \widetilde{V})$ и $\mathbf{M}_{m\times n}$.
\end{predl}
\dok По определению матрица $A$ несет в себе полную информацию об образах базисных векторов.
Остается воспользоваться теоремой \ref{t8_1_1}.
%что фиксация базисов в $V$ и $\widetilde{V}$ дает взаимно-однозначное соответствие между
%$\varphi \in L(V, \widetilde{V})$ и $\mathbf{M}_{m\times n}$.
\edok

\otstup 

Выведем формулу координатной записи линейного отображения.

\begin{theor}\label{t8_3_1}
Пусть $\varphi \rsootv{\bazis{e}, \bazis{f}} A$.
Если $\vek{a} = \bazis{e} X$, $\varphi (\vek{a}) = \bazis{f} Y$, то 
$$\boxed{Y=AX}.$$
\end{theor}
\dok Так как $\vek{a} = \sum\limits_{i=1}^n x_i \vek{e}_i$, 
то $\varphi (\vek{a}) = \sum\limits_{i=1}^n x_i \varphi(\vek{e}_i)$. Заменим в этом равенстве векторы на их координатные
столбцы в базисе $\bazis{f}$, получим:
$Y= \sum\limits_{i=1}^n x_i a_{\bullet i}$, где, как обычно, $a_{\bullet i}$ обозначает $i$-й столбец матрицы $A$.
Правая часть последнего равенства равна $AX$, что и требовалось уствновить.
\edok

\otstup

Формула из предыдущей теоремы фактически эквивалентна определению матрицы линейного отображения.
Более, точно, справедливо следующее %предложение.
%
%\begin{predl}\label{p8_3_2}
\begin{sled}
Пусть дано 
отображение $\varphi :V\to \widetilde{V}$ (априори не известно, что оно линейное) 
и матрица $A \in M_{m\times n}$.
Пусть $\forall$ $\vek{a} \in V$ координатные столбцы $X$ и $Y$ векторов
$\vek{a} = \bazis{e} X$ и $\varphi (\vek{a}) = \bazis{f} Y$ связаны 
равенством $Y=AX$. Тогда $\varphi \in L(V, \widetilde{V})$, причем
$\varphi \rsootv{\bazis{e}, \bazis{f}} A$.
\end{sled}
%\end{predl}
\dok Возьмем $\psi \in L(V, \widetilde{V})$ такое, что $\psi \rsootv{\bazis{e}, \bazis{f}} A$ (такое 
$\psi$ существует, согласно предложению \ref{p8_3_111}). Тогда $\varphi$ и $\psi$ имеют одну и ту же координатную запись $Y=AX$, т.е. 
$\varphi = \psi$.
\edok

\subsection{Матрица перехода и матрица преобразования}

Переход от базиса $\bazis{e}=(\vek{e}_1, \ldots, \vek{e}_n)$ к $\bazis{e}'=(\vek{e}'_1, \ldots, \vek{e}'_n)$  
формально не связан с  линейными преобразованиями. Однако, согласно теореме \ref{t8_1_1}, 
по паре базисов $\bazis{e}$ и $\bazis{e}'$ можно определить единственное линейное отображение 
$\varphi: V\to V$ такое, что $\varphi (\vek{e}_i) = \vek{e}'_i$ для $i=1, \ldots, n$
(при этом $\varphi $ является  изоморфизмом --- см. теорему \ref{t_isom1}). 
Отметим следующую связь между матрицей линейного преобразования и матрицей перехода.

%Матрица этого изоморфизма в базисе $\bazis{e}$ будет совпадать с матрицей перехода от $\bazis{e}$ к $\bazis{e}'$.


\begin{predl}\label{p8_3_112}
Пусть $\dim V=n<\infty$, $\bazis{e}$ и $\bazis{e}'$ --- базисы в $V$.
Пусть изоморфизм $\varphi: V\to V$ таков, что $\varphi (\vek{e}_i) = \vek{e}'_i$ для $i=1, \ldots, n$;
$\varphi \rsootv{\bazis{e}, \bazis{e}} A$. Тогда $A$ --- матрица перехода от $\bazis{e}$ к $\bazis{e}'$.
\end{predl}
\dok Достаточно сопоставить определения матрицы линейного отображения и матрицы перехода.
\edok




\subsection{Связь между операциями над отображениями и матрицами}

Предложение \ref{p8_3_111} может быть усилено следующим образом.

\begin{theor}\label{p8_3_333}
Пусть $\bazis{e}$ и $\bazis{f}$ --- базисы  пространств $V$ и $\widetilde{V}$ соответственно.
Тогда отображение $\varphi \rsootv{\bazis{e}, \bazis{f}} A$ является изоморфизмом
линейных пространств $L(V, \widetilde{V})$ и $\mathbf{M}_{m\times n}$.\\
\end{theor}
\dok Непосредственная проверка.
\edok

\begin{sled}
$\dim L(V, \widetilde{V}) = mn$.
\end{sled}
%\dok
%\edok

\otstup

Аналогично, для векторного пространства $V$ над $\mathbb{C}$ 
пространство $\overline{L}(V, \widetilde{V})$ (со вторым способом определения отображения на константу)
изоморфно $\mathbf{M}_{m\times n}(\mathbb{C})$; изоморфизм устанавливается как
$\varphi \mapsto \overline{A}$.


\begin{predl}\label{p8_3_4}
Пусть $\varphi \in L(V, \widetilde{V})$, $\widetilde{\varphi} \in
L(\widetilde{V}, \widehat{V})$, $\bazis{e}$, $\bazis{f}$, $\bazis{g}$ 
--- базисы  пространств $V$, $\widetilde{V}$, $\widehat{V}$ соответственно.
Пусть $\varphi \rsootv{\bazis{e}, \bazis{f}} A$,
 $\widetilde{\varphi} \rsootv{\bazis{f}, \bazis{g}} \widetilde{A}$.
Тогда $ \widetilde{\varphi} \varphi \rsootv{\bazis{e}, \bazis{g}} \widetilde{A} A$.
\end{predl}
\dok Пусть $\vek{a}\in V$ --- произвольный вектор, $\vek{a} = \bazis{e}X$, 
$\varphi(\vek{a}) = \bazis{f}Y$, $\widetilde{\varphi}(\varphi(\vek{a})) = \bazis{g}Z$.
 Тогда дважды пользуясь теоремой \ref{t8_3_1}, имеем
$Y=AX$, $Z=\widetilde{A}Y$, откуда $Z= \widetilde{A}(AX)=(\widetilde{A}A)X$.
Отсюда следует требуемое
(см. следствие из теоремы \ref{t8_3_1}).
\edok

\begin{sled1}
Пусть $\varphi \rsootv{\bazis{e}, \bazis{f}} A$.
Тогда $\varphi$ --- изоморфизм $\Leftrightarrow$ $A$ обратима.
Если $\varphi$ --- изоморфизм, то
$\varphi ^{-1} \rsootv{\bazis{f}, \bazis{e}} A^{-1}$.
\end{sled1}
\dok Следует из того, что $I_V \rsootv{\bazis{e}, \bazis{e}} E$.
\edok

\begin{sled2}
Пусть $\varphi \in L(V, V)$ и $\varphi \rsootv{\bazis{e}, \bazis{e}} A$.
Тогда  для любого многочлена $p$ имеем $p(\varphi) \rsootv{\bazis{e}, \bazis{e}} p(A)$.
\end{sled2}

\begin{sled3}
Пусть $\bazis{e}$ --- базис  пространства $V$.
Тогда отображение $\varphi \rsootv{\bazis{e}, \bazis{e}} A$ является изоморфизмом 
алгебр $L(V, V)$ и $\mathbf{M}_{n\times n} (\mathbb{F})$.\\
В частности группа %$L(V, V)^*$
(по умножению) изоморфизмов $V\to V$ изоморфна $GL_n(\mathbb{F})$ (группе невырожденных матриц).
\end{sled3}

\otstup

Установленное соответствие между операциями над линейными отображенями и операциями над матрицами
может быть применено в обе стороны: некоторые задачи об отображениях могут быть сведены к вопросам о матрицах, и наоборот (см., например, упражнения в конце следующего параграфа).

Многие понятие могут быть определены одновременно для линейных операторов и для матриц.
Напрмер: говорят о {\it ранге отображения}, имея в виду ранг матрицы этого отображения
(ниже мы увидим, что этот ранг не зависит от выбора базисов);
{\it невырожденным} называют оператор
$\varphi \in L(V, V)$, матрица которого невырожденная (это условие эквивалентно тому, что $\varphi$ --- изоморфизм);
c другой стороны, 
 можем говорить об аннулирующих и минмальных многочленах для матриц из $\mathbf{M}_{n\times n} (\mathbb{F})$.




\subsection{Изменение матрицы при замене базиса}

\begin{theor}\label{t8_3_2}
Пусть в $V$ выбраны базисы $\bazis{e}$ и $\bazis{e}'$, связанные матрицей перехода $S$:
$\bazis{e}'= \bazis{e} S$;
в $\widetilde{V}$ выбраны базисы $\bazis{f}$ и $\bazis{f}'$, связанные матрицей перехода $R$:
$\bazis{f}'= \bazis{f} R$.
Пусть $\varphi \in L(V, \widetilde{V})$ таково, что 
$\varphi \rsootv{\bazis{e}, \bazis{f}} A$ и $\varphi \rsootv{\bazis{e'}, \bazis{f'}} A'$.
%имеет матрицу $A$ в базисах $\bazis{e}$, $\bazis{f}$, и матрицу
%$A'$ в базисах $\bazis{e}'$, $\bazis{f}'$.
Тогда  $$\boxed{A' = R^{-1}AS}.$$
В частности, если $V=\widetilde{V}$, $\bazis{e}= \bazis{f}$ и $\bazis{e}'= \bazis{f}' $, то 
$$\boxed{A' = S^{-1}AS}.$$
\end{theor}
\dok Пусть $\vek{a}\in V$ --- произвольный вектор. Пусть $\vek{a}=\bazis{e}X = \bazis{e}'X'$, 
$\varphi(\vek{a})=\bazis{f}Y = \bazis{f}'Y'$. Тогда по теореме \ref{t8_3_1}  имеем $Y=AX$, $Y'=A'X'$ и
(по теореме \ref{t7_3_2}, глава \ref{lin_prostr}) $X=SX'$, $Y=RY'$. 
Отсюда $RY' = ASX'$ $\Rightarrow$ $Y' = R^{-1}ASX'$ или $Y' = (R^{-1}AS)X'$.
Получаем требуемое: $R^{-1}AS=A'$ (см. следствие из теоремы \ref{t8_3_1}).
\edok

\begin{sled}
Ранг матрицы линейного отображения $\varphi \in L(V, \widetilde{V})$ не зависит от выбора базисов в пространствах $V$ и $\widetilde{V}$.
\end{sled}
\dok
Следует из того, что $A'$ получается из $A$ домножением слева и справа на невырожденные матрицы.
\edok

\otstup

Инвариантная характеризация ранга матрицы линейного отображения дается ниже в следствии из теоремы \ref{t8_2_111}.

\otstup

Заметим также, что матрицу $S^{-1}AS$ иногда называют {\it подобной} или {\it сопряженной} матрице $A$.
Нетрудно проверить, что отношение подобия --- это отношение эквивалентности на множестве
$\mathbf{M}_{n\times n}(\mathbb{F})$. Это согласуется с тем фактом, что подобные матрицы соответствуют
одному и тому же оператору, в разных базисах.


\subsection{Примеры}

Пусть $V = U_1 \bigoplus U_2$. 
Введем базис  $\bazis{e} = (\vek{e}_1, \vek{e}_2, \ldots , \vek{e}_n)$ в $V$, {\it согласованный}  $U_1 \bigoplus U_2$, 
так, что $\vek{e}_1, \ldots , \vek{e}_k$ --- базис в $U_1$, а $\vek{e}_{k+1}, \ldots , \vek{e}_n$ --- базис в $U_2$.\\
Пусть  $\varphi: V\to V$ --- проектирование на $U_1$ вдоль $U_2$. 
Тогда (по определению матрицы линейного отображения): $\varphi \rsootv{\bazis{e}, \bazis{e}} \begin{pmatrix}
E_k & O \\
O & O
\end{pmatrix}.$\\
Пусть  $\psi: V\to V$ --- отражение относительно $U_1$ вдоль $U_2$. 
Тогда $\psi \rsootv{\bazis{e}, \bazis{e}} \begin{pmatrix}
E_k & O \\
O & -E_{n-k}
\end{pmatrix}.$

%В СТАНДАРТНОМ БАЗИСУ УМНОЖЕНИЕ НА ФИКС, МАТРИЦУ


\example{III.1.
Пусть $V= \mathbf{P}_n = \lin{1, x, x^2, \ldots, x^n}$ и 
$\bazis{e} = (1, x, x^2, \ldots, x^n)$ --- стандартный базис в $V$.
Оператор дифференцирования $d: V\to V$ имеет в базисе $\bazis{e}$ матрицу 
$\begin{pmatrix}
0 & 1 & 0 & \ldots & 0\\
0 & 0 & 2 & \ldots & 0\\
 &  & \ldots &  & \\
0 & 0 & 0 & \ldots & n\\
0 & 0 & 0 & \ldots & 0
\end{pmatrix}$
(поскольку $(x^{k+1})' = (k+1)x^k$, имеем $d(e_{i+1}) = (i+1)e_i$). %, $k=1, 2, \ldots, n$.
}



