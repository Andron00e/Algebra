\documentclass[12pt, a4paper]{book}%{article}

\usepackage[top=30pt, left=35pt, right=35pt, bottom=35pt]{geometry}
\geometry{a4paper,portrait}

\RequirePackage{amsfonts, amsmath, amssymb, amsthm, array, hhline}
\RequirePackage[T2A]{fontenc}
\RequirePackage[utf8]{inputenc}
\RequirePackage[russian]{babel}
\RequirePackage{soul}

\RequirePackage{graphicx, wrapfig}
\graphicspath{{media/}}
\DeclareGraphicsExtensions{.png,.jpg}

\RequirePackage{pgfpages}
\
\pagestyle{empty}
%\pagestyle{headings}
%\advance\textwidth35mm
%\advance\textheight65mm
%\advance\voffset-20mm
%\advance\voffset-20mm
%\advance\hoffset-20mm
%\parindent=2cm

\sloppy

%%%%%%%%%%%%
%%%перенос в формулах
%%%%%%%%%%%%%
\newcommand*{\hm}[1]{#1\nobreak\discretionary{}%
            {\hbox{$\mathsurround=0pt #1$}}{}}
\binoppenalty=10000
\relpenalty=10000

%%%%%%%%%%%%%%%
%%%Функции и стандартные значки
%%%%%%%%%%%%%%%

\def\logar{\mathop{\rm log}\nolimits}
\def\tg{\mathop{\rm tg}\nolimits}
\def\ctg{\mathop{\rm ctg}\nolimits}
\def\arctg{\mathop{\rm arctg}\nolimits}
\def\arcsin{\mathop{\rm arcsin}\nolimits}
\def\arccos{\mathop{\rm arccos}\nolimits}

\def\rg{\mathop{\rm rg}\nolimits}
\def\dim{\mathop{\rm dim}\nolimits}
\def\Ker{\mathop{\rm Ker}\nolimits}
\def\Im{\mathop{\rm Im}\nolimits}
\def\Id{\mathop{\rm Id}\nolimits}
\def\I{\mathop{\rm I}\nolimits}
\def\det{\mathop{\rm det}\nolimits}
\def\tr{\mathop{\rm tr}\nolimits}
\def\pr{\mathop{\rm pr}\nolimits}
\def\diag{\mathop{\rm diag}\nolimits}
\def\deg{\mathop{\rm deg}\nolimits}
\def\Sol{\mathop{\rm Sol}\nolimits}
\def\ord{\mathop{\rm ord}\nolimits}
\def\Hom{\mathop{\rm Hom}\nolimits}
\def\Sym{\mathop{\rm Sym}\nolimits}
\def\Alt{\mathop{\rm Alt}\nolimits}
\def\exp{\mathop{\rm exp}\nolimits}
\def\ln{\mathop{\rm ln}\nolimits}
\def\vol{\mathop{\rm vol}\nolimits}

%%%%%%%%
%%больше - меньше
%%%%%%%%

\renewcommand{\leq}{\leqslant}
\renewcommand{\geq}{\geqslant}
%\newcommand{\rg}{{\mathbf{rg }}}

%%%%%%%%%%
%%вектор
%%равно в ранге (для теории групп)
%%%%%%%%%%

\newcommand*{\eqrg}[1]{\stackrel{#1}{=}}
\newcommand*{\vek}[1]{\mathbf{#1}}
\newcommand*{\bazis}[1]{\mathrm{#1}}

\newcommand*{\stolbec}[1]{
 %\left(
 \begin{pmatrix}
#1 \end{pmatrix}
%\right)
}

\newcommand*{\lin}[1]{
\langle
#1 \rangle}


%\newcommand*{\rsootv}[1]{
%\begin{smallmatrix}
%\longleftrightarrow \\
%#1
%\end{smallmatrix}}

\newcommand*{\rsootv}[1]{
\begin{smallmatrix}
\longrightarrow \\
#1
\end{smallmatrix}}



\newcommand*{\otstup}{\vspace{3mm}}

%%%%%%%%%%
%%оформление определений и примеров
%%%%%%%%%%

\newcommand*{\defin}[1]{\begin{flushleft}
\hbox{%
\vrule \hspace{.1em} \vrule \hspace{.5em}\parbox{.95\textwidth}%
{\begin{opr}
{#1}
\end{opr}
}}
\end{flushleft}}

\newcommand*{\example}[1]{\begin{flushleft}
\hbox{%
\hspace{.5em} \vrule \hspace{.5em}\parbox{.95\textwidth}%
{#1}}
\end{flushleft}}


%%%%%%%%%%%%
%%%нумерация формул, предложений, теорем
%%%
%%%метки глав.
%%%начало метки любой теоремы tm_n_k, где m --- номер главы, n --- номер параграфа
%%%начало метки любого предложения pm_n_k, где m --- номер главы, n --- номер параграфа
%%%начало метки любой выносной формулы eqm_n_k, где m --- номер главы, n --- номер параграфа
%%%%%%%%%%%%



\newtheorem{theor}{Теорема}[section]
\renewcommand{\thetheor}{\arabic{section}.\arabic{theor}}

\newtheorem{predl}{Предложение}[section]
\renewcommand{\thepredl}{\arabic{section}.\arabic{predl}}

\newtheorem{lemm}{Лемма}[section]
\renewcommand{\thelemm}{\arabic{section}.\arabic{lemm}}


%\newcounter{opred}[chapter]
%\newcommand{\opr}{\par\refstepcounter{opred}\noindent%
%\textbf{Определение \arabic{opred}. }}

\newcounter{opred}[chapter]
\newcommand{\opr}{\par\refstepcounter{opred}\noindent%
\textbf{Определение.}}

\newcounter{zadacha}[chapter]
\newcommand{\zad}{\par\refstepcounter{zadacha}\noindent%
\textbf{Задача \arabic{zadacha}. }}

\newcommand{\prim}{\par \noindent%\refstepcounter{zadacha}%
\textbf{Пример. }}

\newcommand{\prims}{\par \noindent%\refstepcounter{zadacha}%
\textbf{Примеры. }}

\newcommand{\dok}{$\vartriangleright$\hspace{2mm}}
\newcommand{\edok}{$\square$\hspace{2mm}}
%\newcommand{\eopr}{$\square$}
\newcommand{\dokright}{$\boxed{\Rightarrow}\hspace{2mm}$}
\newcommand{\dokleft}{$\boxed{\Leftarrow}\hspace{2mm}$}


\renewcommand{\contentsname}{Содержание}
\renewcommand{\chaptername}{Глава}
%\renewcommand{\refname}{Литература}
\renewcommand{\bibname}{Литература}
\renewcommand{\indexname}{Предметный указатель}
\renewcommand{\figurename}{Рис.}

\renewcommand{\thesection}{$\large {\mathbf{\S}}$ \arabic{section}. \hspace{-8mm}}
\renewcommand{\thesubsection}{}%{\arabic{subsection}. \hspace{-7mm}}

\renewcommand{\theequation}{\arabic{chapter}.\arabic{equation}}

\renewcommand{\theparagraph}{\hspace{-7mm}}
%\renewcommand{\thesection}{\arabic{section}}
%\renewcommand{\section}{\@startsection{section}{1}%
%{0pt}{-3.5ex plus -1ex minus -.2ex}%
%{2.3ex plus.2ex }{\normalfont\Large}}


%\newcommand{\l@abcd}[2]{\hbox to\textwidth{#1\dotfill #2} }

\makeindex


\begin{document}

\title{Введение в линейную алгебру}
\author{ Кожевников Павел Александрович}

\date{\today}

\maketitle


%\LARGE

%\pagestyle{headings}
\pagestyle{plain}

\newtheorem*{sled}{Следствие}%[chapter]
\newtheorem*{sled1}{Следствие 1}%[chapter]
\newtheorem*{sled2}{Следствие 2}%[chapter]
%\newtheorem*{sled}{Следствие}%[chapter]
\newtheorem*{sled3}{Следствие 3}%[chapter]
\newtheorem*{sled4}{Следствие 4}%[chapter]
\newtheorem*{sled5}{Следствие 5}%[chapter]
\newtheorem*{lem}{Лемма}%[chapter]
\newtheorem*{lem1}{Лемма 1}%[chapter]
\newtheorem*{lem2}{Лемма 2}%[chapter]

\newtheorem*{zamech}{Замечание}%[chapter]

%\renewcommand{\thesled}{\arabic{sled}}

%\tableofcontents

\newpage

\chapter{Векторные пространства}\label{lin_prostr}

\section{Векторные пространства и подпространства}

\subsection{Аксиомы и их следствия}

{\it Бинарная операция} на множестве $Z$ --- это некоторое отображение $\varphi : Z \times Z \to Z$,
таким образом для каждой упорядоченной пары $(a, b)$ элементов множества $Z$ определен элемент $\varphi (a, b)\in Z$ --- 
{\it результат} применения операции $\varphi$ к $a$ и $b$.
Ниже будут встречаться бинарные операции в записи, более привычной для арифметических действий. Скажем, операцию можно обозначить
<<$+$>>  и тогда вместо $\varphi (a, b)$ писать $a+b$.
{\it Унарная операция} на множестве $Z$ --- это просто отбражение $Z \to Z$. Скажем, умножение на число 2 
формально можно считать
таким отображением $\psi : \mathbb{F} \to \mathbb{F}$, что $\forall x\in \mathbb{F}$ $ \psi (x) = 2x$.

Далее предполагаем, что $V$ --- произвольное множество, а $\mathbb{F}$ --- 
некоторое поле (скажем, поле $\mathbb{R}$), при этом  на $V$  определена бинарная операция, которую называем
{\it сложением} и обозначаем <<$+$>>, а также  для каждого $\lambda \in \mathbb{F}$ на 
$V$ определена унарная операция, которую называем {\it умножением} на $\lambda$. 
Результат применения умножения на $\lambda$ к  $\vek{a} \in V$ обозначаем $\lambda \cdot \vek{a}$ или $\lambda \vek{a}$.

\defin{$V$ называется  {\it векторным  пространством (или линейным пространством) 
над полем $\mathbb{F}$)}, если
выполнены следующие свойства V1---V8 (аксиомы векторного пространства):\\
V1. $(\vek a + \vek b)+\vek c = \vek a + (\vek b+\vek c)$ \,\,\,\,\,\, ($\forall$ $\vek{a}, \vek{b}, \vek{c} \in V$);\\
V2. $\exists$ $\vek{0}\in V$ $\forall$ $\vek{a} \in V$:   $\vek a + \vek{0}= \vek{0} + \vek{a} =\vek a$;\\
V3. $\forall$ $\vek{a} \in V$ $\exists$ $\vek x \in V$: $\vek x+ \vek{a}= \vek a+ \vek{x} = \vek{0}$;\\ 
V4. $\vek a + \vek b = \vek b + \vek a$ \,\,\,\,\,\, ($\forall$ $\vek{a}, \vek{b} \in V$);\\
V5. $\lambda(\vek a + \vek b)=\lambda \vek a+\lambda \vek b$ \,\,\,\,\,\, ($\forall$ $\vek{a}, \vek{b} \in V$, $\forall$ $\lambda \in \mathbb{F}$ );\\
V6. $(\lambda+\mu)\vek a=\lambda \vek a+\mu \vek a$ \,\,\,\,\,\, ($\forall$ $\vek{a} \in V$, $\forall$ $\lambda, \mu \in \mathbb{F}$ );\\
V7. $1\cdot \vek a=\vek a$  \,\,\,\,\,\, ($\forall$ $\vek{a} \in V$);\\
V8. $(\lambda \mu)\vek a = \lambda (\mu \vek a)$ \,\,\,\,\,\, ($\forall$ $\vek{a} \in V$, $\forall$ $\lambda, \mu \in \mathbb{F}$ ).
}


На протяжении главы $V$ будет обозначать векторное пространство.
Элементы векторного пространства $V$ будем называть {\it векторами} (независимо от их природы).
Элементы поля $\mathbb{F}$ будем иногда называть {\it константами} или {\it скалярами}.

Заметим, что аксиомы V1---V4 повторяют аксиомы абелевой группы.
Аксиомы V1 и V4 называются аксиомами {\it ассоциативности} и {\it коммутативности} сложения.
Вектор $\vek{0}$ из аксиомы V2 (нейтральный элемент относительно сложнения)
 называется {\it нулевым} вектором.\index{Вектор!нулевой} 
Вектор $\vek{x}$ из аксиомы V3  называется {\it противоположным} 
вектором для вектора $\vek{a}$ и обозначатся $-\vek{a}$.
Используя пртивоположный вектор, можно определить операцию {\it вычитания} равенством $\vek{a}-\vek{b}:= \vek{a}+(-\vek{b})$,
в частности, V2 принимает вид $\vek{a}-\vek{a}=\vek{0}$.

%V5 и V6 можно назвать свойством линейности, или дистрибутивности (по векторам и по константам соответственно),
%V7 --- условие нормировки, V8 --- смешанная ассоциативность умножения.\\
%В аксиомах V1---V4 встречается лишь операция сложения. Множество $V$ с операцией ''$+$'', удовлетворяющее только %аксиоме V1, называется
%{\it полугруппой}; %аксиомам V1 и V2 --- {\it полугруппой с единицей}
%удовлетворяющее аксиомам V1---V3 --- {\it группой};
%аксиомам V1---V4 --- {\it коммутативной}, или {\it абелевой} группой.


Из аксиом следуют привычные правила, которыми мы пользуемся, скажем, при работе с векторами в геометрии.
Например, аксиомы V1 и V4 обеспечивают то, что результат вычисления
суммы векторов $\sum\limits_{i=1}^{k} \vek{a}_i$ 
%линейной комбинации $\sum\limits_{i=1}^{k} \lambda_i \vek{a_i}$ 
не зависит от порядка выполнения операций.
Отметим следствия аксиом V1---V4.

\begin{predl}\label{sled_aksiom}
%1. Результат вычисления
%линейной комбинации $\sum\limits_{i=1}^{k} \lambda_i \vek{a_i}$ не зависит от порядка выполнения операций;
1). Нулевой вектор единственный;\\
2). $\forall$ $\vek{a}\in V$ противоположный ему вектор единственный;\\
3). (закон сокращения)
$\vek{a}+\vek{b}=\vek{a}+\vek{c}$
$\Leftrightarrow$ $\vek{b}=\vek{c}$ \,\,\,\,\,\, ($\forall$ $\vek{a}, \vek{b}, \vek{c} \in V$).
\end{predl}
\dok
1). Пусть $\vek{0}$ и $\vek{0'}$ --- два нулевых вектора. Тогда $\vek{0} = \vek{0} + \vek{0'}= \vek{0'}$, то есть $\vek{0}$ и $\vek{0'}$ совпадают.
\\
2). Пусть векторы $\vek{x}$ и $\vek{y}$ оба являются противоположными для вектора $\vek{a}$.  
Тогда $\vek{x} = \vek{x} + \vek{0}  = \vek{x} + (\vek{a} + \vek{y})  = (\vek{x} + \vek{a}) + \vek{y} = \vek{0}+\vek{y} \hm=\vek{y}$, 
то есть $\vek{x}$ и $\vek{y}$ совпадают.
\\
3). $\vek{a}+\vek{b}=\vek{a}+\vek{c}$ $\Rightarrow$ $-\vek{a}+(\vek{a}+\vek{b})=-\vek{a}+(\vek{a}+\vek{c})$
$\Rightarrow$ $(-\vek{a}+\vek{a})+\vek{b}=(-\vek{a}+\vek{a})+\vek{c}$
$\Rightarrow$ $\vek{0}+\vek{b}=\vek{0}+\vek{c}$
$\Rightarrow$ $\vek{b}=\vek{c}$.
Обратное следствие очевидно.
\edok

\otstup
Аксиомы V5 и V6 можно назвать свойством линейности, или дистрибутивности (по векторам и по константам соответственно),
V7 --- <<условие нормировки>>. Зафиксируем еще некоторые следствия аксиом.

\begin{predl}\label{sled_aksiom1}
1) $0\cdot \vek{a}  = \lambda \cdot \vek{0}= \vek{0}$ \,\,\,\,\,\, ($\forall$ $\vek{a}\in V$, $\forall$ $\lambda\in \mathbb{F}$);\\
2) $-(\lambda \vek{a})=(-\lambda )\vek{a} = \lambda (-\vek{a})$ \,\,\,\,\,\, ($\forall$ $\vek{a}\in V$, $\forall$ $\lambda\in \mathbb{F}$);\\
3)  $\lambda(\vek{a}-\vek{b}) = \lambda \vek{a}- \lambda \vek{b}$ \,\,\,\,\,\, ($\forall$ $\vek{a}, \vek{b}\in V$, $\forall$ $\lambda\in \mathbb{F}$);\\
4)  $(\lambda - \mu) \vek{a} = \lambda \vek{a}- \mu \vek{a}$\,\,\,\,\,\, ($\forall$ $\vek{a} \in V$, $\forall$ $\lambda , \mu \in \mathbb{F}$) .
\end{predl}
\dok
1)  $0\cdot \vek{a} = (0+0)\cdot \vek{a} = 0\cdot \vek{a} + 0\cdot \vek{a}$. 
С другой стороны, $0\cdot \vek{a} =\vek{0}+0\cdot \vek{a}$. По закону сокращения 
(см. предложение \ref{sled_aksiom}, 3) $0\cdot \vek{a}=\vek{0}$.
Аналогично 
$\lambda \cdot \vek{0} = \lambda (\vek{0}+\vek{0}) = \lambda \cdot \vek{0} + \lambda \cdot \vek{0}$, откуда $\lambda \cdot \vek{0}=\vek{0}$.
\\
2). $(-\lambda )\vek{a} + \lambda \vek{a} = (-\lambda  + \lambda) \vek{a} = 0\cdot \vek{a}  = \vek{0}$, поэтому 
$(-\lambda )\vek{a}$ является противоположным для $\lambda \vek{a}$, то есть равен $-(\lambda \vek{a})$.
Аналогично 
$\lambda (-\vek{a}) + \lambda \vek{a} = \lambda (-\vek{a}+ \vek{a}) = \lambda \cdot \vek{0}  = \vek{0}$, поэтому 
$\lambda (-\vek{a}) = -(\lambda \vek{a})$.
\\
3). Вытекает из утверждения 2) и определения вычитания векторов: 
$\lambda(\vek{a}-\vek{b}) = \lambda(\vek{a}+(-\vek{b})) = \lambda \vek{a}+ \lambda (-\vek{b}) =   \lambda \vek{a}- \lambda \vek{b}$.
\\
4). Аналогично 3).
\edok


\subsection{Подпространства}

\defin{
Непустое подмножество $U$ векторного пространства $V$ называется {\it подпространством}, если
$\forall$ $\vek{a}, \vek{b} \in U$, $\forall$ $\lambda\in \mathbb{F}$ выполнено:\\
П1. $\vek{a}+\vek{b} \in U$;\\
П2. $\lambda \vek{a} \in U$.
}

Тот факт, что $U$ является подпространством в векторном пространстве $V$, будем обозначать $U\leq V$.

Свойства П1 и П2 означают, что подпространство является подмножеством, замкнутым относительно
операций сложения и умножения на число, тем самым, подпространство само является векторным
пространством относительно операций в объемлющем векторном пространстве.
Любое векторное пространство $V$ содержит {\it тривиальные} подпространства $V$ и $O= \{ \vek{0} \}$ ({\it нулевое} подпространство).

\begin{predl}\label{nul_podprostr}
Пусть $U\leq V$, тогда $\vek{0} \in U$.
\end{predl}
\dok Пусть $\vek{a} \in V$, тогда, согласно П2, $\vek{0} =0\cdot \vek{a} \in V$.
\edok

\begin{predl}\label{minus_podprostr}
Пусть $U\leq V$ и $\vek{a}\in U$ тогда $-\vek{a} \in U$.
\end{predl}
\dok Достаточно в П2 подставить $\lambda = -1$.
\edok

%{\footnotesize Види}
\otstup
Видим, что подпространтво --- это подгруппа абелевой группы $(V, +)$, 
замкнутая относительно умножения на константы.


\begin{predl}\label{cap}
Пересечение подпространств является подпространством.
\end{predl}
\dok Пусть $U_i\leq V$ для $i\in I$, где $I$ --- некоторое множество индексов; $U=\bigcap\limits_{i\in I} U_i$.
Проверим П1 для множества $U$ (П2 проверяется аналогично).

Пусть $\vek{a}, \vek{b} \in U$, тогда $\vek{a}, \vek{b} \in U_i$ ($\forall$ $i\in I$).
Так как $U_i$ --- подпространство, то $\vek{a}+ \vek{b} \in U_i$ ($\forall$ $i\in I$), тем самым $\vek{a}+ \vek{b} \in U$, что и требовалось.
\edok

\otstup

{\bf Упражнение.}
Пусть $U_1\leq V$, $U_2\leq V$. Докажите, что объединение $U_1\cup U_2$ является подпространством
$\Leftrightarrow$ $U_1\subset U_2$ или $U_2\subset U_1$.



\subsection{Линейные комбинации}

Пусть $\vek{a}_1, \vek{a}_2, \ldots, \vek{a}_k \in V$,
$\lambda_1, \lambda_2, \ldots , \lambda_k\in \mathbb{F}$.

\defin{
Сумма $\sum\limits_{i=1}^{k} \lambda_i \vek{a}_i$
 называется
{\it линейной комбинацией}\index{Линейная!комбинация} векторов
$\vek{a}_1, \vek{a}_2, \ldots, \vek{a}_k$ с {\it коэффициентами}\index{Коэффициент!линейной!комбинации}
$\lambda_1, \lambda_2, \ldots , \lambda_k$.
}

Линейная комбинация $\sum\limits_{i=1}^{k} \lambda_i \vek{a_i}$ называется {\it тривиальной}, 
если все ее коэффициенты равны 0, то есть $\lambda_1=\ldots = \lambda_k = 0$.
В противном случае линейная комбинация называется {\it нетривиальной}.\index{Линейная!комбинация!нетривиальная}
%Если $|\lambda_1|+|\lambda_2|+ \ldots +|\lambda_k| >0$
%(то есть хотя бы один из коэффициентов не равен $0$), то говорят, что
%линейная комбинация (\ref{eq7_2_1}) {\it нетривиальная}
Ясно, что тривиальная линейная комбинация равна $\vek{0}$.

Количество слагаемых $k$ иногда называют {\it длиной} линейной комбинации. 
Можно позволить себе работать и с пустой линейной комбинацией ($k=0$), полагая ее значение равным $\vek{0}$.

Линейную комбинацию условимся записывать также в следующем компактном виде:
$\sum\limits_{i=1}^{k} \lambda_i \vek{a}_i =
\bazis{a} \lambda$, где $\bazis{a} = (\vek{a}_1\,  \vek{a}_2\,  \ldots \, \vek{a}_k)$ --- строка векторов, а
$\lambda  = \stolbec{\lambda_1 \\ \lambda_2 \\ \vdots \\ \lambda_k}$ --- столбец коэффициентов.
Это согласуется с правилом умножения матриц (умножаем строку на столбец).

%{\it footnotesize Другой вариант сокращенной записи --- тензорное суммирование. }

В том случае, когда $\vek{b}\in V$ равен некоторой линейной комбинации векторов
$\vek{a}_1, \vek{a}_2, \ldots, \vek{a}_k$ говорят, что
$\vek b$ {\it раскладывается} по векторам $\vek{a}_1, \vek{a}_2, \ldots, \vek{a}_k$
или {\it линейно выражается} через векторы $\vek{a}_1, \vek{a}_2, \ldots, \vek{a}_k$.



\begin{predl}\label{podpr_lin_komb}
Пусть $U\leq V$. Тогда  $\forall$ $\vek{a}_1, \vek{a}_2, \ldots, \vek{a}_k \in U$ и
$\forall$  $\lambda_1, \lambda_2, \ldots , \lambda_k\in \mathbb{F}$:
 $\sum\limits_{i=1}^{k} \lambda_i \vek{a}_i \in U$.
\end{predl}
\dok
Достаточно многократно применить П1 и П2
\edok

\otstup

Таким образом, условие замкнутости относительно взятия любой линейной комбинации %из последнего предложения 
эквивалентно определению подпространства (в П1 и П2 мы видим частные случаи линейных комбинаций: $\vek{a}+\vek{b}$ и $\lambda \vek{a}$).


\subsection{Линейная оболочка}

Важное понятие линейной оболочки, которое определим ниже, 
 позволяет, в частности, конструировать подпространства.

\defin{
{\it Линейной оболочкой} системы векторов $\vek{a}_1, \ldots, \vek{a}_k$ 
называется множество всех векторов, которые линейно выражаются через $\vek{a}_1, \ldots, \vek{a}_k$.
}

Линейная оболочка векторов $\vek{a}_1, \ldots, \vek{a}_k$ обозначается  $\lin{\vek{a}_1, \ldots, \vek{a}_k}$.
Формальная запись опеделения: \\
$\lin{\vek{a}_1, \ldots, \vek{a}_k} : = \left\{ \sum\limits_{i=1}^{k} \lambda_i \vek{a}_i \, | \,
 \lambda_1, \lambda_2, \ldots , \lambda_k \in \mathbb{F} \right\}$.

Несложно распространить определение на произвольные (в том числе бесконечные) системы векторов.

\defin{
{\it Линейной оболочкой} системы векторов $\mathcal{A}$ 
называется множество всех векторов, каждый из которых линейно выражается через несколько векторов из $\mathcal{A}$.\\
}

%Линейная оболочка множества $\mathcal{A}$ обозначается $\lin{\mathcal{A}}$.
Обозначение линейной оболочки: $\lin{\mathcal{A}}$. Очевидно, для любого подмножества
$\mathcal{A}\subset V$ выполнено $\mathcal{A}\subset \lin{\mathcal{A}}$.
Условимся считать, что $\lin{\varnothing} = O$.

Отметим, что везде рассматриваются линейные комбинации из конечного 
количества слагаемых, хотя длина линейной комбинации не ограничивается.
Формально, \\
$\lin{\mathcal{A}} = \left\{ \sum\limits_{i=1}^{k} \lambda_i \vek{a}_i \, | \,
k\in \mathbb{Z}^+, \, \vek{a}_1, \vek{a}_2, \ldots , \vek{a}_k \in \mathcal{A},
\, \lambda_1, \lambda_2, \ldots , \lambda_k \in \mathbb{F}\right\}$.


\begin{predl}\label{lin_ob}
$\lin{\mathcal{A}}$ является подпространством
для любой системы векторов $\mathcal{A}$.
\end{predl}
\dok Проверим для $\lin{\mathcal{A}}$ свойство П1 (П2 проверяется аналогично). 
Пусть $\vek{a}, \vek{b} \in \lin{\mathcal{A}}$. Это означает, что 
$\vek{a} = \sum\limits_{i=1}^{k} \alpha_i \vek{a}_i$, $\vek{b} = \sum\limits_{j=1}^{m} \beta_j \vek{b}_j$, 
где $\vek{a}_i, \vek{b}_j \in \mathcal{A}$, $\alpha_i, \beta_j \in \mathbb{F}$. Тогда 
$\vek{a} + \vek{b}= \sum\limits_{i=1}^{k} \alpha_i \vek{a}_i+ \sum\limits_{j=1}^{m} \beta_j \vek{b}_j$.
В правой части линейная комбинация длины $k+m$ векторов из $\mathcal{A}$, значит 
$\vek{a} + \vek{b}\in \lin{\mathcal{A}}$.
\edok

\otstup


Пусть подмножество $\mathcal{A}\subset V$ и подпространство $U\leq V$ таковы, что $\mathcal{A}\subset U$.
Предложение \ref{podpr_lin_komb} показывает, что тогда и $\lin{\mathcal{A}} \subset U$. 
Таким образом, предложение \ref{lin_ob} по сути означает, что %для любого подмножества $\mathcal{A}\subset V$ его линейная оболочка 
$\lin{\mathcal{A}}$ --- это минимальное (по включению) подпространство, 
содержащее $\mathcal{A}$.
\otstup

{\bf Упражнение.}
Подмножество $\mathcal{A}\subset V$ является подпространством 
$\Leftrightarrow$ $\mathcal{A} =\lin{\mathcal{A}}$.

%Ясно, что $\mathcal{A} \subset \lin{\mathcal{A}}$, причем  $\mathcal{A} = \lin{\mathcal{A}}$)
%$\Leftrightarrow$ $\mathcal{A}$ является подпространством.


%В частности, $\lin{\mathcal{A}} = \lin{\lin{\mathcal{A}}}$.
%упр. на подпространство: (аксиома замыкания...

%Упражнение.
%Отметим также, что если $\mathcal{B} \subset \lin{\mathcal{A}}$,
%то $\lin{\mathcal{B}} \subset \lin{\mathcal{A}}$.

%и вообще --- выполнены аксиомы операции замыкания...
%это потом используется в теореме о рангах... --- дважды линейно выражать...


\subsection{Примеры}

\example{I. %(Геометрия)
Пусть $O$ --- фиксированная точка геометрического пространства (начало отсчета).
%Геометрический пример векторного пространства --- р
Множество %$V_3$ 
всех радиус-векторов (с обычными операциями сложения векторов и умножения на число) --- пример векторного пространства над $\mathbb{R}$.
Радиус-вектор можно отождествить с точкой пространства --- концом этого радиус-вектора.
Тогда нетривиальные подпространства --- это прямые и плоскости, проходящие через $O$.\\
Если векторы $\vek{a}_1, \ldots, \vek{a}_k$ коллинеарны и среди них есть хотя бы один ненулевой, то $\lin{\vek{a}_1, \ldots, \vek{a}_k}$ 
--- это прямая. \\
Если же векторы $\vek{a}_1, \ldots, \vek{a}_k$ компланарны, но не коллинеарны, то $\lin{\vek{a}_1, \ldots, \vek{a}_k}$ 
--- это плоскость.
}

\example{%II. (Матрицы)\\
II.1. Множество $\mathbf{M}_{m\times n}(\mathbb{F})$ матриц $m\times n$ (заполненных константами)
--- векторное пространство относительно операций сложения матриц и умножения на число.\\
$\mathbb{F}^n$ отождествляем с множеством столбцов $\mathbf{M}_{n\times 1}(\mathbb{F})$.\\
Пусть
$\mathbf{M}_{n\times n}^{+} (\mathbb{F})= \{A\in \mathbf{M}_{n\times n}\,|\, A^T=A \}$ ---
множество симметричных матриц $n\times n$,\\
$\mathbf{M}_{n\times n}^{-} (\mathbb{F}) = \{A\in \mathbf{M}_{n\times n}\,|\, A^T=-A \}$ --- множество 
кососимметричных матриц $n\times n$.\\
Тогда $\mathbf{M}_{n\times n}^{+} (\mathbb{F}) 
\leq \mathbf{M}_{n\times n}(\mathbb{F})$, $\mathbf{M}_{n\times n}^{-} (\mathbb{F})\leq \mathbf{M}_{n\times n} (\mathbb{F})$.
}
\example{II.2.
Для матрицы $A\in \mathbf{M}_{m\times n}$ рассмотрим  однородную систему линейных уравнений (СЛУ) $AX=O$ (где $X\in \mathbf{M}_{n\times 1}$ --- столбец неизвестных)
 и множество ее решений $U= \Sol(AX=O)$. Тогда $U\leq \mathbf{M}_{n\times 1}$.
% --- подпространство в векторном пространстве $\mathbb{F}^n = M_{n\times 1}$. 
\\
Если $A_iX=O$, $i=1, 2, \ldots , k$ --- несколько однородных СЛУ относительно $X\in \mathbf{M}_{n\times 1}$, 
то СЛУ, представляющая собой объединение этих систем, имеет в качестве множества решений пересечение $\bigcap\limits_{i=1}^{k} \Sol(A_iX=O)$.
}

\example{%III. (Функции)\\
III.1. Пусть $S$ --- некоторое множество. Множество  
$\mathbf{F}(S)$ всех функций $f: S \to \mathbb{R}$ --- векторное пространство над $\mathbb{R}$
относительно операций сложения функций и умножения функции на число.\\
Пусть $\mathbf{F}^{+}(\mathbb{R}) = \{f\in \mathbf{F}(\mathbb{R}) \, |\, \forall\, x\in \mathbb{F} \,\,f(-x)=f(x)\}$ --- 
множество всех четных функций, аналогично\\
$\mathbf{F}^{-}(\mathbb{R}) = \{f\in \mathbf{F}(\mathbb{R}) \, |\, \forall\, x\in \mathbb{R} \,\, f(-x)=-f(x)\}$ --- 
множество всех нечетных функций. \\Тогда 
$\mathbf{F}^{+}(\mathbb{R})\leq \mathbf{F}(\mathbb{R})$, $\mathbf{F}^{-}(\mathbb{R})\leq \mathbf{F}(\mathbb{R})$.
}
\example{III.2.
Если $S$ --- интервал числовой прямой, имеется цепочка вложенных подпространств
$\mathbf{F}(S)\geq \mathbf{C}(S) \geq \mathbf{C}^1(S) \geq \ldots \geq \mathbf{C}^k(S) \geq \ldots \geq \mathbf{C}^{\infty}(S) \geq 
\mathbf{P}(S)$, \\где 
$\mathbf{C}(S)$ --- множество непрерывных функций $S \to \mathbb{R}$, 
$\mathbf{C}^k(S)$ --- множество функций, имеющих непрерывную $k$-ю производную,
$\mathbf{P}(S)$ --- множество полиномиальных функций (многочленов).\\ 
Множество (формальных) многочленов $\mathbf{P}$ может быть задано линейной оболочкой: 
$\mathbf{P} = \lin{1, x, x^2, x^3, \ldots}$.\\
$\mathbf{P} \geq \mathbf{P}_n=\lin{1, x, x^2, \ldots, x^n}$, где 
$\mathbf{P}_n$ --- множество многочленов $f$ степени $\deg f\leq n$.
}
\example{III.3.
В пространстве $\mathbf{F}(\mathbb{N})=\{(f_1, f_2, \ldots) \, | \, f_i\in \mathbb{R}\}$ 
всех последовательностей есть подпространство ограниченных последовательностей, а в нем --- подпространство сходящихся 
последовательностей.\\
<<Бесконечная система линейных уравнений>> $f_{i+1}=f_i+f_{i-1}$, $i=2, 3, \ldots$, задает подпространство {\it фибоначчиевых} последовательностей.
}
\example{III.4.
Рассмотрим линейное однородное дифференциальное уравнение $x^{(n)}+ a_{n-1}(t)x^{(n-1)}+\ldots + a_0(t)x=0$ относительно неизвестной функции 
$x(t)$ (где $a_i(t)$ --- данные непрерывные функции $\mathbb{R}\to \mathbb{R}$). 
Множество решений этого уравнения --- подпространство в пространстве всех  
 функций $\mathbb{R}\to \mathbb{R}$.
%Подпространствами (в соответсвующих пространствах) являются и множество решений однородной системы диффуров, УРЧПов
}

\example{IV.
Если $\mathbb{F}$ --- подполе некоторого поля $\mathbb{K}$, то операции в $\mathbb{K}$
индуцируют на $\mathbb{K}$ структуру векторного пространства над $\mathbb{F}$.
%размерность --- степень расширения. --- позже или можно не говорить
}

\section{Линейная зависимость. Размерность и ранг. Базис}

\subsection{Линейная зависимость}

%%%%%%%%%%%%%%%%%
%%% сразу символическую матричную запись линейной комбинации???
%%%%%%%%%%%%%%%%%

Продолжаем работать в векторном пространстве $V$ над полем $\mathbb{F}$.


\defin{
Система векторов  $\vek{a}_1, \vek{a}_2, \ldots, \vek{a}_k $
называется {\it линейно зависимой}\index{Линейная!зависимость},
если некоторая их нетривиальная линейная комбинация равна
$\vek{0}$, и {\it линейно независимой} в противном случае.
}

Полагают, что пустая система векторов линейно независима
(формально это согласуется с определением).


\begin{predl}\label{p7_2_1}
Система векторов $\vek{a}_1, \vek{a}_2, \ldots, \vek{a}_k \in V$ ($k\geq 2$)
линейно зависима $\Leftrightarrow$ 
среди векторов $\vek{a}_1, \vek{a}_2, \ldots, \vek{a}_k$ {\bf найдется} вектор, который
%$\exists $ $i\in \{1, 2, \ldots, k\}$: вектор
линейно выражается через остальные $k-1$ векторов этой системы.
\end{predl}
\dok \dokright Пусть $\lambda_1 \vek{a}_1 +\lambda_2 \vek{a}_2+ \ldots + \lambda_k \vek{a}_k
=\vek{0}$, и не все коэффициенты равны $0$,
скажем $\lambda_k \neq 0$. Тогда поделим равенство на $-\lambda_k$
и перенесем $\vek{a}_k$ в другую часть; получим
$\vek{a}_k = \sum\limits_{i=1}^{k-1}\mu_i \vek{a}_i$, где
$\mu_i=-\dfrac{\lambda_i}{\lambda_k}$, $i=1, 2, \ldots , k-1$.

\dokleft Пусть, скажем, вектор $\vek{a}_k$ раскладывается по векторам
$\vek{a}_1, \vek{a}_2, \ldots, \vek{a}_{k-1} $:
$\vek{a}_k = \sum\limits_{i=1}^{k-1}\mu_i \vek{a}_i$. Тогда
$\mu_1 \vek{a}_1 +\mu_2 \vek{a}_2+ \ldots + \mu_{k-1} \vek{a}_{k-1}-\vek{a}_k$ --- нетривиальная
линейная комбинация, равная $\vek{0}$.
\edok

\begin{predl}\label{p7_2_2}
1) Если в конечной системе векторов
имеется некоторая линейно зависимая подсистема, то и вся система линейно зависима.

2) Любая подсистема линейно независимой системы линейно независима.
\end{predl}
\dok 1) Пусть, скажем, для системы векторов $\vek{a}_1, \vek{a}_2, \ldots, \vek{a}_k$
ее подсистема $\vek{a}_1, \vek{a}_2, \ldots, \vek{a}_m$ ($m\leq k$)
линейно зависима. Тогда существует нетривиальная линейная комбинация
$\sum\limits_{i=1}^{m}\mu_i \vek{a}_i$, равная $\vek{0}$.
Значит, 
$\mu_1 \vek{a}_1 +\mu_2 \vek{a}_2+ \ldots + \mu_{m} \vek{a}_{m} +
0\cdot \vek{a}_{m+1} + \ldots + 0\cdot \vek{a}_k$ --- нетривиальная линейная комбинация, равная $\vek{0}$.

2) Это переформулировка утверждения 1).
\edok

\begin{sled}
Любая система векторов, содержащая $\vek{0}$, является линейно зависимой.
\end{sled}

\begin{predl}\label{p7_2_3}
Пусть векторы $\vek{a}_1, \vek{a}_2, \ldots, \vek{a}_k $ и $\vek{b}$ таковы, что 
$\vek{b}\in \lin{\vek{a}_1, \vek{a}_2, \ldots, \vek{a}_k }$.
Тогда коэффициенты $\lambda_1, \lambda_2, \ldots , \lambda_k$
в разложении
$\vek{b}= \sum\limits_{i=1}^{k} \lambda_i \vek{a}_i$
определяются однозначно $\Leftrightarrow$ 
система $\vek{a}_1, \vek{a}_2, \ldots, \vek{a}_k $ линейно независима.
\end{predl}
\dok
\dokright
Предположим, что напротив, 
система $\vek{a}_1, \vek{a}_2, \ldots, \vek{a}_k $ линейно зависима и существует нетривиальная линейная комбинация
$\sum\limits_{i=1}^{k} \alpha_i \vek{a_i}$, равная $\vek{0}$. 
Тогда можно прибавить эту линейную комбинацию к имеющейся линейной комбинации, равной $\vek{b}$, 
и получить новое линейное выражение: 
$\vek{b}= \sum\limits_{i=1}^{k} (\lambda_i+\alpha_i) \vek{a_i}$ (оно действительно хотя бы в одном
коэффициенте отличается от разложения
$\vek{b}= \sum\limits_{i=1}^{k} \lambda_i \vek{a_i}$).

\dokleft
Пусть система $\vek{a}_1, \vek{a}_2, \ldots, \vek{a}_k $ линейно независима, и предположим, что наряду с разложением
$\vek{b}= \sum\limits_{i=1}^{k} \lambda_i \vek{a}_i$
имеется разложение
$\vek{b}= \sum\limits_{i=1}^{k} \mu_i \vek{a}_i.$
Вычитая из первого равенства второе,  получаем
$\sum\limits_{i=1}^{k} (\lambda_i -\mu_i) \vek{a}_i=\vek{0}$.
Так как $\vek{a}_1, \vek{a}_2, \ldots, \vek{a}_k$ --- линейно независимая система, то левая часть
полученного равенства --- тривиальная линейная комбинация, откуда
$\lambda_i =\mu_i$, $i=1, 2, \ldots , k$.
\edok

\otstup

Пусть $\bazis{a} = (\vek{a}_1, \vek{a}_2, \ldots , \vek{a}_k)$ --- строка векторов, 
где $\vek{a}_1, \vek{a}_2, \ldots , \vek{a}_k$ --- линейно независимая система.
Тогда предыдущее предложение можно интерпретировать как закон сокращения:
 $\bazis{a} \lambda = \bazis{a} \mu$ $\Rightarrow$ $\lambda = \mu$.
Этот закон выполнен как для столбцов $\lambda , \mu \in \mathbf{M}_{k\times 1}$, так и для матриц $\lambda , \mu \in \mathbf{M}_{k\times m}$.
Обратим внимание на то, что если отказаться от условия линейной независимости системы
$\vek{a}_1, \vek{a}_2, \ldots , \vek{a}_k$, то следствие $\bazis{a} \lambda = \bazis{a} \mu$ $\Rightarrow$ $\lambda = \mu$,
вообще говоря, неверно.

\subsection{Ранг и размерность}

\defin{
Целое неотрицательное число $r$ называется {\it рангом}\index{Ранг}
непустой системы (возможно, бесконечной) $\mathcal{A}$ векторов из $V$
если в системе $\mathcal{A}$ найдется линейно
независимая подсистема из $r$  векторов, а любая подсистема из $r+1$ векторов является
линейно зависимой.  
}
\defin{
Будем говорить, что $\mathcal{A}$ имеет бесконечный ранг, если для любого $r\in \mathbb{N}$
в $\mathcal{A}$ найдется линейно независимая подсистема из $r$ векторов.
}

Обозначение для ранга: $\rg \mathcal{A}$. 
 В частности,
 $\rg(\vek{a}_1, \vek{a}_2, \ldots, \vek{a}_k)$ --- ранг конечной системы
векторов $\vek{a}_1, \vek{a}_2, \ldots, \vek{a}_k$.

В том случае, когда $\mathcal{A}$ является подпространством в $V$, более употребительное
название для ранга ---
{\it размерность}\index{Размерность}. Обозначение для размерности --- $\dim \mathcal{A}$.
Итак, если $U\leq V$, то $\boxed{\dim U = \rg U}$.
Пространство размерности $k$ называют $k$-{\it мерным}.
Если $\dim V < \infty$, то пространство $V$ называют {\it конечномерным}, иначе --- {\it бесконечномерным}.

Очевидно, система из одного нулевого вектора имеет ранг 0,
а ранг любой конечной системы из $k$ векторов не превосходит $k$.

\begin{predl}\label{p7_2_4}
Конечная система векторов $\vek{a}_1, \vek{a}_2, \ldots, \vek{a}_k $ линейно независима $\Leftrightarrow$
$\rg (\vek{a}_1, \vek{a}_2, \ldots, \vek{a}_k) = k$.
\end{predl}
\dok Сразу следует из определения.
\edok

\otstup

\begin{predl}\label{p7_2_5} Пусть $\mathcal{A}_1, \mathcal{A}_2$ --- две системы векторов из
$V$, причем
$\rg \mathcal{A}_1 = r_1$, $\rg \mathcal{A}_2 = r_2$. Тогда
$\rg(\mathcal{A}_1 \bigcup \mathcal{A}_2) \leq r_1+r_2$.
\end{predl}
\dok 
Пусть это не так, и в объединении наборов
$\mathcal{A}_1$ и $\mathcal{A}_2$ нашлась
линейно независимая подсистема из $r_1+r_2+1$ векторов.
Но (по определению ранга и предложению \ref{p7_2_2})
среди этих векторов не более $r_1$ векторов из $\mathcal{A}_1$
и не более $r_2$ векторов из $\mathcal{A}_2$. Противоречие.
\edok

\begin{sled}\label{sled7_2_5} Пусть $\mathcal{A}_i$, $i=1, 2, \ldots, k$ --- системы векторов из
$V$, причем
$\rg \mathcal{A}_i = r_i$. Тогда
$\rg(\mathcal{A}_1 \bigcup \ldots \bigcup \mathcal{A}_k) \leq \sum\limits_{i=1}^k r_i$.
\end{sled}



\begin{predl}\label{p5_2_5} 
Пусть $\mathcal{A}, \mathcal{B}$ --- две системы векторов из $V$, причем $\mathcal{A} \subset \mathcal{B}$ и
$\rg \mathcal{B} = r$. Тогда $\rg \mathcal{A} \leq r$.
\end{predl}
\dok Сразу следует из определения.
\edok

\otstup

Предыдущее предложение почти очевидно: если к системе векторов $\mathcal{A}$ добавить
некоторые векторы (расширить $\mathcal{A}$ до системы $\mathcal{B}$), то ранг не уменьшится.
Оказывается, ранг не изменится, если к $\mathcal{A}$ добавлять линейные комбинации векторов из
$\mathcal{A}$ (и наоборот, ранг не изменится, если из системы векторов удалить вектор,
который раскладывается по оставшимся векторам).
%
%Может ли ранг при этом остаться неизменным, и если да, то при каком условии?
%Оказывается, при добавлении к системе $\mathcal{A}$ некоторых столбцов ранг не изменится
%тогда и только тогда,
%когда каждый из добавляемых столбцов линейно выражается через столбцы системы
%$\mathcal{A}$.
Обобщение этого факта составляет содержание следующей основной теоремы о рангах.
%Теперь мы докажем основную теорему о рангах.
Доказательству теоремы предпошлем две леммы (которые являются частными
случаями теоремы).

\begin{lem1} Пусть $\mathcal{A}$ --- система векторов из $V$, 
$\rg \mathcal{A} = r<\infty $ и $\vek{a}_1, \ldots , \vek{a}_r$ ---  
линейно независимая подсистема векторов из $\mathcal{A}$.
Тогда любой вектор из $\mathcal{A}$ раскладывается по $\vek{a}_1, \ldots , \vek{a}_r$.
\end{lem1}
\dok Пусть $\vek{a}$ --- произвольный вектор из $\mathcal{A}$.
%Достаточно доказать, что $A$ линейно выражается через столбцы $A_1, $ $A_2, \hm\ldots , $ $A_r$.
%
%Это утверждение очевидно, если $A$ совпадает с одним из столбцов
%$A_1, A_2, \hm\ldots , A_r$.
Из определения ранга следует, что система из $r+1$ векторов  $\vek{a}, $ $\vek{a}_1, \ldots , \vek{a}_r$
линейно зависима. Тогда найдутся
числа $\mu, $ $\lambda_1, \hm\ldots , $ $\lambda_r\in \mathbb{R}$, не все равные нулю и такие, что
%$|\mu|+ \sum\limits_{i=1}^{r}|\lambda_i|>0$ и
$\mu \vek{a}+ \sum\limits_{i=1}^{r} \lambda_i \vek{a}_i = \vek{0}$. При этом $\mu \neq 0$, иначе система
$\vek{a}_1, \ldots , \vek{a}_r$ была бы линейно зависимой. Отсюда
$\vek{a} = - \sum\limits_{i=1}^{r} \dfrac{\lambda_i}{\mu} \vek{a}_i$.
\edok

\otstup

\begin{lem2} Пусть $\vek{a}_1, \ldots , \vek{a}_k$, $\vek{b}$ --- такие векторы из $V$, что 
$\vek{b}\in \lin{\vek{a}_1, \hm\ldots , \vek{a}_k}$.
Тогда
$\rg (\vek{a}_1, \hm\ldots , \vek{a}_k) \hm= \rg (\vek{a}_1, \hm\ldots , \vek{a}_k, \vek{b})$.
\end{lem2}
\dok %Утверждение очевидно, если $B$ совпадает с одним из столбцов
%$A_1, A_2, \hm\ldots , A_k$.
Положим $r=\rg (\vek{a}_1, \hm\ldots , \vek{a}_k)$ ($r\leq k$).
Предположим, что утверждение неверно, и в системе $\vek{a}_1, \hm\ldots , \vek{a}_k, \vek{b}$ нашлась
линейно независимая подсистема из $r+1$ векторов. Тогда один из этих
$r+1$ векторов --- это $\vek{b}$ (так как в системе $\vek{a}_1, \hm\ldots , \vek{a}_k$ нет линейно независимой подсистемы из
$r+1$ векторов).
Итак, пусть для определенности $\vek{b}, \vek{a}_1, \hm\ldots , \vek{a}_r$ --- линейно независимая система.
Из предложения \ref{p7_2_2} следует, что система $\vek{a}_1, \hm\ldots , \vek{a}_r$  линейно независима,
значит, по лемме 1 каждый из векторов $\vek{a}_1, \hm\ldots , \vek{a}_k$
лежит в $\lin{\vek{a}_1, \hm\ldots , \vek{a}_r}$.
%равен линейной комбинации векторов-столбцов $A_1, A_2, \hm\ldots , A_r$.
По условию $\vek{b}$ равен линейной комбинации векторов $\vek{a}_1, \hm\ldots , \vek{a}_k$:
$\vek{b} = \sum\limits_{i=1}^{k}\lambda _i \vek{a}_i$.
Подставив в это выражение %$B$ по векторам-столбцам $A_1, A_2, \hm\ldots , A_k$
вместо векторов  $\vek{a}_1, \hm\ldots , \vek{a}_k$ их разложения по векторам $\vek{a}_1, \hm\ldots , \vek{a}_r$, получим, что
$\vek{b}$ раскладывается по векторам $\vek{a}_1, \hm\ldots , \vek{a}_r$.
Но это противоречит линейной независимости
системы $\vek{b}, \vek{a}_1, \hm\ldots , \vek{a}_r$  (см. предложение \ref{p7_2_1}).
\edok

\otstup

\begin{theor}[основная теорема о рангах]\label{t5_2_1}\index{Теорема!основная о рангах}
Пусть $\mathcal{A}$ и $\mathcal{B}$ --- две такие системы векторов из $V$, что $\mathcal{A} \subset \mathcal{B}$.
Пусть $\rg \mathcal{A} = r< \infty$. Тогда $\rg \mathcal{B} = r$ $\Leftrightarrow$
любой вектор из $\mathcal{B}$ принадлежит $\lin{\mathcal{A}}$.
%линейно выражается через несколько %(конечное число)
%векторов-столбцов из $\mathcal{A}$.
\end{theor}
\dok
\dokright В системе $\mathcal{A} $ зафиксируем некоторую линейно независимую подсистему
$\vek{a}_1, \ldots, \vek{a}_r$ из $r$ векторов.
Так как $\vek{a}_1, \ldots, \vek{a}_r\hm\in \mathcal{B}$ и $\rg \mathcal{B} = r$,
то по лемме 1 (примененной к системе $\mathcal{B}$) любой вектор из $\mathcal{B}$ лежит в
$\lin{\vek{a}_1, \ldots, \vek{a}_r}$ и, следовательно, в $\lin{\mathcal{A}}$.

\dokleft Предположим, что утверждение неверно, и в системе $\mathcal{B}$ нашлась
линейно независимая подсистема из $r+1$  векторов $\vek{b}_1, \vek{b}_2, \hm\ldots, \vek{b}_{r+1}$.
Каждый из них линейно выражается через несколько (конечное число) векторов из
$\mathcal{A}$, поэтому   можно выбрать конечную
систему векторов $\vek{a}_1, \vek{a}_2, \hm\ldots , \vek{a}_l$ из $\mathcal{A}$,
через которые линейно выражается каждый из векторов
$\vek{b}_1, \vek{b}_2, \hm\ldots, \vek{b}_{r+1}$.
Имеем $\rg (\vek{a}_1, \vek{a}_2, \hm\ldots , \vek{a}_l, \vek{b}_1, \vek{b}_2, \hm\ldots, \vek{b}_{r+1}) \hm\geq
\rg (\vek{b}_1, \vek{b}_2, \hm\ldots, \vek{b}_{r+1}) \hm= r\hm+ 1$.
С другой стороны, применяя многократно лемму 2, имеем:
$\rg (\vek{a}_1, \vek{a}_2, \hm\ldots , \vek{a}_l, \vek{b}_1, \vek{b}_2, \hm\ldots, \vek{b}_{r+1}) \hm=
\rg (\vek{a}_1, \vek{a}_2, \hm\ldots , \vek{a}_l, \vek{b}_1, \vek{b}_2, \hm\ldots, \vek{b}_{r}) \hm=
\rg (\vek{a}_1, \vek{a}_2, \hm\ldots , \vek{a}_l, \vek{b}_1, \vek{b}_2, \hm\ldots, \vek{b}_{r-1}) \hm= \hm\ldots \hm=
\rg (\vek{a}_1, \vek{a}_2, \hm\ldots , \vek{a}_l) \hm\leq \rg \mathcal{A} \hm= r$. Противоречие.
\edok

%\otstup

%В ПРИМЕРАХ БУДЕТ!!!!!
%\begin{sled2}
%Для любой системы  $\mathcal{A}$  векторов-столбцов из $\mathbf{M}_{m\times 1}$
%ее ранг определен, причем $\rg \mathcal{A}\hm\leq m$. %$\rg \mathbf{M}_{1\times m} = m$.
%\end{sled2}
%\dok
%Очевидно, $\rg \mathcal{A}\hm\leq \rg \mathbf{M}_{m\times 1}$.
%Согласно предложению \ref{p5_2_3'}, в множестве $\mathbf{M}_{m\times 1}$ всех столбцов
%высоты $m$ имеется базисная подсистема из $m$ столбцов.
%Отсюда $\rg \mathbf{M}_{m\times 1} = m$.
%%Значит,
%%нельзя выбрать линейно независимую систему из более чем $m$ столбцов.
%\edok

%\otstup

\begin{sled1}
Для любой системы векторов $\mathcal{A} $ из $V$
выполнено $\rg \mathcal{A} = \dim \lin{\mathcal{A}}$.
\end{sled1}

\begin{sled2}
Пусть даны подпространства $U\leq W\leq V$ такие, что $\dim U = \dim W <\infty $. Тогда $U=W$.
\end{sled2}


%\otstup

\defin{
%Упорядоченную конечную 
Подсистему $\vek{a}_1, \ldots , \vek{a}_k$
системы векторов $\mathcal{A}$ будем называть {\it базисной}, %для подмножества $\mathcal{A}\subset V$,
если\\
1. $\vek{a}_1, \ldots , \vek{a}_k$ --- линейно независима,\\
2. любой вектор из $\mathcal{A}$ принадлежит линейной оболочке $\lin{\vek{a}_1, \ldots , \vek{a}_k}$.
}


%\begin{sled1}[описание базисных подсистем]
%Пусть $\mathcal{A}$ --- система векторов из $V$,
%и $\rg \mathcal{A} = r<\infty $.
%Тогда базисными для $\mathcal{A}$ являются
%в точности линейно независимые подсистемы из $r$ векторов.
%%Подсистема $A_1, $ $A_2, $ $\hm\ldots ,$ $k$ в $\mathcal{A}$ является базисной
%%$\Leftrightarrow$
%%она линейно независима и $k=r$.
%\end{sled1}


\begin{theor}\label{t7_2_2}
Пусть $\rg \mathcal{A} = r<\infty $. Тогда подсистема $\vek{a}_1, \vek{a}_2, \ldots , \vek{a}_k$ системы $\mathcal{A}$ является базисной для $\mathcal{A}$
$\Leftrightarrow$ $\vek{a}_1, \vek{a}_2, \ldots , \vek{a}_k$ линейно независима и  $k=r$.
\end{theor}
\dok Сразу следует из основной теоремы \ref{t5_2_1}.
\edok

\begin{predl}\label{p7_2_6}
Пусть $\rg \mathcal{A} = r<\infty $. Пусть $\vek{a}_1, \vek{a}_2, \ldots , \vek{a}_k$ --- линейно независимая подсистема 
системы $\mathcal{A}$. 
%Пусть $\rg \mathcal{A} = r< \infty$.
Тогда систему $\vek{a}_1, \vek{a}_2, \ldots , \vek{a}_k$ можно дополнить $r-k$ векторами до
базисной подсистемы. 
%линейно независимой системы $\vek{a}_1, \vek{a}_2, \ldots , \vek{a}_k, \vek{a}_{k+1}, \ldots , \vek{a}_r$
%векторов из $\mathcal{A}$.
\end{predl}
\dok
Если $k<r$, то $k= \rg (\vek{a}_1, \vek{a}_2, \ldots , \vek{a}_k ) < \rg \mathcal{A}$,
и по основной теореме \ref{t5_2_1} в $\mathcal{A}$ найдется вектор $\vek{a}_{k+1}$,
который не выражается линейно через $\vek{a}_1, \vek{a}_2, \ldots , \vek{a}_k$.
Тогда $\rg (\vek{a}_1, \vek{a}_2, \ldots , \vek{a}_{k+1} ) > k$, следовательно %(см. предложение \ref{p7_2_4})
$\vek{a}_1, \vek{a}_2, \ldots , \vek{a}_{k+1}$ --- линейно независимая подсистема.
Продолжая процесс добавления векторов, в конце концов придем к базисной подсистеме.
\edok

\otstup

{\bf Упражнение.}$^*$ Теорию ранга и размерности можно развить, начиная с другого (эквивалентного)
определения ранга: рангом подмножества $\mathcal{A}\subset V$ можем называть 
минимальное $r$ такое, что существует подмножество $\mathcal{B}$ такое, что $|\mathcal{B}|=r$ 
и $\mathcal{A} \subset \lin{\mathcal{B}}$.

\subsection{Базис}

%Будем рассматривать лишь конечные базисы.

\defin{ Упорядоченный конечный набор векторов $\vek{e}_1, \vek{e}_2, \ldots , \vek{e}_n$
называется {\it базисом} векторного пространства $V$, если \\
B1. $\vek{e}_1, \vek{e}_2, \ldots , \vek{e}_n$ --- линейно независимая система;\\
B2. $V= \lin{\vek{e}_1, \vek{e}_2, \ldots , \vek{e}_n}$.
}

Видим, что определение базиса согласуется с определением базисной подсистемы.

Базисы часто будем для краткости обозначать одной буквой, имея в виду упорядоченный набор векторов или 
строку из векторов, например: $\bazis{e} = (\vek{e}_1, \vek{e}_2, \ldots, \vek{e}_n)$.

\begin{theor}[Описание базисов]\label{t7_3_1}
Пусть $\dim V = n< \infty$. Тогда 
упорядоченный набор векторов  $\vek{e}_1, \ldots , \vek{e}_k$ является базисом пространства $V$
$\Leftrightarrow$ система $\vek{e}_1, \ldots , \vek{e}_k$ линейно независима и  $k=n$.
%1. Любой упорядоченный набор из $n$ линейно независимых векторов является базисом; \\
%2. Любой базис содержит ровно $n$ векторов.
\end{theor}
\dok Следует из теоремы \ref{t7_2_2}.
%определения базиса и основной теоремы о рангах.
\edok
\otstup

Существование базисов в векторных пространствах конечной размерности
следует из предыдущей теоремы. Если $\dim V = n<\infty$, то каждый базис пространства $V$
содержит ровно $n$ векторов.
Если $\dim V = \infty$, то конечного базиса в $V$ не существует
(в этом курсе мы не будем заниматься понятием бесконечного базиса).


\begin{predl}\label{p7_3_1}
Пусть $\dim V = n < \infty$, и
$\vek{e}_1, \vek{e}_2, \ldots , \vek{e}_k$ --- линейно независимая система векторов.
Тогда эту систему можно дополнить до
базиса $\vek{e}_1, \vek{e}_2, \ldots , \vek{e}_k, \vek{e}_{k+1}, \ldots , \vek{e}_n$
пространства~$V$.
\end{predl}
\dok
Следует из предложения \ref{p7_2_6}.
\edok

\otstup

\subsection{Координаты}

\defin{
Пусть в векторном пространстве $V$ зафиксирован базис
$\bazis{e} = (\vek{e}_1, \vek{e}_2, \ldots, \vek{e}_n)$.
Коэффициенты $x_1, x_2, \ldots , x_n$ в разложении
$\vek{a}= x_1\vek{e}_1+x_2 \vek{e}_2+ \ldots+x_n \vek{e}_n$
вектора $\vek{a}\in V$ по этому базису называются
{\it координатами}\index{Координаты!вектора} вектора $\vek{a}$ в базисе $\bazis{e}$.
}

Из предложения \ref{p7_2_3} следует, что упорядоченный набор координат
данного вектора $\vek{a}$ в данном базисе однозначно определен.
Упорядоченный набор координат
$(x_1, x_2, \ldots, x_n)$ удобно записывать в виде столбца:
$X = \stolbec{x_1\\ x_2\\ \vdots \\ x_n}$. Этот столбец называется {\it координатным столбцом}\index{Столбец!координатный}
вектора $\vek{a}$ в базисе~$\bazis{e}$.
Для любого упорядоченного
набора координат имеется вектор из $V$ именно с таким набором координат.
Таким образом, если в векторном
пространстве $V$ зафиксирован базис $\bazis{e}$, %$ = (\vek{e}_1, \vek{e}_2, \ldots, \vek{e}_n)$,
то имеется взаимно-однозначное соответствие между множеством $V$ и 
множеством $\mathbf{M}_{n\times 1}$ (вектору сопоставлятся координатный столбец).
Запись $\vek{a} = \bazis{e} X$
будет означать, что вектор $\vek{a}$ имеет координатный столбец  $X$  в базисе
$\bazis{e}$. 

{\footnotesize Также  разложения вида
$\sum\limits_{i=1}^nx_i\vek{e}_i$
можно записывать в виде так называемого {\it тензорного суммирования} --- следующей договоренности, предложенной Эйнштейном: 
наличие одинаковых верхнего и нижнего индекса означает суммирования по этому индексу (от 1 до $n=\dim V$).
Так, если в координатах $x_i$ вместо нижних индексов использовать верхние, то запись $\sum\limits_{i=1}^nx^i\vek{e}_i$
 можно сократить до $x^i\vek{e}_i$.
}
%(эта запись согласуется с символическим умножением матриц:
%$(\vek{e}_1 \, \vek{e}_2 \, \ldots \, \vek{e}_n) \stolbec{
%x_1\\ x_2\\ \vdots \\ x_n}$ ).

Следующее предложение показывает, что записи вида
$\vek{a} = \bazis{e}X $ согласуются с дистрибутивностью матричного умножения.



\begin{predl}[линейность сопоставления координат]\label{p7_3_2}
Пусть в $V$ зафиксирован базис $\bazis{e} = (\vek{e}_1, \vek{e}_2, \ldots, \vek{e}_n)$.
Тогда при сложении векторов соответствующие координаты складываются,
а при умножении вектора на число $\lambda\in \mathbb{R}$ соответствующие
координаты умножаются на $\lambda$.
\end{predl}
\dok По условию $\vek{a}=\bazis{e}X = \sum\limits_{i=1}^nx_i\vek{e}_i$,
$\vek{b}=\bazis{e}Y = \sum\limits_{i=1}^ny_i\vek{e}_i$.\\
Сложив равенства, имеем
$\vek{a}+\vek{b}=\sum\limits_{i=1}^n(x_i+y_i)\vek{e}_i = \bazis{e} (X+Y) $.\\
Умножив первое равенство на $\lambda$, имеем
$\lambda \vek{a}=\sum\limits_{i=1}^n(\lambda x_i )\vek{e}_i = \bazis{e} (\lambda X)$.
\edok

\otstup

Итак, при фиксации базиса $\bazis{e}$ равенство $\vek{a} = \bazis{e} X$ возникает важное линейное взаимно-однозначное соответствие (ниже для таких соответствий вводится термин <<изоморфизм>>) между $V$ и $\mathbf{M}_{n\times 1}$. % --- сопоставление вектору его координатного столбца.
Эта <<координатизация>> дает возможность вопросы о линейных операциях в $V$ перевести на матричный язык
(вместо абстрактных элементов $V$ можно заниматься столбцами). В частности, отметим такое следствие
предложения \ref{p7_3_2} (оно согласуется с общими свойствами изоморфизма --- см. главу \ref{lin_otobr}).

\begin{sled}\label{sled7_3_2}
Пусть в $V$ зафиксирован базис $\bazis{e}$, и в этом базисе векторы $\vek{a}_1, \ldots, \vek{a}_k$
имеют координатные столбцы $X_1,\ldots, X_k$ соответственно: $\vek{a}_i = \bazis{e} X_i$, $i=1, \ldots, k$.
Тогда система векторов $\vek{a}_1, \ldots, \vek{a}_k$ линейно зависима $\Leftrightarrow$
система столбцов $X_1, \ldots, X_k$ линейно зависима.
\end{sled}
\dok 
Согласно предложению \ref{p7_3_2}, 
$\sum\limits_{i=1}^k \lambda_i\vek{a}_i = \vek{0}$ $\Leftrightarrow$ 
$\sum\limits_{i=1}^k \lambda_iX_i = O$.
\edok


\subsection{Замена базиса и координат}

Пусть $\bazis{e} = (\vek{e}_1, \vek{e}_2, \ldots , \vek{e}_n)$ и
$\bazis{e}' = (\vek{e}_1', \vek{e}_2', \ldots , \vek{e}_n')$ --- два базиса
в векторном пространстве $V$ ($n=\dim V$).
%(Условно назовем их {\it старый} и {\it новый}.)

\defin{Матрица $S$ размера $n\times n$, $j$-ый столбец которой равен координатному столбцу
вектора $\vek{e}_j'$ в базисе $\bazis{e} $ ($j=1, 2, \ldots, n$), называется {\it матрицей перехода } от базиса
$\bazis{e}$ к базису~$\bazis{e}'$.
}

Определение означает, что столбцы $s_{\bullet 1}, \ldots , s_{\bullet n}$ матрицы перехода от 
$\bazis{e}$ к $\bazis{e}'$ удовлетворяют равенствам $\vek{e}_1' = \bazis{e}s_{\bullet 1}$, \ldots, 
$\vek{e}_n' = \bazis{e}s_{\bullet n}$.
Эти равенства можно записать в виде  $\boxed{\bazis{e}' = \bazis{e} S}$.
По сути это компактная запись определения матрицы перехода. 

\begin{predl}\label{p7_3_222}
Пусть в пространстве $V$ выбран базис $\bazis{e}$. Матрица $S \in \mathbf{M}_{n\times n}$
является матрицей перехода от $\bazis{e}$ к некоторому базису $\bazis{e}'$ тогда и только тогда, 
когда $S$ невырожденная.
\end{predl}
\dok Согласно следствию из предложения \ref{p7_3_2}, $S$ является 
матрицей перехода к некоторому базису $\bazis{e}'$ тогда и только тогда, 
когда столбцы $S$ образуют базис в $\mathbf{M}_{n\times 1}$.
\edok

\begin{predl}\label{p7_3_223}
Пусть в пространстве $V$ выбран базис $\bazis{e}$. Матрица 
перехода от $\bazis{e}$ к $\bazis{e}$ равна единичной матрице $E_n$.
\end{predl}
\dok Следует из определения матрицы перехода.
\edok

\begin{theor}\label{t7_3_2}
Пусть $\dim V=n<\infty $ и $\vek{a}\in V$ имеет в базисах
$\bazis{e}$ % = (\vek{e}_1, \vek{e}_2, \ldots , \vek{e}_n)$ 
и
$\bazis{e}'$ % = (\vek{e}_1', \vek{e}_2', \ldots , \vek{e}_n')$ 
координатные столбцы
$X$ и $X'$. Тогда $$\boxed{X = SX '},$$ где $S$ --- матрица перехода от
базиса $\bazis{e}$ к базису $\bazis{e}'$.
\end{theor}
\dok
По условию $\vek{a} = \bazis{e}X = \bazis{e}'X'$.
Так как $\bazis{e}' = \bazis{e} S$, имеем
$\bazis{e}X = (\bazis{e}S) X' = \bazis{e} (S X')$. В последнем равенстве используется 
ассоциативность умножения матриц, она верна и в случае <<необычной>> матрицы $\bazis{e}$,
элементы которой --- векторы.
Далее из закона сокращения (здесь используем, что $\bazis{e}$ --- линейно независимая система)
получаем $X=SX'$.
\edok

\begin{predl}\label{p7_3_3}
 Пусть $\bazis{e}$, $\bazis{e}'$, $\bazis{e}''$ --- три базиса в $V$.
Пусть $S$ --- матрица перехода от $\bazis{e}$ к $\bazis{e}'$,
а $R$ --- матрица перехода от $\bazis{e}'$ к $\bazis{e}''$. Тогда
матрица перехода от $\bazis{e}$ к $\bazis{e}''$ равна $SR$.
\end{predl}
\dok
$\bazis{e}''=\bazis{e}'R = (\bazis{e}S)R = \bazis{e}(SR)$.
\edok

\otstup

\begin{sled} Если матрица перехода от базиса $\bazis{e}$ к базису $\bazis{e}'$ равна $S$, то 
 матрица перехода от базиса $\bazis{e}'$ к базису $\bazis{e}$ равна  $S^{-1}$.
\end{sled}
\dok Положив в предложении \ref{p7_3_3} $\bazis{e}''=\bazis{e}$, с учетом предложения \ref{p7_3_223}
получим $SR=E$.
\edok

\subsection{Примеры}

\example{I.
Понятие базиса на плоскости и в 
 пространстве согласуется с определением из курса геометрии:
базисы на плоскости --- упорядоченные пары неколлинеарных векторов
(т.е. плоскость является двумерным пространством);
базисы в пространстве --- упорядоченные тройки некомпланарных векторов
(т.е. геометрическое пространство является трехмерным).
}

\example{II.1.
{\it Стандартный базис} в $\mathbf{M}_{m\times n}$ образуют матрицы, элементы которых ---
все нули, кроме одной единицы. Тогда числа, записанные в ячейках произвольной 
матрицы $A\in \mathbf{M}_{m\times n}$, --- это коэффициенты в разложении $A$ по стандартному базису.
Отсюда $\dim \mathbf{M}_{m\times n} = mn$. 
}

\otstup

{\bf Упражнение.} Найдите $\dim \mathbf{M}^+_{n\times n}(\mathbb{R})$ и $\dim \mathbf{M}^-_{n\times n}(\mathbb{R})$.
Укажите некоторые базисы этих подпространств.

\otstup

\example{II.2.
Пространство столбцов $V=\mathbf{M}_{n\times 1} (\mathbb{F})$ можно назвать
{\it стандартным} $n$-мерным пространством над полем $\mathbb{F}$ 
(в согласии с замечаниями о <<координатизации>>). \\
Базисы в $V$ --- во взаимно-однозначном соответствии 
с множеством $GL_n (\mathbb{F})$ невырожденных матриц:
каждая невырожденная матрица $A = (a_{\bullet 1}\,\, \ldots \,\, a_{\bullet n})$ определяет базис
$(a_{\bullet 1}, \ldots  , a_{\bullet n})$.
В частности, 
верхнетреугольная матрица с ненулевыми константами на главной диагонали определяет так называемый
{\it треугольный базис} простанства $V$. \\
Та же матрица $A$ совпадает с матрицей перехода от стандартного базиса к базису 
$(a_{\bullet 1}, \ldots ,a_{\bullet n})$.\\
%Так группа невырожденных матриц $GL_n (\mathbb{F})$  естественно действует на множестве базисов.\\
}

\example{II.3.
Подпространство $U\leq \mathbb{R}^n = \mathbf{M}_{n\times 1}(\mathbb{R})$ 
может быть задано
как линейная оболочка нескольких столбцов: $U=\lin{a_{\bullet 1}, \ldots, a_{\bullet k}}$.
Базис $U$ в этом случае можно найти, выполнив элементарные преобразования столбцов (они не изменяют их линейную оболочку) до ступенчатого вида.
Имеем $\dim U =\rg A$, где $A$ --- матрица,  $(a_{\bullet 1}\,\, \ldots \,\, a_{\bullet k})$.
}

\example{II.4.
Подпространство $U\leq \mathbb{R}^n$ может быть задано
как множество решений некоторой однородной СЛУ (системы линейных уравнений) $AX=O$, т.е. $U=\Sol(AX=O)$.
При этом  если $\rg A = r$, то $\dim U = n-r$. Определение базиса в $U = \Sol(AX=O)$ 
совпадает с определением ФСР (фундаментальной системы решений) СЛУ $AX=O$.\\
В указанном виде может быть задано любое подпространство $U\leq \mathbb{R}^n$ 
(имеется алгоритм получения СЛУ, для которой данная линейная оболочка столбцов явлется
множеством решений; см. также предложение \ref{10_2_5} \, главы \ref{evkl_prostr}).
}

%\example{II.3.
%Ранг матрицы $A\in \mathbf{M}_{m\times n}$ --- это ранг системы ее столбцов или системы ее строк 
%(теорема о ранге матрицы обеспечивает равенство столбцового и строчного ранга).
%Ранг можно найти, приводя матрицу элементарными преобразованиями строк к ступенчатому виду.
%Ранг матрицы равен количеству ненулевых строк в ступенчатом виде, а ведущие элементы строк в 
%ступенчатом виде помечают номера столбцов, которые образуют в исходной матрице базисную подсистему столбцов 
%(конечно, базисная подсистема столбцов не обязательно единственная). 
%}

\example{III.1.
Набор степеней $1, x, x^2, \ldots, x^n$ --- базис в пространстве $\mathbf{P}_n$ 
(формальных) многочленов степени не выше $n$. Так,  $\dim \mathbf{P}_n = n+1$.
(Пространство $\mathbf{P}$ всех многочленов уже не является конечномерным.)
\\ 
Также набор функций $1, x, x^2, \ldots, x^n$ является базисом в пространстве полиномиальных 
функций $\mathbb{R} \to \mathbb{R}$ степени не выше $n$.
Действительно, если линейная 
комбинация  $\sum\limits_{i=0}^n \lambda_ix^i$ равна нулевой функции, то 
все коэффициенты равны 0, иначе многочлен степени $k$ будет иметь больше чем $k$ корней.\\
}
\example{III.2.
Рассмотрим  пространство фибоначчиевых последовательностей $V = \{(x^1, x^2, \ldots) \, | \, x^{i+1}=x^i+x^{i-1}, \, i=2, 3, \ldots \}$.
Каждая последовательность из $V$ однозначно задается первыми двумя членами $x^1, x^2$. 
Рассмотрим две последовательности $e_1 = (1, 0, 1, 1, 2, \ldots )$ и $e_2 = (0, 1, 1, 2, 3, \ldots )$. 
Они  образуют {\it стандартный базис} в $V$: каждая последовательность $(x^1, x^2, \ldots) \in V$ совпадает с линейной комбинацией $x^1e_1+x^2e_2$
(поскольку совпадают первые два члена). В частности, $\dim V = 2$.\\
В $V$ можно обнаружить известные последовательности --- геометрические прогрессии.
Действительно $ (1, \lambda, \lambda ^2, \ldots) \in V$ $\Leftrightarrow$ $\lambda^{i+1}=\lambda^{i}+\lambda^{i-1}$, $i=2, 3, \ldots$.
Последнее множество равенств эквивалентно единственному равенству $\lambda^2=\lambda+1$, откуда $\lambda_{1,2} = \frac{1\pm \sqrt{5}}{2}$.
Найденные прогрессии $g_i = (1, \lambda_i, \lambda_i ^2, \ldots)$, $i=1, 2$, образуют базис в $V$.\\
Линейное выражение (обычной) последовательности Фибоначчи $f=(1, 1, 2, 3, 5, 8, \ldots)$ через $g_i$ 
позволит найти явную формулу для чисел Фибоначчи.\\
Равенство $f=c_1g_1+c_2g_2$ эквивалентно системе равенств для первых двух членов: \\
$1=c_1+c_2$, $1=c_1\lambda_1+c_2\lambda_2$. Отсюда $c_1=\frac{\sqrt{5}+1}{2\sqrt{5}}=\frac{\lambda_1}{\sqrt{5}}$, 
$c_2=\frac{\sqrt{5}-1}{2\sqrt{5}}=-\frac{\lambda_2}{\sqrt{5}}$. И окончательно, $n$-е число Фибоначчи
равно $c_1 \lambda_1^{n-1} + c_2 \lambda_2^{n-1}= \frac{\lambda_1^{n} - \lambda_2^{n}}{\sqrt{5}}$.
}

\example{IV.
Пусть $\mathbb{F}$ --- конечное поле из $q$ элементов (например, $\mathbb{F} = \mathbb{Z}_p$ для
некоторого простого $p$), а $V$ --- векторное пространство над $\mathbb{F}$, $\dim V = n$.\\
Фиксация любого базиса определяет биекцию $V \to \mathbf{M}_{n\times 1}(\mathbb{F}) = \mathbb{F} ^n$,
откуда $|V| = q^n$.\\
Далее, выбор (упорядоченной) линейно независимой системы $\vek{a}_1, \ldots, \vek{a}_k$ 
в $V$ можно осуществить, последовательно выбирая 
$\vek{a}_1\neq \vek{0}$, $\vek{a}_2\notin \lin{ \vek{a}_1}$, $\vek{a}_3\notin \lin{ \vek{a}_1, \vek{a}_2}$, и т.д.
Для выбора $\vek{a}_{i+1}$ имеется $(q^n-q^{i})$ возможностей (все векторы из $V$, исключая лежащие в 
$i$-мерном подпространстве $\lin{ \vek{a}_1, \ldots, \vek{a}_{i}}$). Отсюда количество таких линейно независимых систем 
равно $\prod\limits_{i=1}^k (q^n-q^{i})$.\\
В частности, количество базисов в $V$ (равное также $|GL_n(\mathbb{F})|$) равно
$\prod\limits_{i=1}^n (q^n-q^{i})$.\\
Линейно независимая система $\vek{a}_1, \ldots, \vek{a}_k$ определяет подпространство
$\lin{\vek{a}_1, \ldots, \vek{a}_k}$, и каждое $k$-мерное подпространство определяется таким образом
столькими способами, сколько в нем базисов. Отсюда находим количество $k$-мерных подпространств в $V$ как
$\dfrac{\prod\limits_{i=1}^k (q^n-q^{i})}{\prod\limits_{i=1}^k (q^k-q^{i})}$.
}




%О РАЗРЕШЕНИИ ЛИНЕЙНЫХ РЕККУРЕНТ Жорданов базис и пр. собственные функции. Оператор сдвига
%то же про диффуры.

%\example{III.3.
%Каждому решению линейного однородного ДУ $x^{(n)}+ a_{n-1}(t)x^{(n-1)}+\ldots + a_0(t)x=0$,
%сопоставим {\it столбец начальных условий} $\stolbec{x(0)\\x'(0) \\ \ldots \\ x^{(n-1)}(0)$.
%В том случае, если $a_i(t)$ --- непрерывные функции $\mathbb{R}\to \mathbb{R}$,
%теорема существования и единственности решения задачи Коши из курса дифференциальных уравнений
%доказывает, что это сопоставление биективно, поэтому является изоморфизмом пространства решений данного ДУ и $\mathbb{R}^n$.
%В частности, пространство решений $n$-мерно.
%}


\section{Сумма подпространств. Прямая сумма %
%и прямое дополнение
}

\subsection{Сумма подпространств}

\defin{Пусть $\mathcal{A}_1, \mathcal{A}_2, \ldots , \mathcal{A}_k$ ---
подмножества в $V$. {\it Суммой} (по Минковскому)
подмножеств $\mathcal{A}_1, \mathcal{A}_2, \ldots , \mathcal{A}_k$
называется множество
$\{\vek{a}_1 + \vek{a}_2 + \ldots + \vek{a}_k \, | \, \vek{a}_i \in \mathcal{A}_i \}$, $i=1, \ldots, k$.
}

Обозначение для суммы подмножеств: $\mathcal{A}_1+ \mathcal{A}_2+ \ldots +\mathcal{A}_k$
или $\sum\limits_{i=1}^{k} \mathcal{A}_i$.
Из определения ясно, что $\mathcal{A}_1+ \mathcal{A}_2 = \mathcal{A}_2+ \mathcal{A}_1$,
$(\mathcal{A}_1+ \mathcal{A}_2)+ \mathcal{A}_3 \hm= \mathcal{A}_1+ (\mathcal{A}_2+ \mathcal{A}_3)$.
%Но некоторые другие свойства суммы векторов для суммы множеств не выполняются, скажем,
%из $\mathcal{A}+\mathcal{A}=\mathcal{A}$ не следует, что $\mathcal{A}=\{o\}$. 


Далее будем заниматься почти всегда только суммами подпространств.

%\begin{predl}\label{p7_4_1}
%Сумма нескольких (конечного числа) подпространств является подпространством.
%\end{predl}
%\dok
%\edok

Следующее предложение связывает понятия суммы и линейной оболочки.

\begin{predl}\label{p7_4_2}
Пусть $U_i = \lin{\mathcal{A}_i}$, $i=1, 2, \ldots , k$.
Тогда
$$\sum\limits_{i=1}^k U_i=
\lin{\mathcal{A}_1 \cup \mathcal{A}_2 \cup \ldots \cup \mathcal{A}_k}.$$
\end{predl}
\dok По определению, $\sum\limits_{i=1}^k U_i$ --- множество элементов вида $\sum\limits_{i=1}^k \vek{a}_i$, где $\vek{a}_i$ может быть значением произвольной линейной комбинации векторов из $\mathcal{A}_i$, а значит множество сумм $\sum\limits_{i=1}^k \vek{a}_i$ совпадает с множеством значений линейных комбинации векторов из 
$\mathcal{A}_1 \cup \mathcal{A}_2 \cup \ldots \cup \mathcal{A}_k$.
\edok

\otstup

\begin{sled}
Сумма нескольких (конечного числа) подпространств является подпространством.
\end{sled}

Видим, что сумма $\sum\limits_{i=1}^k U_i$ подпространств $U_1, \ldots , U_k$ --- это минимальное по включению подпространство,
содержащее каждое из подпространств $U_1, \ldots,  U_k$.


\begin{predl}\label{p7_4_3}
Пусть $U_i\leq V$, $i=1, 2, \ldots , k$.
Тогда $$\dim \left(\sum\limits_{i=1}^k U_i \right) \leq \sum\limits_{i=1}^k \dim U_i.$$
\end{predl}
\dok
По предыдущему предложению, $\sum\limits_{i=1}^k U_i =
\lin{U_1 \cup U_2 \cup \ldots \cup U_k}$.
Но по основной теореме \ref{t5_2_1} и предложению
\ref{p7_2_5} имеем $\dim \lin{U_1 \cup U_2 \cup \ldots \cup U_k} \hm=
\rg(U_1 \cup U_2 \cup \ldots \cup U_k) \leq
\sum\limits_{i=1}^k \dim U_i$.
\edok

\otstup

Для операции сложения подмножеств выполнены далеко не все свойства, которыми обладает операция сложения векторов.
%Оперируя с суммой подпространств, нужно иметь в виду следующие предостережения.
Например, вообще говоря не выполнен закон сокращения, скажем, для любого подпространства $U\leq V$
справедливо равенство $U+U=U$.
Также не работает аналогия между суммой подпространств и теоретико-множественным
объединением: скажем не всегда
$(U_1+U_2)\cap U_3 \hm= (U_1\cap U_3)+(U_2\cap U_3)$
в отличие от теоретико-множественного тождества $(U_1\cup U_2)\cap U_3 = (U_1\cap U_3)\cup (U_2\cap U_3)$.

\otstup

{\bf Упраженение.} Приведите пример подпространств $U_1, U_2, U_3$ некоторого векторного пространства $V$ таких, что
$(U_1+U_2)\cap U_3 \hm\neq (U_1\cap U_3)+(U_2\cap U_3)$. 

\subsection{Прямая сумма}

Важен следующий специальный случай суммы подпространств $U_i\leq V$.

\defin{Сумма $U$ подпространств $U_1, U_2, \ldots , U_k$ называется (внутренней) {\it прямой суммой},
если $\forall \vek{a}\in U$ имеется единственный набор
$\vek{a}_i\in U_i$ ($i=1, 2, \ldots , k$) такой, что
$\vek{a} = \sum\limits_{i=1}^{k} \vek{a}_i$.
}


Обозначение для прямой суммы: $U_1 \bigoplus U_2 \bigoplus \ldots \bigoplus U_k$
или  $\bigoplus \limits_{i=1}^{k} U_i$.
Иногда говорят, что $U$ {\it разложено в прямую сумму} подпространств $U_1, U_2, \ldots , U_k$.
Пример разложения в прямую сумму: 
$V=\lin{\vek{e}_1} \bigoplus \lin{\vek{e}_2} \bigoplus \ldots
 \bigoplus \lin{\vek{e}_n}$, где  $\vek{e}_1, \vek{e}_2, \ldots , \vek{e}_n$ --- некоторый базис в $V$.


\otstup
%%%%%%%%%%%%%%%%
%%ВНЕШЯЯ ПРЯМАЯ СУММА
%%%%%%%%%%%%%%%%

{\footnotesize Наряду с внутренней прямой суммой, которой в основном будем заниматься, 
можно рассмотреть {\it внешнюю прямую сумму} векторных пространств $U_1, U_2, \ldots , U_k$ 
как декартово произведение $U_1\times  U_2\times \ldots \times U_k$, на котором 
операции сложения и умножения на константу определены <<покомпонентно>>. Конструкции
внутренней и внешней прямой суммы по сути эквивалетны (связаны естественным изоморфизмом).}



%Если $U = \bigoplus \limits_{i=1}^{k} U_i$, ниже обозначаем $\overline{U_i}=U_1+\ldots +U_{i-1}+U_{i+1}+\ldots +U_k$.
В случае рассмотрения суммы $U = \sum\limits_{i=1}^{k} U_i$ через $\overline{U_i}$ обозначаем сумму всех рассматриваемых пространств, 
за исключением $U_i$, т.е. $\overline{U_i}=U_1+\ldots +U_{i-1}+U_{i+1}+\ldots +U_k$.

\begin{theor}[критерий-1 прямой суммы]\label{t7_4_1}
Пусть $U_i\leq V$, $i=1, \ldots, k$.
Сумма подпространств $U_1, \ldots, U_k$ --- прямая сумма 
$\Leftrightarrow$
$U_i\cap \overline{U_i} = O$, $i=1, 2, \ldots , k$.
\end{theor}
\dok \dokright Предположим противное, скажем условие $U_i\cap \overline{U_i} = O$ не выполнено для $i=1$.
Это значит, что  нашелся ненулевой вектор $\vek{a}_1 \in U_1 \cap  \overline{U_1}$. 
Имеем $\vek{a}_1 = \vek{a}_2+\ldots + \vek{a}_k$ для некоторых $\vek{a}_i\in U_i$, $i=2, \ldots, k$.
Но тогда $\vek{0} = \vek{0} +\ldots + \vek{0} = \vek{a}_1 +(- \vek{a}_2)+\ldots + (-\vek{a}_k)$ --- два различных разложения нулевого вектора в сумму векторов из $U_i$
вопреки определению прямой суммы.

\dokleft Предположим противное, сумма $U=\sum\limits_{i=1}^{k} U_i$ не является прямой суммой, 
тогда для некоторого вектора $\vek{a}\in U$ найдутся два различных разложения
$\vek{a} = \sum\limits_{i=1}^{k} \vek{a}_i = \sum\limits_{i=1}^{k} \vek{b}_i$, где $\vek{a}_i\in U_i$, $\vek{b}_i\in U_i$, $i=1, \ldots, k$.
Разложения различны, значит хотя бы для одного $i$ имеем $\vek{a}_i\neq \vek{b}_i$, скажем $\vek{a}_1\neq \vek{b}_1$.
Но тогда $\vek{a}_1 - \vek{b}_1 = \sum\limits_{i=2}^{k} (\vek{b}_i-\vek{a}_i)$. В левой части равенства --- ненулевой вектор из $U_1$, а в правой части --- вектор из
$\overline{U_1}$, поэтому $U_1\cap \overline{U_1} \neq  O$. Противоречие.
\edok 

\otstup
Указанным критерием часто удобно пользоваться в случае двух подпространств. 
Этот случай выделим отдельным следствием.

\begin{sled}
Пусть $U_1\leq V$, $U_2\leq V$. Тогда $U_1+U_2$ --- прямая сумма $\Leftrightarrow$ $U_1\cap U_2 = O$.
\end{sled}

\otstup


В отличие случая двух подпространств, 
условие тривиальности попарных пересечений
$U_1\cap U_2 = U_2\cap U_3 = U_3\cap U_1 = O$ не является достаточным для того, чтобы сумма подпространств
$U_1+U_2+U_3$ была прямой суммой. 
Контрпримером могут служить три различных одномерных пространства,
лежащие в двумерном пространстве.

\otstup


{\bf Упражнение.}
Сумма подпространств $U_1, \ldots, U_k$ --- прямая сумма 
$\Leftrightarrow$
$\forall$ $\vek{a}_i\in U_i$, $\vek{a}_i\neq \vek{0}$
система $\vek{a}_1, \ldots, \vek{a}_k$ линейно независима.

\otstup

{\bf Упражнение.} Пусть $U_i\leq V$ таковы, что 
$U_1\oplus U_2\oplus U_3$ --- прямая сумма. Тогда $U=U_2\oplus U_3$ --- прямая сумма и 
$U_1\oplus U$ --- также прямая сумма.
%Не зависит от порядка; прямую сумму можно определить последовательно через
%сумму двух.

\otstup

{\bf Упражнение.} Пусть $U_i\leq V$, $i=1, \ldots, k$, --- подпространства, для которых выполнено: 
$U_j\cap (\sum\limits_{i=1}^{j-1} U_i) = O$ для всех $j=2, 3, \ldots, k$.
Тогда $\sum\limits_{i=1}^{k} U_i$ --- прямая сумма.


\begin{theor}[критерий-2 прямой суммы]\label{t7_4_2}
Пусть $U_i\leq V$, $\dim U_i=n_i<\infty$, $\bazis{e}^{(i)}$ --- базис в~$U_i$, $i=1, \ldots, k$;
$U=\sum \limits_{i=1}^{k} U_i$. Тогда следующие условия эквивалентны:\\
1) $U= \bigoplus \limits_{i=1}^{k} U_i$;\\
2) система из $\sum \limits_{i=1}^{k}n_i$ векторов $\bigcup\limits_{i=1}^{k} \bazis{e}^{(i)}$ --- базис
в  $U$;\\
3) $\dim U = \sum \limits_{i=1}^{k} n_i$.
\end{theor}
\dok Имеем  $U_i = \lin{\bazis{e}^{(i)}}$, тогда согласно предложению 
\ref{p7_4_2}, $U = \lin{\bazis{e}}$, где $\bazis{e}=\bigcup\limits_{i=1}^{k} \bazis{e}^{(i)}$.

2) $\Leftrightarrow$ 3)
По  следствию из основной теоремы \ref{t5_2_1} имеем $\dim U=\rg \bazis{e}$.
Значит (см. предложение \ref{p7_2_4}) $\dim U = \sum \limits_{i=1}^{k} n_i$ 
$\Leftrightarrow$  $\rg \bazis{e} = \sum \limits_{i=1}^{k} n_i$
$\Leftrightarrow$  система $\bazis{e}$ линейно независима.

1) $\Rightarrow$ 2) 
Предположим противное, и система $\bazis{e}$ линейно зависима.
Запишем нетривиальную линейную  комбинацию векторов  из $\bazis{e}$, равную $\vek{0}$:
$\sum \limits_{i=1}^{k}  \vek{\ell}_i=\vek{0}$, где $\vek{\ell}_i$ --- линейная комбинация векторов из $\bazis{e}^{(i)}$.
Хотя бы одна из линейных комбинаций $\vek{\ell}_i$  нетривиальная, пусть это~$\vek{\ell}_1$, тем самым $\vek{\ell}_1\neq \vek{0}$.
Тогда  $\vek{\ell}_1= - \vek{\ell}_2-\ldots - \vek{\ell}_k$. Здесь $\vek{\ell}_i$ равно некоторому вектору из $U_i$, 
%причем $\vek{l}_1$ равно ненулевому  вектору из $U_1$, так как это значение нетривиальной линейной комбинации 
%линейно независимых векторов $\bazis{e}^{(1)}$.
значит $\vek{\ell}_1\in U_1\cap \overline{U_1}$  в противоречие с
теоремой \ref{t7_4_1}.

2) $\Rightarrow$ 1) 
Предположим противное, $U$ не являетяся прямой суммой подпространств $U_i$, и скажем 
(см. теорему \ref{t7_4_1}) $\exists$ $\vek{a}\neq \vek{0}$: $\vek{a}\in U_1\cap \overline{U_1}$.
Тогда $\vek{a}= \vek{\ell}_1= \vek{\ell}_2+\ldots +\vek{\ell}_k$, где $\vek{\ell}_i\in U_i$, т.е. $\vek{\ell}_i$ равно некоторой линейной комбинации векторов из $\bazis{e}^{(i)}$.
Перенося в левую часть, получаем $\vek{\ell}_1- \vek{\ell}_2-\ldots -\vek{\ell}_k = \vek{0}$ (в левой части нетривиальная линейная комбинация, поскольку 
уже $\vek{\ell}_1$ --- нетривиальная линейная комбинация), откуда
$\bazis{e}$ --- линейно зависимая система. Противоречие. 
\edok

\subsection{Прямое дополнение. Проекции}


\defin{
Если  $U_1\leq V$ и $U_2\leq V$ таковы, что $U_1 \bigoplus U_2 = V$, 
то подпространство $U_2$ называют {\it прямым дополнением} подпространства
$U_1$ (в векторном пространстве $V$).
}

Подпространства $U_1$ и $U_2$ входят в определение симметрично, поэтому: $U_1$ --- прямое дополнение для $U_2$ 
$\Leftrightarrow$ $U_2$ --- прямое дополнение для $U_1$.  
Отметим, что прямое дополнение ничего общего не имеет с понятием теоретико-множественного дополнения.


\begin{predl}\label{p7_4_4}
Пусть $\dim V = n< \infty$. Тогда сумма размерностей подпространства и любого его прямого
дополнения равна $n$.
\end{predl}
\dok
Следует из теоремы \ref{t7_4_2}.
\edok

\begin{predl}\label{p7_4_5}
Пусть $\dim V = n< \infty$. Для любого подпространства $U\leq V$ существует прямое дополнение.
\end{predl}
\dok
Выберем в $U$ базис $\vek{e}_1, \ldots, \vek{e}_k$. %, тогда $U=\lin{\vek{e}_1, \ldots, \vek{e}_k}$.
Согласно предложению  \ref{p7_3_1}, систему $\vek{e}_1, \ldots, \vek{e}_k$ можно дополнить до
базиса $\vek{e}_1, \ldots, \vek{e}_k, \vek{e}_{k+1}, \ldots ,\vek{e}_{n}$ пространства $V$.
Тогда из теоремы \ref{t7_4_2} вытекает, что  подпространство $W=\lin{\vek{e}_{k+1}, \ldots, \vek{e}_n}$ таково, что $U\oplus W = V$.
\edok

\otstup

Заметим, что для одного пространства может существовать много прямых дополнений
(достаточно посмотреть на геометрический пример $U\leq V$, где $\dim U=1$, $\dim V=2$). 

Если $V=U_1 \bigoplus U_2$, то, согласно определению прямой суммы, 
 любой вектор $\vek{a} \in V$ однозначно представляется в виде суммы
$\vek{a} = \vek{a}_1 + \vek{a}_2$, где $\vek{a}_i\in U_i$, $i=1, 2$.
 Вектор $\vek{a}_1$ называется 
{\it проекцией} вектора $\vek{a}$ на подпространство $U_1$ {\it вдоль} $U_2$ (или {\it параллельно} $U_2$).
%вдоль подпространства $\overline{U_i}=U_1+\ldots +U_{i-1}+U_{i+1}+\ldots +U_k$}.

\otstup

{\footnotesize
Для $U\leq V$ определяется {\it фактор-пространство} $V/U$ как
фактор-множество относительно эквивалентности $\vek{a}\sim \vek{b}$ $\Leftrightarrow$ $\vek{b}-\vek{a}\in U$;
элементы $V/U$ --- смежные классы вида $\vek{a} +U$ с естественно определенными операциями
$(\vek{a} +U)+(\vek{b} +U):=(\vek{a}+\vek{b} )+U$, $\lambda (\vek{a} +U) := \lambda \vek{a} +U$.
Имеется естественный изоморфизм $W\to V/U$,
где $W$ --- прямое дополнение подпространства $U\leq V$; он задается как $\vek{a} \mapsto \vek{a} +U$.
}

\subsection{Формула размерностей суммы и пересечения}


\begin{theor}[формула Грассмана]\label{t7_4_3}
Пусть $U_1\leq V$, $U_2\leq V$.
Тогда 
$$\boxed{\dim (U_1+U_2) + \dim (U_1\cap U_2)= \dim U_1+ \dim U_2}.$$
\end{theor}
\dok Достаточно рассмотреть случай $\dim U_i<\infty $ (иначе в обеих частях формулы --- бесконечность).

Согласно предложению \ref{p7_4_5}, можем выбрать  $W\leq U_2$ так, что 
\begin{equation}\label{eqW1}
(U_1\cap U_2)\oplus W = U_2.
\end{equation}
Тогда  $U_1+U_2 = U_1+ ((U_1\cap U_2) +  W) = (U_1+ (U_1\cap U_2)) +  W = U_1+W $.
Кроме того, поскольку $W\leq U_2$, имеем  $U_1\cap W = U_1\cap U_2 \cap W = (U_1\cap U_2) \cap W = O$
(из (\ref{eqW1}) по следствию из теоремы \ref{t7_4_1}). Получаем, что $U_1+W $ --- прямая сумма:
\begin{equation}\label{eqW2}
U_1+  U_2 = U_1 \oplus W.
\end{equation}
 Согласно %предложению \ref{p7_4_4}, 
теореме \ref{t7_4_2}, из равенств (\ref{eqW1}) и (\ref{eqW2}) следует, что \\
$\dim W = \dim U_2 - \dim (U_1\cap U_2) = \dim (U_1+ U_2) - \dim U_1$, откуда следует требуемая формула размерностей.
%утверждение теоремы.
\edok

\otstup 
{\bf Упражнение.} Пусть $\dim V=4$. Существуют ли подпространства $U_1$ и $U_2$ такие, что 
$\dim U_1 = \dim U_2 = 3$ и $\dim (U_1\cap U_2)=1$?

\subsection{Примеры}


\example{I.1. Сумма двух непараллельных отрезков --- параллелограмм. (Здесь, как обычно, точки отождествляем
с концами радиус-векторов.) %Что представляет собой сумма двух многоугольников? Многоугольника и круга?
}
\example{I.2.
Пусть $V$ ---  геометрическое трехмерное векторное пространство,
$U_1\leq V$, $\dim U_1=1$ (т.е. $U_1$ --- прямая, проходящая через $O$),
 $U_2\leq V$, $\dim U_2=2$ (т.е. $U_2$ --- плоскость, проходящая через $O$).
Если $U_1 \not \hspace{-1mm} \subset U_2$, то $V=U_1\oplus U_2$. \\
Понятно, как геометрически разложить радиус-вектор $\overrightarrow{OA}$ в сумму проекций: через $A$ проведем прямую, параллельную $U_1$; пусть
она пересекает $U_2$  в точке $A_2$. Тогда $\vek{a}=\vek{a}_1+\vek{a}_2$, где $\vek{a}_1 = \overrightarrow{A_2A}$, $\vek{a}_2 = \overrightarrow{OA_2}$. 
}

\example{II.1.
Множество решений совместной СЛУ (над $\mathbb{R}$) $AX=b$ с $n$ неизвестными имеет вид $\Sol(AX=b) = \{X_0\}+\Sol(AX=O)$.
Если  $r=\rg A$, то $\Sol(AX=b)$ --- это  $(n-r)$-мерная плоскость (или $(n-r)$-мерное {\it линейное многообразие}) 
в пространстве $\mathbb{R}^n$.
}
%
%\example{Пусть $\mathcal{A}_1 = \{\vek{b} \}$ --- множество из одного вектора,
%а $\mathcal{A}_2 = U$ --- подпространство размерности $k$. Тогда
%$\mathcal{A}_1+ \mathcal{A}_2 = \{\vek{b} \} + U$ (то есть сдвиг подпространства $U$ на вектор
%$\vek{b}$) --- $k$-мерное {\it линейное многообразие} или {\it $k$-мерная} плоскость.
%}
%
\example{II.2. Пусть $\mathbf{M}_{n\times n} = \mathbf{M}_{n\times n} (\mathbb{F})$, где 
$\mathbb{F}$ --- поле характеристики, не равной 2. Тогда\\
$\mathbf{M}_{n\times n} = \mathbf{M}_{n\times n}^{+} \bigoplus \mathbf{M}_{n\times n}^{-}$.\\
Действительно, $A=\dfrac{A+A^T}{2}+\dfrac{A-A^T}{2}$ и %очевидно, 
$\mathbf{M}_{n\times n}^{+} \cap \mathbf{M}_{n\times n}^{-} = O$.
%где $\mathbf{M}_{n\times n}^{+}= \{A\in \mathbf{M}_{n\times n}| A^T=A \}$ ---
%подпространство симметричных матриц,
%$\mathbf{M}_{n\times n}^{-} = \{A\in \mathbf{M}_{n\times n}| A^T=-A \}$ --- подпространство кососимметричных
%симметричных матриц.
}
\example{II.3.
Пусть $A\in \mathbf{M}_{n\times m}(\mathbb{R})$, $A=(a_{\bullet 1} \ldots a_{\bullet m})$.
Положим $U\hm=\lin{a_{\bullet 1} \ldots a_{\bullet m}}$, $\dim U = \rg A = r$.
Пусть $W = \Sol(A^TX=O)$. Тогда $\dim W = n-r$. Кроме того
$U\cap W = O$. Действительно, 
если $Y= \stolbec{y_1\\y_2\\ \vdots \\y_n}$ таков, что $ Y\in U\cap W$, то  есть $Y^TY = O$, 
откуда $\sum\limits_{i=1}^n y_i^2 = 0$, значит $Y=O$.\\
(Отметим, что естественное объяснение этого факта получится ниже, см. предложение
\ref{10_2_5} из главы \ref{evkl_prostr}.)
}

\example{III.1.
Несложно показать, что $\mathbf{F} = \mathbf{F}^{+} \bigoplus
\mathbf{F}^{-}$, где
$\mathbf{F}^{+}$, $\mathbf{F}^{-}$ --- подпространства
четных и нечетных функций $\mathbb{R}\to \mathbb{R}$.\\
Заметим, что $e^x=\ch x + \sh x$, при этом $f(x) =\ch x$ --- четная функция, а $g(x) =\sh x$ --- нечетная функция.
Поэтому $f(x) = \ch x$ --- это проекция функции $y(x) = e^x$ на $\mathbf{F}^{+}$ вдоль $\mathbf{F}^{-}$.
}




\section{Понятие аффинного пространства}

Идея конструкции абстрактного точечного (аффинного) пространства
---  в сопоставлении \\
<<точка $\leftrightarrow$ радиус-вектор>>.


\subsection{Определение и свойства}



{\it Аффинным пространством}, ассоциированным с векторным пространством $V$, называется множество $S$
(элементы которого будем называть \{точками\})  с отображением $S\times V\to S$ (откладываение от точки вектора, 
обозначать будем <<+>>), такие что\\
A1)  $p+(\vek{a}+\vek{b}) = (p+\vek{a})+\vek{b} $ \,\,\,\,\,\, ($\forall p\in S$, $\forall \vek{a}, \vek{b} \in V$ );\\
A2)  $p+\vek{0} = p$; \,\,\,\,\,\, ($\forall p\in S$);\\
A3)  $\forall p, q\in S$  существует единстивенный $\vek{a}\in V$ такой, что $p+ \vek{a} = q$.\\

\otstup

Вектор $\vek{a}$ из А3) называем {\it вектором, соединяющим точки} $p$ и $q$, и обозначаем $\overline{pq}$.

\otstup

Видим <<правило треугольника>>: $\overline{pq}+\overline{qr}=\overline{pr}$.

\otstup

$V$ естественно является аффинным пространством, ассоциированным с $V$.

\otstup

Наоборот, если в афинном пространстве зафиксировать точку $o$ (<<начало отсчета>>), 
то возникает биекция ({\it векторизация})  $S\to V$:
$$p \mapsto \overline{op}.$$ 

(зависит от выбора начала отсчета)

\otstup

$(o, \bazis{e}) $ --- {\it аффинная система координат} в $S$ (или {\it репер})

(в аналитической геометрии было: ДСК)

\otstup

Координаты точки $p\in S$ в ДСК
$(o, \bazis{e}) $ --- \\
это координаты вектора $\overline{op}$ в базисе $\bazis{e}$.


\otstup

координаты точки $p+\vek{a}$ ???
\\
\\
координаты вектора  $\overline{pq}$ 

\otstup

Замена координат: $$X=SX'+\gamma$$

\otstup
%барицентрические линейные комбинации, барики...

\subsection{Примеры подмножеств, конструкций}


{\it Плоскость}, или {\it линейное подмногообразие} ---
$$p+U, $$
где $U\leq V$.

\\
\\

Размерность 
$k$-мерная плоскость ($k=0$ --- точка, $k=1$ --- прямая, $k=n-1$ --- гиперплоскость). 

\otstup


Любые $k+1$ точек из $S$ принадлежат некоторой плоскости размерности $\leq k$.

$p_0 + \lin{\overline{p_0p_1}, \overline{p_0p_2}, \ldots, \overline{p_0p_k}}$.

%Критерий плоскости --- принадлежность целиком прямой

\otstup

{\footnotesize

Аффинная зависимость, независимость системы из $k+1$ точек.

Связь с линейной зависимостью векторов.

\otstup

Линейные комбинации точек 
(чтобы была корректность, сумма коэффициентов должна равняться 1, пример --- центр масс.)


\otstup
паралелелепипед, симмлекс.

выпуклые линейные комбинации, выпуклая оболочка.
}
 %аффинные пр-ва


%ГЛАВА
%\ref{lin_otobr}

\chapter{Линейные отображения}\label{lin_otobr}


В этой главе рассматриваются векторные пространства 
 $V$, $\widetilde{V}$ и т.д. над произвольным полем $\mathbb{F}$.
(При этом для обозначения операции сложения  в разных пространствах 
мы позволяем себе использовать один и тот же символ <<$+$>>; то же касается нулевых векторов, и т.д.)

В ситауциях, когда используется специфика поля $\mathbb{F}$, оговариваем это отдельно. 
(Например, для $\mathbb{F} = \mathbb{C}$ умножение линейного отображения на константу может быть определено
особым способом.)


\section{Определение. Операции над линейными отображениями. Изоморфизм}

\subsection{Определение и его следствия}


\defin{
Отображение $\varphi: V \to \widetilde{V}$ называется {\it линейным},
если $\forall$ $\vek{a}, \vek{b} \in  V$ и $\forall$ $\lambda\in \mathbb{F}$
выполняются равенства
\\
L1. $\varphi(\vek{a} + \vek{b}) = \varphi(\vek{a}) +  \varphi(\vek{b}) $,
\\
L2. $\varphi(\lambda \vek{a}) = \lambda  \varphi(\vek{a})$.
}


Линейное отображение также называют {\it гомоморфизмом} векторных пространств.
Множество всех линейных отображений $V \to \widetilde{V}$ обычно
обозначают $\Hom(V; \widetilde{V})$.
Мы чаще будем использовать чуть более короткое обозначение $L(V, \widetilde{V})$.

Линейное отображение $V \to V$ (т.е. в частном случае $\widetilde{V}=V$)
называют также  {\it линейным преобразованием} или {\it линейным оператором}.

Линейное отображение $V \to \mathbb{F}$
называют также  {\it линейным функционалом} или {\it линейной функцией}.
Это соответствует случаю $\dim \widetilde{V} = 1$, в этом случае 
$\widetilde{V}$ можно отождествить с полем констант~$\mathbb{F}$.


\example{
Примером линейного отображения является {\it нулевое} отображение $0: V \to \widetilde{V}$ такое, что
$\forall$ $\vek{a} \in  V$ выполнено $0(\vek{a})=\vek{0}$.
}



\begin{predl}\label{p8_1_1}
Пусть $\varphi \in L( V, \widetilde{V})$. Тогда $\forall$ $\vek{a}_i\in V$ и $\forall$ $\lambda_i\in \mathbb{F}$ выполнено
 \begin{equation}\label{eqLin}
\boxed{\varphi \left(\sum\limits_{i=1}^k \lambda _i \vek{a}_i \right) =
\sum\limits_{i=1}^k \lambda _i \varphi (\vek{a}_i)}.
\end{equation}
\end{predl}
\dok
Следует из многократного применения L1, L2.
\edok

\otstup

Отметим, что L1 и L2 представляют собой частные случаи фомулы (\ref{eqLin}).
Зафиксируем несложные, но важные следствия определения и предыдущего предложения.

\begin{predl}\label{p8_1_101}
Пусть $\varphi \in L( V, \widetilde{V})$. Тогда
\\
1). $\varphi (\vek{0}) = \vek{0}$;
\\
2). $\forall$ $\vek{a}\in V$ выполнено $\varphi(-\vek{a})=-\varphi(\vek{a})$.
\end{predl}
\dok
1). Следует из L2 для $\lambda= 0$.\\
2). Следует из L2 для $\lambda= -1$.
\edok

\begin{predl}\label{p8_1_102}
1). Если $\vek{a}_1, \vek{a}_2, \ldots , \vek{a}_k$  --- линейно зависимая система векторов, то
$\varphi(\vek{a}_1), \varphi(\vek{a}_2), \ldots , \varphi(\vek{a}_k)$
 --- тоже линейно зависимая система векторов.
%(имеется
%нетривиальная линейная комбинация, равная нулю, с тем же набором коэффициентов).
\\
2). $\forall$ $\mathcal{A}\subset V$ выполнено $\rg \varphi (\mathcal{A}) \leq \rg \mathcal{A}$.
\end{predl}
\dok
1).  Следует из предложения \ref{p8_1_1} и пункта 1) предложения \ref{p8_1_101}.\\
2). Следует из 1).
\edok

\begin{predl}[образ подпространства]\label{p8_1_103}
Пусть $\varphi \in L( V, \widetilde{V})$, $U\leq V$ и $U=\lin{\mathcal{A}}$. Тогда \\
1). $\varphi (U)\leq \widetilde{V}$, более того, $\varphi (U)=\lin{\varphi ( \mathcal{A} ) }$.
В частности, если $\vek{e}_1, \vek{e}_2, \ldots , \vek{e}_n$ --- базис в $V$, 
то (напомним, что $\varphi(V)$ также обозначается $\Im \varphi$)
$$\boxed{\Im \varphi =  \lin{\varphi(\vek{e}_1), \varphi(\vek{e}_2), \ldots , \varphi(\vek{e}_n)}}.$$
%В частности, если $U=\lin{\vek{a}_1, \vek{a}_2, \ldots , \vek{a}_k}$, то $\varphi(U)=\lin{\varphi(\vek{a}_1), \varphi(\vek{a}_2), \ldots , \varphi(\vek{a}_k)}$.
\\
2). $\dim \varphi(U) \leq \dim U$.
\end{predl}
\dok
1). Следует из предложения \ref{p8_1_1}.\\
2). Следует из 1) (и является частным случаем пункта 2 предложения \ref{p8_1_102}). 
\edok

\otstup

В следующем предложении отметим отдельно свойства линейного вложения (инъективного отображения).

\begin{predl}\label{p8_1_1000}
Пусть $\varphi \in L( V, \widetilde{V})$ и $\varphi$ инъективно. Тогда
\\
1). Если $\vek{a}_1, \vek{a}_2, \ldots , \vek{a}_k$  --- линейно независимая система векторов, то
$\varphi(\vek{a}_1), \varphi(\vek{a}_2), \ldots , \varphi(\vek{a}_k)$
 --- тоже линейно независимая система векторов.\\
2). $\forall$ $\mathcal{A}\subset V$ выполнено $\rg \varphi (\mathcal{A}) = \rg \mathcal{A}$.
В частности, для $U\leq V$ выполнено $\dim \varphi (U) = \dim U$.
\end{predl}
\dok
1). Пусть $\sum\limits_{i=1}^k \lambda _i \varphi (\vek{a}_i) = \vek{0}$, тогда
$\varphi \left(\sum\limits_{i=1}^k \lambda _i \vek{a}_i\right) = \vek{0}$. Но  $\varphi (\vek{0}) = \vek{0}$, 
поэтому в силу инъективности, $\sum\limits_{i=1}^k \lambda _i \vek{a}_i = \vek{0}$. Отсюда, поскольку 
$\vek{a}_1, \vek{a}_2, \ldots , \vek{a}_k$  --- линейно независимая система, имеем
$\lambda_1=\ldots = \lambda _k = 0$, то есть $\sum\limits_{i=1}^k \lambda _i \varphi (\vek{a}_i)$ --- тривиальная линейная комбинация.\\
2). Следует из 1).
\edok

\otstup 

%\begin{sled}\label{rgAfA}
%Пусть $\varphi \in L( V, \widetilde{V})$, $\mathcal{A}\subset V$.
%Тогда $\rg \varphi (\mathcal{A}) \leq \rg \mathcal{A}$. Если, кроме того, 
%$\varphi$ инъективно, то $\rg \mathcal{A} = \rg \varphi (\mathcal{A}) $.
%\end{sled}
%%\dok
%%\edok

Следующая теорема показывает, что
линейное отображение определено, причем единственным образом, образами базисных векторов.


\begin{theor}\label{t8_1_1}
Пусть $\bazis{e} = (\vek{e}_1, \vek{e}_2, \ldots , \vek{e}_n)$ --- базис в $V$,
и $\vek{c}_1, \vek{c}_2, \ldots , \vek{c}_n$ --- фиксированные векторы из~$\widetilde{V}$. Тогда \\
1). существует единственное линейное отображение $\varphi :V\to \widetilde{V}$ такое, что
$\varphi(\vek{e}_i) = \vek{c}_i$, $i=1, 2, \ldots , n$. \\
2). При этом $\varphi$ инъективно $\Leftrightarrow$ система $\vek{c}_1, \vek{c}_2, \ldots , \vek{c}_n$ линейно независима.
\end{theor}
\dok 1). {\it Единственность}. Пусть $\vek{a}\in V$ разложен по базису $\bazis{e}$: $\vek{a}= \sum\limits_{i=1}^n x_i\vek{e}_i$.
Тогда, в силу (\ref{eqLin}), однозначно получаем 
\begin{equation}\label{eqLin1}
\varphi(\vek{a})= \sum\limits_{i=1}^n x_i\varphi (\vek{e}_i) = \sum\limits_{i=1}^n x_i \vek{c}_i.
\end{equation}
{\it Существование}. Очевидно, что формула (\ref{eqLin1}) удовлетворяет условию $\varphi(\vek{e}_i) = \vek{c}_i$, $i=1, 2, \ldots , n$.
Достаточно доказать, что она действительно определяет линейное отображение.
Для произвольных векторов $\vek{a}, \vek{b} \in V$, таких, что $\vek{a}= \sum\limits_{i=1}^n x_i\vek{e}_i$, $\vek{b}= \sum\limits_{i=1}^n y_i\vek{e}_i$,
по определению отображения $\varphi$ (с учетом предложения \ref{p7_3_2} главы \ref{lin_prostr}) имеем: 
$\varphi(\vek{a})= \sum\limits_{i=1}^n x_i \vek{c}_i$, 
$\varphi(\vek{b})=  \sum\limits_{i=1}^n y_i \vek{c}_i$,
$\varphi(\vek{a}+\vek{b})= \sum\limits_{i=1}^n (x_i+y_i) \vek{c}_i$, 
$\varphi(\lambda \vek{a})= \sum\limits_{i=1}^n (\lambda x_i) \vek{c}_i$.
Теперь легко видеть, что  L1 и L2 выполнены.\\
2). \dokright Следует из предложения \ref{p8_1_1000}.\\
\dokleft Пусть $\varphi (\vek{a}) = \varphi (\vek{b})$ для некоторых векторов $\vek{a}\neq \vek{b}$, тогда  
$\varphi (\vek{a}- \vek{b}) = \vek{0}$. Разложим ненулевой вектор $\vek{a}- \vek{b}$ по базису $\bazis{e}$:
$\vek{a}- \vek{b} = \sum\limits_{i=1}^n \lambda_i \vek{e}_i$.
Тогда $\varphi (\vek{a}- \vek{b}) = \sum\limits_{i=1}^n \lambda_i \vek{c}_i = \vek{0}$. Получаем, что нетривиальная линейная
комбинация векторов $\vek{c}_1, \vek{c}_2, \ldots , \vek{c}_n$ равна $\vek{0}$. Противоречие.
\edok

\otstup

Утверждение теоремы \ref{t8_1_1} дает понимание, насколько мы свободны в определении линенйного отображения,
оно бывает полезно при конструировании линейных отображений с заданными свойствами.

\otstup

{\bf Упражнение.} Может ли для некоторых $\varphi \in L(V, V)$ и $\vek{a}\in V$ выполняться
одновременно условия $\varphi (\vek{a})\neq \vek{0}$, $\varphi (\varphi (\vek{a}))= \vek{0}$?


\subsection{Изоморфизм. Композиция (произведение). Обратное отображение. }

\defin{
Отображение $\varphi: V\to \widetilde{V}$ называется {\it изоморфизмом}, если оно
линейно и биективно.
}

\defin{
Векторные пространства $V$ и $\widetilde{V}$ называются
{\it изоморфными}, если существует изоморфизм $\varphi: V\to \widetilde{V}$.
}

Тот факт, что $V$ и $\widetilde{V}$ изоморфны, обозначаем $V\cong \widetilde{V}$.



\begin{predl}\label{p8_1_2}
1). Пусть $\varphi \in L(V, \widetilde{V})$, $\psi \in L(\widetilde{V}, \widehat{V})$. Тогда  
%$\varphi : V\to \widetilde{V}$ и $\psi : \widetilde{V}\to \widehat{V}$
%--- линейные отображения, то
$\psi  \varphi \in L(V, \widehat{V})$.\\
2). Если кроме того $\varphi$ и $\psi$ --- изоморфизмы, то
$\psi  \varphi$ --- также изоморфизм.
\end{predl}
\dok 1). Проверим L1 для отображения $\psi  \varphi$. Из определения композиции и условия L1 для отображений $\psi $ и $\varphi$ 
имеем: $\forall$ $\vek{a}, \vek{b}\in V$ выполнено \\ $(\psi  \varphi)(\vek{a} + \vek{b}) = 
\psi  (\varphi(\vek{a} + \vek{b})) = \psi  (\varphi(\vek{a}) + \varphi (\vek{b}))  = 
\psi  (\varphi(\vek{a})) + \psi  (\varphi (\vek{b}))  = 
(\psi  \varphi)(\vek{a}) + (\psi  \varphi) (\vek{b}). $ \\
Аналогично проверяется L2.\\
2). Следует из 1) и того, что композиция биективных отображений биективна.
\edok

\otstup

Говорят, что два преобразования $\varphi ,\psi \in L(V, V)$  {\it перестановочные} (или {\it коммутируют}), если $\varphi \psi= \psi \varphi $.
Отметим, что композиция линейных преобразований вообще говоря не подчиняется тождеству $\varphi \psi= \psi \varphi $.
Нетрудно привести соответствующие примеры (например, используя замечание после теоремы \ref{t8_1_1}).
%примеры двух преобразований $\varphi ,\psi \in L(V, V)$, которые не являются {\it перестановочными} (или {\it не коммутируют}),
%т.е. для которых $\varphi \psi\neq \psi \varphi $.

\begin{predl}\label{p8_1_3}
Если отображение $\varphi : V\to \widetilde{V}$  --- изоморфизм, то
$ \varphi ^{-1}$ --- также изоморфизм.
\end{predl}
\dok
Для данных векторов $\vek{a}, \vek{b} \in \widetilde{V}$ однозначно определены векторы
$\vek{c}= \varphi ^{-1} (\vek{a})$ и $\vek{d}= \varphi ^{-1} (\vek{b})$.

Так как $\varphi(\vek{c} + \vek{d}) = \varphi(\vek{c}) +  \varphi(\vek{d}) =
\vek{a}+\vek{b} $, то $\vek{c} + \vek{d} = \varphi ^{-1}(\vek{a}+\vek{b})$,
то есть $\varphi ^{-1}(\vek{a}+\vek{b}) = \varphi ^{-1}(\vek{a})+\varphi ^{-1}(\vek{b})$.

Далее, $\varphi(\lambda \vek{c}) = \lambda \varphi(\vek{c}) = \lambda \vek{a}$, откуда
$\varphi ^{-1}(\lambda \vek{a}) = \lambda \vek{c} = \lambda \varphi ^{-1}(\vek{a})$.
\edok

\otstup

Для целого неотрицательного $k$ определим $k$-ую степень преобразования $\varphi: V\to V$
как
$\varphi ^k= {\underbrace
{\varphi  \varphi \ldots \varphi}_{k \, \, \mbox{\scriptsize букв} \, \, \varphi}}$ при $k>0$
и как тождественное преобразование $I_V$ при $k=0$.
При этом справедливы равенства $\varphi ^{m+k}= \varphi ^m \varphi^k$
и $\varphi^{mk}= (\varphi ^m)^k$.
Если кроме того $\varphi$ --- изоморфизм, то
можно определить $k$-ую степень и для отрицательных $k$ как
$\varphi^{k}= (\varphi ^{-1})^{-k}$.
Нетрудно проверить, что в этом случае равенства $\varphi ^{m+k}= \varphi ^m \varphi ^k$
и $\varphi ^{mk}= (\varphi ^m)^k$ остаются в силе для всех $m, k\in \mathbb{Z}$.

\otstup

Очевидно $V\cong V$ (пример изоморфизма --- тождественное преобразование $I_V\, : \, V\to V$).
Предложения \ref{p8_1_2} и \ref{p8_1_3} показывают, что отношение <<быть изоморфными>>
симметрично и транзитивно, т.е. 
$V\cong \widetilde{V}$ $\Rightarrow$ $\widetilde{V} \cong V$ и
$V\cong \widetilde{V}$, $\widetilde{V} \cong \widehat{V}$ $\Rightarrow$ $V\cong \widehat{V}$.
Тем самым, все векторные пространства можно мыслить разбитыми на классы эквивалентности (так что 
два пространства из одного класса эквивалентности изоморфны, а из разных --- не изоморфны). 
Следующая теорема дает классификацию конечномерных векторных пространств.

\begin{theor}\label{t_isom}
Пусть $\dim V<\infty $, $\dim  \widetilde{V} <\infty $. 
Тогда $$V\cong \widetilde{V} \,\,\, \Leftrightarrow \,\,\,  \dim V = \dim \widetilde{V}.$$
\end{theor}
\dok \dokright 
Если $\varphi \,:\, V\to \widetilde{V}$ --- изоморфизм, то $\varphi$ инъективно и сюръективно.
Значит, согласно предложению \ref{p8_1_1000},
имеем $\dim V = \dim (\varphi (V)) = \dim \widetilde{V}$.

\dokleft Пусть $\dim V = \dim \widetilde{V} =  n$.
В предложении \ref{p7_3_2} главы \ref{lin_prostr} мы фактически сталкивались с изоморфизмом
 --- сопоставлением вектору его координатного столбца (в фиксированном базисе). Значит, у нас есть
<<эталонное>> $n$-мерное пространство $\mathbb{F}^n = \mathbf{M}_{n\times 1}$, так что 
$V\cong \mathbf{M}_{n\times 1}$, $\widetilde{V} \cong \mathbf{M}_{n\times 1}$, и следовательно,
$V\cong \widetilde{V}$.
\edok

\otstup

В следующей теореме соберем условия, эквивалентные изоморфности данного линейного отображения
в конечномерном случае.

\begin{theor}\label{t_isom1}
Пусть $\dim V = \dim  \widetilde{V}=n <\infty $, $\bazis{e} = (\vek{e}_1, \vek{e}_2, \ldots, \vek{e}_n )$ --- базис в $V$.
Пусть $\varphi \in L(V, \widetilde{V})$. Тогда следующие условия эквивалентны:\\
1) $\varphi$ --- изоморфизм;\\
2) $\varphi$ --- инъекция;\\
3) $\varphi$ --- сюръеция;\\
4)  $\varphi (\vek{e}_1), \varphi (\vek{e}_2), \ldots, \varphi ( \vek{e}_n )$ --- базис в $\widetilde{V}$.
\end{theor}
\dok 
1) $\Rightarrow$ 2) Очевидно.

2) $\Rightarrow$ 3) Согласно предложению \ref{p8_1_1000}, $\dim (\varphi (V)) = \dim V = n$. Значит, по следствию 2 из теоремы
\ref{t5_2_1}, имеем $\varphi (V)=\widetilde{V}$, то есть $\varphi$ сюръективно.

3) $\Rightarrow$ 4) Как мы знаем (см. предложение \ref{p8_1_103}), $\varphi (V) = \lin{\varphi (\vek{e}_1), \varphi (\vek{e}_2), \ldots, \varphi ( \vek{e}_n )}$.
Из того, что $\dim \varphi (V) =n$, следует, что $\rg (\varphi (\vek{e}_1), \varphi (\vek{e}_2), \ldots, \varphi ( \vek{e}_n ) )  =n$,
значит, система векторов $\varphi (\vek{e}_1), \varphi (\vek{e}_2), \ldots, \varphi ( \vek{e}_n )$ линейно независима, т.е. является базисом в $V$.

4) $\Rightarrow$ 1)  Согласно теореме \ref{t8_1_1}, $\varphi$ инъективно.
А поскольку $\varphi (V) = \lin{\varphi (\vek{e}_1), \varphi (\vek{e}_2), \ldots, \varphi ( \vek{e}_n )}$, 
$\varphi$ сюръективно.
\edok




\subsection{Линейные операции на $\Hom(V; \widetilde{V})$.}

\defin{
{\it Суммой}  отображений $\varphi, \psi \in L(V, \widetilde{V})$ называется такое
отображение $\eta: V\to \widetilde{V}$, что $\forall \vek{a} \in V$ выполнено
$$\eta (\vek{a}) = \varphi (\vek{a}) + \psi  (\vek{a}).$$
}

\defin{
{\it Произведением} отображения $\varphi \in L(V, \widetilde{V})$ на константу $\lambda\in \mathbb{F}$
называется такое отображение $\psi: V\to \widetilde{V}$, что $\forall \vek{a} \in V$ выполнено 
$$\psi (\vek{a}) = \lambda \varphi (\vek{a}).$$ 
}


Результат операции сложения и умножения на число называется, как обычно, суммой и произведением на
число,  и обозначаются обычным образом:
$\varphi  + \psi $ и $\lambda \varphi$. Таким образом, определения означают выполнение 
следующих равенств, которые выглядят вполне естественно:
$$(\varphi  + \psi) (\vek{a}) = \varphi (\vek{a}) + \psi  (\vek{a});$$
$$ (\lambda \varphi) (\vek{a}) = \lambda \varphi (\vek{a}).$$ 


При $\mathbb{F} = \mathbb{C}$ умножение линейного отображения на константу может быть определено
особым способом (и использованием комплексного сопряжения). {\it Второй вариант} определения: 
$$(\lambda \varphi) (\vek{a}) = \overline{\lambda} \varphi (\vek{a}).$$

%Так, для пространства над $\mathbb{C}$ имеются два варианта определения $\lambda \varphi$.
%Когда мы желаем указать, что в множестве $L(V, \widetilde{V})$ выбран второй вариант определения $\lambda \varphi$,
%пишем $\overline{L}(V, \widetilde{V})$.

\begin{predl}\label{p8_1_4}
Если $\varphi, \psi \in L(V, \widetilde{V})$, то
$\varphi + \psi \in L(V, \widetilde{V})$ и
$\lambda \varphi \in L(V, \widetilde{V})$.
\end{predl}
\dok Проверяется непосредственно. Например, проверим L2 для $\lambda \varphi$ со вторым вариантом определения.
Имеем $(\lambda \varphi)(\mu \vek{a}) = \overline{\lambda} \varphi(\mu \vek{a}) =  \overline{\lambda} \mu  \varphi(\vek{a}) =
\mu (\overline{\lambda}   \varphi(\vek{a})) = \mu (\lambda   \varphi)(\vek{a}).$ Тем самым, $\lambda \varphi$ удовлетворяет L2.
\edok

\begin{predl}\label{p8_1_5}
Множество $L(V, \widetilde{V})$ является векторным пространством относительно
введенных выше операций сложения и умножения на число. 
\end{predl}
\dok Проверяется непосредственно. 
\edok

\otstup

При $\mathbb{F} = \mathbb{C}$ множество $L(V, \widetilde{V})$ наделяется структурой
векторного пространства двумя способами, в соответствии с двумя способами определения 
умножения на константу. В случае рассмотрения второго варианта 
%определения пробозначаем векторное пространство
%линейных отображений $V\to \widetilde{V}$ с первым вариантом определения умножения на число,
вместо $L(V, \widetilde{V})$ пишем $\overline{L}(V, \widetilde{V})$.
% --- то же пространство со вторым вариантом
%определения умножения на число.

%Имеется следующая
%связь между композиции с линейными операциями:
% сложением и умножением на число:

\begin{predl}\label{p8_1_6}
Пусть $\varphi, \psi \in L(V, \widetilde{V})$,
$\widetilde{\varphi}, \widetilde{\psi} \in L(\widetilde{V}, \widehat{V})$.
Тогда\\
1. $\widetilde{\varphi} (\varphi + \psi) = \widetilde{\varphi} \varphi + \widetilde{\varphi}
\psi$;\\
2. $(\widetilde{\varphi} +\widetilde{\psi}) \varphi  =
\widetilde{\varphi} \varphi +\widetilde{\psi} \varphi $;\\
3. $\lambda (\widetilde{\varphi} \varphi) = (\lambda \widetilde{\varphi}) \varphi =
\widetilde{\varphi} (\lambda \varphi)$.
\end{predl}
\dok  Проверяется непосредственно.
\edok

%\otstup




\subsection{Функции от операторов.}

На множестве линейных операторов $L(V, V)$ определены операции 
сложения, умножения на константы и умножения. Предложения \ref{p8_1_5} и \ref{p8_1_6}
вметсте с условием ассоциативности умножения (которая выполнена для композиции произвольных отображений)
означают, что $L(V, V)$ имеет структуру {\it ассоциативной алгебры}.
Поэтому $L(V, V)$ можем называть алгеброй линейных операторов на пространстве $V$.

Если $\varphi \in L(V, V)$, а $f\in \mathbb{F}[X]$ --- некоторый многочлен,
так что $f(x) = \sum \limits_{i=0}^{m}a_ix^i$,
то положим $f(\varphi) = \sum \limits_{i=0}^{m}a_i \varphi^i$ (напомним, что $\varphi ^0 = I_V$ --- тождественное
преобразование).
Преобразование $f(\varphi)$ перестановочно с $\varphi$.
Более общо, если $f$ и $g$ --- два многочлена, то $f(\varphi)$ и $g(\varphi)$ ---
перестановочные преобразования.

{\footnotesize Формально, при фиксированном $\varphi \in L(V, V)$ подстановка  $f\mapsto f(\varphi)$ задает гомоморфизм алгебр $\mathbb{F}[X] \to L(V, V)$. Образ --- множество многочленов от $\varphi$ --- является
коммутативной подалгеброй в $L(V, V)$. Ниже мы говорим о ядре этого гомоморфизма.}

\otstup

Говорят, что многочлен $f\in \mathbb{F}[X]$ является {\it аннулирующим} для оператора $\varphi \in L(V, V)$
(или $f$ аннулирует $\varphi$), если $f(\varphi) = 0$ (нулевой оператор).
При $\dim V<n$ для данного $\varphi \in L(V, V)$ имеются ненулевые аннулирующие многочлены.
Отложив доказательство этого факта, установим структуру множества многочленов, аннулирующих $\varphi$.


\defin{
Ненулевой многочлен $\mu \in \mathbb{F}[X]$ называется {\it минимальным} многочленом оператора $\varphi \in L(V, V)$, если
$\mu$ аннулирует $\varphi$ и имеет минимальную степень среди всех аннулирующих $\varphi$ многочленов.
}

\begin{predl}\label{min_mn}
Пусть  $\mu \in \mathbb{F}[X]$ 
 --- минимальный многочлен оператора $\varphi \in L(V, V)$ и $f\in \mathbb{F}[X]$. 
Тогда $f$ аннулирует $\varphi$ $\Leftrightarrow$  $f\, \vdots \, \mu$ 
(делится в кольце $\mathbb{F}[X]$).
\end{predl}
\dok  Разделим $f$ на $\mu$  с остатком: $f = q\mu + r$, где $q,r \in \mathbb{F}[X]$, $\deg r < \deg \mu$.
Тогда равенство остается верным после подстановки $\varphi$: $f (\varphi) = q(\varphi)\mu (\varphi)+ r(\varphi)$.
Поскольку $\mu (\varphi) = 0$, имеем $f (\varphi) =  r (\varphi)$. \\
Значит,  $f (\varphi)= 0$ $\Leftrightarrow$  $r (\varphi)=0$.
Условие $r (\varphi)=0$ означает, что $r$ аннулирует $\varphi$ и его степень меньше степени минимального многочлена, 
т.е. $r=0$ (нулевой многочлен). Таким образом, 
$f (\varphi)= 0$ $\Leftrightarrow$  $r=0$ $\Leftrightarrow$ $f\, \vdots \, \mu$. 
\edok

\begin{sled}\label{min_mn_ed}
Для оператора $\varphi \in L(V, V)$ минимальный многочлен единственный, 
с точностью до умножения на ненулевую константу.
\end{sled}
\dok  Если  $\mu$ и $\mu '$ --- оба минимальные многочлены для $\varphi$, 
то, согласно предложению \ref{min_mn}, $\mu' \, \vdots \, \mu$ и $\mu \, \vdots \, \mu'$. 
\edok


%%%%%%%%%%%%%
%%ВОПРОС --- ПОЧЕМУ МИНИМАЛЬНЫЙ МНОГОЧЛЕН НЕ МОЖЕТ ИЗМЕНИТЬСЯ ПРИ РАСШИРЕНИИ ПОЛЯ
%%%%%%%%%%%%%



\otstup

Многочлен являются частным случаем степенного ряда, или более общо, ряда Лорана. 
Можно определять функции от оператора $\varphi$ через значение соответствующего операторного 
ряда $\sum a_i \varphi ^i$, правда  
во мноих случаях для обеспечения сходимости такого ряда 
требуются дополнительные условия на $\varphi$
(кроме того, сходимость  можно понимать 
в разных  смыслах). Так, для линейных операторов $\varphi : V\to V$, где $V$ --- 
пространство над $\mathbb{R}$ или $\mathbb{C}$, можно говорить об
$\exp (\varphi)$ и т.д. 

%При определенных условиях можно ввести функции от преобразования, например $\exp (\varphi)$

%(О перестановочности (вспомнить, когда будет разговор о матрицах) преобразований

\subsection{Примеры}


Пусть $V = U_1 \bigoplus U_2$. Тогда для каждого вектора $\vek{a}$ имеется единственное
разложение $\vek{a} = \vek{a}_1 + \vek{a}_2$, где
$\vek{a}_1$ --- проекция $\vek{a}$ на $U_1$ вдоль $U_2$,
$\vek{a}_2$ --- проекция $\vek{a}$ на $U_2$ вдоль $U_1$.

\example{
Отображение $\varphi: V\to V$ такое, что $\varphi(\vek{a}) = \vek{a}_1$,
называется {\it проектированием} на $U_1$ вдоль (или параллельно) $U_2$. 
}
\example{
Отображение $\psi: V\to V$ такое, что $\varphi(\vek{a}) = \vek{a}_1 - \vek{a}_2$,
называется {\it отражением}  (или {\it симметрией}) относительно $U_1$ вдоль (или параллельно) $U_2$.
}

Легко проверить, что отражение  и проектирование --- 
линейные преобразования, причем отражение является изоморфизмом.


\example{I.
Пусть  $\varphi: \mathbf{R}^3\to \mathbf{R}^3$ --- 
движение, для которого начало координат $O$ --- неподвижная точка (отождествляем точки с радиус-векторами),
например, $\varphi$ --- поворот вокруг оси (прямой, проходящей через $O$).\\
Линейные биективные преобразования плоскости $\varphi: \mathbf{R}^2\to \mathbf{R}^2$ ---
 %$\mathbf{R}^2 \to \mathbf{R}^2$
аффинные преобразовани, для которых $O$ --- неподвижная точка (см. ниже параграф \ref{aff}).
%Тогда $\varphi$ --- линейное и биективное преобразование.
}

\example{II.1.
Пусть $A\in \mathbf{M}_{m\times n}$ --- фиксированная матрица.
Определим $\varphi \in L(\mathbf{M}_{n\times p}, \mathbf{M}_{m\times p})$
правилом $\varphi (X) = AX$. Нетрудно проверить, что $\varphi$ линейно (ниже увидим, что в некотором смысле к этому примеру можно 
свести любое линейное отображение конечномерных пространств). \\
Аналогично можно определить отображение домножение справа на фиксированную матрицу.
}

\example{II.2.
Транспонирование матриц $m\times n$ --- пример изоморфизма
$\mathbf{M}_{m\times n}\to \mathbf{M}_{n\times m}$.
}

\example{III.1.
Пусть $\mathbf{C}[a,b]$ --- пространство всех непрерывных функций, определенных на отрезке $[a,b]$. 
Для различных отрезков $[a,b]$ пространства $\mathbf{C}[a,b]$ изоморфны. Например, изоморфизм между 
пространствами $\mathbf{C}[-2,0]$, $\mathbf{C}[0,1]$ определяется %$\varphi \in L(\mathbf{C}[-2,0], \mathbf{C}[0,1])$, определенное 
правилом $\varphi (f(x)) \hm= f(2x-2)$. % является изоморфизмом.
}

\example{III.2.
Пусть $V= \mathbf{C}^{1}(\mathbb{R})$. 
{\it Дифференцирование}  %$\dfrac{d}{dx}: Vbf{P} \to \mathbf{P}$ задается правилом $\varphi (f(x)) = f'(x)$.
$d: V \to V$ задается правилом $(d (f))(x) = f'(x)$. Нетрудно видеть, что $d$ --- линейное преобразование.
Можно определить {\it линейный дифференциальный оператор} ---  многочлен от $d$.\\
В случае функций многих переменных (для $V= \mathbf{C}^{1}(\mathbb{R}^n)$) можно рассмотреть линейные операторы взятия частной производной $\dfrac{\partial}{\partial x_i}$ (эти операторы --- не перестановочные, 
но становятся таковыми при оганичении на подпространство $\mathbf{C}^{2}(\mathbb{R}^n)$).\\
Линейное отображение $\mathbf{C} (\mathbb{R}) \to \mathbf{C} (\mathbb{R})$ {\it интегрирование} с переменным верхним пределом %(на множестве многочленов степени не выше $n$)
%$\psi: \mathbf{P}_n \to \mathbf{P}_{n+1} $ 
задается правилом $\varphi (f(x)) = \int\limits_{0}^{x} f(t)\, dt$.
%СУЖЕНИЕ....
%ИНТЕГРИРОВАНИЕ С ПОСТОЯННЫМ ПРЕДЕЛОМ....
%Урматы=Кер
%(Здесь $P$ и $P_n$ --- пространства многочленов и многочленов степени, не превосходящей $n$, соответственно.)
}

\example{III.3.
Пусть  $V=\mathbf{F}(\mathbb{N})=\{(a_1, a_2, \ldots) \, | \, a_i\in \mathbb{F}\}$ --- пространство числовых последовательностей.
{\it Оператор сдвига} $\varphi : V\to V$ определяется равенством $\varphi (a_1, a_2, \ldots) = (a_2, a_3, \ldots)$.
Определим {\it оператор первой разности} $\Delta = \varphi - I_V$, так что
$\Delta (a_1, a_2, \ldots) = (a_2-a_1, a_3-a_2, \ldots)$.\\
Для $k\in \mathbb{N}$ степень $\Delta ^k$ называют {\it оператором $k$-ой разности}.
}

\example{IV.
Пусть  $\mathbb{F}$ --- подполе некоторого поля $\mathbb{K}$, а $\varphi : \mathbb{K}\to \mathbb{K}$
--- изоморфизм поля $\mathbb{K}$ на себя (автоморфизм), при котором все элементы 
поля $\mathbb{F}$ неподвижны. Тогда $\varphi $ является линейным оператором $\mathbb{K}\to \mathbb{K}$, 
где $\mathbb{K}$ рассматривается как линейное пространство над $\mathbb{F}$.
}



\section{Матрица линейного отображения}\label{matr_lin_otobr}

В этом параграфе занимаемся только конечномерными векторными пространствами. Полагаем
$\dim V = n$, $\dim \widetilde{V} = m$, $\dim \widehat{V} = p$.

\subsection{Определение. Координатная запись линейного отображения}

\defin{
Пусть в пространствах $V$ и $\widetilde{V}$ зафиксированы базисы
$\bazis{e}=(\vek{e}_1, \vek{e}_2, \ldots , \vek{e}_n)$ и
$\bazis{f}=(\vek{f}_1, \vek{f}_2, \ldots , \vek{f}_m)$ соответственно.
{\it Матрицей линейного отображения} $\varphi \in L(V, \widetilde{V})$
в паре базисов $\bazis{e}$ и $\bazis{f}$
называется матрица $A\in \mathbf{M}_{m\times n}$, столбцы $a_{\bullet 1}, a_{\bullet 2}, \ldots , a_{\bullet n}$
которой --- координатные столбцы соответственно
векторов $\varphi (\vek{e}_1), \varphi (\vek{e}_2), \ldots , \varphi (\vek{e}_n)$
в базисе $\bazis{f}$.
}

Матрицу линейного преобразования $\varphi \in L(V, V)$ в паре совпадающих базисов $\bazis{e}$ и $\bazis{e}$
будем также называть короче: матрица $\varphi$ в базисе $\bazis{e}$.

Компактно определение можно записать как $(\varphi (\vek{e}_1)\, \varphi (\vek{e}_2)\, \ldots \, \varphi (\vek{e}_n)) = \bazis{f} A$.
Тот факт, что $A$ --- матрица линейного отображения $\varphi$
в паре базисов $\bazis{e}$ и $\bazis{f}$, будем обозначать
$\varphi \rsootv{\bazis{e}, \bazis{f}} A$.


\begin{predl}\label{p8_3_111}
Пусть в пространствах $V$ и $\widetilde{V}$ зафиксированы базисы
$\bazis{e}$ и $\bazis{f}$ соответственно.
Тогда отображение $\varphi \rsootv{\bazis{e}, \bazis{f}} A$ --- биекция между $L(V, \widetilde{V})$ и $\mathbf{M}_{m\times n}$.
\end{predl}
\dok По определению матрица $A$ несет в себе полную информацию об образах базисных векторов.
Остается воспользоваться теоремой \ref{t8_1_1}.
%что фиксация базисов в $V$ и $\widetilde{V}$ дает взаимно-однозначное соответствие между
%$\varphi \in L(V, \widetilde{V})$ и $\mathbf{M}_{m\times n}$.
\edok

\otstup 

Выведем формулу координатной записи линейного отображения.

\begin{theor}\label{t8_3_1}
Пусть $\varphi \rsootv{\bazis{e}, \bazis{f}} A$.
Если $\vek{a} = \bazis{e} X$, $\varphi (\vek{a}) = \bazis{f} Y$, то 
$$\boxed{Y=AX}.$$
\end{theor}
\dok Так как $\vek{a} = \sum\limits_{i=1}^n x_i \vek{e}_i$, 
то $\varphi (\vek{a}) = \sum\limits_{i=1}^n x_i \varphi(\vek{e}_i)$. Заменим в этом равенстве векторы на их координатные
столбцы в базисе $\bazis{f}$, получим:
$Y= \sum\limits_{i=1}^n x_i a_{\bullet i}$, где, как обычно, $a_{\bullet i}$ обозначает $i$-й столбец матрицы $A$.
Правая часть последнего равенства равна $AX$, что и требовалось уствновить.
\edok

\otstup

Формула из предыдущей теоремы фактически эквивалентна определению матрицы линейного отображения.
Более, точно, справедливо следующее %предложение.
%
%\begin{predl}\label{p8_3_2}
\begin{sled}
Пусть дано 
отображение $\varphi :V\to \widetilde{V}$ (априори не известно, что оно линейное) 
и матрица $A \in M_{m\times n}$.
Пусть $\forall$ $\vek{a} \in V$ координатные столбцы $X$ и $Y$ векторов
$\vek{a} = \bazis{e} X$ и $\varphi (\vek{a}) = \bazis{f} Y$ связаны 
равенством $Y=AX$. Тогда $\varphi \in L(V, \widetilde{V})$, причем
$\varphi \rsootv{\bazis{e}, \bazis{f}} A$.
\end{sled}
%\end{predl}
\dok Возьмем $\psi \in L(V, \widetilde{V})$ такое, что $\psi \rsootv{\bazis{e}, \bazis{f}} A$ (такое 
$\psi$ существует, согласно предложению \ref{p8_3_111}). Тогда $\varphi$ и $\psi$ имеют одну и ту же координатную запись $Y=AX$, т.е. 
$\varphi = \psi$.
\edok

\subsection{Матрица перехода и матрица преобразования}

Переход от базиса $\bazis{e}=(\vek{e}_1, \ldots, \vek{e}_n)$ к $\bazis{e}'=(\vek{e}'_1, \ldots, \vek{e}'_n)$  
формально не связан с  линейными преобразованиями. Однако, согласно теореме \ref{t8_1_1}, 
по паре базисов $\bazis{e}$ и $\bazis{e}'$ можно определить единственное линейное отображение 
$\varphi: V\to V$ такое, что $\varphi (\vek{e}_i) = \vek{e}'_i$ для $i=1, \ldots, n$
(при этом $\varphi $ является  изоморфизмом --- см. теорему \ref{t_isom1}). 
Отметим следующую связь между матрицей линейного преобразования и матрицей перехода.

%Матрица этого изоморфизма в базисе $\bazis{e}$ будет совпадать с матрицей перехода от $\bazis{e}$ к $\bazis{e}'$.


\begin{predl}\label{p8_3_112}
Пусть $\dim V=n<\infty$, $\bazis{e}$ и $\bazis{e}'$ --- базисы в $V$.
Пусть изоморфизм $\varphi: V\to V$ таков, что $\varphi (\vek{e}_i) = \vek{e}'_i$ для $i=1, \ldots, n$;
$\varphi \rsootv{\bazis{e}, \bazis{e}} A$. Тогда $A$ --- матрица перехода от $\bazis{e}$ к $\bazis{e}'$.
\end{predl}
\dok Достаточно сопоставить определения матрицы линейного отображения и матрицы перехода.
\edok




\subsection{Связь между операциями над отображениями и матрицами}

Предложение \ref{p8_3_111} может быть усилено следующим образом.

\begin{theor}\label{p8_3_333}
Пусть $\bazis{e}$ и $\bazis{f}$ --- базисы  пространств $V$ и $\widetilde{V}$ соответственно.
Тогда отображение $\varphi \rsootv{\bazis{e}, \bazis{f}} A$ является изоморфизмом
линейных пространств $L(V, \widetilde{V})$ и $\mathbf{M}_{m\times n}$.\\
\end{theor}
\dok Непосредственная проверка.
\edok

\begin{sled}
$\dim L(V, \widetilde{V}) = mn$.
\end{sled}
%\dok
%\edok

\otstup

Аналогично, для векторного пространства $V$ над $\mathbb{C}$ 
пространство $\overline{L}(V, \widetilde{V})$ (со вторым способом определения отображения на константу)
изоморфно $\mathbf{M}_{m\times n}(\mathbb{C})$; изоморфизм устанавливается как
$\varphi \mapsto \overline{A}$.


\begin{predl}\label{p8_3_4}
Пусть $\varphi \in L(V, \widetilde{V})$, $\widetilde{\varphi} \in
L(\widetilde{V}, \widehat{V})$, $\bazis{e}$, $\bazis{f}$, $\bazis{g}$ 
--- базисы  пространств $V$, $\widetilde{V}$, $\widehat{V}$ соответственно.
Пусть $\varphi \rsootv{\bazis{e}, \bazis{f}} A$,
 $\widetilde{\varphi} \rsootv{\bazis{f}, \bazis{g}} \widetilde{A}$.
Тогда $ \widetilde{\varphi} \varphi \rsootv{\bazis{e}, \bazis{g}} \widetilde{A} A$.
\end{predl}
\dok Пусть $\vek{a}\in V$ --- произвольный вектор, $\vek{a} = \bazis{e}X$, 
$\varphi(\vek{a}) = \bazis{f}Y$, $\widetilde{\varphi}(\varphi(\vek{a})) = \bazis{g}Z$.
 Тогда дважды пользуясь теоремой \ref{t8_3_1}, имеем
$Y=AX$, $Z=\widetilde{A}Y$, откуда $Z= \widetilde{A}(AX)=(\widetilde{A}A)X$.
Отсюда следует требуемое
(см. следствие из теоремы \ref{t8_3_1}).
\edok

\begin{sled1}
Пусть $\varphi \rsootv{\bazis{e}, \bazis{f}} A$.
Тогда $\varphi$ --- изоморфизм $\Leftrightarrow$ $A$ обратима.
Если $\varphi$ --- изоморфизм, то
$\varphi ^{-1} \rsootv{\bazis{f}, \bazis{e}} A^{-1}$.
\end{sled1}
\dok Следует из того, что $I_V \rsootv{\bazis{e}, \bazis{e}} E$.
\edok

\begin{sled2}
Пусть $\varphi \in L(V, V)$ и $\varphi \rsootv{\bazis{e}, \bazis{e}} A$.
Тогда  для любого многочлена $p$ имеем $p(\varphi) \rsootv{\bazis{e}, \bazis{e}} p(A)$.
\end{sled2}

\begin{sled3}
Пусть $\bazis{e}$ --- базис  пространства $V$.
Тогда отображение $\varphi \rsootv{\bazis{e}, \bazis{e}} A$ является изоморфизмом 
алгебр $L(V, V)$ и $\mathbf{M}_{n\times n} (\mathbb{F})$.\\
В частности группа %$L(V, V)^*$
(по умножению) изоморфизмов $V\to V$ изоморфна $GL_n(\mathbb{F})$ (группе невырожденных матриц).
\end{sled3}

\otstup

Установленное соответствие между операциями над линейными отображенями и операциями над матрицами
может быть применено в обе стороны: некоторые задачи об отображениях могут быть сведены к вопросам о матрицах, и наоборот (см., например, упражнения в конце следующего параграфа).

Многие понятие могут быть определены одновременно для линейных операторов и для матриц.
Напрмер: говорят о {\it ранге отображения}, имея в виду ранг матрицы этого отображения
(ниже мы увидим, что этот ранг не зависит от выбора базисов);
{\it невырожденным} называют оператор
$\varphi \in L(V, V)$, матрица которого невырожденная (это условие эквивалентно тому, что $\varphi$ --- изоморфизм);
c другой стороны, 
 можем говорить об аннулирующих и минмальных многочленах для матриц из $\mathbf{M}_{n\times n} (\mathbb{F})$.




\subsection{Изменение матрицы при замене базиса}

\begin{theor}\label{t8_3_2}
Пусть в $V$ выбраны базисы $\bazis{e}$ и $\bazis{e}'$, связанные матрицей перехода $S$:
$\bazis{e}'= \bazis{e} S$;
в $\widetilde{V}$ выбраны базисы $\bazis{f}$ и $\bazis{f}'$, связанные матрицей перехода $R$:
$\bazis{f}'= \bazis{f} R$.
Пусть $\varphi \in L(V, \widetilde{V})$ таково, что 
$\varphi \rsootv{\bazis{e}, \bazis{f}} A$ и $\varphi \rsootv{\bazis{e'}, \bazis{f'}} A'$.
%имеет матрицу $A$ в базисах $\bazis{e}$, $\bazis{f}$, и матрицу
%$A'$ в базисах $\bazis{e}'$, $\bazis{f}'$.
Тогда  $$\boxed{A' = R^{-1}AS}.$$
В частности, если $V=\widetilde{V}$, $\bazis{e}= \bazis{f}$ и $\bazis{e}'= \bazis{f}' $, то 
$$\boxed{A' = S^{-1}AS}.$$
\end{theor}
\dok Пусть $\vek{a}\in V$ --- произвольный вектор. Пусть $\vek{a}=\bazis{e}X = \bazis{e}'X'$, 
$\varphi(\vek{a})=\bazis{f}Y = \bazis{f}'Y'$. Тогда по теореме \ref{t8_3_1}  имеем $Y=AX$, $Y'=A'X'$ и
(по теореме \ref{t7_3_2}, глава \ref{lin_prostr}) $X=SX'$, $Y=RY'$. 
Отсюда $RY' = ASX'$ $\Rightarrow$ $Y' = R^{-1}ASX'$ или $Y' = (R^{-1}AS)X'$.
Получаем требуемое: $R^{-1}AS=A'$ (см. следствие из теоремы \ref{t8_3_1}).
\edok

\begin{sled}
Ранг матрицы линейного отображения $\varphi \in L(V, \widetilde{V})$ не зависит от выбора базисов в пространствах $V$ и $\widetilde{V}$.
\end{sled}
\dok
Следует из того, что $A'$ получается из $A$ домножением слева и справа на невырожденные матрицы.
\edok

\otstup

Инвариантная характеризация ранга матрицы линейного отображения дается ниже в следствии из теоремы \ref{t8_2_111}.

\otstup

Заметим также, что матрицу $S^{-1}AS$ иногда называют {\it подобной} или {\it сопряженной} матрице $A$.
Нетрудно проверить, что отношение подобия --- это отношение эквивалентности на множестве
$\mathbf{M}_{n\times n}(\mathbb{F})$. Это согласуется с тем фактом, что подобные матрицы соответствуют
одному и тому же оператору, в разных базисах.


\subsection{Примеры}

Пусть $V = U_1 \bigoplus U_2$. 
Введем базис  $\bazis{e} = (\vek{e}_1, \vek{e}_2, \ldots , \vek{e}_n)$ в $V$, {\it согласованный}  $U_1 \bigoplus U_2$, 
так, что $\vek{e}_1, \ldots , \vek{e}_k$ --- базис в $U_1$, а $\vek{e}_{k+1}, \ldots , \vek{e}_n$ --- базис в $U_2$.\\
Пусть  $\varphi: V\to V$ --- проектирование на $U_1$ вдоль $U_2$. 
Тогда (по определению матрицы линейного отображения): $\varphi \rsootv{\bazis{e}, \bazis{e}} \begin{pmatrix}
E_k & O \\
O & O
\end{pmatrix}.$\\
Пусть  $\psi: V\to V$ --- отражение относительно $U_1$ вдоль $U_2$. 
Тогда $\psi \rsootv{\bazis{e}, \bazis{e}} \begin{pmatrix}
E_k & O \\
O & -E_{n-k}
\end{pmatrix}.$

%В СТАНДАРТНОМ БАЗИСУ УМНОЖЕНИЕ НА ФИКС, МАТРИЦУ


\example{III.1.
Пусть $V= \mathbf{P}_n = \lin{1, x, x^2, \ldots, x^n}$ и 
$\bazis{e} = (1, x, x^2, \ldots, x^n)$ --- стандартный базис в $V$.
Оператор дифференцирования $d: V\to V$ имеет в базисе $\bazis{e}$ матрицу 
$\begin{pmatrix}
0 & 1 & 0 & \ldots & 0\\
0 & 0 & 2 & \ldots & 0\\
 &  & \ldots &  & \\
0 & 0 & 0 & \ldots & n\\
0 & 0 & 0 & \ldots & 0
\end{pmatrix}$
(поскольку $(x^{k+1})' = (k+1)x^k$, имеем $d(e_{i+1}) = (i+1)e_i$). %, $k=1, 2, \ldots, n$.
}




\section{Образ и ядро}


\subsection{Образ}

В предложении \ref{p8_1_103} мы видели, что при линейном отображении $\varphi \in L(V, \widetilde{V})$ образ $\varphi (U)$ подпространства $U\leq V$
является подпространством в  $\widetilde{V}$.
Для образа отображения $\varphi $ наряду с обозначением $\varphi (V)$ используют обозначение $\Im \varphi$.
Очевидно, $\varphi \in L(V, \widetilde{V})$ является сюръективным $\Leftrightarrow$ $\widetilde{V} = \Im \varphi$.

%\begin{theor}\label{t8_2_1}
%Пусть $\varphi \in L(V, \widetilde{V})$, и $\bazis{e} = (\vek{e}_1, \vek{e}_2, \ldots , \vek{e}_n)$
%--- базис в  $V$. Тогда 
%$\Im \varphi = \lin{\varphi(\vek{e}_1), \varphi(\vek{e}_2), \ldots , \varphi(\vek{e}_n) }.$
%\end{theor}
%\dok  Следует из \ref{p8_1_101}.
%\edok
%%ЭТО ПОВТОР!!

\begin{theor}[координатное описание образа]\label{t8_2_111}
Пусть $\varphi \in L(V, \widetilde{V})$,  $\bazis{e}$, $\bazis{f}$ --- базисы в $V$ и $\widetilde{V}$.
Пусть $\varphi \rsootv{\bazis{e}, \bazis{f}} A$.
Для вектора $\vek{b} \in \widetilde{V}$, $\vek{b}=\bazis{f}Y$ выполнено:
$\vek{b} \in \Im \varphi$ $\Leftrightarrow$ $Y\in \lin{a_{\bullet 1}, a_{\bullet 2}, \ldots, a_{\bullet n}}$.
\end{theor}
\dok  Следует из сопоставления определения матрицы линейного отображения и предложения \ref{p8_1_103}.
\edok

Иначе говоря, теорема утверждает, что в терминах координатных столбцов $\Im \varphi$ задается 
как линейная оболочка столбцов матрицы линейного отображения.
Это, в частности, дает еще одно объяснение того факта, что 
$\rg A$ не зависит от выбора базисов в пространствах $V$ и $\widetilde{V}$.

\begin{sled}
В условиях теоремы $\boxed{\dim \Im \varphi = \rg A} $.
\end{sled}


\subsection{Ядро}

Покажем вначале, что при линейном отображении полный прообраз подпространства является подпространством.

\begin{predl}\label{proobr}
Пусть $\varphi \in L(V, \widetilde{V})$ и  $\widetilde{U}\leq \widetilde{V}$.
Тогда 
%1. $\varphi (U)\leq \widetilde{V}$; \\
$\{ \vek{a} \in V \, | \, \varphi (\vek{a}) \in \widetilde{U} \} \leq V$.
%$\varphi ^{-1}(\widetilde{U})$ 
%Полный прообраз подпространства $\widetilde{V}$
%--- подпространство в $V$.
\end{predl}
\dok Положим $U=\{ \vek{a} \in V \, | \, \varphi (\vek{a}) \in \widetilde{U} \}$ и проверим для $U$ свойства П1 и П2.\\
Пусть $ \vek{a}, \vek{b}$ --- произвольные векторы из $U$. Тогда 
$\varphi(\vek{a})\in \widetilde{U}$, $\varphi(\vek{b})\in \widetilde{U}$. Из свойства П1 для 
$\widetilde{U}$ получаем, что $\varphi(\vek{a})+\varphi(\vek{b})\in \widetilde{U}$, то есть
$\varphi(\vek{a}+\vek{b})\in \widetilde{U}$. Но последнее включение означает, что $\vek{a}+\vek{b}\in U$.\\
П2 проверяется аналогично.
\edok

\defin{{\it Ядром} линейного отображения $\varphi \in L(V, \widetilde{V})$
называется  подмножество $\{ \vek{a} \in V \, | \, \varphi (\vek{a}) = \vek{0}\}$.
}

Обозначение для ядра --- $\Ker \varphi$. %Для образа отображения $\varphi $ наряду с обозначением $\varphi (V)$ используют обозначение $\Im \varphi$.
Опеределение можно переформулировать так: $\Ker \varphi$ --- это полный прообраз нулевого подпространства,
поэтому из предложения \ref{proobr} вытекает


\begin{sled}\label{Ker}
Если $\varphi \in L(V, \widetilde{V})$, то $\Ker \varphi \leq V$. %и $\Im \varphi \leq \widetilde{V}$. % --- подпространства в $V$ и $\widetilde{V}$ соответственно.
\end{sled}




\otstup 

Для выяснения, является ли $\varphi$ инъективным, полезен следующий критерий.

\begin{predl}\label{p8_2_2}
$\varphi \in L(V, \widetilde{V})$ инъективно $\Leftrightarrow$ $\Ker \varphi = O$.
\end{predl}
\dok 
Следующее доказательство похоже на доказательство пункта 2 теоремы \ref{t8_1_1}.\\
\dokright Пусть $\vek{a}\in \Ker \varphi $, то есть $\varphi  (\vek{a})=\vek{0}$.
Но также $\varphi  (\vek{0})=\vek{0}$, и поскольку $\varphi$ инъективно, имеем $\vek{a}=\vek{0}$.\\
\dokright Пусть $\varphi  (\vek{a})=\varphi  (\vek{b})$. Тогда $\varphi  (\vek{a}-\vek{b})=\vek{0}$.
С учетом $\Ker \varphi = O$ имеем $\vek{a}-\vek{b}=\vek{0}$, то есть $\vek{a}=\vek{b}$.
Тем самым, $\varphi  (\vek{a})=\varphi  (\vek{b})$ $\Rightarrow$ $\vek{a}=\vek{b}$, и инъективность доказана.
\edok


\begin{theor}[координатное описание ядра]\label{t8_2_122}
Пусть $\varphi \in L(V, \widetilde{V})$,  $\bazis{e}$, $\bazis{f}$ --- базисы в $V$ и $\widetilde{V}$.
Пусть $\varphi \rsootv{\bazis{e}, \bazis{f}} A$.
Для вектора $\vek{a} \in V$, $\vek{a}=\bazis{e}X$ выполнено:
$\vek{a} \in \Ker \varphi$ $\Leftrightarrow$ $AX =O$. 
\end{theor}
\dok  Достаточно сопоставить определение ядра с  формулой $Y=AX$ (см. теорему \ref{t8_3_1}).
\edok

\otstup

Иначе говоря, теорема утверждает, что в терминах координатных столбцов $\Ker \varphi$ задается как 
общее решение  $\Sol (AX=O)$ однородной СЛУ с матрицей коэффициентов $A$.

\begin{sled}
В условиях теоремы $\boxed{\dim \Ker \varphi = n - \rg A}$.
\end{sled}




\subsection{Связь между размерностями ядра и образа}

Установим формулу, связывающую размерности  ядра и образа.


\begin{theor}\label{t8_2_2}
Пусть $\varphi \in L(V, \widetilde{V})$ и $\dim V =n < \infty $.
Тогда 
\begin{equation}\label{Ker_Im}
\boxed{\dim \Ker \varphi + \dim \Im \varphi = n}.
\end{equation}
\end{theor}
\dok Можно считать, что $\dim \widetilde{V}<\infty $ (иначе заменим $\widetilde{V}$ на любое конечномерное подпространство, содержащее $\Im \varphi$, 
замена  не влияет на $\Im$ и $\Ker $). 
Рассмотрим матрицу $A$ отображения $\varphi$ в некоторых базисах.
По следствию из теоремы \ref{t8_2_111} имеем $\dim \Im \varphi = \rg A$,
а по следствию из теоремы \ref{t8_2_122} имеем $\dim \Ker \varphi = n - \rg A$.
Отсюда вытекет нужная формула размерностей.
\edok

\begin{sled}\label{sootv_mezhdu_podpr}
Пусть $\varphi \in L(V, \widetilde{V})$, $\dim V<\infty $ и  $U\leq V$. Тогда
\begin{equation}\label{dimUphiU}
\dim U \leq \dim \varphi (U) +  \dim \Ker \varphi,
\end{equation}
причем равенство  достигается $\Leftrightarrow$
$U\geq \Ker \varphi$.
\end{sled}
\dok Ограничение $\varphi$ на $U$ (т.е. $\varphi  |_{U}  \in L(U, \widetilde{V})$),
обозначим $\psi$. Применив (\ref{Ker_Im}) к $\psi$,
имеем $ \dim \varphi (U) +  \dim \Ker \psi = \dim U $, при этом 
$\Ker \psi = (\Ker \varphi )\cap U \leq \Ker \varphi$,
откуда следуют нужные утверждения.
\edok

\otstup 

Отметим, что (\ref{Ker_Im}) находится в согласии с эквивалентностью условий 1), 2), 3) 
в теореме \ref{t_isom1}.

\otstup 

{\footnotesize
Мы доказали теорему \ref{t8_2_2}, перейдя c <<языка отображений>> на <<матричный язык>>. Более <<концептуально>> ---
получить формулу размерностей (\ref{Ker_Im}) как следствие 
{\it теоремы о гомоморфизме}  $V / \Ker \varphi \cong \Im \varphi$ (это стандартная теорема 
курса алгебры для многих алгебраических систем), которая вытекает из естественной биекции
$\vek{a}+ \Ker \varphi \mapsto \varphi (\vek{a})$.
Как следствие теоремы о гомоморфизме можно получить биекцию между подпространствами в $V$, содержашими $\Ker \varphi$, и подпространствами в $\Im \varphi$ (что соответсвует случаю равенства в (\ref{dimUphiU})).
}

\otstup 

В следующем упражнении намечен еще один путь доказательства формулы из теоремы \ref{t8_2_2} без привлечения матрицы линейного отображения.

\otstup

{\bf Упражнение.} Пусть $\varphi \in L(V, \widetilde{V})$ и $\dim V =n < \infty $.
Пусть $\vek{e}_{r+1}, \ldots, \vek{e}_{n}$ --- базис в $\Ker \varphi$. Дополним его до базиса 
$\vek{e}_{1}, \vek{e}_{2}, \ldots, \vek{e}_{n}$ пространства $V$. 
Докажите, что тогда $\Im \varphi = \lin{\varphi (\vek{e}_{1}), \ldots, \varphi (\vek{e}_{r})}$, причем 
$\varphi (\vek{e}_{1}), \ldots, \varphi (\vek{e}_{r})$ --- линейно независимая система.

\otstup

Предлагаем еще одно упражнение о простейшем виде матрицы линенйного отображения.

\otstup

{\bf Упражнение.} 
Пусть $\varphi \in L(V, \widetilde{V})$. Докажите, что в пространствах $V$ и $\widetilde{V}$ можно выбрать базисы $\bazis{e}$ и $\bazis{f}$
так, что %матрица отображения $\varphi$ будет иметь {\it простейший} вид
$$\varphi \rsootv{\bazis{e}, \bazis{f}} \begin{pmatrix}
E_r & O \\
O & O
\end{pmatrix}.$$
(Утверждение последнего упражнения не следует путать со случаем выбора одного и того же базиса $\bazis{e} = \bazis{f}$
для матрицы линейного преобразования, как обычно будет происходить в главе \ref{structure}.)




\subsection{Примеры}

Пусть $V = U_1 \bigoplus U_2$. Пусть  $\varphi: V\to V$ --- проектирование на $U_1$ вдоль $U_2$. 
Тогда $\Im \varphi = U_1$, $\Ker \varphi = U_2$.

%Отображение $\psi: V\to V$ такое, что $\varphi(\vek{a}) = \vek{a}_1 - \vek{a}_2$,
%называется {\it отражением}  (или {\it симметрией}) относительно $U_1$ вдоль $U_2$.

\example{III.1.
Пусть $V= \mathbf{C}^{\infty}(\mathbb{R})$. 
Общее решение $U$ линейного дифференциального уравнения
$x^{(n)}+a_{n-1}(t)x^{(n-1)}+\ldots + a_{1}(t)x' + a_0(t)x =0 $ (где $a_i(t)$ --- непрерывные функции $\mathbb{R}\to \mathbb{R}$) 
является ядром линейного дифференциального оператора $d^n+a_{n-1}d^{n-1}+\ldots +a_{1}d + a_0d^0 $.\\
Теорема существования и единственности из курса дифференциальных уравнений устанавливает изоморфизм
между решением $x\in U$ и столбцом {\it начальных условий } $\stolbec{x(t_0)\\ x'(t_0) \\ \ldots \\ x^{(n-1)}(t_0)}$,
тем самым, $\dim U = n$.
}

\example{III.2.
Пусть  $V=\mathbf{F}(\mathbb{N})=\{(a_1, a_2, \ldots) \, | \, a_i\in \mathbb{F}\}$ --- пространство числовых последовательностей.\\
Фибоначчиевы последовательности могут быть заданы как ядро линейного оператора $\varphi ^2- \varphi - \varphi ^0$, где 
$\varphi : V\to V$ --- оператор сдвига.
}



\subsection{Ядро и образ произведения отображений}

Начнем с двух почти очевидных предложений.



\begin{predl}
Пусть $\varphi \in L(V, \widetilde{V})$, $\psi \in L(\widetilde{V}, \widehat{V})$.
Тогда $$\Im (\psi \varphi ) \leq \Im \psi .$$
\end{predl}
\dok Это частный случай включения $\psi (U) \leq \psi(V)$ для $U= \varphi (V)$.
\edok


\begin{predl}\label{Ker_psi_phi}
Пусть $\varphi \in L(V, \widetilde{V})$, $\psi \in L(\widetilde{V}, \widehat{V})$.
Тогда $$\Ker (\psi \varphi ) \geq \Ker  \varphi .$$
\end{predl}
\dok Если $\vek{a}\in \Ker  \varphi$, то $\varphi (\vek{a}) = \vek{0}$
$\Leftrightarrow$  $(\psi \varphi)  (\vek{a}) = \psi (\varphi  (\vek{a})) = \vek{0}$
$\Leftrightarrow$ $\vek{a}\in \Ker (\psi \varphi )$.
\edok

\otstup

{\bf Упражнение.}
Переформулируйте и докажите теоремы об оценке ранга произведения матриц
$\rg (AB) \leq \rg A$, $\rg (AB) \leq \rg B$
в терминах линейных отображений.
%\\
%Получите доказательство теоремы о ранге произведения матриц, используя 
%линейные отображения.

\otstup
 


Зафиксируем еще несколько фактов, которые будем  использовать далее.

\begin{predl}\label{Ker_psi_phi_leq}
Пусть $\varphi \in L(V, \widetilde{V})$, $\psi \in L(\widetilde{V}, \widehat{V})$, $\dim V<\infty $.
Тогда 
\begin{equation}\label{Kerpsivarphileq}
\dim \Ker (\psi \varphi ) \leq \dim \Ker  \varphi + \dim \Ker  \psi.
\end{equation}
\end{predl}
\dok 
Пусть $U=\Ker (\psi \varphi ) $. Тогда $\varphi (U) \leq \Ker \psi$.
Согласно следствию из теоремы \ref{t8_2_2},  имеем 
$\dim U = \dim \varphi (U) +  \dim \Ker \varphi \leq  \dim \Ker  \psi + \dim \Ker  \varphi  $, 
что и требовалось.
\edok 


\otstup 
Отметим, что равенство в 
 (\ref{Kerpsivarphileq})
достигается $\Leftrightarrow$ $\varphi (U) = \Ker \psi $, или, поскольку $U$ является 
полным прообразом  $\Ker \psi $ относительно $\varphi$, эквивалентно, $ \Ker \psi \leq \Im \varphi$.

\otstup


{\bf Упражнение.}
а) Для матриц $A\in \mathbf{M}_{m\times n}$, 
$B\in \mathbf{M}_{n\times k}$ 
докажите {\it неравенство Сильвестра} $$\rg (AB) \geq \rg A +  \rg B - n.$$

б) Каков максимальный ранг матрицы $A\in \mathbf{M}_{n\times n}$, если $A^2=O$?

в)$^*$ Для матриц $A\in \mathbf{M}_{m\times n}$, 
$B\in \mathbf{M}_{n\times k}$,  $C\in \mathbf{M}_{k\times l}$ 
докажите {\it неравенство Фробениуса} (обобщающее неравенство Сильвестра) 
$$\rg (ABC) + \rg B \geq \rg (AB) +  \rg (BC).$$

\otstup


В качестве следствия  предложения \ref{Ker_psi_phi} имеем

\begin{predl}\label{sled_Ker_psi_phi}
Пусть $\varphi \in L(V, V)$, $\psi \in L(V, V)$,
причем $\varphi \psi = \psi \varphi $.
Тогда 
\begin{equation}\label{Kerpsiphigeq}
\Ker (\psi \varphi ) \geq \Ker  \varphi + \Ker  \psi .
\end{equation}
\end{predl}
\dok Так как $\Ker (\psi \varphi ) \geq \Ker  \varphi  $ и 
$\Ker (\psi \varphi )  = \Ker (\varphi \psi ) \geq \Ker  \psi  $, то \\
$\Ker (\psi \varphi ) \geq \lin{\Ker  \varphi, \Ker  \psi}  = \Ker  \varphi + \Ker  \psi$.
\edok

\begin{sled}\label{sledsled_Ker_psi_phi}
Пусть  $\varphi \in L(V, V)$, $\psi \in L(V, V)$, $\dim V <\infty$, 
причем $\varphi \psi = \psi \varphi $.
Если $\Ker  \varphi \cap  \Ker  \psi = O$, то 
$\Ker (\psi \varphi ) =\Ker  \varphi \oplus \Ker  \psi .$
\end{sled}
\dok Из (\ref{Kerpsiphigeq}), при условии $\Ker  \varphi \cap  \Ker  \psi = O$,
следует, что $\Ker (\psi \varphi ) \geq \Ker  \varphi \oplus \Ker  \psi $ и
$\dim \Ker (\psi \varphi ) \geq \dim \Ker  \varphi + \dim \Ker  \psi$.
Тогда из (\ref{Kerpsivarphileq}) вытекет требуемое.
\edok

\begin{predl}\label{Ker_polynom}
Пусть $\varphi \in L(V, V)$, а $f_1, \ldots, f_k\in \mathbb{F}[X]$ --- попарно взаимно простые многочлены.
Тогда $\sum\limits_{i=1}^k \Ker (f_i(\varphi) )$ --- прямая сумма. \\
Если, кроме того,   $\dim V<\infty$, то $\Ker (f_1(\varphi) ) \oplus \ldots \oplus 
\Ker (f_k(\varphi) ) = \Ker (f_1(\varphi)\ldots f_k(\varphi) )$.
\end{predl}
\dok Докажем в случае  $k=2$. Этого будет достаточно, так как случай произвольного $k$ 
можно разобрать,  применив утверждение для $k=2$ к многочленам
$f_1$ и $f_2\ldots f_k$, далее к $f_2$ и $f_3\ldots f_k$ и т.д.\\
Так как $f_1$ и $f_2$ взаимно просты, существуют многочлены 
$g_1, g_2\in \mathbb{F}[X]$  такие, что $g_1f_1+g_2f_2=1$.
Тогда из этого равенсва многочленов  имеем 
$g_1 (\varphi)f_1(\varphi)+g_2(\varphi)f_2(\varphi)=I_V$.\\
Применяя это операторное равенство для $\vek{a}\in \Ker (f_1(\varphi) ) \cap \Ker (f_2(\varphi) ) $, 
получаем $\vek{0}=\vek{a}$, значит 
$\Ker (f_1(\varphi) ) \cap \Ker (f_2(\varphi) ) =O$, тем самым утверждение про прямую сумму доказано.
\\
Второе утверждение сразу получаем в силу следствия из предложения \ref{sled_Ker_psi_phi}.
\edok



%Расширить  Ker произведения... Re
%не-ва для рангов.
% Ker p(f) + Ker q(f)


\section{Аффинные преобразования (на плоскости).}\label{aff}

\subsection{Определение и основные свойства}


%\newpage

$S$ --- аффинное пространство.

далее рассматриваем двумерный случай: $S=\mathcal{P}$ (плоскость) 

$V$ --- двумерное векторное пространство (радиус-векторы на плоскости).


%Пусть зафиксировано $\varphi \in L(V, V)$.



%\section{Линейные и аффинные преобразования плоскости.}
\defin{
Преобразование плоскости $f: \mathcal{P}\to \mathcal{P}$ называется 
линейным (линейно-аффинным) преобразованием, если \\$\exists$ $\varphi\in L(V,V)$ такое, что 
$\forall$ $ M, N \in \mathcal{P}$ выполнено
$$\overrightarrow{f(M)f(N)} = \varphi (\overrightarrow{MN}).$$
} 



Соответствующее линейное отображение $\varphi\in L(V,V)$ обозначим
$\widetilde{f}$; иногда $\widetilde{f}$ называют {\it дифференциалом} отображения $f$.


\defin{
Линейное преобразование $f: \mathcal{P}\to \mathcal{P}$ называется {\it аффинным}, если
оно биективно.
}


\begin{predl}\label{p4_2_3}.\\
Пусть $f: \mathcal{P}\to \mathcal{P}$ --- линейное преобразование.
Тогда \\ $f$ аффинно $\Leftrightarrow$ $\widetilde{f}$ биективно.
\end{predl}


\begin{predl}\label{p4_2_4}.\\
Если $f: \mathcal{P} \to \mathcal{P}$ --- аффинное преобразование, то
$f^{-1}$  также аффинное преобразование, причем $\widetilde{f^{-1}} = \widetilde{f}^{-1}$.
\end{predl}
%\dok
%\edok


\begin{predl}\label{p4_2_5}.\\
1) Если $f: \mathcal{P} \to \mathcal{P}$ и $g: \mathcal{P} \to \mathcal{P}$
--- линейные преобразования, то $gf$ также линейное, причем
$\widetilde{gf} = \widetilde{g}\widetilde{f}$.\\
2) Если $f: \mathcal{P} \to \mathcal{P}$ и $g: \mathcal{P} \to \mathcal{P}$
--- аффинные преобразования, то $gf$ также аффинное.
\end{predl}

\otstup

Как видим,  аффинные  преобразования образуют группу относительно композиции.




\subsection{Координатная запись линейного преобразования.}

Зафиксируем ДСК $(O, \bazis{e})$.

Пусть в определении линейного преобразования $f : \mathcal{P} \to \mathcal{P}$
дифференциал $\widetilde{f} : V\to V$ имеет матрицу
$A$, a $f(O)$ имеет координатный столбец
$C$. 

%Расписывая в
%координатах и в матричном виде, имеем следующее соотношение между координатными столбцами


\begin{predl}\label{p4_2_555}
Пусть $X$ и $Y$ --- координатные столбцы точки $M$  ее образа $f(M)$. Тогда
\begin{equation}
\boxed{Y=AX+C}.
\end{equation}
\end{predl}

Видим, что $C$ --- это координатный столбец для $f(O)$.
Координатные записи для $f$ ($Y=AX+C$) и $\widetilde{f}$ 
($Y=AX$) отличаются <<сдвигом на $C$>>.
Иначе, $f$ может быть представлено в виде композиции 
преобразования с неподвижной точкой $O$ и сдвига (параллельного переноса).


%Геом. смысл столбцов матрицы $A$ и столбца $C$ (?)

%
%Комментарий:
%Обозначение в  задачнике: $X^*$ вместо $Y$.

%Как видим, любое линейное преобразование может быть разложено в произведение $gh$,
%где $h$ --- преобразование

%$$
%\begin{cases}
%x^{*}=a_1 x+b_1y\cr
%y^{*}=a_2 x+b_2y,
%\end{cases}
%$$
%для которого $O$ --- неподвижная точка, а $g$ --- параллельный перенос на вектор
%$\vek{c}$.


\begin{predl}\label{p4_2_6}(критерий аффинности)
Пусть 
дифференциал $\widetilde{f}$ линейного преобразования $f: \mathcal{P} \to \mathcal{P}$ 
имеет в базисе $\bazis{e}$ матрицу $A$. 
Тогда \\ $f$ аффинно (то есть биективно) $\Leftrightarrow$ $|A|\neq 0$.
\end{predl}
%\dok Следует из предложения \ref{p4_2_3}.
%\edok
%\footnote{
%Можно показать, что если $\varphi$ не биективно, то образ $\Im \varphi$ --- либо прямая (множество
%векторов, коллинеарных данному), либо точка (нулевой вектор).
%}


%\otstup

%Последняя теорема находится в согласии с предыдущим предложением: если рассмотреть
%линейное преобразование, для которого $|A| = 0$, то параллелограмм переходит в
%отрезок или точку, то есть в "вырожденный параллелограмм" площади 0.
%Итак, мы увидели, что модуль $|A|$ является коэффициентом изменения площадей, а знак
%$|A|$ говорит об изменении или сохранении ориентации (любого) базиса.
%%Из доказанной теоремы ясно, что величина
%$\delta = \begin{vmatrix}
%a_1 & b_1\\ a_2 & b_2
%\end{vmatrix}$
%не зависит от выбора декартовой системы координат, в которой $f$ задается формулами (*).




\begin{theor}\label{t4_2_2}
%Существование и единственность линейного преобразования, заданного
%образами трех точек, не лежащих на одной прямой.
%(Преобразование аффинно $\Leftrightarrow$ три образа не лежат на одной прямой.)
Пусть даны точки $A, B, C, K, L, M \in \mathcal{P}$, причем точки $A$, $B$ и $C$ не лежат на одной
прямой.\\
Тогда существует единственное линейное преобразование $f:\mathcal{P} \to \mathcal{P}$ такое, что
$f(A)=K$, $f(B)=L$, $f(C)=M$; при этом $f$ аффинно $\Leftrightarrow$ $K$, $L$, $M$ не лежат на
одной прямой.
\end{theor}
%\dok
%\edok

%Даже нужнее для дальнейшего:
%существование и единственность линейного преобразования, заданного
%образом точки и двух неколлинеарных векторов.

%\zad Чем может являться $\Im f$ для линейного преобразования $f:\mathcal{P} \to \mathcal{P}$?

\begin{theor}[связь аффинного преобразования с заменой координат]
Пусть $f:\mathcal{P} \to \mathcal{P}$ --- аффинное преобразование.
Тогда если $M$ имеет координатный столбец
$X$ в ДСК
$(O, \vek{e_1}, \vek{e_2})$, то
$f(M)$ имеет тот же координатный столбец
$X$
в ДСК
$(f(O), \widetilde{f}(\vek{e_1}), \widetilde{f}(\vek{e_2}))$.
\end{theor}
%\footnote{Заметим, что здесь $(f(O), \widetilde{f}(\vek{e_1}), \widetilde{f}(\vek{e_2}))$ --- действительно
%ДСК, так как
%векторы $\widetilde{f}(\vek{e_1}) \nparallel$ не коллинеарны.
%}
%\dok
%\edok

%%%%%%%%%%%%%%%%%%
%%%%%%%%%%%%%%%%%%

\subsection{Примеры}


\example{
Пусть $l$ --- прямая. Выберем ПДСК ($O, \vek{e_1}, \vek{e_2}$)
так, что $O\in l$, $\vek{e_1}\parallel l$. 
%Тогда преобразование $f:\mathcal{P} \to \mathcal{P}$,
%заданное в рассматриваемой системе координат формулами
$$
\begin{cases}
y_1=x_1\cr
y_2=\lambda x_2.
\end{cases}
$$
%что это?
--- при $\lambda = 0$ (ортогональным) проектированием на прямую $l$;\\
%(это линейное, но не аффинное преобразование);\\
%при $\lambda \neq 0 $ --- аффинными преобразованиями:\\
--- при $\lambda = 1$ --- тождественное преобразование;\\
--- при $\lambda = -1$ --- {\it осевая симметрия} (или {\it отражение}) относительно прямой $l$;\\
--- при $\lambda> 0$ --- {\it сжатие} к прямой $l$ с коэффициентом
$\lambda$ (слово "сжатие" употребляется и в случае $\lambda >1$, а иногда оно заменяется на слово
"растяжение");\\
--- при $\lambda< 0$ --- композиция симметрии относительно $l$ и сжатия к $l$.\\
}


\example{
 Пусть $M$ --- некоторая точка. Выберем ДСК
($O, \vek{e_1}, \vek{e_2}$)
так, что $O=M$. 
%Тогда преобразование $f:\mathcal{P} \to \mathcal{P}$,
%заданное в рассматриваемой системе координат формулами
$$
\begin{cases}
y_1=\lambda x_1\cr
y_2=\lambda x_2.
\end{cases}
$$
--- при $\lambda\neq 0$ ---  гомотетия с коэффициентом $\lambda$.
}


\example{
Пусть $\vek{c}$ --- некоторый вектор, имеющий координаты
$\begin{pmatrix}
c_1 \\c_2
\end{pmatrix}$. в некоторой системе координат ($O, \vek{e_1}, \vek{e_2}$).
$$
\begin{cases}
y_1=x_1+c_1\cr
y_2= x_2+c_2
\end{cases}
$$
--- параллельный перенос.
}

\example{
Рассмотрим преобразование $f:\mathcal{P} \to \mathcal{P}$ поворота на угол $\varphi$ (против часовой стрелки
вокруг точки $M$). Выберем прямоугольную систему координат ($O, \vek{e_1}, \vek{e_2}$)
так, что $O=M$. %Пусть точка
$$
\begin{cases}
y_1=\cos \varphi \, x_1 - \sin \varphi \, x_2\cr
y_2=\sin \varphi \, x_1 + \cos \varphi \, x_2.
\end{cases}
$$
%и поэтому является аффинным преобразованием.
%(вспомним матрицу поворота)
}
%\footnote{Композицией осевой симметрии, параллельного переноса и поворота можно получить любое движение
%(см. $\S...$), а композицией движения и гомотетии --- любое преобразование подобия.
%Таким образом, в группе аффинных преобразований имеются вложенные подгруппы: всех преобразований подобия
%и всех движений.}

Из указанных примеров композицией можно получать новые (на самом деле все) аффинные преобразования.
Известно, что всякое афинное преобразование может быть разложено 
в виде произведения движения (=композиция поворота или отражения и сдвига) и сжатий к перпендикулярным осям.


\subsection{Геометрические свойства}



\begin{predl}\label{p4_3_1}
%Следствия (геометрические) для аффинных преобразований:
Пусть $f: \mathcal{P}\to \mathcal{P}$ --- аффинное преобразование. Тогда\\
1) образ прямой --- прямая, образ отрезка --- отрезок;\\
2) параллельные прямые переходят в параллельные прямые;\\
3) отношение длин параллельных отрезков сохраняется;\\
4) образ центрально-симметричной фигуры --- центрально-симметричная фигура.
\end{predl}
%\dok
%\edok


%Следующее теорема в некотором смысле обобщает предыдущее предложение.

\begin{theor}\label{t4_2_1}(изменение площадей)
Пусть 
дифференциал $\widetilde{f}$ линейного преобразования $f: \mathcal{P} \to \mathcal{P}$ 
имеет в базисе $\bazis{e}$ матрицу $A$. 
Тогда \\ 
$$\boxed{S(f(\Pi )) = |\det A| \cdot S(\Pi ) }.$$
\end{theor}
%\footnote{Из доказанной теоремы можно вывести, что
%при аффинном преобразовании площадь (то есть мера Жордана) любой (измеримой по Жордану) фигуры
%изменяется в $|\det A|$ раз.
%Например, исходя из формулы площади круга можно получить формулу площади эллипса %%$\dfrac{x^2}{a^2}+\dfrac{y^2}{b^2}=1$:
%$S=\pi ab$, где $a$ и $b$ --- длины его полуосей (см. далее о связи аффинных преобразований и кривых
%второго порядка).
%}
%\dok
%Следует из теоремы \ref{t4_1_1}.
%\edok


Таким образом, модуль $\det A$ отвечает за изменение площадей, а 
знак $\det A$ --- за сохранение или изменение ориентации.
Это согласуется с предложением о критерии аффинности (случай $\det A = 0$ соответсвует 
вырождению образа парлеллограмма).

%\subsection{Аффинные преобразования и кривые второго порядка.}


\begin{theor}
Образом алгебраической кривой порядка $n$ при аффинном преобразовании является
алгебраическая кривая порядка $n$.
\end{theor}
%\dok
%\edok



%%%%%%%%%%%%%%%%%%%
%%%%%%%%%%%%%%%%%%%%
%%%%%%%%%%%%%%%%%% % аффинные преобразования плоскости (опять экскурс в аффинные пр-ва)


%ГЛАВА
%\ref{structure}

\chapter{Структура линейного преобразования.}\label{structure}

В этой главе $V$ --- векторное пространство над полем $\mathbb{F}$,
а $\varphi$ --- фиксированное линейное преобразование $V\to V$.


\section{Инвариантные подпространства}

%В этом параграфе $V$ --- векторное пространство над полем $\mathbb{F}$.
Важную информацию о линейном преобразовании $\varphi \in L(V,V)$ можно узнать, изучая инвариантные
подпространства (то есть инвариантные относительно $\varphi$
подмножества $V$, являющиеся подпространствами). %--- см. определение ??????).
Напомним, что $U \subset V$ называется инвариантным относительно $\varphi$, 
если $\varphi (U)\subset U$, т.е. если $\forall$ $\vek{a} \in U$
выполнено $\varphi(\vek{a}) \in U$.
Всякий раз, когда имеется инвариантное относительно $\varphi \in L(V,V)$ подпространство $U\leq V$,
можем говорить об ограничении (или  сужении) $\varphi | _{U} \in L(U, U)$.
{\footnotesize Кроме того, условие инвариантности $U$ относительно $\varphi$ дает возможность
корректно определить {\it фактор-оператор} $\breve{\varphi} \in L(V/U,V/U)$ по естественному 
правилу  $\breve{\varphi} (\vek{a}+U) = \varphi (\vek{a}) +U$.}


\begin{predl}\label{p8_5_0}
Пусть  $U\leq V$, $U=\lin{\vek{a}_1, \ldots, \vek{a}_k}$.
Тогда $U$ инвариантно относительно $\varphi \in L(V,V)$ $\Leftrightarrow$ $\varphi (\vek{a}_1), \ldots, \varphi (\vek{a}_k) \in U$. 
\end{predl}
\dok Следует из предложения \ref{p8_1_103} главы \ref{lin_otobr}.
\edok

\begin{predl}\label{p8_5_1}
Пусть подпространства $U_i\leq V$, $i=1, \ldots, k$, инвариантны относительно
$\varphi \in L(V,V)$. Тогда $\sum\limits_{i=1}^k U_i$ и
$\bigcap\limits_{i=1}^k U_i$ тоже инвариантны относительно
$\varphi$.
\end{predl}
\dok %TO BE PROVED
\edok

\otstup

Отметим, что предыдущее предложение верно и для произвольных подмножеств
$U_1, U_2, \ldots , U_k \subset V$.

%\begin{zamech}
%Предыдущее предложение верно и для произвольных подмножеств
%$U_1, U_2, \ldots , U_k \subset V$.
%\end{zamech}

\begin{predl}\label{p8_5_2}
Пусть $\varphi \in L(V,V)$,  $\psi \in L(V,V)$ таковы, что $\varphi  \psi =  \psi \varphi$.
Тогда $\Ker \psi$, $\Im \psi$ инвариантны относительно $\varphi$.
%В частности, собственное подпространство $V_{\lambda}$ для $\varphi$
%инвариантно относительно $\varphi$.
\end{predl}
\dok 1).  Пусть $\vek{a}\in \Ker \psi$, так что $\psi (\vek{a}) = \vek{0}$.
Тогда $\psi (\varphi (\vek{a})) = \varphi (\psi (\vek{a}))  = \varphi (\vek{0}) = \vek{0}$, 
откуда $\varphi ( \vek{a} )\in \Ker \psi$.\\
2). Пусть $\vek{b}\in \Im \psi$, так что $\vek{b} = \psi (\vek{a})$ для некоторого вектора $\vek{a}$.
Тогда $\varphi ( \vek{b} ) = \varphi ( \psi (\vek{a})) = \psi (\varphi  (\vek{a})) $, 
откуда $\varphi ( \vek{b} )\in \Im \psi$.
\edok

\begin{sled}\label{p8_5_222}
Пусть $\varphi \in L(V,V)$ и $f\in \mathbb{F}[X]$.
Тогда $\Ker f(\varphi)$, $\Im  f(\varphi)$ инвариантны относительно~$\varphi$.
\end{sled}

\begin{predl}\label{p8_5_3}
Пусть $\varphi \in L(V,V)$ и $\lambda_0 \in \mathbb{F}$.
Пусть подпространство
$U\leq  V$ таково, что $U \geq \Im (\varphi - \lambda _0)$.
Тогда $U$ инвариантно относительно $\varphi$.
\end{predl}
\dok Для любого вектора $\vek{a}$ 
выполнено  $(\varphi - \lambda _0)(\vek{a}) \in \Im (\varphi - \lambda _0)\leq U$.
Тогда если $\vek{a}\in U$, то $\varphi (\vek{a}) = (\varphi - \lambda _0)(\vek{a}) + \lambda _0 \vek{a} \in U$
(оба слагаемых принадлежат в $U$).
\edok

\otstup

Предыдущее предложение согласуется с соответствием между инвариантными 
подпространствами сопряженных операторов (см. главу 6).


\begin{predl}\label{p8_5_4}
Пусть $\varphi \in L(V,V)$ --- изоморфизм. 
Пусть подпространство $U\leq V$ инвариантно относительно $\varphi \in L(V,V)$ и $\dim U<\infty $.
Тогда $U$ инвариантно относительно $\varphi ^{-1}$.
\end{predl}
\dok Сужение $\varphi|_U: U\to U$ является инъекцией, поэтому (см. теорему \ref{t_isom1} главы
\ref{lin_otobr}) является изоморфизмом. Значит  $(\varphi|_U)^{-1}: U\to U$ --- сужение $\varphi ^{-1}$ на $U$.
\edok

\otstup

Иногда вид матрицы линейного преобразования говорит о наличии некоторых инвариантных подпространств.
%линейный оператор --- термин???

\begin{predl}\label{p8_5_5}
Пусть  $\dim V = n <\infty $ и
$\varphi \in L(V,V)$ имеет в базисе $\bazis{e} = (\vek{e}_1,  \vek{e}_2, \ldots , \vek{e}_n)$
матрицу $A=(a_{ij})$. Подпространство
$U_k = \lin{ \vek{e}_1,  \vek{e}_2, \ldots , \vek{e}_k}$ инвариантно относительно $\varphi$
$\Leftrightarrow$
$a_{ij}=0$ для всех $i=k+1, \ldots, n$, $j=1, 2, \ldots, k$
(то есть матрица $A$ имеет блочно-треугольный вид
$\begin{pmatrix} B & C \\ O & D \end{pmatrix}$, где $O\in \mathbf{M}_{(n-k)\times k}$
--- нулевая матрица).\\
Кроме того, в таком случае $B$ --- матрица сужения $\varphi |_{U_k}$ в базисе
$\vek{e}_1,  \vek{e}_2, \ldots , \vek{e}_k$ пространства $U_k$.
\end{predl}
\dok По определению матрицы линейного отображения, $a_{ij}=0$ для всех $i=k+1, \ldots, n$ $\Leftrightarrow$
$\varphi (\vek{e}_j) \in \lin{\vek{e}_1, \vek{e}_2, \ldots, \vek{e}_k}$. Остается воспользоваться предложением \ref{p8_5_0}.
\edok


\otstup
{\footnotesize В условиях предложения \ref{p8_5_5} матрица $D$ соответствует матрице фактор-оператора
(в базисе $\vek{e}_i+U$, $i=k+1, \ldots, n$).
}


\begin{sled}
Пусть $\varphi \in L(V,V)$, $\varphi \rsootv{\bazis{e}, \bazis{e}} A$.
Тогда $A$ верхнетреугольная $\Leftrightarrow$ все подпространства  $\lin{\vek{e}_1, \ldots , \vek{e}_k}$, $k=1, \ldots, n$, инвариантны относительно $\varphi$.
\end{sled}
%\dok 
%\edok
%<<флаг>>

Аналогично предложению \ref{p8_5_5}, матрица $A$ (где $\varphi \rsootv{\bazis{e}, \bazis{e}} A$) имеет блочно-треугольный вид
$\begin{pmatrix} B & O \\ C & D \end{pmatrix}$, где $O\in \mathbf{M}_{k\times (n-k)}$
%в том и только в том случае, когда 
$\Leftrightarrow$
%подпространство
$\lin{ \vek{e}_{k+1},  \vek{e}_{k+2}, \ldots , \vek{e}_n}$ инвариантно относительно~$\varphi$.
Соответственно,  блочно-диагональная структура $A$ (с квадратными блоками по диагонали) означает
разложение $V$  в прямую сумму инвариантных подпространств, каждое из которых --- линейная оболочка нескольких 
подряд идущих базисных векторов. В частности, 
$A$ диагональна тогда  и только тогда, когда все одномерные подпространства $\lin{{e}_i}$, $i=1, \ldots, n$, 
инвариантны относительно $\varphi$. 

В последнем случае структура оператора $\varphi$ вполне понятна: рассматриваемый  базис таков, что
на каждом из них $\varphi$ действует умножением на константу. 
%$n$ базисных векторовДальнейшее во многом будет 
%Дальнейшая теория строится во многом исходя из желания по возможности найти для оператора 
%такую структуру или понять....


\begin{predl}\label{p8_5_66}
Пусть  $\dim V = n <\infty $, $\varphi \in L(V,V)$ и $f\in \mathbb{F}[X]$  
--- многочлен степени $k$ такой, что оператор $f(\varphi)$ вырожден.
Тогда существует ненулевое подпростанство размерности не больше $k$, инвариантное относительно $\varphi$.
\end{predl}
\dok Пусть $\vek{a}$ --- ненулевой вектор из $\Ker f(\varphi)$.
Покажем, что подпространство $U = \lin{\vek{a}, \varphi(\vek{a}), \varphi ^2(\vek{a}), \ldots , 
\varphi ^{k-1}(\vek{a})}$ инвариантно относительно $\varphi$.
Условия из  предложения \ref{p8_5_0}, очевидно выполнены для всех порождающих $U$ 
векторов, кроме возможно, $\varphi ^{k-1}(\vek{a})$. Итак, 
нужно проверить, что $\varphi ^{k}(\vek{a}) \in U$. \\
Пусть $f(x) = \alpha _k x^k + \alpha _{k-1} x^{k-1} + \ldots  + \alpha _{1} x + \alpha _0$, $\alpha_k\neq 0$.
Тогда \\ $(f(\varphi))(\vek{a}) = 
\alpha _k \varphi ^k (\vek{a}) + \alpha _{k-1} \varphi ^{k-1} (\vek{a})
+ \ldots  + \alpha _{1} \varphi (\vek{a}) + \alpha _0 \vek{a} = \vek{0}$, 
откуда видим, что $\varphi ^k (\vek{a})$
линейно выражается через $\vek{a}, \varphi(\vek{a}), \varphi ^2(\vek{a}), \ldots , 
\varphi ^{k-1}(\vek{a})$, т.е.  $\varphi ^k (\vek{a})\in U$.
% = -\frac{\alpha _{k-1}}{\alpha _{k}} \varphi ^{k-1} (\vek{a}) - \ldots 
%-\frac{\alpha _{0}}{\alpha _{k}} \vek{a} \in \lin{\vek{a}, \varphi(\vek{a}), \varphi ^2(\vek{a}), \ldots , 
%\varphi ^{k-1}(\vek{a})} = U.$
\edok


\otstup

Ниже как следствие предложения \ref{p8_5_66} мы установим существование инвариантных подпространств
малой размерности в случаях $\mathbb{F}=\mathbb{C}$ ($\dim =1$) и $\mathbb{F}=\mathbb{R}$ ($\dim \leq 2$).



\section{Собственные векторы. Диагонализируемость.}

В этом параграфе полагаем, что поле $\mathbb{F}$ --- либо $\mathbb{R}$, либо $\mathbb{C}$.
{\footnotesize Хотя теорию несложно обобщить на случай произвольного поля $\mathbb{F}$, используя
наряду с $\mathbb{F}$ {\it алгебраическое замыкание}  --- расширение
$\mathbb{K}$ поля $\mathbb{F}$, для которого каждый многочлен из $\mathbb{K}[X]$ имеет корень.}


\subsection{Собственные значения. Собственные векторы и собственные подпространства.}


%Следующие определения связаны с изучением одномерных инвариантных подпространств.

\defin{
Константа $\lambda_0 \in \mathbb{F}$ называется {\it собственным значением}
преобразования~$\varphi$, если $\exists $ $\vek{a} \in V$ такой, что $ \vek{a} \neq \vek{0}$ и
\begin{equation}\label{eigen}
\varphi ( \vek{a}) = \lambda _0 \vek{a}. 
\end{equation}
}

\defin{
Если для $ \vek{a} \neq \vek{0}$ выполнено (\ref{eigen}), 
то вектор $\vek{a}$ называется {\it собственным вектором}
преобразования $\varphi$, отвечающим собственному значению $\lambda _0$.
}

\begin{predl}\label{p8_5_6}
Ненулевой вектор $\vek{a}$ является собственным вектором преобразования $\varphi$
$\Leftrightarrow$
одномерное подпространство
$\langle \vek{a} \rangle$ инвариантно относительно $\varphi$.
\end{predl}
\dok  В силу предложения \ref{p8_5_0}, условие 
инвариантности $\langle \vek{a} \rangle$ эквивалентно выполнению (\ref{eigen})
для некоторого $\lambda_0 \in \mathbb{F}$.
\edok


\begin{predl}\label{p8_5_7}
Ненулевой вектор $\vek{a}$ является собственным вектором преобразования $\varphi$,
отвечающим собственному значению $\lambda_0$
$\Leftrightarrow$
$\vek{a} \in \Ker (\varphi -\lambda_0)$.
\end{predl}
\dok  Достаточно заметить, что (\ref{eigen}) эквивалентно условию $(\varphi - \lambda _0)(\vek{a})=\vek{0}$.
\edok

\otstup
Отметим, что пара предложений выше согласуется с предложением \ref{p8_5_66}: наличие одномерных 
инвариантных подпространств для $\varphi$ связано с вырожденностью оператора 
$\varphi - \lambda _0$ (многочлена от $\varphi$ первой степени).

\defin{
Пусть $\lambda_0$ --- собственное значение преобразования $\varphi \in L(V, V)$.
Подпространство $\Ker (\varphi -\lambda_0)$ называется {\it собственным подпространством}
для преобразования $\varphi$ (отвечающим собственному значению $\lambda_0$).
}

Собственное подпространство условимся обозначать $V_{\lambda_0}$.
Таким образом, по определению $V_{\lambda_0} = \Ker (\varphi -\lambda_0)$ ---
содержит все собственные векторы,  отвечающие собственному значению
$\lambda_0$, и $\vek{0}$.

{\footnotesize Иногда, в других курсах и книгах,  собственными подпространствами 
называют также и подпространства $\Ker (\varphi -\lambda_0)$.
Отметим, что введенные формально новые понятия, такие как собственное подпространство,
по сути лишь новая терминология для важных частных случаев определенных ранее понятий.}


\begin{predl}\label{p8_5_8}
Пусть $\lambda _1, \lambda _2, \ldots , \lambda _k$ --- различные собственные значения для
$\varphi \in L(V, V)$. Тогда $\sum\limits_{i=1}^{k} V_{\lambda _i} $ --- прямая сумма.
\end{predl}
\dok Это предложение сразу следует из предложения \ref{Ker_polynom} главы
\ref{lin_otobr} --- достаточно рассмотреть линейные многочлены $f_i(x) = x- \lambda _i$.

 Приведем еще одно, независимое, доказательство. 
Предположим, противное, и пусть, скажем, $\vek{a}_1\in V_{\lambda _1}\cap \sum\limits_{i=2}^{k} V_{\lambda _i}$, 
$\vek{a}_1\neq \vek{0}$ (см. теорему \ref{t7_4_1}, глава \ref{lin_prostr}).
Тогда $\vek{a}_1 = \sum\limits_{i=2}^{k} \vek{a}_i$ для некоторых $ \vek{a}_i\in V_{\lambda _i}$.
Применим к этому равенству преобразование $\psi = \prod\limits_{i=2}^{k}(\varphi-\lambda _i)$.
Так как $(\varphi-\lambda _i)(\vek{a}_i)=\vek{0}$ и в произведении $\prod\limits_{i=2}^{k}(\varphi-\lambda _i)$ преобразования перестановочны, то
$\psi (\vek{a}_i)=\vek{0}$ для $i=2, \ldots, k$. С другой стороны, $\psi (\vek{a}_1 ) = \prod\limits_{i=2}^{k}(\lambda_1-\lambda _i)\vek{a}_1\neq \vek{0}$.
Противоречие.
\edok

\subsection{Характеристический многочлен}

Далее в этом параграфе полагаем $\dim V = n<\infty$.

\defin{
Пусть  $\varphi$ имеет в некотором базисе матрицу $A$.
Функция $|A-\lambda E|$ переменной $\lambda$ называется
{\it характеристическим многочленом} оператора $\varphi$.
}

Для характеристического многочлена примем обозначение $\chi_{\varphi}(\lambda)$.

\defin{Уравнение
$\chi_{\varphi}(\lambda) = 0$ называется
{\it характеристическим уравнением}, а его (комплексные) корни
--- {\it характеристическими числами}  преобразования $\varphi$.
}

Характеристическое уравнение 
дает условие вырожденности оператора $\varphi - \lambda$; это важное 
соображение еще будет зафиксировано ниже (см. теорему \ref{t8_5_1}).
А сперва сделаем некоторые наблюдения про $\chi_{\varphi}(\lambda)$.

\begin{predl}\label{p8_5_0000}
$\chi_{\varphi}(\lambda)$ является многочленом степени $n$:
\begin{equation}\label{chi1}
\chi_{\varphi}(\lambda) = (-1)^n\lambda ^n +a_1 \lambda ^{n-1} + a_2 \lambda ^{n-2} + \ldots +
a_{n-1} \lambda + a_n,
\end{equation}
где $a_i\in \mathbb{F}$.
При этом $$a_1 = (-1)^{n-1} \tr A, \,\,\,\,\,\,   a_n = |A|.$$
\end{predl}
\dok 
Формулу вида (\ref{chi1}) можно получить, пользуясь явным разложением определителя $|A-\lambda E|$.
Отсюда сразу можно получить  $a_n = \chi_{\varphi}(0) = |A|$.\\
Далее, диагональные элементы матрицы $|A-\lambda E|$ --- линейные функции от $\lambda $
(равные $a_{ii}-\lambda$), а внедиагональные элементы --- константы, 
поэтому в явном разложении определителя все слагаемые, кроме произведения диагональных элементов 
будут содержать $\lambda $  в степени не превышающей $n-2$:\\ 
$$\chi_{\varphi}(\lambda) = \prod\limits_{i=1}^n (a_{ii}-\lambda) + g(\lambda), $$
где $\deg g\leq n-2$. Отсюда в $\chi_{\varphi}(\lambda)$ коэффициент  при $\lambda ^n$ 
равен $(-1)^n$, а коэффициент при $\lambda ^{n-1}$ равен $a_1 = (-1)^{n-1} \sum\limits_{i=1}^n a_{ii} =
(-1)^{n-1} \tr A$. 
\edok

\otstup

Теперь разложим (над полем комплексных чисел) многочлен $\chi_{\varphi}(\lambda)$
на линейные множители:
\begin{equation}\label{chi2}
\chi_{\varphi}(\lambda) = (-1)^n \prod\limits_{i=1}^n (\lambda - \mu _i),
\end{equation}
где $\mu_i\in \mathbb{C}$ --- характеристические числа.

Среди $\mu_i$ могут быть равные числа. Группируя равные характеристические числа, примем далее
такие обозначения: пусть $\lambda _1, \lambda _2, \ldots , \lambda _k $
--- попарно различные характеристические числа, а $s_1, s_2, \ldots , s_k$ ---
их (алгебраические) {\it кратности}, так что $\sum\limits_{i=1}^k s_i = n$. Тем самым, формула 
(\ref{chi2}) принимает вид
\begin{equation}\label{chi3}
\chi_{\varphi}(\lambda) = (-1)^n \prod\limits_{i=1}^k (\lambda - \lambda _i)^{s_i}.
\end{equation}

\begin{predl}\label{p8_5_00}
1). Сумма (с учетом кратности) всех характеристических чисел равна $\tr A$.\\
2). Произведение (с учетом кратности) всех характеристических чисел равно $|A|$.
\end{predl}
\dok Достаточно приравнять  
соответствующие коэффициенты в (\ref{chi1}) и (\ref{chi2}) (или же воспользоваться теоремой Виета).
\edok

\otstup

Отметим следующий важный частный случай.

\begin{predl}\label{p8_5_99}
Пусть  матрица $A$ оператора $\varphi$ (в некотором базисе) --- верхнетреугольная. Тогда
характеристические числа совпадают с числами, расположенными на диагонали матрицы~$A$. 
\end{predl}
\dok Это ясно, поскольку  определитель верхнетреугольной матрицы $A-\lambda E$
равен произведению диагональных элементов.
\edok


\begin{sled}%\label{p8_5_99}
Пусть в случае $\mathbb{F}=\mathbb{R}$ оператор $\varphi$ в некотором базисе 
имеет верхнетреугольную матрицу. Тогда все характеристические числа $\varphi$
вещественные. 
\end{sled}


\otstup

Следующее предложение обосновывает корректность обозначения
$\chi_{\varphi}(\lambda)$. % (а не $\chi_{A}(\lambda)$).

\begin{predl}\label{p8_5_9}
Характеристический многочлен 
преобразования $\varphi$ не зависит от выбора базиса в $V$.
\end{predl}
\dok Нужно доказать, что если $\varphi \rsootv{\bazis{e}, \bazis{e}} A$ 
и $\varphi \rsootv{\bazis{e'}, \bazis{e'}} A'$, 
то $|A-\lambda E|= |A'-\lambda E|$. 
По теореме \ref{t8_3_2} 
$A'=S^{-1}AS$, поэтому \\ $|A'-\lambda E| = |S^{-1}AS-\lambda E| = |S^{-1}AS-\lambda S^{-1}ES| =
|S^{-1}(A-\lambda E) S| = 
|S^{-1}|\cdot |A-\lambda E|\cdot | S| = $ \\
$=|S^{-1}|\cdot | S| \cdot |A-\lambda E| = |S^{-1}  S| \cdot |A-\lambda E| = |E| \cdot |A-\lambda E|=|A-\lambda E|.$
\edok

\begin{sled}
Определитель, след, набор характеристических чисел (с учетом кратностей) 
матрицы оператора  $\varphi$ не зависят от выбора базиса.
\end{sled}

%Следующие теоремы проясняют связь между характеристическим многочленом и предыдущим разделом.

\otstup

\begin{theor}\label{t8_5_1}
%$V$ --- векторное пространство над $\mathbb{R} (\mathbb{C})$, %$\dim V = n$,
Для константы $\lambda_0 \in \mathbb{F}$ следующие условия эквивалентны:
$\lambda_0$ --- собственное значение для $\varphi$
$\Leftrightarrow$ $\lambda_0$ --- характеристическое число (т.е. корень $\chi_{\varphi}$).
%корень характеристического многочлена $\chi_{\varphi}(\lambda)$.
\end{theor}
\dok $\lambda_0$ --- собственное значение $\Leftrightarrow$
$\Ker (\varphi - \lambda_0) \neq O$ $\Leftrightarrow$ (переходя к координатам)
СЛУ $(A - \lambda_0 E)X=O$ имеем ненулевое решение $\Leftrightarrow$
квадратная матрица $(A - \lambda_0 E)$ вырожденная 
$\Leftrightarrow$ $|A - \lambda_0 E|=0$.
\edok

\otstup

Таким образом, в случае $\mathbb{F} = \mathbb{C}$ <<собственное значение>>
и <<характеристическое число>> --- это одно и то же. 
В случае $\mathbb{F} = \mathbb{R}$ разница только в том, что характеристическое число может не принадлежать
$\mathbb{R}$; в этом случае собственные значения --- это в точности
вещестенные характеристические числа.

\otstup

Теперь мы готовы зафиксировать результат об инвариантных подпространствах малой размерности.

\begin{theor}\label{t8_5_222}
1). Пусть $\mathbb{F} = \mathbb{C}$. Тогда для $\varphi$ существует одномерное инвариантное подпространство
(или эквивалентно, существует собственный вектор).\\
2). Пусть $\mathbb{F} = \mathbb{R}$ и $n$ нечетно. Тогда для $\varphi$ существует одномерное инвариантное подпространство.\\
3). Пусть $\mathbb{F} = \mathbb{R}$. Тогда для $\varphi$ существует 
ненулевое инвариантное подпространство размерности не выше 2.
\end{theor}
\dok 
1) Следует из теоремы \ref{t8_5_1},  поскольку $\chi_{\varphi}$ имеет (комплексный) корень.\\
2) Следует из теоремы \ref{t8_5_1},  поскольку при нечетном $n$ 
многочлен $\chi_{\varphi}\in \mathbb{R}[X]$ имеет вещественный корень.\\
3) Если $\chi_{\varphi}\in \mathbb{R}[X]$ имеет вещественный корень, найдем, как и в 
пункте 2), одномерное инвариантное подпространство.\\ 
 Иначе, пусть $\lambda_0\in \mathbb{C} \setminus \mathbb{R}$ --- корень $\chi_{\varphi}$.
Многочлен $f (x) = x^2 - (\lambda_0 + \overline{\lambda_0})x + \lambda_0 \overline{\lambda_0}$ 
принадлежит $\mathbb{R}[X]$ (отметим, что  $\overline{\lambda_0}$ --- тоже корень $\chi_{\varphi}$). При этом оператор $f(\varphi)$ вырожден, так как
его матрица $f(A) = (A-\lambda _0 E) (A-\overline{\lambda_0} E)$ вырождена (здесь, как обычно, 
$A$ --- матрица оператора $\varphi$). Действительно, $\det f(A) = 
|A-\lambda _0 E|\cdot |A-\overline{\lambda_0} E| = 0$. Теперь нужное утверждение следует из предложения \ref{p8_5_66}.
\edok

%(В СОГЛАСИИ С предложением \ref{p8_5_66} --- ПОДПРОСТРАНСТВА МАЛЫХ РАЗМЕРНОСТеЙ


\otstup


\begin{theor}\label{t8_5_2}
%Пусть $V$ --- векторное пространство над $\mathbb{R} (\mathbb{C})$, %$\dim V = n$, и
Пусть $\lambda_i \in \mathbb{F}$.
 является
корнем кратности $s_i$ характеристического многочлена $\chi_{\varphi}(\lambda)$ преобразования
$\varphi$.
Тогда $$1\leq \dim V_{\lambda _i} \leq s_i.$$
\end{theor}
\dok Пусть $\dim V_{\lambda _i}=s$. Выберем в $\lambda _i$ базис $(\vek{e}_1, \ldots, \vek{e}_s)$ 
и дополним его до базиса $\bazis{e}= (\vek{e}_1, \ldots, \vek{e}_n)$ в пространстве $V$.
Тогда $\varphi \rsootv{\bazis{e}, \bazis{e}} A$, где $A$ имеет блочный вид $\begin{pmatrix}
\lambda _i E_s & C \\
O & D
\end{pmatrix}.$ Пользуясь формулой детерминанта с углом нулей, имеем\\ $\chi_{\varphi}(\lambda) = |A-\lambda E| = |\lambda _i E_s- \lambda E_s|\cdot 
|D- \lambda E_{n-s}| = 
(\lambda_i-\lambda)^s p(\lambda)$, где $p$ --- многочлен. \\ Тем самым, $s_i\geq s$.
\edok

\otstup

 Величину $\dim V_{\lambda _i}$ иногда называют {\it геометрической кратностью}
собственного значения $\lambda _i$. 
Таким образом, в теореме \ref{t8_5_2} утверждается, что
геометрическая кратность корня не превышает алгебраической кратности.
%дает 



\subsection{Диагонализируемость}

\defin{
Преобразование $\varphi$ называется {\it диагонализируемым}, если в  $V$ существует базис,
в котором матрица $\varphi$ имеет диагональный вид.
}

Ранее мы видели (см. следствие из предложения \ref{p8_5_99}), что 
в случае  наличие характеристических 
чисел, лежащих  вне поля $\mathbb{F}$, 
препятствует не только диагонализируемости, но даже треугольному виду матрицы преобразования.
Поэтому в критерии ниже полагаем, что все характеристические числа оператора $\varphi$
принадлежат $\mathbb{F}$ (в случае $\mathbb{F}=\mathbb{C}$ это условие не несет в себе никаких ограничений,
т.е. выполнено всегда).


\begin{theor}[критерий-1 диагонализируемости]\label{t8_5_3}
% $V$ --- векторное пространство над $\mathbb{R} (\mathbb{C})$. %, $\dim V = n$.
Пусть оператор $\varphi$
имеет попарно различные характеристические числа
$\lambda_1, \lambda_2, \ldots , \lambda_k\in \mathbb{F}$  кратностей
$s_1, s_2, \ldots , s_k$ соответственно. Тогда следующие условия эквивалентны:\\
1) $\varphi$ диагонализируем;\\
2) В $V$ существует базис из собственных векторов; \\ % $\varphi$;
3) $\dim V_{\lambda _i} = s_i$ для $i=1, 2, \ldots, k$;\\
%(в частности, в случае $V$ над $\mathbb{R}$ необходимо все $\lambda_1, \lambda_2, \ldots , \lambda_k$
%вещественные).\\
4) $V = \bigoplus \limits_{i=1}^k V_{\lambda _i}$. % является прямой суммой собственных подпространств.
\end{theor}
\dok 1) $\Leftrightarrow$ 2). Очевидно следует из определения матрицы линейного преобразования.\\
2) $\Rightarrow$ 3). %Докажем например, что $\dim V_{\lambda _1} = s_1$. 
В базисе из собственных векторов пусть $t_i$ векторов соответствуют $\lambda _i$ ($i=1, \ldots, k$),
так что $t_1+\ldots +t_k = n$.
%Обозначим для удобства $\vek{e}_1, \ldots,  \vek{e}_{t_1}$ собственные векторы из данного базиса, 
%которые отвечают $ 
Тогда $V_{\lambda _i}$ содержит линейно независимую систему из $t_i$ векторов,
откуда $\dim V_{\lambda _i} \geq  t_i$. 
С учетом теоремы \ref{t8_5_2}, имеем
$$t_i\leq \dim V_{\lambda _i} \leq  s_i, \,\,\,\,\,\,\,i=1, \ldots, k.$$
Суммируем эти неравенства. Поскольку $s_1+\ldots +s_k =  t_1+\ldots +t_k = n$, 
все неравенства должны обращаться в равенства, т.е. $\dim V_{\lambda _i} =  s_i$ для всех $i=1, \ldots, k.$\\
3) $\Rightarrow$ 4). Следует из теоремы  \ref{p8_5_8} с учетом $s_1+\ldots +s_k =  n$.\\
4) $\Rightarrow$ 2). В каждом $V_{\lambda_i}$ выберем базис. Объединение 
этих базисов будет нужным базисом в $V$ (см. критерий-2 прямой суммы --- теорема \ref{t7_4_2}
главы \ref{lin_prostr}), и этот базис составлен из собственных векторов. 
\edok

\otstup



\begin{sled}
%Пусть $V$ --- векторное пространство над $\mathbb{R}$, $\varphi \in L(V, V)$.
Если $\chi_{\varphi} (\lambda )$ имеет $n$ различных вещественных корней, принадлежащих 
$\mathbb{F}$, то $\varphi$ диагонализируемо.
\end{sled}

\otstup

Это следствие 
очерчивает достаточно большой класс диагонализируемых операторов.

\otstup

Вообще, определять диагонализируем оператор или нет, часто бывает удобно, 
используя условие 3) из теоремы \ref{t8_5_3}.

\otstup

Посмотрим, что означает условие диагонализируемости в переводе <<на матричный язык>>.
%Пусть $\varphi \in L(V, V)$ имеет матрицу $A$ в некотором базисе.
В силу формулы изменения матрицы при замене базиса (см. теорему 
\ref{t8_3_2} главы \ref{lin_otobr}),
для диагонализируемости матрицы $A$ (матрицы оператора $\varphi$ в некотором базисе)
необходимо и достаточно существование такой невырожденной
матрицы $S$, что  матрица $S^{-1}AS$ (подобная матрице $A$) диагональна.

%Поэтому иногда говорят о диагонализируемости квадратной матрицы $A$.


\subsection{Примеры}

%Не все преобразования диагонализируемы. 

\example{I.
Преобразование $\mathbb{R}^2\to \mathbb{R}^2$
поворота на угол, не кратный $\pi$, не диагонализиремо  (нет собственных векторов, 
или поскольку характеристические числа не вещественны).\\
Сжатие к прямой (проходящей через $O$), проектирование на прямую, отражение относительно прямой, 
напротив, диагонализируемые.
}

\example{II.
Матрица $\begin{pmatrix} \lambda_0 & 1 \\ 0 & \lambda_0 \end{pmatrix}$ 
не диагонализируема (как для $\mathbb{F}=\mathbb{R}$, так и для $\mathbb{F}=\mathbb{C}$), 
так как $\lambda _0$ --- корень кратности $2$, но $\dim V_{\lambda _0} = 1$.\\
(Объяснение без использования теоремы \ref{t8_5_3} такое: если бы это преобразование было диагонализируемым,
то диагональный вид был бы нулевой матрицей, и значит, преобразование было бы нулевым, что неверно.)
Этот пример --- частный случай жордановой клетки, общая теория на этот счет --- в следующем параграфе.
}

\example{III.
Пусть $V= \mathbf{P}_n = \lin{1, x, x^2, \ldots, x^n}$ и 
Оператор дифференцирования $d: V\to V$ имеет единственное характеристическое число $0$ кратности $n+1$ 
(см. матрицу $d$ в стандартном базисе --- пример в конце параграфа  \ref{matr_lin_otobr}
\,\,\,\,\,\,\,\,\,\,).
При $n\geq 1$ преобразование $d$ не является диагонализируемым, как и любой линейный 
дифференциальный оператор вида $d-\lambda I_V$.
}





\section{Структура недиагонализируемых операторов.}


В этом параграфе $V$ --- векторное пространство над полем $\mathbb{F}$,
$\dim V = n<\infty$, а $\varphi$ --- фиксированное линейное преобразование $V\to V$.
Предполагается (если не оговаривается противное), что все характеристические числа оператора $\varphi$
принадлежат $\mathbb{F}$ (для $\mathbb{F} = \mathbb{C}$ это выполнено всегда).

Характеристические числа обозначаем $\lambda_1, \lambda_2, \ldots , \lambda_k$,
а 
$s_1, s_2, \ldots , s_k$ --- их кратности соответственно.

\subsection{Треугольный вид}


\begin{lemm}\label{trtr}
Существует базис, в котором матрица $\varphi$ верхнетреугольная, с заданным порядком 
расстановки характеристических чисел по диагонали.
\end{lemm}
\dok СХЕМА. Требуется найти флаг инвариантных подпространств $\lin{\vek{e}_1, \ldots, \vek{e}_k}$, $k=1, \ldots, n$.

Индукция. Найдем $(n-1)$-мерное инвариантное подпространство $U\geq \Im(\varphi - \lambda _i)$ 
(см. предложение \ref{p8_5_3}). Возьмем любой $\vek{e}_n\notin U$. Тогда $\varphi (\vek{e}_n) = \lambda _i \vek{e}_n + \vek{b}$, где  $\vek{b}\in U$. 
%фактор-оператор - умножение на $\lambda _i$
Тогда для ограничения $\varphi|_U : U\to U$ характеристический многочлен равен 
$\dfrac{\chi_{\varphi}(\lambda)} {  \lambda _i -\lambda }$.
Применяя предположение индукции к $U$, находим в $U$ нужные базисные векторы
$\vek{e}_1, \ldots, \vek{e}_{n-1}$.
\edok

%В вещественном случае предыдущая теорема верна тогда и только тогда, когда
%у $\varphi$  все характеристические числа вещественные.


\otstup

{\bf Упражнение.}
Пусть $f\in \mathbb{F}[X]$. Найдите характеристические числа (и их кратности)
оператора $f(\varphi)$ (зная характеристические числа (и их кратности)
оператора $\varphi$.


\subsection{Корневые подпространства. Теорема Гамильтона-Кэли.}

Определяемые ниже {\it корневые подпространства} можно мыслить как обобщение собственных 
подпространств $V_{\lambda_i} = \Ker (\varphi - \lambda_i)$.

Для каждого характеристического числа $\lambda_i$
рассмотрим систему вложенных подпространств:
\begin{equation}\label{Ker^}
O\leq  \Ker (\varphi - \lambda_i)\leq \Ker (\varphi - \lambda_i)^2 \leq 
\Ker (\varphi - \lambda_i)^3\leq \ldots  
\end{equation}
Так как $\dim V<\infty$, в этой цепочке начиная с некоторого шага наступит стабилизация.
Обозначим $m_i$ номер этого шага, так что 
$\Ker (\varphi - \lambda_i)^{m_{i}-1} %\stackrel{<}{\neq} 
\neq \Ker (\varphi - \lambda_i)^{m_{i}} = \Ker (\varphi - \lambda_i)^{m_{i}+1} = \ldots$.

Подпространство $\Ker (\varphi - \lambda_i)^{m_{i}} $ назовем
{\it корневым}, отвечающим собственному значению $\lambda_i$.
Обозначаем это корневое подпространство 
$V^{\lambda_i}$.
Равество $V^{\lambda_i} = \Ker (\varphi - \lambda_i)^{m} $ будет верно для 
всех $m\geq m_i$, и наоборот, $m_i$ --- минимальное натуральное $m$, 
для которого верно это равенство.



Следующие утверждения о $\Ker (\varphi - \lambda_i)^{t_i}$ (в частности о $V^{\lambda _i}$) 
можно сравнить с теоремами \ref{p8_5_8}, \ref{t8_5_2} и \ref{t8_5_3}
о $V_{\lambda _i}$.

\begin{lemm}\label{p8_5_88888}
Для любых  $t_i\in \mathbb{Z}_{+}$ сумма 
$\sum\limits_{i=1}^{k} \Ker (\varphi - \lambda_i)^{t_i} $ --- прямая сумма.\\
В частности, $\sum\limits_{i=1}^{k} V^{\lambda _i} $ --- прямая сумма.
\end{lemm}
\dok Это предложение сразу следует из предложения \ref{Ker_polynom} главы
\ref{lin_otobr} --- достаточно рассмотреть многочлены $f_i(x) = (x- \lambda _i)^{t_i}$.
\edok

\begin{lemm}\label{Kers_i}
$\dim \Ker (\varphi - \lambda_i)^{s_i} \geq s_i.$
\end{lemm}
\dok
СХЕМА. 
Согласно лемме \ref{trtr}, существует базис
$\bazis{e} = (\vek{e}_1, \ldots, \vek{e}_n)$, в котором матрица оператора $\varphi$ верхнетреугольная, 
причем константны, равные $\lambda_i$, расположены в первых 
$s_i$ диагональных клетках. Это будет означать что для $j=1, \ldots, s_i$ 
выполнено $\varphi (\vek{e}_j)=  \lambda_i \vek{e}_j + \sum\limits_{x=1}^{j-1} c_x \vek{e}_x$, 
или $(\varphi - \lambda_i)(\vek{e}_j ) \in \lin{\vek{e}_1,\ldots , \vek{e}_{j-1} }$;
отcюда $(\varphi - \lambda_i)^j(\vek{e}_j )=0$, в частности, 
$\vek{e}_j \in \Ker (\varphi - \lambda_i)^{s_i}$ для $j=1, \ldots, s_i$.
Требуемое доказано, поскольку в $\Ker (\varphi - \lambda_i)^{s_i} $ нашлась линейно независимая система 
из $s_i$ векторов.
\edok

\begin{sled}
$\dim V^{\lambda _i} \geq s_i.$
\end{sled}


%%"'ВЫПУКЛОСТЬ РАЗМЕРНОСТЕЙ ЯДЕР?? из жнф следует..


\begin{theor} Справедливы следующие утверждения:
\begin{equation}\label{oplusV^}
1).\,\,\,\,\,\,\,\, \boxed{V\, \, = \, \, \bigoplus\limits_{i=1}^k \,\, V^{\lambda_i} };
\end{equation}
$$2).\,\,\,\,\,\,\,\, \boxed{\dim V^{\lambda_i} =  s_i};$$
3) $m_i\leq s_i$  (т.е. стабилизация в (\ref{Ker^}) наступает на $s_i$-м шаге или ранее), $i=1, \ldots, k$
\end{theor}
\dok
1), 2). Из леммы \ref{p8_5_88888}, с учетом последнего следствия $\dim V^{\lambda _i} \geq s_i$,
размерность прямой суммы $\bigoplus\limits_{i=1}^k \dim V^{\lambda_i}$  не меньше $\sum\limits_{i=1}^k s_i=n = \dim V$, значит эта сумма обязана совпадать с $V$, а во все неравенства обязаны обращаться в равенство.
\\
3). Следует из 2) с учетом леммы \ref{Kers_i}.
\edok

\begin{predl}\label{minmn} 
Минимальным многочленом для оператора $\varphi$ является многочлен 
$$\mu(\lambda) = \prod_{i=1}^k (\lambda - \lambda _i)^{m_i}.$$
%(где $m_i$
\end{predl}
\dok Из предложения \ref{Ker_polynom} главы \ref{lin_otobr} следует, что 
\begin{equation}\label{keroplus}
 \Ker (\prod \limits_{i=1}^{k} (\varphi - \lambda_i)^{t_i}) \,\, = \,\,  \bigoplus\limits_{i=1}^{k} 
\Ker (\varphi - \lambda_i)^{t_i}. 
\end{equation}
Если $t_i=m_i$, $i=1, \ldots, k$, то в правой части (\ref{keroplus}) прямая сумма корневых подпространств, т.е. $V$,
значит $\prod \limits_{i=1}^{k} (\varphi - \lambda_i)^{m_i}$ --- нулевой оператор, т.е. 
$\mu(\lambda)$ действительно аннулирует $\varphi$.

Согласно предложению \ref{min_mn} главы \ref{lin_otobr}, минимальный многочлен для $\varphi$ 
является делителем $\mu$, а значит имеет вид $\prod_{i=1}^k (\lambda - \lambda _i)^{t_i}$,
где  $t_i\leq m_i$, $i=1, \ldots, k$. 

Пусть хотя бы одно из этих неравенств строгое, например, 
$t_1< m_1$. Тогда $\dim \Ker (\varphi - \lambda_i)^{t_1} <s_1$ и
$\dim \Ker (\varphi - \lambda_i)^{t_i} \leq s_i$, $i=2, \ldots, k$. Отсюда следует, что 
размерность прямой суммы в правой части (\ref{keroplus}) строго меньше $\sum\limits_{i=1}^k s_i = n$, 
а значит $\Ker (\prod \limits_{i=1}^{k} (\varphi - \lambda_i)^{t_i}) \neq V$. Следовательно, при  
$t_1< m_1$ многочлен $\prod_{i=1}^k (\lambda - \lambda _i)^{t_i}$ не аннулирует $\varphi$.
\edok 

\begin{predl}[критерий-2 диагонализируемости]\label{kr_diag2}
Следующие условия на $\varphi$ эквивалентны:\\
1) $\varphi$ диагонализируем;\\
2) $V_{\lambda_i} = V^{\lambda_i}$ для всех $i=1, 2, \ldots, k$; \\ % $\varphi$;
3) $\mu_{\varphi}$ раскладывается (в $\mathbb{F}[X]$) на различные линейные множители.
%(не имеет кратных корней. %если кто-то аннулирует без кратных корней (распад. на лин. множители, %то диагоналихируемо
\end{predl}
\dok
1) $\Leftrightarrow$ 2). Достаточно составить условие 4) в критерии-1 диагонализируемости (теорема \ref{t8_5_3}), 
(\ref{oplusV^}) и очевидные включения $V_{\lambda_i} \leq V^{\lambda_i}$.\\
2) $\Leftrightarrow$ 3). Условие 2) означает, что $m_i=1$ для всех $i=1, 2, \ldots, k$.
Остается воспользоваться предыдущим предложением \ref{minmn}.
\edok

\otstup

{\bf Упражнение.} Оператор $\varphi\in L(\mathbb{R}^n, \mathbb{R}^n)$ 
удовлетворяет равенству $\varphi ^3 = \varphi$. Докажите, что $\varphi$
диагонализируем.

\otstup


{\bf Упражнение.}
Пусть $\varphi$ диагонализируем, а $U\leq V$ --- инвариантное подпространство. 
Докажите, что оператор $\varphi |_U$ ограничения на $U$ тоже диагонализируем,
{\footnotesize как и фактор-оператор}.


\begin{theor}[Теорема Гамильтона-Кэли]\label{HK} 
Для всякого $\varphi \in L(V, V)$ выполнено $$\chi_{\varphi}(\varphi) = 0. $$
\end{theor}
\dok Из описания минимального многочлена $\mu_{\varphi}$ (предложение \ref{minmn}) следует, что
$\chi_{\varphi}$ делится на $\mu_{\varphi}$, и значит, многочлен $\chi_{\varphi}$ аннулирует ${\varphi}$.
\edok

\otstup

Итак, в случае $\mathbb{F}=\mathbb{C}$ теорема Гамильтона-Кэли доказана для всех операторов, 
а в случае $\mathbb{F}=\mathbb{R}$ --- верна также для всех операторов, но доказана для 
операторов, удовлетворяющих условию, сформулировнному в начале параграфа: все характеристические числа оператора $\varphi$ принадлежат $\mathbb{R}$. Устранить этот недостаток можно, посмотрев на теорему Гамильтона-Кэли как
на формальное тождество для матриц $A\in \mathbf{M}_{n\times n}(\mathbb{F})$.
Так как это тождество верно для всех матриц из  $\mathbf{M}_{n\times n}(\mathbb{C})$, то оно верно 
и для всех матриц из  $\mathbf{M}_{n\times n}(\mathbb{R}) \subset \mathbf{M}_{n\times n}(\mathbb{C})$.

\otstup


{\bf Упражнение.} %$^*$
Матрица $A\in \mathbf{M}_{n\times n}$ такова, что $A^k=O$ при некотором $k\in \mathbb{N}$.  
Докажите, что $A^n=O$.

\otstup


{\bf Упражнение.}$^*$ %позже????
Пусть $\mu \in \mathbb{R}[X]$ --- минимальный многочлен матрицы 
$A\in \mathbf{M}_{n\times n}(\mathbb{R})$. Докажите, что $\mu$ является минимальным 
многочленом для $A$, при рассмотрении $A$ как $A\in \mathbf{M}_{n\times n}(\mathbb{C})$
(и соответственно минимальный среди многочленов над $\mathbb{C}$).\\
Попробуйте обобщить это утверждение для произвольного поля и его расширения 
$\mathbb{F}\subset \mathbb{K}$.
%Лучше додумать бы.... автоморфизм поля сделать и использовать единственность \mu



\subsection{ЖНФ: определения и формулировки.}

%Жордановы клетки и жордановы цепочки.


{\it Жордановой клеткой}, отвечающей константе $\lambda_0$, называют квадратную  матрицу вида
%{\footnotesize
%Пусть $J_t(\lambda_0)$ --- матрица $t\times t$, у которой по главной диагонали --- одинаковые числа,
%на следующей диагонали над главной --- единицы, а остальные элементы --- нули:
$$\begin{pmatrix} \lambda_0 & 1 & 0 &  0 & \ldots & 0 \\ 
0& \lambda_0 & 1 & 0  & \ldots & 0 \\
0& 0& \lambda_0 & 1 &  \ldots & 0 \\
& & &\ldots  & &\\
& & &\ldots  & &\\
0& 0& \ldots & 0 & 0 &\lambda_0 
\end{pmatrix}.$$
%Жорданову клетку $t\times t$ с константами $\lambda_0$ обозначим 
%$J_t(\lambda_0)$.

Блочно-диагональная матрица, у которой каждый диагональный блок является жордановой клеткой, называется {\it жордановой матрицей}, или матрицей, имеющей {\it жорданову нормальную форму} (жнф).
Очевидно, жордановы матрицы являются верхнетреугольными.

Базис, в котором $\varphi$ имеет жнф,  называют {\it жорданов базис}.


<<Прочитаем>> жнф (по определению матрицы линейного преобразования) и поймем, что значит жорданов базис.
Каждая жорданова клетка $t\times t$ соответствует так называемой {\it жордановой цепочке} длины $t$: 
$$ \vek{0} \leftarrow \vek{e}_1 \leftarrow \vek{e}_2 \leftarrow \ldots \leftarrow \vek{e}_t. $$\\
Здесь стрелочкой обозначено применение оператора $\varphi - \lambda_i$.
Каждая жорданова цепочка начинается с собственного вектора, следующие векторы называется
{\it присоединенными} ($\vek{e}_{i+1}$ --- присоединенный для $\vek{e}_i$).

Основными утверждениями здесь являются следующие.

\begin{theor}[существование ЖНФ]
В $V$ существует базис, в котором матрица $\varphi$ --- жорданова.
\end{theor}

Теорему существования можно переформулировать так: существует базис, являющийся объединением жордановых цепочек.

\begin{theor}[единственность  ЖНФ]
Для данного $\varphi$ единственна ЖНФ  (с точностью до перестановки жордановых клеток по диагонали).
\end{theor}

Теорему единственности можно переформулировать так: 
 для каждого $\lambda _i$ количество жордановых цепочек данной длины 
для жорданова базиса определено однозначно.

%Условие единственности жнф нив каком виде не означает единственности, что жорданов базис е

Единственность жнф выполнена несмотря на то, что, 
как увидим в доказательстве сущесвтвования, 
при конструировании жорданова базиса может быть достаточно степеней свободы.



\subsection{Доказательство существования ЖНФ}

Достаточно в каждом корневом подпространстве $V^{\lambda_i}$ построить базис,
являющийся объединением жордановых цепочек. Далее в доказательстве существования
рассматриваем нильпотентный оператор $\psi : V^{\lambda_i} \to V^{\lambda_i}$, 
являющийся ограничением оператора $\varphi - \lambda_i$ на $V^{\lambda_i}$.

\otstup

СХЕМА: построение жордановых цепочкек <<сверху вниз>> в каждом корневом подпространстве,
исходя из <<башни>>  \ref{Ker^}, которую обозначим используя $\psi$:
\begin{equation} %\label{Ker^}
O\leq  \Ker \psi \leq \Ker \psi ^2 \leq 
\Ker \psi  ^3 \leq \ldots  \leq \Ker \psi  ^{m_i} = V^{\lambda_i}.
\end{equation}


Индуктивно выберем подпространства $W_{m_i+1}=O$, далее $W_{m_i}$,  $W_{m_i-1}$ и т.д.
такие, что 
\begin{equation} \label{Woplus}
W_{t+1} \oplus \Ker \psi ^t = \Ker \psi ^{t+1}.
\end{equation}
По $W_{t+1}$ определим $W_t$ ($t\geq 1$) так. 

Если для $W_{t+1}$ верно (\ref{Woplus}), то для $\psi (W_{t+1})$ выполнено:

i)  $\psi (W_{t+1}) \leq \Ker \psi ^t $;\\
(это следует только из того, что $W_{t+1} \leq \Ker \psi ^{t+1}$.)

ii) $\dim (\psi (W_{t+1})) =  \dim W_{t+1}$;\\
(это верно, так как $W_{t+1}\cap \Ker \psi = O$.)

iii) $\psi (W_{t+1}) \cap \Ker \psi ^{t-1} = O $.
(proof)

Из iii) следует, что можно определить $W_t\geq \psi (W_{t+1}) $ так, чтобы выполнялось 
(\ref{Woplus}), т.е. $W_{t} \oplus \Ker \psi ^{t-1} = \Ker \psi ^{t}.$

\otstup

Пользуясь определенными подпространствами $W_t$, несложно построить жордановы цепочки <<сверху вниз>>.
На $t$-м <<слое>>, имея базис в $\psi (W_{t+1})$, дополним его до базиса в $W_t$,
применяя $\psi$, <<спускаем>> на этаж ниже, получая базис в $\psi (W_{t})$.


\otstup

{\bf Упражнение.}
a) Докажите, что если для некотрого $t$ выполнено $\Ker (\varphi - \lambda_i)^t = 
\Ker (\varphi - \lambda_i)^{t+1}$, то 
$\Ker (\varphi - \lambda_i)^t =  V^{\lambda_i}$.

б) Докажите, что последовательность $\dim (\Ker \varphi ^t)  - \dim (\Ker \varphi ^{t-1})$, $t=1, 2, \ldots$, 
невозрастающая.


\subsection{Доказательство единственности жнф}

Пусть дан жорданов базис $B$. 
Сперва поймем, что множество $B_{\lambda _i}$ векторов из $B$, принадлежащих цепочкам, отвечающим данному $\lambda _i$, 
имеет мощность $s_i$.
(Иначе говоря, сумма длин жордановых цепочек, отвечающих $\lambda _i$,  равна $s_i$.)

Пусть $|B_{\lambda _i}| = s_i'$. Так как $B_{\lambda _i} \subset V^{\lambda _i}$,
то $s_i ' \leq \dim V^{\lambda _i} = s_i$.  С другой стороны, всего векторов в жордановом базисе $n$, 
т.е. $\sum\limits_{i=1}^k s_i' = n = \sum\limits_{i=1}^k s_i$. Значит, каждое нерваенство
$s_i ' \leq  s_i$ обращается в равенство.

Кроме того, мы видим, что
$\lin{B_{\lambda _i} } = V^{\lambda _i} $.

\otstup

Далее, как в доказательстве существования
рассматриваем $\psi : V^{\lambda_i} \to V^{\lambda_i}$, 
являющийся ограничением оператора $\varphi - \lambda_i$ на $V^{\lambda_i}$.
$\psi$ действует на жордановых цепочках как <<спуск на один этаж>>.

Поэтому $\psi (V^{\lambda_i}) $ равно линейной оболочке  всех векторов из $B_{\lambda _i} $, 
не являющихся последними векторами своих жордановых цепочек.
Аналогично, $\psi ^2(V^{\lambda_i}) $ равно линейной оболочке  всех векторов из $B_{\lambda _i} $, 
не являющихся последними и предпоследними векторами своих жордановых цепочек, и т.д.

Обозначим  $c_d$ количество жордановых цепочек длины $d$, отвечающих $\lambda _i$. Их наблюдений выше имеем равенства:
$$ \dim V^{\lambda _i} - \dim (\Im \psi ) = c_1+c_2+c_3 \ldots + c_{m_i},   $$
$$ \dim (\Im \psi ) - \dim (\Im \psi ^2) = c_2+c_3 \ldots + c_{m_i},   $$
$$ \dim (\Im \psi ^2) - \dim (\Im \psi ^3) = c_3 \ldots + c_{m_i},   $$
и т.д.
Отсюда можно выразить $c_i$ через (инвариантные) харакетристики $\varphi$.


\otstup

%{Минимальный многочлен в терминах жнф.}

Отметим, что для величины $m_i$ (шаг, на котором происходит стабилизация ядер в (\ref{Ker^}))
теперь есть еще два эквивалетных описания: %максимальная высота векторов из $V^{\lambda _i}$
максимальная длина жордановой цепочки, отвечающей $\lambda _i$, в жордановом базисе,
или максимальный размер жордановой клетки, отвечающей $\lambda _i$, в жнф.
В этих терминах можно теперь формулировать, например, 
описание минимального многочлена (см. предложение \ref{kr_diag2}).



\subsection{О связи с теорией линейных дифференциальных уравнений}

Пусть $V_0= \mathbf{C}^{\infty}$ и $d: V_0 \to V_0$ --- оператор дифференцирования. 

Множество $V\leq V_0$ решений однородного дифф.ур. 
$$y^{(n)}+a_{n-1}y^{(n-1)}+\ldots +a_1y'+a_0y=0$$
--- это $V = \Ker p(d)$, где $p(x) = x^{n}+a_{n-1}x^{n-1}+\ldots +a_1x+a_0$.

\otstup

$V$ инвариантно относительно $d$, поэтому рассмотрим $d$ как сужение $d: V\to V$.\\
Тогда $p$ --- аннулирующий многочлен для $d$. \\

\otstup

Для каждого корня $\lambda_i$ (кратности $s_i$) многочлена $p$ попробуем найти собственный вектор $f$ оператора $d$:\\
$d(f) = \lambda_i f$ --- ответ единственный (с точностью до пропорциональности): $f(x) = e^{\lambda_i x} $.

\otstup

Ясно, что $f\in V$, и так как 
$\Ker (d-\lambda_i)^{s_i} \leq  \Ker p(d)=V$, \\
для корня $\lambda_i$ вся жорданова цепочка длины $s_i$ лежит в $V$:

$e^{\lambda_i x} \leftarrow (1+x) e^{\lambda_i x} \leftarrow (1+x+\dfrac{x^2}{2}) e^{\lambda_i x}
\leftarrow (1+x+\dfrac{x^2}{2}+\dfrac{x^3}{3!}) e^{\lambda_i x} \leftarrow \ldots$.

\otstup

Из теории ДУ известно, что $\dim V = n$, 
поэтому найденный нами жорданов базис для оператора~$d$ ---это базис в пространстве решений $V$.


\subsection{Связь с теорией линейных рекуррент}


Пусть $V_0$ --- пространство всех последовательностей комплексных чисел $f=(f_0, f_1, f_2, \ldots)$ \\
 $\delta: V_0 \to V_0$ --- оператор сдвига: $\delta (f) = g$ так что $g_i=f_{i+1}$. 

\otstup

Рассматриваем подпространство $V\leq V_0$ \\решений однородной рекурренты степени $n$:
$$f_{k+n}+a_{k-1}f_{k+n-1}+\ldots +a_1f_{k+1}+a_0f_k=0.$$
$\dim V = n$ и это $V = \Ker p(\delta)$, \\где $p(x) = x^{n}+a_{n-1}x^{n-1}+\ldots +a_1x+a_0$.

\otstup

$V$ инвариантно относительно $\delta$, поэтому рассмотрим $\delta$ как сужение $\delta: V\to V$.\\
Тогда $p$ --- аннулирующий многочлен для $\delta$. \\

\otstup

Для каждого корня $\lambda_i$ (кратности $s_i$) многочлена $p$ попробуем найти собственный вектор 
$f$ оператора $\delta$:\\
$\delta(f) = \lambda_i f$ --- ответ единственный (с точностью до пропорциональности): \\
$f = (1, \lambda_i, \lambda_i^2, \ldots) $ или $f_k = \lambda_i^k$ --- геометрическая прогрессия.

\otstup

Ясно, что $f\in V$, и так как 
$\Ker (\delta-\lambda_i)^{s_i} \leq  \Ker p(\delta)=V$, \\
 для корня $\lambda_i$ имеется одна жорданова цепочка
и одна <<строго растущая>> башня $\Ker (\delta-\lambda_i) \leq \ldots \leq \Ker (\delta-\lambda_i)^{s_i}$.\\

\otstup

Для предъявления базиса (на этот раз не будем брать жорданов базис)
в корневом подпространстве достаточно взять по вектору
<<с каждого этажа>>.

\otstup

Возьмем $g\in V_0$ вида  $g_k = q_{t-1}(k)\lambda_i^k$, где $q_r$ --- многочлен степени $r$.\\
такое $g\in \Ker (\delta-\lambda_i)^{t} \setminus \Ker (\delta-\lambda_i)^{t-1}$.


 
 % Жорданка 





\chapter{Билинейные и квадратичные формы}\label{kvadr_formy}

На протяжении всей этой главы $V$ обозначает данное векторное пространство над полем $\mathbb{R}$ или $\mathbb{C}$.
Будем одновременно развивать теорию {\it билинейных} форм в случае пространства над $\mathbb{R}$ и 
{\it полуторалинейных} форм в случае пространства над $\mathbb{C}$. 
В большинстве случаев определения, формулировки и доказательства аналогичны.
Знак комплексного сопряжения при работе в пространстве над $\mathbb{R}$ можно игнорировать.
Терминология для пространства над $\mathbb{C}$ приводится в скобках.

\section{Билинейные формы. Матрица билинейной формы}\label{matr_bilin_formy}


\subsection{Определение}

\defin{
Отображение $\beta: V \times V \to \mathbb{R}$ ($\beta: V \times V \to \mathbb{C}$) называется {\it билинейным} ({\it полуторалинейным}),
или {\it билинейной (полуторалинейной) функцией}, или {\it билинейной (полуторалинейной) формой} на пространстве $V$,
если $\forall$~$\vek{a}, \vek{a}_1, \vek{a}_2, \vek{b}, \vek{b}_1, \vek{b}_2 \in  V$ и $\forall$ $\lambda\in \mathbb{R}$ ($\mathbb{C}$)
выполняются равенства
\\
B1.1. $\beta(\vek{a}_1 + \vek{a}_2, \vek{b}) = \beta(\vek{a}_1, \vek{b}) + \beta(\vek{a}_2, \vek{b}) $,
\\
B1.2. $\beta(\lambda \vek{a}, \vek{b}) = \lambda \beta(\vek{a}, \vek{b}) $,
\\
B2.1. $\beta(\vek{a}, \vek{b}_1 + \vek{b}_2) = \beta(\vek{a}, \vek{b}_1) + \beta(\vek{a}, \vek{b}_2) $,
\\
B2.2. $\beta(\vek{a}, \lambda \vek{b}) = \overline{\lambda} \beta(\vek{a},  \vek{b}) $.
\\
}

Множество всех билинейных форм на пространстве $V$ над полем $\mathbb{F}$
обозначают  $\Hom (V, V; \mathbb{F})$.
Мы будем пользоваться более коротким обозначением $\mathcal{B} (V)$
для множества всех билинейных (полуторалинейных) форм на пространстве $V$. 



\subsection{Матрица и билинейной формы. Координатная запись}

Если $\dim V=n<\infty$ и в $V$ зафиксирован некоторый базис $\bazis{e}=(\vek{e}_1, \vek{e}_2, \ldots , \vek{e}_n)$,
то билинейной (полуторалинейной) форме можно сопоставить матрицу $n\times n$ следующим образом.

\defin{{\it Матрицей билинейной (полуторалинейной) формы} $\beta \in \mathcal{B} (V)$
в базисе $\bazis{e}$ 
называется матрица $B= (b_{ij}) \in \mathbf{M}_{n\times n}$ такая, что $(b_{ij}) = \beta (\vek{e}_i, \vek{e}_j )$
для всех $i=1, \ldots, n$, $j=1, \ldots, n$.
}

Тот факт, что $B$ --- матрица билинейной (полуторалинейной) формы $\beta$
в базисе $\bazis{e}$ будем обозначать $\beta \rsootv{\bazis{e}} B$.
Посмотрев на определение, матрицу $B$ можно неформально назвать таблицей билинейного умножения.



\begin{theor}[координатная запись]\label{t9_1_1}
Пусть $\beta \in \mathcal{B} (V)$ и $\beta \rsootv{\bazis{e}} B$. 
Пусть $\vek{a}= \bazis{e} X$, $\vek{b}= \bazis{e} Y$.
Тогда $$\boxed{ \beta (\vek{a}, \vek{b}) = X^T B \overline{Y}} .$$
\end{theor}
\dok Раскроем $\beta (\vek{a}, \vek{b}) = \beta (\sum \limits_{i=1}^n  x_i \vek{e}_i, \sum \limits_{i=j}^n  y_j \vek{e}_j) $, пользуясь линейностью по первому и (полулинейностью)
по второму аргументам: \\
$\beta (\vek{a}, \vek{b}) = \sum \limits_{i=1}^n \sum \limits_{j=1}^n x_i \overline{y_j} \beta (\vek{e}_i, \vek{e}_j) 
= \sum \limits_{i=1}^n \sum \limits_{j=1}^n x_i b_{ij} \overline{y_j} $. Полученная двойная сумма и есть единственный элемент матрицы  $X^T B \overline{Y}$ 
(эту матрицу размера $1\times 1$ мы отождествляем с числом, записанным в единственной ее ячейке).
\edok

\otstup

Формула из предыдущей теоремы фактически эквивалентна определению матрицы билинейной формы.
Более, точно, справедливо следующее %предложение.
%

\begin{predl}\label{p9_1_2}
%\begin{zamech}
Пусть дано отображение $\beta :V\times V \to \mathbb{R}$ ($\beta :V\times V \to \mathbb{C}$)
(априори не известно, что билинейное) 
и матрица $B \in M_{n\times n}$.
Пусть $\forall$ $\vek{a}, \vek{b} \in V$, имеющих координатные столбцы $X$ и $Y$  в базисе $\bazis{e}$, 
выполнено $\beta (\vek{a}, \vek{b}) = X^T B \overline{Y}$. Тогда $\beta \in \mathcal{B} (V)$, причем
$\beta \rsootv{\bazis{e}} B$.
%\end{zamech}
\end{predl}
\dok Непосредственно проверяется, что отображение $\beta$, заданное как  $\beta (\vek{a}, \vek{b}) = X^T B \overline{Y}$,
удовлетворяет равенствам B1.1---B2.2 из опеределения, поэтому $\beta \in \mathcal{B} (V)$.

Кроме того, $\vek{e}_i = \bazis{e} E_{\bullet i}$ (где $E_{\bullet i}$ --- $i$-й столбец единичной матрицы, и из правил перемножения матриц
$\beta (\vek{e}_i, \vek{e}_j) = E_{\bullet i} ^T B \overline{E_{\bullet j}} =b_{ij}$.
Поэтому в самом деле $\beta \rsootv{\bazis{e}} B$.
\edok

\otstup

В зависимости от ситуации удобно пользоваться как определением матрицы билинейной формы, так и координатной записью $X^T B \overline{Y}$.

%%%%%%%!!!
%Идеология та же, что и для линейных отображениях --- матрица несет в себе информацию об образах на парах базисных векторов.
%?? Ср. с линейными отобр.


Как следствие предложения \ref{p9_1_2} получаем, что соответствие $\beta \rsootv{\bazis{e}} B$ (зависящее от выбора базиса $\bazis{e}$)
является взаимно-однозначным соответствием между $\mathcal{B} (V)$ и $\mathbf{M}_{n\times n}$.

{\footnotesize Отметим, что после введения на множестве $\mathcal{B} (V)$ естественных операций сложения и умножения на константу, 
$\mathcal{B} (V)$ превращается в векторное пространство, при этом соответствие $\beta \rsootv{\bazis{e}} B$ является изоморфизмом. %!! над $C$????
}

\subsection{Изменение матрицы при замене базиса}

\begin{theor}\label{t9_1_2}
Пусть в $V$ выбраны базисы $\bazis{e}$ и $\bazis{e}'$, связанные матрицей перехода $S$:
$\bazis{e}'= \bazis{e} S$.
Пусть $\beta \in \mathcal{B} (V)$ таково, что 
$\beta \rsootv{\bazis{e}} B$ и $\beta \rsootv{\bazis{e}'} B'$.
%имеет матрицу $A$ в базисах $\bazis{e}$, $\bazis{f}$, и матрицу
%$A'$ в базисах $\bazis{e}'$, $\bazis{f}'$.
Тогда  $$\boxed{B'=S^TB\overline{S}}.$$
\end{theor}
\dok Пусть $\vek{a}, \vek{b} \in V$ --- произвольные векторы. Пусть $\vek{a}=\bazis{e}X = \bazis{e}'X'$, 
$\vek{b}=\bazis{e}Y = \bazis{e}'Y'$.
Тогда по теореме \ref{t9_1_1}  имеем $\beta (\vek{a}, \vek{b}) = X^T B \overline{Y}$ и $\beta (\vek{a}, \vek{b}) = X'^T B \overline{Y'}$
Подставляя  $X=SX'$, $Y=SY'$ (см. теорему \ref{t7_3_2}, глава \ref{lin_prostr}),
 имеем $\beta (\vek{a}, \vek{b}) = (SX')^T B \overline{SY'} = X'^T S^T B \overline{S} \overline{Y'} = X'^T (S^T B \overline{S}) \overline{Y'} $. 
В силу предложения \ref{p9_1_2} получаем требуемое: $B'=S^TB\overline{S}$.
\edok

\begin{sled1}
В обозначениях теоремы $\rg B = \rg B'$, т.е. ранг матрицы билинейной формы не зависит от выбора базиса
\end{sled1}
\dok 
Достаточно заметить, что матрица перехода невырожденная, а умножение на невырожденную матрицу не меняет ранг.
\edok

\begin{sled2}
В обозначениях теоремы определители $|B|$ и $|B'|$ отличаются на положительный вещественный множитель.
\end{sled2}
\dok 
По правилу произведения определителей: 
 $|B'|=|S^T|\cdot |B| \cdot |\overline{S}|  = |S|\cdot |B| \cdot \overline{|S|} = |z|^2 \cdot |B|$, где $z=|S|$.
\edok

Следствие 1 позволяет корректно ввести ранг $\rg \beta$ билинейной формы.
%Следствие 2 будет использоваться в такой ситуации: если.....

В случае, если $S$ --- {\it элементарная матрица}, преобразование $B\to S^TB\overline{S}$ соответствует 
следующим {\it двойным элементарным преобразовниям}: выполнеяется элементарное преобразование строк (ему соответствует домножение слева на матрицу $S^T$),
а затем {\it соответствующее} элементарное преобразование столбцов строк (ему соответствует домножение справа на матрицу $\overline{S}$).
Виды двойных элементарных преобразований: прибавим к $i$-й строке $j$-ю, умноженную на $\lambda$, 
 а затем прибавим к $i$-му столбцу $j$-й, умноженный на $\overline{\lambda}$;
поменяем местами $i$-ю и $j$-ю строки, а затем поменяем местами $i$-й и $j$-й столбцы; 
$i$-ю строку умножим на $\lambda$, а затем $i$-й столбец умножим на $\overline{\lambda}$.

% вещественно и
%положительно (отрицательно), то $\det A'$ вещественно и положительно (отрицательно)
%в любом базисе.


\subsection{Сужение}

Если $U\leq V$ и $\beta \in \mathcal{B} (V)$, то можно рассмотреть {\it сужение} 
$\beta \mid_{U\times U}: U\times U \to \mathbb{R}$. %($f\mid_{U\times U}: U\times U \to \mathbb{C}$).
Очевидно, сужение билинейной формы на подпространство $U$ является билинейной формой на подпространстве $U$.

Для матрицы $B$ обозначим через $B_k$ левую верхнюю угловую подматрицу $k\times k$, так что $B_k = (b_{ij})$
для всех $i=1, \ldots, k$, $j=1, \ldots, k$. В частности, $B_1=(b_{11})$, $B_n=B$.

Если в пространстве $V$ выбран базис 
$\bazis{e}=(\vek{e}_1, \vek{e}_2, \ldots , \vek{e}_n)$, то обозначим через $\bazis{e}^{(k)}$ упорядоченную систему $(\vek{e}_1, \ldots , \vek{e}_k)$ --- это базис пространства
$U_k = \lin{\vek{e}_1, \ldots , \vek{e}_k}$. (При $k=n$ имеем $U_k=V$.)

\begin{predl}\label{p9_1_1}
Пусть $\beta \in \mathcal{B} (V)$ и $\beta \rsootv{\bazis{e}} B$.
Тогда $\beta \mid_{U\times U} \rsootv{\bazis{e}^{(k)}} B_k $.
\end{predl}
\dok Сразу следует из определения.
\edok

\otstup
Примеры (скалярное произведение, на функциях с плотностью), пр-во минковского. 

\section{Симметричные билинейные и квадратичные формы}


\subsection{Симметричные билинейные формы}


\defin{
Билинейная (полуторалинейная) форма $\beta$ на пространстве $V$ называется {\it симметричной} ({\it эрмитовой} или {\it эрмитово симметричной}),
если $\forall$ $\vek{a}, \vek{b} \in V$
$$\beta (\vek{a}, \vek{b}) = \overline{\beta (\vek{b}, \vek{a})}.$$
}

Множество всех симметричных (эрмитовых) билинейных (полуторалинейных) форм на пространстве $V$ обозначаем $\mathcal{B}_{sym}(V)$.

Отметим, сразу, что $\forall$ $\vek{a} \in V$ значение $\beta (\vek{a}, \vek{a})$ вещественно
(это утверждение содержательно для комплексного пространства).

\begin{theor}\label{t9_2_1}
Пусть $\dim V<\infty$, $\bazis{e}$ --- базис в $V$. Пусть $\beta \in  \mathcal{B}(V)$, $\beta \rsootv{\bazis{e}} B$.
Тогда $\beta\in \mathcal{B}_{sym}(V)$ $\Leftrightarrow$ $B^{*}=B$ (где, как обычно,  $B^{*} = \overline{B^T}$).
\end{theor}
\dok 
\dokright Надо доказать, что $\overline{b_{ji}}=b_{ij}$ или что $\overline{\beta(\vek{e}_j, \vek{e}_i)}=\beta(\vek{e}_i, \vek{e}_j)$ для всевозможных пар индексов.
Но это сразу следует из определения.

\dokleft Пусть $\vek{a}, \vek{b} \in V$ --- произвольные векторы, $\vek{a}=\bazis{e}X$, $\vek{b}=\bazis{e}Y$.
Тогда $\beta (\vek{a}, \vek{b}) = X^T B \overline{Y}$.
Также $\beta (\vek{b}, \vek{a}) = Y^T B \overline{X}$ или (транспонируем матрицу $1\times 1$)
$\beta (\vek{b}, \vek{a}) = \overline{X}^T B^T Y = \overline{ X^T B^{*} \overline{Y} } = \overline{X^T B \overline{Y}}$. 
Отсюда $\overline{\beta (\vek{b}, \vek{a})}=\beta (\vek{a}, \vek{b})$, что и требовалось.
\edok


%{\bf Упражнение.}
%Определите {\it кососимметричные билинейные формы}, найдите условия на матрицу, эквивалентные кососимметричности формы.


\subsection{Квадратичные формы}


\defin{Пусть $\beta\in \mathcal{B}_{sym}(V)$. Отображение $k: V\to \mathbb{R}$ ($k: V\to \mathbb{C}$), заданное правилом
$k(\vek{a})=\beta (\vek{a}, \vek{a})$ называется {\it квадратичной (эрмитовой) формой} или {\it квадратичной функцией},
порожденной билинейной формой $\beta$.
}

Множество всех (эрмитовых) квадратичных форм обозначим $\mathcal{K}(V)$.

Определение дает возможность говорить о (эрмитово) симметричной матрице (эрмитовой) квадратичной форме.
Координатная запись квадратичной формы имеет вид $k(\vek{a}) = X^TB\overline{X}$ или 
$k(\vek{a}) = \sum \limits_{i=1}^n \sum \limits_{j=1}^n b_{ij} x_i  \overline{x_j} $, 
где $X = \stolbec{x_1\\ x_2\\ \vdots \\ x_n}$ --- координатный столбец вектора $\vek{a}$.
Заметим, что диагональные элементы  матрицы $B$ --- это значения квадратичной формы на базисных векторах:
$b_{ii}=k(\vek{e}_i)$.


\begin{zamech}
В вещественном случае породить квадратичную форму можно было бы и произвольной (не обязательно симметричной) билинейной формой,
однако это не изменило бы запас квадратичных форм: билинейная форма с матрицей $B$ порождает ту же форму, что и симметричная билинейная форма с матрицей
$\dfrac{B+B^T}{2}$.
\end{zamech}


\begin{theor}\label{t9_2_2}
%(Восстановление симметричной билинейной формы по соответствующей квадратичной.)
Данная (эрмитово) квадратичная форма $k$ порождается ровно одной билинейной формой $\beta\in \mathcal{B}_{sym}(V)$.
\end{theor}
\dok Достаточно показать, как значение $\beta (\vek{a}, \vek{b})$ на заданных векторах $\vek{a}$, $\vek{b}$ определяется только по значениям квадратичной формы.

1) Доказательство для $\mathbb{R}$. 
Заметим, что \\ $k(\vek{a}+\vek{b}) = \beta (\vek{a}+\vek{b}, \vek{a}+\vek{b}) = \beta (\vek{a}, \vek{a}) + \beta (\vek{a}, \vek{b})
+ \beta (\vek{b}, \vek{a})+\beta (\vek{b}, \vek{b}) = k (\vek{a}) + 2\beta (\vek{a}, \vek{b})+k (\vek{b})$, поэтому
$\beta (\vek{a}, \vek{b})= \dfrac{k(\vek{a}+\vek{b})-k(\vek{a})-k(\vek{b})}{2}$.

2) Доказательство для $\mathbb{C}$. 
Имеем \\
$k(\vek{a}+\vek{b}) = \beta (\vek{a}+\vek{b}, \vek{a}+\vek{b}) = \beta (\vek{a}, \vek{a}) + \beta (\vek{a}, \vek{b}) + \beta (\vek{b}, \vek{a})+\beta (\vek{b}, \vek{b}) 
= k (\vek{a}) + \beta (\vek{a}, \vek{b})+\beta (\vek{b}, \vek{a})+k (\vek{b})$, 
отсюда $$\beta (\vek{a}, \vek{b})+\beta (\vek{b}, \vek{a}) = k(\vek{a}+\vek{b})-k (\vek{a})-k (\vek{b}).$$
Далее \\
$k(\vek{a}+i\vek{b}) = \beta (\vek{a}+i\vek{b}, \vek{a}+i\vek{b}) = \beta (\vek{a}, \vek{a}) -i \beta (\vek{a}, \vek{b}) + i \beta (\vek{b}, \vek{a})+\beta (\vek{b}, \vek{b}) 
= k (\vek{a}) -i (\beta (\vek{a}, \vek{b})-\beta (\vek{b}, \vek{a}))+k (\vek{b})$, 
отсюда $$\beta (\vek{a}, \vek{b})-\beta (\vek{b}, \vek{a}) = ik(\vek{a}+i\vek{b})+ik (\vek{a})+ik (\vek{b}).$$
Складывая получаенные равенства, получаем выражение $\beta (\vek{a}, \vek{b})$ только через значения 
$k(\vek{a}+\vek{b})$, $k(\vek{a}+i\vek{b})$, $k(\vek{a})$, $k(\vek{b})$.
\edok

\begin{sled}
Тождественно нулевая (эрмитова) квадратичная форма порождается лишь тождественно нулевой (эрмитово) симметричной формой.
\end{sled}

\otstup

Теорема \ref{t9_2_2} показывает, что в определении фактически устанавливается взаимно-однозначное соответствие между множествами $\mathcal{B}_{sym}(V)$ и $\mathcal{K}(V)$. 
Ниже отождествляем эти множества.





\section{Диагональный и канонический вид квадратичной формы}


\defin{
Говорят, что (эрмитова) квадратичная форма $k$ имеет {\it диагональный вид} в базисе $\bazis{e}$ конечномерного пространства $V$,
если матрица формы $k$ в базисе $\bazis{e}$ диагональна.
}

\defin{
Диагональный вид (эрмитовой) квадратичной формы называется {\it каноническим}, если 
каждый диагональный элемент матрицы равен одному из чисел $1, 0, -1$.
}


Диагональный вид формы $k$ в базисе $\bazis{e}$ означает, что $k(\vek{a}) = \sum\limits_{i=1}^n d_i x_i\overline{x_i}$ или 
$k(\vek{a}) = \sum\limits_{i=1}^n d_i |x_i|^2$, где $d_i\in \mathbb{R}$.
Заметим, что от диагонального вида легко перейти к каноническому, выполнив простую замену координат: $x_i = x_i'/\sqrt{|d_i|}$ для всех $i$ с условием $d_i\neq 0$.

Докажем следующую основную теорему.


\begin{theor}\label{t9_3_1}
Пусть $\dim V=n<\infty$,  $k \in \mathcal{K} (V)$. Тогда существует базис, в котором $k$ имеет диагональный вид. 
\end{theor}
\dok 
%Доказательство проведем индукцией по $n$. База $n=1$ очевидна. Предположим теорема верна для размерностей меньших $n$. Докажем утверждение для формы $k$ на 
%$n$-мерном пространстве $V$.
%Пусть $\bazis{e}$ --- некоторый данный базис и $k \rsootv{\bazis{e}} B$.
%Достаточно научиться приходить к базису $\bazis{e}'}$, в котором у матрицы $B'$ формы $k$ 
%все элементы первой строки и первого столбца, за исключением возможно $b'_{11}$ равны нулю, т.е. $\beta().
%Далее мы можем зафиксировать базисный вектор $\vek{e}'_1$ и продолжить аналогичныете же действия с подматрицей $(n-1)\times (n-1)$, полученной отбрасыванием первой строки и первого столбца.
Достаточно научиться выполнять двойные элементарные преобразования, приводящие к матрице $B'$, у которой 
все элементы первой строки и первого столбца, за исключением возможно $b'_{11}$ равны нулю. 
Далее с матрицей $B'$ можно продолжить аналогичные двойные элементарные преобразования,
не затрагивающие первые строку и столбец, и т.д.

1) Пусть $b_{11}\neq 0$. Последовательно проведем для всех $m=2, 3, \ldots, n$, для которых $b_{m1}\neq 0$,  следующие двойные элементарные преобразовния:
вычтем из $m$-й строки 1-ю строку, умноженную на $b_{m1}/b_{11}$, а затем вычтем из $m$-го столбца 1-й столбец, умноженный на $b_{1m}/b_{11} = \overline{b_{m1}}/b_{11}$.
Двойное элементарное преобразование соответствует следующей операции с матрицами: $B\to C^TB\overline{C} $, где $C$ --- элементарная матрица, т.е. соответствует замене базиса.
После указанной серии двойных элементарных преобразований все элементы перой строки и первого столбца, за исключением $b_{11}$ станут равными нулю.

2) Пусть $b_{11}=0$, но для некоторого $m$ верно $b_{m1}\neq 0$. Тогда сведем ситуацию к случаю 1), предварительно выполнив следующее двойное элементарное преобразование:
прибавим к $1$-й строке $m$-ю строку, умноженную на $\overline{b_{m1}}$, а затем прибавим к $1$-му столбцу $m$-й столбец, умноженный на $b_{m1} = \overline{b_{1m}}$.
После этого в левом верхнем углу окажется число $2|b_{m1}|^2\neq 0$.
\edok

\begin{sled}
Пусть $\dim V=n<\infty$,  $k \in \mathcal{K} (V)$. Тогда существует базис, в котором $k$ имеет канонический вид.
\end{sled}

Алгоритм, описанный в доказательстве теоремы \ref{t9_1_2}, позволяет также вести <<протокол>>, т.е. отслеживать преобразования базиса.
Координатные столбцы исходного базиса образуют единичную матрицу (это исходная матрица перехода). Далее при каждом двойном преобразовании с текущей матрицей перехода
$S$ проделываем только столбцовое преобразование.
\section{Знакоопределенные формы. Индексы инерции}

\defin{
(Эрмитова) квадратичная форма $k$ на пространстве $V$ называется {\it положительно определенной}, если
$\forall \vek{a} $ $\in V$, $\vek{a}\neq \vek{0}$,  выполнено $k (\vek{a})>0$.
}

\defin{
(Эрмитова) квадратичная форма $k$ на пространстве $V$ называется {\it положительно полуопределенной}, если
$\forall \vek{a} $ $\in V$  выполнено $k (\vek{a})\geq 0$.
}

Аналогично определяются {\it отрицательно определенные} и {\it отрицательно полуопределенные} формы.
Конечно, существуют квадратичные формы, 
не принадлежащие ни к одному из определенных типов, например $x_1^2-x_2^2$.


\otstup

{\bf Упражнение.} Форма $k$ положительно определена и имеет в некотором базисе матрицу $B$. 
Может ли диагональный элемент матрицы $B$ быть неположительным?


\otstup

Следующая несложная теорема показывает, как выяснить по диагональному виду формы, является ли она положительно определенной, положительно полуопределенной, и т.д.


\begin{theor}\label{t9_4_1}
Пусть $k\in \mathcal{K}(V)$, $\bazis{e}$ --- базис в $V$ и $k \rsootv{\bazis{e}} \diag (d_1, d_2, \ldots, d_n)$. Тогда\\
$k$ положительно определена $\Leftrightarrow$  $d_i>0$ для всех $i=1, \ldots, n$;\\
$k$ положительно полуопределена $\Leftrightarrow$  $d_i\geq 0$ для всех $i=1, \ldots, n$;\\
$k$ отрицательно определена $\Leftrightarrow$  $d_i<0$ для всех $i=1, \ldots, n$;\\
$k$ отрицательно полуопределена $\Leftrightarrow$  $d_i\leq 0$ для всех $i=1, \ldots, n$.
\end{theor}
\dok Докажем для положительно определенных форм (в остальных случаях доказательство аналогично).

\dokright
Имеем  $d_i = k(\vek{e}_i)>0$ (из опрделения положительной определенности).

\dokleft Пусть все $d_i>0$ и $\vek{a}\neq \vek{0}$ --- произвольный вектор, $X = \stolbec{x_1\\ x_2\\ \vdots \\ x_n}$ --- его координатный столбец.
Тогда $k(\vek{a}) = \sum\limits_{i=1}^n d_i |x_i|^2 >0$, поскольку хотя бы одна координата $x_i$ ненулевая. 
\edok

\begin{sled}
Определитель матрицы положительно определенной формы положителен.
\end{sled}
\dok 
Определитель матрицы положительно определенной формы в базисе, где она имеет диагональный вид, положителен.
А значит, по следствию 2 из теоремы \ref{t9_1_2}, определитель положиетелен и для произвольного базиса.
\edok



\defin{
{\it Положительным  индексом инерции} квадратичной формы $k$
называется наибольшее целое число $p$, для которого
существует такое подпространство  $U\leq V$, $\dim U = p$, что сужение
$k \mid_U$ является положительно определенной формой.
}

Аналогично определяется отрицательный индекс инерции. Положительный и отрицательный индекс формы $k$ обозначаем $p$ (или $p(k)$) и $q$ (или $q(k)$) соответственно.
Очевидно, форма $k$ положительно определена $\Leftrightarrow$ $p(k) = n$ (где $n=\dim V$), форма $k$ положительно полуопределена 
$\Leftrightarrow$ $q(k) = 0$.


\begin{theor}[об индексах инерции]\label{t9_4_2}
Пусть $k\in \mathcal{K}(V)$, $\bazis{e}$ --- базис в $V$ и $k \rsootv{\bazis{e}} \diag (d_1, d_2, \ldots, d_n)$. Тогда
$p(k)$ равно количеству положительных чисел среди $d_1, d_2, \ldots, d_n$, а $q(k)$ --- количеству отрицательных чисел среди $d_1, d_2, \ldots, d_n$.
\end{theor}
\dok Пусть $p'$ --- количество положительных среди чисел $d_1, \ldots, d_n$. Докажем, что $p=p'$, где $p=p(k)$. Для отрицательного индекса инерции рассуждения будут аналогичны.

Можно считать, что $p'$ первых $d_i$ положительны (этого можно добиться перестановкой векторов в базисе $\bazis{e}$),
т.е. $d_1>0, \ldots, d_{p'}>0$, $d_{p'+1}\leq 0, \ldots,  d_n\leq 0$.
Положим $U_{p'} = \lin{\vek{e}_1, \ldots, \vek{e}_{p'}}$, $W_{p'} = \lin{\vek{e}_{p'+1}, \ldots, \vek{e}_{n}}$.
Заметим, что  $k \mid_{U_{p'}}$  положительно определена, а $k \mid_{W_{p'}}$ отрицательно полуопределена.
Так как $\dim U_{p'}=p'$, имеем $p\geq p'$. 

Предположим, что $p>p'$, тогда рассмотрим подпространство $U\leq V$ такое, что $\dim U=p$ и $k \mid_{U}$  положительно определена.
Имеем $\dim U+\dim W_{p'} = p+(n-p') >n $. Но $\dim (U+W_{p'}) \leq \dim V = n$, значит по теореме \ref{t7_4_3} главы \ref{lin_prostr}
получаем $\dim (U\cap W_{p'})>0$, то есть найдется ненулевой вектор $\vek{a}\in U\cap W_{p'}$.
Поскольку $\vek{a}\in U$, имеем $k(\vek{a})>0$, а так как $\vek{a}\in W_{p'}$, имеем $k(\vek{a})\leq 0$. Противоречие.
\edok

\begin{sled}
Сумма индексов инерции $p(k)+q(k)$ равна рангу $\rg k$.
\end{sled}

Для выяснения, является ли данная форма положительно определенной, без приведения ее к диагональному виду, иногда применяется следующий критерий.


\begin{theor}[критерий Сильвестра]\label{t9_4_3}
Пусть $k\in \mathcal{K}(V)$, $\bazis{e}$ --- базис в $V$ и $k \rsootv{\bazis{e}} B$. Тогда
$k$ положительно определена $\Leftrightarrow$ $|B_i|>0$ для всех $i=1, 2, \ldots, n$.
\end{theor}
\dok 
\dokright
Если $k$ положительно определена, то, очевидно $k\mid_U$ тоже положительно определена для любого подпространства $U\leq V$.
Поскольку $B_i$ --- матрица сужения $k$ на подпространство $U_i = \lin{\vek{e}_1, \ldots, \vek{e}_i}$ (см. предложение \ref{p9_1_1}), 
достаточно применить следствие из теоремы \ref{t9_4_1}.

\dokleft
 Предположим противное и найдем минимальное $m$, для которого форма  $\tilde{k} = k\mid_{U_m}$ не является положительно определенной
(где $U_m = \lin{\vek{e}_1, \ldots , \vek{e}_m}$).
По выбору $m$ получаем, что форма $k\mid_{U_{m-1}} = \tilde{k} \mid_{U_{m-1}}$ положительно определена, значит $p(\tilde{k})\geq m-1$. Так
как $p(\tilde{k}) < m$ (иначе $\tilde{k}$ была бы положительно определенной), имеем $p(\tilde{k}) =  m-1$.
%, т.е. положительный индекс инерции $p(\tilde{k})\leq m-1$.
Пусть $D=\diag (d_1, \ldots, d_m)$ %$\sum\limits_{i=1}^{m} d_i|x_i|^2$ 
--- диагональный вид (в некотором базисе) формы $\tilde{k}$. Тогда среди чисел  $d_1, \ldots, d_m$ ровно $m-1$ положительных --- все кроме одного, 
значит $|D| = d_1d_2\ldots d_m\leq 0$. Но это противоречит условию $|B_m|>0$, поскольку $D$ и $B_m$ --- матрицы одной и той же формы $\tilde{k}$ в разных базисах
(см. следствие 2 из теоремы \ref{t9_1_2}).
\edok

\otstup

Чтобы выяснить, является ли данная форма $k$ отрицательно определенной, можно проверить на положительную определенность форму $(-k)$. 

\otstup

%Следствие: критерий для отрицательно определенных форм \\
%(Вытекает из того, что $k$ отрицательно определена $\Leftrightarrow$ $-k$ положительно определена.)

{\bf Упражнение.}
Пусть $k\in \mathcal{K}(V)$, $\bazis{e}$ --- базис в $V$ и $k \rsootv{\bazis{e}} B$. Тогда
$k$ положительно полуопределена $\Rightarrow$ $|B_i|\geq 0$ для всех $i=1, 2, \ldots, n$.
Обратное утверждение неверно.


{\bf Упражнение.}
Пусть $k\in \mathcal{K}(V)$,  положительно определена. Тогда сумма всех элементов ее матрицы
больше 0.
%(значение на векторе (1,1, ..., 1)  (ясно из координатного расписывания в сумму)

{\bf Упражнение.}
%Задачи на оценивание положительного индекса инерции (большое изотропное подпространство напр. дано)
Пусть положительный индекс инерции формы $k$ не равен 0. Докажите, что 
существует базис, в котором в матрице формы $k$ все диагональные элементы положительные
(а внедиагональные --- какие угодно).



\section{Кососимметричные формы}

В этом параграфе $V$ --- векторное пространство над $\mathbb{R}$.


\defin{
Билинейная (полуторалинейная) форма $\beta$ на пространстве $V$ называется {\it кососимметричной},
если $\forall$ $\vek{a}, \vek{b} \in V$
$$\beta (\vek{a}, \vek{b}) = -\beta (\vek{b}, \vek{a}).$$
}


Множество всех симметричных (эрмитовых) билинейных (полуторалинейных) форм на пространстве $V$ обозначаем 
$\mathcal{B}_{Alt}(V)$.

\otstup

Для $\beta \in \mathcal{B}_{Alt}(V)$, очевидно,
 $\beta (\vek{a}, \vek{a}) = 0$. 


\begin{theor}\label{t9_2_1111}
Пусть $\dim V<\infty$, $\bazis{e}$ --- базис в $V$. Пусть $\beta \in  \mathcal{B}(V)$, $\beta \rsootv{\bazis{e}} B$.
Тогда $\beta\in \mathcal{B}_{Alt}(V)$ $\Leftrightarrow$ $B^{T}=-B$.% (где, как обычно,  $B^{*} = \overline{B^T}$).
\end{theor}
\dok Аналогично доказательству теоремы \ref{t9_2_1}.
\edok

%\dok 
%\dokright Надо доказать, что $\overline{b_{ji}}=b_{ij}$ или что $\overline{\beta(\vek{e}_j, \vek{e}_i)}=\beta(\vek{e}_i, \vek{e}_j)$ для всевозможных пар индексов.
%Но это сразу следует из определения.

%\dokleft Пусть $\vek{a}, \vek{b} \in V$ --- произвольные векторы, $\vek{a}=\bazis{e}X$, $\vek{b}=\bazis{e}Y$.
%Тогда $\beta (\vek{a}, \vek{b}) = X^T B \overline{Y}$.
%Также $\beta (\vek{b}, \vek{a}) = Y^T B \overline{X}$ или (транспонируем матрицу $1\times 1$)
%$\beta (\vek{b}, \vek{a}) = \overline{X}^T B^T Y = \overline{ X^T B^{*} \overline{Y} } = \overline{X^T B \overline{Y}}$. 
%Отсюда $\overline{\beta (\vek{b}, \vek{a})}=\beta (\vek{a}, \vek{b})$, что и требовалось.
%\edok



\begin{theor}[канонический вид]\label{t9_3_1111}
Пусть $\dim V=n<\infty$,  $\beta\in \mathcal{B}_{Alt}(V)$. 
Тогда существует базис, в котором $\beta$ имеет блочно-диагональный вид, в котором по диагонали
встречаются блоки вида  \\$(0)$ или 
$\begin{pmatrix} 0 & -1 \\ 
1& 0
\end{pmatrix}.$
\end{theor}
\dok CХЕМА: ДВОЙНЫМИ ЭЛЕМЕНТАРНЫМИ ПРЕОБРАЗОВАНИЯМИ.
\edok
  %кососимметричные формы


\chapter{Евклидовы и унитарные пространства}\label{evkl_prostr}



\section{Скалярное произведение. Матрица Грама}%\label{matr_grama}

\subsection{Определения}

Билинейную (эрмитово) симметричную положительно
определенную функцию называют также {\it скалярным произведением}.

\defin{{\it Евклидовым пространством} называется векторное пространство
над $\mathbb{R}$, в котором зафиксировано скалярное произведение.
}

\defin{{\it Унитарным пространством} называется векторное пространство
над $\mathbb{C}$, в котором зафиксировано скалярное произведение.
}

Будем развивать теорию евклидовых и унитарных пространств параллельно (все случаи, когда
имеются отличия, будем отмечать). В этой главе, если не оговорено противное, 
предполагается, что мы работаем в евклидовы (унитарные) пространстве~$\mathcal{E}$.
% как правило будем обозначать


Для обозначения скалярного произведения векторов $\vek{a}$ и $\vek{b}$
будем использовать $(\vek{a},\vek{b})$ (опуская $\beta$ в записи $\beta(\vek{a}, \vek{b})$).


\defin{{\it Длиной}, или {\it нормой} вектора
$\vek{a}$ называют величину $\sqrt{ (\vek{a},\vek{a})}$.
}

Длину обозначают $|\vek{a}|$ или $||\vek{a}||$.
Из положительной определенности скалярного произведения
вытекает, что $\forall \, \vek{a} \in \mathcal{E}$ выполнено
$|\vek{a}|\geq 0$, причем $|\vek{a}|= 0$
$\Leftrightarrow$ $\vek{a} = \vek{0}$.

Операцию деления вектора на его длину (т.е. переходу к единичному вектору того же направления)
иногда называют нормированием (<<отнормируем вектор>>).


\defin{Говорят, что векторы $\vek{a}$ и $\vek{b}$ ортогональны, если $(\vek{a}, \vek{b})=0$.
}

Обозначение для ортогональности векторов: $\vek{a}\perp \vek{b}$. Нетрудно заметить, что существует единственный вектор, 
ортогональный любому вектору --- это $\vek{0}$.

\otstup

Заметим, что любое подпространство $U$ евклидова (унитарного) пространства
также естественным образом
становится евклидовым (унитарным) (в качестве скалярного произведения на $U$
выступает сужение скалярного произведения на объемлющем пространстве).

\subsection{Матрица Грама}

\defin{{\it Матрицей Грама} системы векторов
$\vek{a}_1, \vek{a}_2, \ldots, \vek{a}_k$ называется матрица
$(\gamma _{i j}) \in \mathbf{M}_{k\times k}$ такая, что
$\gamma _{i j} = (\vek{a}_i, \vek{a}_j)$.
}

Таким образом, матрица Грама представляет собой  <<таблицу умножения>> базисных векторов.
Обозначение для матрицы Грама: $\Gamma (\vek{a}_1, \vek{a}_2, \ldots, \vek{a}_k)$.

Отметим, что понятие матрицы Грама не является абсолютно  новым.
Скажем, если $\bazis{e} = (\vek{e}_1, \vek{e}_2, \ldots, \vek{e}_n)$ --- базис, то
матрица $\Gamma (\vek{e}_1, \vek{e}_2, \ldots, \vek{e}_n)$ совпадает
с матрицей билинейной формы скалярного произведения в базисе $\bazis{e}$ (см. определение из $\S$ 
\ref{matr_bilin_formy} главы 
\ref{kvadr_formy}).
Более общо, если $\vek{a}_1, \vek{a}_2, \ldots, \vek{a}_k$ --- линейно независимая система
векторов, то $\Gamma (\vek{a}_1, \vek{a}_2, \ldots, \vek{a}_k)$
совпадает с матрицей скалярного произведения на подпространстве
$U = \lin{\vek{a}_1, \vek{a}_2, \ldots, \vek{a}_k}$ (в базисе $\vek{a}_1, \vek{a}_2, \ldots, \vek{a}_k$).

Следующая теорема говорит о том, что зная матрицу 
Грама, можно вычислять скалярное произведение, а значит, длины и (как увидим далее) другие метрические характеристики.

\begin{theor}[Cкалярное произведение]\label{t10_1_1}
Пусть $\bazis{e} = (\vek{e}_1, \vek{e}_2, \ldots, \vek{e}_n)$ --- базис
в $\mathcal{E}$, и $\Gamma = \Gamma(\vek{e}_1, \vek{e}_2, \ldots, \vek{e}_n)$ --- матрица Грама.
Если векторы $\vek{a}, \vek{b} \in \mathcal{E}$ таковы, что
$\vek{a} = \bazis{e} X$ и $\vek{b} = \bazis{e} Y$, то $$\boxed{(\vek{a}, \vek{b}) = X^T\Gamma \overline{Y}.}$$
\end{theor}
\dok Это частный случай теоремы \label{t9_1_1} из главы \ref{kvadr_formy}.
\edok

Докажем следующее свойство определителя матрицы Грама, геометрический смысл которого 
связан с объемом (обсуждается ниже).

\begin{predl}\label{p10_1_1}
Если система векторов $\vek{a}_1, \vek{a}_2, \ldots, \vek{a}_k$ линейно независима,
то $| \Gamma (\vek{a}_1, \vek{a}_2, \ldots, \vek{a}_k) | >0$; иначе
$| \Gamma (\vek{a}_1, \vek{a}_2, \ldots, \vek{a}_k) | =0$.
\end{predl}
\dok 
1) Пусть система векторов $\vek{a}_1, \vek{a}_2, \ldots, \vek{a}_k$ линейно независима. Тогда, как было отмечено выше,
$\Gamma (\vek{a}_1, \vek{a}_2, \ldots, \vek{a}_k)$ --- это матрица скалярного произведения на подпространстве
$U = \lin{\vek{a}_1, \vek{a}_2, \ldots, \vek{a}_k}$ (в базисе $\vek{a}_1, \vek{a}_2, \ldots, \vek{a}_k$).
То есть $\Gamma (\vek{a}_1, \vek{a}_2, \ldots, \vek{a}_k)$ --- это матрица положительно определенной билинейной симметричной формы, значит,
$|\Gamma (\vek{a}_1, \vek{a}_2, \ldots, \vek{a}_k)|>0$ по следствию из теоремы \ref{t9_4_1} главы \ref{kvadr_formy}.

2) Пусть система векторов $\vek{a}_1, \vek{a}_2, \ldots, \vek{a}_k$ линейно зависима, т.е. 
$\sum\limits_{i=1}^n \lambda_i \vek{a}_i = \vek{0}$ для некоторого ненулевого столбца $\lambda = \stolbec{\lambda_1\\  \vdots \\ \lambda_n}$.
Домножив это равенство скалярно на каждый из векторов $\vek{a}_j$, $j=1, \ldots, k$, и воспользовавшись линейностью, 
получим $\sum\limits_{i=1}^n \lambda_i (\vek{a}_i, \vek{a}_j) = 0$. Получается, что столбец $\lambda$ 
%Система полученных равенств записывается как матричное равенство
является нетривиальным решением системы линейных уравнений с (квадратной) 
матрицей коэффициентов $ \Gamma (\vek{a}_1, \vek{a}_2, \ldots, \vek{a}_k)^T $. Значит, эта матрица вырожденная, откуда
$| \Gamma (\vek{a}_1, \vek{a}_2, \ldots, \vek{a}_k)| = 0$.
\edok


\otstup 

\begin{sled1}[Неравенство Коши-Буняковского-Шварца (КБШ)]
$\forall$ $\vek{a}, \vek{b}\in \mathcal{E}$ выполнено $$|\vek{a}|\cdot |\vek{b}| \geq |(\vek{a}, \vek{b})|.$$
\end{sled1}
\dok Неравенство $ |\Gamma (\vek{a}, \vek{b})| \geq 0 $ имеет вид 
$(\vek{a}, \vek{a})(\vek{b}, \vek{b}) - (\vek{a}, \vek{b})(\vek{b}, \vek{a}) \geq 0$
$\Leftrightarrow$
$(\vek{a}, \vek{a})(\vek{b}, \vek{b}) - (\vek{a}, \vek{b})\overline{(\vek{a}, \vek{b})} \geq 0$
$\Leftrightarrow$
$|\vek{a}|^2 \cdot |\vek{b}|^2 - |(\vek{a}, \vek{b})|^2 \geq 0 $.
\edok

\otstup
Отметим, что неравенство КБШ позволяет корректно определить угол между векторами Евклидова пространства.

\begin{sled2}[Неравенство треугольника]
$\forall$ $\vek{a}, \vek{b}\in \mathcal{E}$ выполнено $$|\vek{a}|+|\vek{b}| \geq |\vek{a} + \vek{b}| .$$
\end{sled2}
\dok
Имеем %$(|\vek{a}| + |\vek{b}|)^2  = |\vek{a}|^2+ |\vek{b}|^2 +2 |\vek{a}|\cdot |\vek{b}|$,\\
$|\vek{a} + \vek{b}|^2 = (\vek{a} + \vek{b}, \vek{a} + \vek{b}) = |\vek{a}|^2+ |\vek{b}|^2 
+ (\vek{a}, \vek{b}) + (\vek{b}, \vek{a})=  $\\
$= |\vek{a}|^2+ |\vek{b}|^2 
+ (\vek{a}, \vek{b}) + \overline{(\vek{a}, \vek{b})} \leq |\vek{a}|^2+ |\vek{b}|^2 
+ 2 |(\vek{a}, \vek{b})|$, что по неравенству КБШ не больше
чем $ |\vek{a}|^2+ |\vek{b}|^2 + 2 |(\vek{a}|\cdot |\vek{b})|  =  (|\vek{a}| + |\vek{b}|)^2 $.
\edok

\otstup

Пользуясь неравенством треугольника, несложно показать, что 
определив расстояние $\rho(\vek{a}, \vek{b}) = |\vek{a} - \vek{b}|$, 
превращаем $\mathcal{E}$ в метрическое пространство. 


\example{
II. {\it Стандартное} скалярное произведение на пространстве столбцов высоты $n$ определяется как  
$(\vek{a}, \vek{b}) = \sum\limits_{i=1}^{n}x_i\overline{y_i} = X^T\overline{Y}$,  где $\vek{a} = \bazis{e}X$, $\vek{b} = \bazis{e}Y$.
}

\example{
III. В пространстве $C[a, b]$ непрерывных (комплекснозначных) функций, опеределенных на отрезке $[a,b]$,  можно задать скалярное произведение как  \\
$(f, g) = \int\limits_{a}^{b} f(x)\overline{g(x)} \, dx$.
варианты с весами. ПРИМЕРЫ - к БИЛИНЕЙНЫМИ ФУНКЦИЯМ
}

%%СЛУЧАЙНЫЕ ВЕЛИЧИНЫ ---- Корреляция и ковариация --- евклидово пр-во (матожидание квадрата) --- дисперсия...
%% что такое по сути метод наименьших квадратов (монте-карло) 
% перемножение матриц --- в графах
%см. Кострикин-Манин стр. 126

\otstup 

{\bf Упражнение.} а) Пусть $B\in \mathbf{M}_{n\times n} (\mathbb{R})$ --- матрица положительно полуопределенной формы $\beta \in \mathcal{B}(V)$.
Докажите, что $B$ равна матрице Грама $\Gamma (\vek{a}_1, \vek{a}_2, \ldots, \vek{a}_n)$, где 
$\vek{a}_1, \vek{a}_2, \ldots, \vek{a}_n$ --- некоторая система векторов $n$-мерного евклидова пространства
$\mathcal{E}$.\\
б) Докажите, что существует (гомоморфизм) $\varphi\in L(V, \mathcal{E})$ такое, что 
$\beta (\vek{a}, \vek{b}) = (\varphi (\vek{a}), \varphi (\vek{b}))$.

 %матрица Грама
\section{Ортогональные системы векторов. Ортогональное дополнение. Ортогонализация} %\label{ortogonal_vectors}


\subsection{Ортогональные системы векторов. ОНБ}

\defin{Система векторов $\vek{a}_1, \ldots, \vek{a}_k$ в евклидовом (унитарном) пространстве называется {\it ортогональной}, если 
$\vek{a}_i\perp \vek{a}_j$ для всех $1\leq i<j\leq k$.
}

\defin{Ортогональная система векторов $\vek{a}_1, \ldots, \vek{a}_k$ называется {\it ортонормированной}, если 
$|\vek{a}_i|=1$ для всех $1\leq i\leq k$.
}

В частности, можно говорить об ортогональных базисах и ортонормированных базисах (далее используем сокращение ОНБ).

\begin{predl}[теорема Пифагора]\label{p10_2_1}
Пусть $\vek{a}_1, \ldots, \vek{a}_k$ --- ортогональная система векторов. Тогда 
$|\vek{a}_1 + \ldots + \vek{a}_k|^2=|\vek{a}_1|^2+\ldots + |\vek{a}_k|^2$.
\end{predl}
\dok
$|\vek{a}_1 + \ldots + \vek{a}_k|^2=(\vek{a}_1 + \ldots + \vek{a}_k, \vek{a}_1 + \ldots + \vek{a}_k)$.
Раскрывая по линейности с учетом того, что $(\vek{a}_i, \vek{a}_j)=0$ при $i\neq j$, 
получаем $(\vek{a}_1, \vek{a}_1)+\ldots + (\vek{a}_k, \vek{a}_k) = |\vek{a}_1|^2+\ldots + |\vek{a}_k|^2$.
\edok

\otstup

Отметим следующий почти очевидный критерий ортогональности в терминах матрицы Грама.

\begin{predl}\label{p10_2_2} 
Система векторов $\vek{a}_1, \ldots, \vek{a}_k$ является ортогональной $\Leftrightarrow$ $\Gamma(\vek{a}_1, \ldots, \vek{a}_k)$ --- диагональная матрица. \\
Система векторов $\vek{a}_1, \ldots, \vek{a}_k$ является ортонормированной $\Leftrightarrow$ $\Gamma(\vek{a}_1, \ldots, \vek{a}_k)$ --- единичная матрица. 
\end{predl}
\dok Сразу следует из определения матрицы Грама.
\edok

\begin{sled1}
Ортогональная система ненулевых векторов линейно независима.
\end{sled1}
\dok Следует из предыдущего предложения с учетом предложения \ref{p10_1_1}.
{\footnotesize Также это ясно, например, из теоремы Пифагора.}
\edok

\begin{sled2}
Для конечномерного евклидова пространства: ортогональный базис --- это базис, в котором форма скалярного произведения имеет диагональный вид,
ОНБ --- это базис, в котором форма скалярного произведения имеет канонический вид. 
\end{sled2}

\begin{sled3}
В конечномерном евклидовом пространстве существет ОНБ.
\end{sled3}
\dok %ОНБ --- это базис, в котором форма скалярного произведения имеет канонический вид. 
Ввиду следствия 2, это частный случай следствия из теоремы \ref{t9_3_1} главы \ref{kvadr_formy}.
\edok

\begin{sled4}
Пусть $\bazis{e} = (\vek{e}_1, \vek{e}_2, \ldots, \vek{e}_n)$ --- ОНБ в $\mathcal{E}$.
Если векторы $\vek{a}, \vek{b} \in \mathcal{E}$ таковы, что
$\vek{a} = \bazis{e} X$ и $\vek{b} = \bazis{e} Y$, то $$\boxed{(\vek{a}, \vek{b}) = X^T\overline{Y}.}$$
\end{sled4}
\dok 
Это утверждение --- частный случай теоремы \ref{t10_2_1}.
\edok

\subsection{Переход от ОНБ к ОНБ. Ортогональные и унитарные матрицы}

\defin{Матрица $Q\in \mathbf{M}_{n\times n} (\mathbb{C})$ называется {\it унитарной}, если $$Q^{*}Q=E.$$
}

\defin{Матрица $Q\in \mathbf{M}_{n\times n} (\mathbb{R})$ называется {\it ортогональной}, если $$Q^{T}Q=E.$$
}

Множество всех ортогональных и унитарных матриц $n\times n$  обозначаем соответственно $O_n$ и $U_n$.
Сравнивая определения, получаем  $O_n\subset U_n$ и более того, $O_n= U_n\cap \mathbf{M}_{n\times n} (\mathbb{R})$
(т.е. ортогональные матрицы это в точной вещественные унитарные матрицы).


\begin{predl}\label{p10_2_100} 
Для матрицы $Q\in \mathbf{M}_{n\times n} (\mathbb{C}) $
следующие условия эквивалентны: \\
1) $Q\in U_n$;\\
2) $\exists \, Q^{-1} $ и $Q^{-1} = Q^{*}$;\\
3) столбцы матрицы $Q$ образуют ОНБ в унитарном пространстве столбцов $\mathbb{C}^n = \mathbf{M}_{n\times 1}$, наделенном
стандартным скалярным произведением $(X, Y)=X^T\overline{Y}$.
\end{predl}
\dok Очевидно, 1) $\Leftrightarrow$ 2).\\
Пусть $Q_1, \ldots, Q_n$ --- столбцы матрицы $Q$. Равеноство $Q^{*}Q=E$ означает, что $\overline {Q_i^{T}} Q_j = \delta_{ij}$ или 
$Q_i^{T} \overline {Q_j} = \delta_{ij}$. Значит, 1) $\Leftrightarrow$ 3).
\edok

\otstup

Видим, в частности, что обращать ортогональную матрицу легко: достаточно ее транспонировать.
Cвойство 3), возможно, наиболее просто для проверки <<вручную>>, 
является ли матрица $Q$ ортогональной (унитарной).

%Эквивалентные условия: $QQ^{*}=E$; $Q^{*}Q=E$;
%столбцы (строки) --- ортонормированный базис в $\mathbb{R}_n = \mathbf{M}_{n\times 1}$ ($\mathbb{C}_n$) 
%(со стандартным скалярным произведением $(X, Y)=X^T\overline{Y}$.

\begin{predl}\label{p10_2_101} 
Пусть  $Q\in U_n$. Тогда $Q^T \in U_n$, $\overline{Q} \in U_n$,  $Q^{*} \in U_n$.
\end{predl}
\dok 
\edok

\otstup

Видим равноправие строк и столбцов ортогональной (унитарной) матрицы, в частности, что строки ортогональной (унитарной) матрицы --- тоже ОНБ в пространстве строк со стандартным скалярным произведением.

\begin{predl}[Групповое свойство]\label{p10_2_102} 
Пусть  $Q, R\in U_n$. Тогда $QR\in U_n$, $Q^{-1} \in U_n $.
\end{predl}
\dok 
\edok

\begin{predl}\label{p10_2_103} 
Пусть  $Q\in U_n$. Тогда $|\det Q| = 1$.
\end{predl}
\dok 
Имеем $\det (Q^{*}Q)=\det E = 1$. Отсюда $\det Q^{*}\cdot \det Q= 1$ $\Leftrightarrow$
$\overline{\det Q^{T}}\cdot \det Q= 1$ $\Leftrightarrow$
$\overline{\det Q}\cdot \det Q= 1$ $\Leftrightarrow$ $|\det Q|^2 = 1$.
\edok

\otstup

Имеется следующая связь между ортогональными (унитарными) матрицами и переходом от ОНБ к ОНБ.

\begin{theor}\label{t10_2_104} 
Пусть $\dim \mathcal{E}=n<\infty $, $\bazis{e}=(\vek{e}_1, \ldots, \vek{e}_n)$ --- ОНБ в $\mathcal{E}$,
$\bazis{e}'=(\vek{e}'_1, \ldots, \vek{e}'_n)$ --- некоторый базис в $\mathcal{E}$. Пусть $S$ --- матрица перехода от  
базиса $\bazis{e}$ к базису $\bazis{e}'$. Тогда\\
Тогда $\varphi$ является ортогональным  $\Leftrightarrow$ $S$ --- ортогональная (унитарная) матрица.
\end{theor}
\dok 
Это сразу следует из определения матрицы перехода и условия 3) в предложении \ref{p10_2_100}.
\edok

\otstup

Таким образом, ортогональные (унитарные) матрицы --- в точности матриы перехода от ОНБ к ОНБ.
Иногда именно в теорминах матрицы перехода естественно интерпретировать ортогональные (унитарные) матрицы.
Например, можно дать другое доказательство предожения \ref{p10_2_102}.
%УПР.?

\subsection{Ортогональные подпространства}

\defin{Множества $U_1$ и $U_2$ евклидова (унитарного) пространства называются {\it ортогональными}, если 
$\forall \, \vek{a}_1\in U_1$ и $\forall \, \vek{a}_2\in U_2$ выполнено $\vek{a}_1\perp \vek{a}_2$.
}

Ортогональность двух векторов получается как частный случай (вектор считаем одноэлементным подмножеством).
Обозначение для ортогональности прежнее, например $\vek{a}\perp U$ --- значит вектор $\vek{a}$ ортогонален любому вектору из $U$.
Чаще нас будут интересовать ортогональные подпространства.
Почти очевидно, что ортогональные подпространства пересекаются тривиально. Более общий факт дает следующая теорема.

\begin{theor}\label{t10_2_2} 
Пусть $U_1, \ldots, U_k$ --- попарно ортогональные подпространства евклидова (унитарного) пространства $\mathcal{E}$. 
Тогда сумма $U_1 + \ldots + U_k$ является прямой суммой.
\end{theor}
\dok
Предположим противное, пусть, скажем найдется ненулевой $\vek{a}\in U_1\cap (U_2+ \ldots +U_k)$ 
(пользуемся критерием-1 прямой суммы --- см. теорему \ref{t7_4_1} главы \ref{lin_prostr}).
Тогда $\vek{a}=\vek{a}_2+\ldots + \vek{a}_k$, где $\vek{a}_i\in U_i$. Перенесем все слагаемые в левую часть и вычеркнем нулевые векторы, тогда 
получем, что сумма ненулевых попарно ортогональных векторов равна $\vek{0}$. Это противоречит следствию 1 из предложения \ref{p10_2_2}. 
\edok


\begin{predl}[признак ортогональности]\label{p10_2_3} 
Пусть $U_1= \lin{\vek{a}_1, \ldots, \vek{a}_k}$, $U_2= \lin{\vek{b}_1, \ldots, \vek{b}_l}$.
Тогда \\ $U_1 \perp U_2$
$\Leftrightarrow$ $\vek{a}_i\perp \vek{b}_j$, $i=1, 2, \ldots, k$, $j=1, 2, \ldots, l$.
\end{predl}
\dok
\dokright Очевидно из определения ортогональности подпространств.\\
\dokleft Пусть $\vek{a}\in U_1$, $\vek{b}\in U_2$. Тогда существуют разложения
 $\vek{a} = \sum\limits_{i=1}^k \alpha _i \vek{a}_i$,  
$\vek{b} = \sum\limits_{j=1}^l \beta _j \vek{b}_j$. Тогда 
из линейности скалярного произведения получаем $(\vek{a}, \vek{b}) = \sum\limits_{i=1}^k \sum\limits_{j=1}^l  \alpha _i \overline{\beta _j} (\vek{a}_i, \vek{b}_j)= 0$, 
т.е. $\vek{a} \perp \vek{b}$.
\edok



\subsection{Ортогональное дополнение. Ортогональная проекция}

\defin{
{\it Ортогональным  дополнением} подпространства $U$ (в евклидовом (унитарном) пространстве $\mathcal{E}$) называется 
множество всех векторов, ортогональных $U$.
}

 Обозначение для ортогонального дополнения: $U^{\bot}$. Итак,  $U^{\bot} = \{\vek{a} \, | \, \vek{a}\perp U\}$.
Очевидно, $U\perp U^{\bot}$, значит согласно теореме \ref{t10_2_2}, имеем  $U\bigoplus U^{\bot}$.
В случае $\dim U<\infty$ можно усилить так.

\begin{theor}\label{t10_2_3} 
Пусть $U\leq \mathcal{E}$, $\dim U=k<\infty$.
Тогда $$\boxed{U\bigoplus U^{\bot}=\mathcal{E}}.$$
\end{theor}
\dok 
Представим произвольный вектор $\vek{a}\in \mathcal{E}$ в виде суммы $\vek{a}_1+\vek{a}_2$, где $\vek{a}_1\in U$,  $\vek{a}_2\perp U$.

Возьмем в $U$ ортогональный базис $\vek{b}_1, \ldots, \vek{b}_k$ 
и представим произвольный вектор $\vek{a}\in \mathcal{E}$ 
в виде $\vek{a} = \alpha_1\vek{b}_1+\ldots + \alpha_k\vek{b}_k +\vek{c}$. Достаточно подобрать коэффициенты $\alpha_1, \ldots, \alpha_k$ так, чтобы
$\vek{c}=  \vek{a} - \alpha_1\vek{b}_1 - \ldots - \alpha_k\vek{b}_k$ был ортогонален $U$. Последнее равносильно (см. предложение \ref{p10_2_3})
тому, что $(\vek{c}, \vek{b}_i)=0$, $i=1, \ldots, k$. 
Ввиду ортогональности $\vek{b}_i\perp \vek{b}_j$, $i\neq j$, имеем $(\vek{c}, \vek{b}_i)=0$
$\Leftrightarrow$  $(\vek{a}, \vek{b}_i)-\alpha_i (\vek{b}_i, \vek{b}_i)=0$ $\Leftrightarrow$ $\alpha_i = \dfrac{(\vek{a}, \vek{b}_i)}{(\vek{b}_i, \vek{b}_i)}$.
\edok

\begin{sled1}
Если  $\dim \mathcal{E}=n< \infty$,  $U\leq \mathcal{E}$ и $\dim U =k$, то $\dim U^{\bot} = n-k$.
\end{sled1}

\begin{sled2}
Если  $\dim \mathcal{E}=n<\infty$ и $U\leq \mathcal{E}$, то $(U^{\bot})^{\bot} = U$.
\end{sled2}
\dok 
Так как $U\perp U^{\bot}$, то $U \subset (U^{\bot})^{\bot}$. Но по следствию 1, $\dim U =\dim  (U^{\bot})^{\bot} = n-\dim U^{\bot}$.
Значит (см....), $(U^{\bot})^{\bot} = U$.
\edok

{\bf Упражнение.}
%Для подпространств конечномерного евклидова пространства: \\
Для $U_i\leq \mathcal{E}$, $\dim U_i<\infty$, $i=1, 2$ докажите, что $(U_1+U_2)^{\bot} = U_1^{\bot} \cap U_2^{\bot}$.


\begin{sled3}
Пусть  $\dim \mathcal{E}=n< \infty$. Тогда данную ортогональную систему ненулевых векторов можно дополнить до ортогонального базиса.
\end{sled3}
\dok Пусть  $\vek{b}_1, \ldots, \vek{b}_k$ --- данная ортогональная система, положим $U = \lin{\vek{b}_1, \ldots, \vek{b}_k}$.
В силу \ref{p10_2_2}, $\vek{b}_1, \ldots, \vek{b}_k$ ---  ортогональный базис в $U$. Пусть 
$\vek{b}_{k+1}, \ldots, \vek{b}_n$ ---  ортогональный базис в $U^{\bot}$. 
Тогда $\vek{b}_1, \ldots, \vek{b}_k, \vek{b}_{k+1},\ldots, \vek{b}_n$ ---  ортогональный базис в $\mathcal{E}$.
\edok



Формула $U\bigoplus U^{\bot}=\mathcal{E}$  оправдывает термин <<ортогональное дополнение>>. Это частный случай прямого дополнения (см. ....).
В таком случае упростим терминологию и обозначения.

\defin{
{\it Ортогональной проекцией} вектора $\vek{a}$ на подпространство $U\leq \mathcal{E}$ называется проекция $\vek{a}$ на $U$ вдоль $U^{\bot}$.
}

Обозначение для ортогональной проекции: $\pr_U \vek{a}$

\begin{predl}[формула проекции]\label{10_2_4} 
Пусть $\vek{a}\in \mathcal{E}$, $U\leq \mathcal{E}$ 
и $\vek{b}_1, \ldots, \vek{b}_k$ --- ортогональный базис в $U$. Тогда 
$$\boxed{\pr_U \vek{a} = \sum\limits_{i=1}^k \frac{(\vek{a}, \vek{b}_i)}{(\vek{b}_i, \vek{b}_i)} \vek{b}_i.}$$
\end{predl}
\dok 
В доказательстве теоремы \ref{t10_2_3} мы уже нашли нужное выражение для $\pr_U \vek{a}$ 
(в виде $\alpha_1\vek{b}_1+\ldots + \alpha_k\vek{b}_k$, где $\alpha_i = \dfrac{(\vek{a}, \vek{b}_i)}{(\vek{b}_i, \vek{b}_i)}$.
\edok



\otstup

Проекции могут естественно возникать в задачах на экстремум:

\begin{predl}[расстояние до подпространства]\label{10_2_5} 
Пусть $U\leq \mathcal{E}$, $\dim U<\infty$. 
Тогда $\min\limits_{\vek{x}\in U} |\vek{a}-\vek{x}| = |\pr_{U^{\bot}} \vek{a}|$.
\end{predl}
\dok Представим $\vek{a}$ как $\vek{a}=\vek{a}_1+\vek{a}_2$, где $\vek{a}_1 = \pr_U \vek{a}$, $\vek{a}_2 = \pr_{U^{\bot}} \vek{a}$.
Тогда для $\vek{x}\in U$ имеем $|\vek{a}-\vek{x}|^2 = |\vek{a}_1+\vek{a}_2-\vek{x}|^2 = |(\vek{a}_1-\vek{x})+\vek{a}_2|^2 $.
Так как $(\vek{a}_1-\vek{x}) \perp \vek{a}_2$, по теореме Пифагора 
$|(\vek{a}_1-\vek{x})+\vek{a}_2|^2 = |\vek{a}_1-\vek{x}|^2+|\vek{a}_2|^2\geq |\vek{a}_2|^2$.
При этом неравенство обращается в равенство при $\vek{x}=\vek{a}_1$.
\edok

Следующее предложение показывает, что при работе в с координатами в ОНБ очень легко получать описание ортогонального дополнения.

\begin{predl}[Ортогональное дополнение в координатах]\label{10_2_5} 
Пусть $\bazis{e}$ --- ОНБ в $\mathcal{E}$,
$U = \lin{\vek{a}_1, \ldots, \vek{a}_k}$, где $\vek{a}_1, \ldots, \vek{a}_k$ --- векторы c с координатными столбцами $X_1, \ldots, X_k$ в ОНБ $\bazis{e}$,
$\Phi$ --- матрица со столбцами $X_i$: $\Phi = (X_1\, \ldots \, X_k)$.
Тогда $U^{\bot}$ задается (в коодинатах в ОНБ $\bazis{e}$) как $\Sol (\Phi^* X =O)$.
\end{predl}
\dok  Система $\Phi^* X =O$ состоит из уравнений вида $\overline{X_i^T}X = 0$. 
Эти уравнения означают, что вектор $\vek{x}$ с координатным столбцом $X$ ортогонален $\vek{a}_i$, $i=1, \ldots, k$.
В силу предложения \ref{p10_2_3}, это эквивалентно условию $\vek{b}\perp U$.
\edok

%Почему любое подпр-во задается системой в коодинатах????? НАОБОРОТ, если $U$ задано системой.


\begin{sled}[Теорема Фредгольма о совместности СЛУ]
Пусть дана СЛУ $AX=b$. %  (с матрицей коэффициентов $A$ размера $m\times n$. 
Тогда $AX=b$ совместна $\Leftrightarrow$ $\forall \, Y_0\in \Sol(A^{*}Y=O)$ выполнено <<условие ортогональности>>: $b^{*}Y_0=0$.
\end{sled}
\dok 
Рассмотрим пространство $\mathcal{E}=\mathbb{M}_{m\times 1}$ столбцов высоты $m$ со стандартным скалярным произведением.
Столбцы $a_{\bullet 1}, \ldots, a_{\bullet n}$ матрицы $A$ и столбец $b$ лежат в $\mathcal{E}$.
Условие совместности системы $AX=b$ эквивалентно тому, что $b\in U$, где $U = \lin{a_{\bullet 1}, \ldots, a_{\bullet n}}$.
Далее, $U^{\bot} = \Sol(A^{*}Y=O)$, поэтому <<условие ортогональности>>  $\forall \, Y_0\in \Sol(A^{*}Y=O)$
эквивалентно тому, что $b\perp U^{\bot}$. Но очевидно, $b\in U$ $\Leftrightarrow$ $b\perp U^{\bot}$.
\edok


\subsection {Ортогонализация}

Пусть изначально подпространство задано как линейная оболочка произвольной конечной системы векторов: $U = \lin{\vek{a}_1, \ldots, \vek{a}_k}$,
а требуется найти ортогональный (или ОНБ) базис в $U$ (например, чтобы после этого была возможность использовать формулу для $\pr_U \vek{a}$).
Фактически это та же процедура нахождения базиса, в котором форма скалярного произведения имеет диагональный 
(канонический вид).
Но ввиду важности этого частного случая, дадим несколько другое описание прцедуры (с геометрической точки зрения).
Алгоритм называется {\it ортогонализацей Грама-Шмидта}.

Положим $U_i=\lin{\vek{a}_1, \ldots, \vek{a}_i}$, так что $U_{i+1}=U_i\cup \{\vek{a}_{i+1}\}$.
Идея следующая: будем по очереди <<поправлять>> векторы $\vek{a}_{i}$, 
заменяя на $\vek{b}_{i}=\pr_{U_{i-1}^{\bot}} \vek{a}_{i}$ так, 
что $\vek{b}_{i}\perp U_{i-1}$ и $U_i=\lin{\vek{b}_1, \ldots, \vek{b}_i}$.

Получим явные формулы. 
Первый ненулевой вектор $\vek{a}_t$ из списка $\vek{a}_1, \ldots, \vek{a}_k$ является ортогональным базисом в $U_t$.
Пусть на некотором шаге мы уже имеем ортогональный базис $\vek{b}_1, \ldots, \vek{b}_{\ell}$ в $U_m$.
Заменяем $\vek{a}_{m+1}$ на вектор 
$\vek{a}_{m+1}- \pr_{U_m} \vek{a}_{m+1}$, т.е. на \\
$\vek{a}_{m+1}- \sum\limits_{i=1}^{\ell} \dfrac{(\vek{a}_{m+1}, \vek{b}_{i})}{(\vek{b}_{i}, \vek{b}_{i})} \vek{b}_{i}$.
Если последний вектор нулевой (это соответсвует случаю $\vek{a}_{m+1}\in U_m$), то пропустим его, если он ненулевой, 
то объявим $\vek{b}_{\ell+1} $ равным 
$$ \vek{a}_{m+1}- \sum\limits_{i=1}^{\ell} \dfrac{(\vek{a}_{m+1}, \vek{b}_{i})}{(\vek{b}_{i}, \vek{b}_{i})} \vek{b}_{i},$$ 
тем самым достраивая ортогональный базис в $U_{m+1}$.

\otstup

{\bf Упражнение.}
а) Пусть $\bazis{a}=(\vek{a}_1, \vek{a}_2, \ldots, \vek{a}_n)$ --- базис. 
Докажите, что существует ОНБ $\bazis{e}$, такой, что матрица перехода 
от $\bazis{e}$ к $\bazis{a}$ --- верхнетреугольная.\\
б) Докажите, что любую матрицу $A\in \mathbf{M}_{n\times n}(\mathbb{R})$ можно представить в виде
$A=QT$, где $Q\in O_n$, а $T$ --- верхнетреугольная.


\subsection{$k$-мерный объем}

Формально $k$-мерный объем определяется и изучается в теории меры. Тем не менее сделаем несколько важных замечаний на этот счет (говоря об объемах, полагаем $\mathbb{F}=\mathbb{R}$).

Ориентированным $n$-мерным  объемом $\vol_{\pm}$ параллелепипеда, построенного 
на векторах $\vek{a}_1, \vek{a}_2, \ldots, \vek{a}_n$, будем здесь считать (и конечно это согласуется с 
результатами из курса анализа) значение детерминанта матрицы $A$, составленной из координатнных столбцов 
векторов $\vek{a}_1, \vek{a}_2, \ldots, \vek{a}_n$ в заданном ОНБ $\bazis{e}$
(т.е. $A$ --- матрица линейного отображения, переводящего $\vek{e}_i$ в $\vek{a}_i$, $i=1, \ldots, n$).
{\footnote При этом модуль этого детерминанта не зависит от выбора ОНБ, 
так как $\det$ матрицы переходв от ОНБ к ОНБ равен $\pm 1$, см. выкладку для $k$-мерных объемов ниже.}
%Таким образом, если $\vek{a}_1, \vek{a}_2, \ldots, \vek{a}_n$ --- линейно зависимая система, то
%$\vol(\vek{a}_1, \vek{a}_2, \ldots, \vek{a}_n)=0$, 
Можно сразу отметить, что $\vol_{\pm}$ не меняется при стлбцовых элементарных преобразованиях III типа 
(прибавление к столбцу дрогого столбца, умноженного на константу), в частности, не меняется в процессе ортогонализации (без выполнения нормировки).


Имеем $(\vol  (\vek{a}_1, \vek{a}_2, \ldots, \vek{a}_n)) ^2  = \det A^T \cdot \det A = \det (A^T A)$.
Матрица $A^TA$, как несложно видеть, совпадает с матрицей Грама
$\Gamma (\vek{a}_1, \vek{a}_2, \ldots, \vek{a}_n)$.
Так проясняется геометрический смысл 
 определителя $\det (\Gamma (\vek{a}_1, \vek{a}_2, \ldots, \vek{a}_n))$,
который согласуется с предложением \ref{p10_1_1}.

Покажем, что $k$-мерный объем тоже равен
$\sqrt{\det (\Gamma (\vek{a}_1, \vek{a}_2, \ldots, \vek{a}_k))}$.
В частности, для одномерного объема имеем $\vol(\vek{a})=|\vek{a}|$.


Действительно, пусть $A\in \mathbb{M}_{n\times k}(\mathbb{R})$ --- матрица,
 составленная из координатнных столбцов 
векторов $\vek{a}_1, \vek{a}_2, \ldots, \vek{a}_k$ в заданном ОНБ $\bazis{e}$.
Кроме того, рассмотрим некоторое $k$-мерное подпространство $U\supset 
\lin{\vek{a}_1, \vek{a}_2, \ldots, \vek{a}_k}$, и выберем в $U$ некоторый ОНБ
 $\bazis{f} = (\vek{f}_1, \vek{f}_2, \ldots, \vek{f}_k)$,
так что $\bazis{f} =  \bazis{e}C$, где $C\in \mathbb{M}_{n\times k}(\mathbb{R})$ --- матрица,
 составленная из координатнных столбцов 
векторов $\vek{f}_1, \vek{f}_2, \ldots, \vek{f}_k$ в  ОНБ $\bazis{e}$;
в силу ортонормированности системы $\bazis{f}$ имеем $C^TC = E$.
Пусть $A_0\in \mathbb{M}_{k\times k}(\mathbb{R})$ --- матрица,
 составленная из координатнных столбцов 
векторов $\vek{a}_1, \vek{a}_2, \ldots, \vek{a}_k$ в  $\bazis{f}$,
так что $A=CA_0$.
Тогда 
$\det (A^T A) = \det (A_0^TС^T СA_0) = \det (A_0^T A_0)  = 
(\vol  (\vek{a}_1, \vek{a}_2, \ldots, \vek{a}_k)) ^2 $.


Отсюда несложно видеть, что в случае $\vek{a}_i\perp \vek{b}_j$
верно равенство (*)
$\vol  (\vek{a}_1, \vek{a}_2, \ldots, \vek{a}_k, \vek{b}_1, \vek{b}_2, \ldots, \vek{b}_l)
=  \vol  (\vek{a}_1, \vek{a}_2, \ldots, \vek{a}_k) 
\cdot \vol (\vek{b}_1, \vek{b}_2, \ldots, \vek{b}_l) $.

В ОБЩЕМ СЛУЧАЕ НЕР-ВО. Ф-ЛУ ЧЕРЕЗ УГОЛ МЕЖДУ ПОДПРОСТРАНТСВАМИ? (Посмотреть Лин. алгебра в задачах Прасолова)

Многократное применение этого равенства дает формулу вычисления объема {\it
прямоугольного параллелепипеда}, натянутого на попарно ортогональные векторы 
$\vek{a}_1, \vek{a}_2, \ldots, \vek{a}_k$:\\
$\vol  (\vek{a}_1, \vek{a}_2, \ldots, \vek{a}_k) = \prod_{i=1}^k |\vek{a}_i|.$


%ПРАВИЛО КРАМЕРА И ЕГО ГЕОМ, СМЫСЛ.ЧЕРЕЗ ОБЪЕМЫ

Можно показать, что справедлив принцип <<объем равен площадь основания на высоту>>
для вычисления $k$-мерного объема:  
$\vol  (\vek{a}_1, \vek{a}_2, \ldots, \vek{a}_k) = 
\vol  (\vek{a}_1, \vek{a}_2, \ldots, \vek{a}_{k-1})\cdot 
|\pr_{\lin{\vek{a}_1, \vek{a}_2, \ldots, \vek{a}_{k-1}}} \vek{a}_k|.$

Действительно, выберем в $U\supset 
\lin{\vek{a}_1, \vek{a}_2, \ldots, \vek{a}_k}$, ОНБ,
в котором матрица, составленная из координатных столбцов векторов 
$\vek{a}_1, \vek{a}_2, \ldots, \vek{a}_k$, является верхнетреугольной.
Ортогонализуем $\vek{a}_k$, заменяя его на $\vek{a}_k - \pr_{\lin{\vek{a}_1, \vek{a}_2, \ldots, \vek{a}_{k-1}}} \vek{a}_k$, и пользуемся (*).

%НЕРАВЕНСТВО В ОБЩЕМ СЛУЧАЕ --- В процессе ортогонализации длина не увеличивается.

С помощью этого соображения можно выразить расстояние до подпространства через 
объемы (см. расстояние до подпространства --- ниже.)


ОБОБЩЕНИЕ ЧЕРЕЗ ПРОЕКЦИЮ $l$-мерного пар-пипеда на орт. дополнение? 
ВООБЩЕ: ОЦЕНКИ ОБЪЕМА ТЕЛА (Выпуклого ) ЧЕРЕЗ площади проекции (разных размерностей).

%(все таки сводить в угловым минорам? см. Винберг стр. 196, 208.--- кажется, не обязательно.
%Прасолов, Тихомиров = посмотреть это все.... %ортогональное дополнение и пр
\section{Сопряженное преобразование. Самосопряженные преобразования.}

\subsection{Сопряженное преобразование}

Рассмотрим линейное преобразовние $\varphi : \mathcal{E}\to \mathcal{E}$.

\defin{
Линейное преобразование $\psi : \mathcal{E}\to \mathcal{E}$
называют {\it сопряженным} преобразованию $\varphi$, если $\forall$ $\vek{a}, \vek{b}\in \mathcal{E}$ выполнено
$$\boxed{(\varphi(\vek{a}), \vek{b})= (\vek{a}, \psi(\vek{b}))}.$$
}

Обозначение для сопряженного преобразования: $\varphi^{*}$. Из определения неясно, существует ли $\varphi^{*}$ и единственно ли оно. 
Этот недостаток определения будет устранен после следующего предложения.

\begin{predl}\label{p10_3_1}
Пусть $\dim \mathcal{E}<\infty $, $\bazis{e}$ --- ОНБ в $\mathcal{E}$. Пусть $\varphi, \psi \in L(\mathcal{E}, \mathcal{E})$, 
 $\varphi \rsootv{\bazis{e}, \bazis{e}} A$, 
$\psi \rsootv{\bazis{e}, \bazis{e}} B$. Тогда
$$\psi = \varphi ^{*} \Leftrightarrow \boxed{B=A^{*}}.$$
%где $A^{*}=\overline{A^T}$.
\end{predl}
\dok
Пусть $\vek{a}=\bazis{e}X$, $\vek{b}=\bazis{e}Y$. Имеем:
$(\varphi(\vek{a}), \vek{b}) = (AX)^T\overline{Y} = X^T A^T\overline{Y} = \beta_1 (\vek{a}, \vek{b})$, где $\beta_1$  --- билинейная (полуторалинейная) форма с матрицей $A^T$.  \\
$(\vek{a}, \psi(\vek{b})) = X^T\overline{BY} = X^T \overline{B} \overline{Y} = \beta_2 (\vek{a}, \vek{b})$, 
где $\beta_2$  --- билинейная (полуторалинейная) форма с матрицей $\overline{B}$. \\
Теперь утверждение теоремы означает, что равноство форм $\beta_1=\beta_2$ эквивалентно равенству их матриц (в одном базисе).
\edok

{\bf Упражнение.} Сформулируйте аналог предыдущего предложения в произвольном базисе с матрицей Грама $\Gamma$.

\begin{sled1}
Пусть $\dim \mathcal{E}<\infty $. Для данного $\varphi \in L(\mathcal{E}, \mathcal{E})$
существует и единственное $\varphi ^{*}$.
\end{sled1}

\begin{sled2}
Пусть $\dim \mathcal{E}<\infty $, $\varphi, \psi \in L(\mathcal{E}, \mathcal{E})$. Тогда \\
1) $(\varphi ^{*})^{*} = \varphi$;\\
2) $(\varphi \psi)^{*} = \psi^{*} \varphi^{*}$;
3) $\rg (\varphi) = \rg (\varphi ^{*})$;
4) $\overline{\chi_{\varphi} (\lambda)} = \chi_{\varphi ^{*}} (\overline{\lambda})$.
\end{sled2}
\dok Введем ОНБ $\bazis{e}$ и перейдем к матрицам преобразования в этом ОНБ. Свойства 1) --- 3) сразу следуют из соответсвующих свойств для матриц.\\
4) $\overline{\chi_{\varphi} (\lambda)} =\overline{|A-\lambda E|}$
$\chi_{\varphi ^{*}} (\overline{\lambda}) = |A^*-\overline{\lambda} E| = |\overline{A^T-\lambda E}| = 
\overline{|(A-\lambda E)^T|} = \overline{|A-\lambda E|}$.
\edok

\begin{theor}\label{t10_3_1} 
Пусть $U\leq \mathcal{E}$, $\dim \mathcal{E} <\infty$, $\varphi \in L(\mathcal{E}, \mathcal{E})$. Тогда  \\
$U$ инвариантно относительно $\varphi$ 
$\Leftrightarrow$ 
$U^{\bot}$ инвариантно относительно $\varphi ^{*}$.
\end{theor}
\dok 
\dokright
Пусть $\vek{b} \in U^{\bot}$. Требуется понять, что $\varphi ^{*} (\vek{b}) \in U^{\bot}$, то есть что 
$\forall$ $\vek{a} \in U$ выполнено $(\vek{a}, \varphi ^{*} (\vek{b})) = 0$.
Но $(\vek{a}, \varphi ^{*} (\vek{b})) = (\varphi (\vek{a}), \vek{b}) = 0$, так как $\varphi (\vek{a}) \in U$ ввиду инвариантности $U$ относительно $\varphi$.\\
\dokleft Аналогично ввиду $(\varphi ^{*})^{*} = \varphi$ и $(U^{\bot})^{\bot} = U$.
\edok

\begin{theor}[Теорема Фредгольма]\label{t10_3_2} 
Пусть $\dim \mathcal{E}=n <\infty$, $\varphi \in L(\mathcal{E}, \mathcal{E})$. Тогда  \\
$$\boxed{\Ker \varphi ^{*} = (\Im \varphi)^{\bot }}.$$
\end{theor}
\dok  Во-первых можно заметить, что подпространства $\Ker \varphi ^{*}$ и $(\Im \varphi)^{\bot }$ имеют равные размерности.
Действительно, по теор..... $\dim \Ker \varphi ^{*} = n-\rg \varphi ^{*} = n-\rg \varphi$, а по ....
$\dim (\Im \varphi)^{\bot } = n - \dim (\Im \varphi) =  n-\rg \varphi$.

Значит, согласно ....., достаточно доказать включение $\Ker \varphi ^{*} \in  (\Im \varphi)^{\bot }$.
Пусть $\vek{b}\in \Ker \varphi ^{*}$. Покажем, что $\vek{b}\in (\Im \varphi)^{\bot }$, т.е. $\forall$ $\vek{c}\in \Im \varphi$
выполнено $\vek{c}\perp \vek{b}$. Поскольку $\vek{c}\in \Im \varphi$, найдем $\vek{a}\in \mathcal{E}$ такой, что $\vek{c} = \varphi (\vek{a})$.
Тогда  $(\vek{c}, \vek{b}) = (\varphi (\vek{a}), \vek{b}) = (\vek{a}, \varphi ^{*}(\vek{b})) = (\vek{a}, \vek{0}) = 0$, откуда $\vek{c}\perp \vek{b}$, что и требовалось.
\edok

\subsection{Самосопряженные преобразования}

\defin{
Линейное преобразование $\varphi : \mathcal{E}\to \mathcal{E}$ называется {\it самосопряженным}, если $\varphi^{*}=\varphi$.
}

Иначе говоря, $\varphi \in L(\mathcal{E}, \mathcal{E})$ самосопряженное 
$\Leftrightarrow$ $\forall$ $\vek{a}, \vek{b} \in \mathcal{E}$ выполнено $(\varphi(\vek{a}), \vek{b})= (\vek{a}, \varphi(\vek{b}))$.
Тогда нетрудно заметить, что если $U$ --- инвариантное подпространство для самосопряженного $\varphi$, 
то сужение $\varphi \mid_{U} : U \to U$ тоже является самосопряженным преобразованием.

\begin{theor}\label{t10_4_1} 
Пусть $\varphi \in L(\mathcal{E}, \mathcal{E})$, $\bazis{e}$ --- ОНБ, $\varphi \rsootv{\bazis{e}, \bazis{e}} A$.
Тогда  $\varphi$ --- самосопряженное $\Leftrightarrow$ $A^{*}=A$.
\end{theor}
\dok Это частный случай предложения \ref{p10_3_1}.
\edok


\begin{theor}\label{t10_4_2} 
Если $\varphi\in L(\mathcal{E}, \mathcal{E})$ --- самосопряженное преобразование,
то все его характеристические числа вещественные. %корни $\chi_{\varphi}(\lambda)$ --- вещественные.
\end{theor}
\dok 
1) Докажем вначале утверждение для унитарного пространства (над $\mathbb{C}$). Пусть $\lambda_0$ --- характеристическое число, а $\vek{a}$ --- собственный вектор, 
соответствующий собственному значению $\lambda_0$.
Запишем равенство $(\varphi(\vek{a}), \vek{a}) = (\vek{a}, \varphi(\vek{a}))$ (оно следует из определения самосопряженного преобразования).
Так как $\varphi(\vek{a}) = \lambda_0 \vek{a}$, то $(\lambda_0 \vek{a}, \vek{a}) = (\vek{a}, \lambda_0 \vek{a})$, откуда 
$\lambda_0  (\vek{a}, \vek{a}) = \overline{\lambda_0} (\vek{a},  \vek{a})$ $\Rightarrow$ $\lambda_0 = \overline{\lambda_0}$ $\Rightarrow$ $\lambda_0 \in \mathbb{R}$.\\
2) Покажем, как случай евклидова пространства (над $\mathbb{R}$) свести к разобранному. С учетом теоремы \ref{t10_4_1} получаем, что в п.1) доказано, что 
для каждой эрмитово-симметричной матрицы $A$ уравнение $|A-\lambda E|=0$ имеет лишь вещественные корни. Этого достаточно, так как  $\varphi$ в любом ОНБ имеет симметричную матрицу.
(частный случай эрмитово-симметричной).
\edok


\begin{predl}\label{p10_4_3} 
Пусть $\varphi\in L(\mathcal{E}, \mathcal{E})$ --- самосопряженное преобразование,
$\lambda _i\neq \lambda _j$ --- его различные собственные значения. Тогда $V_{\lambda _i}\perp V_{\lambda _j}$
\end{predl}
\dok Пусть $\vek{a}_i\in V_{\lambda _i}$ и $\vek{a}_j\in V_{\lambda _j}$.
Тогда $(\varphi(\vek{a}_i), \vek{a}_j) = (\vek{a}_i, \varphi(\vek{a}_j))$ 
$\Rightarrow$   $(\lambda_i \vek{a}_i, \vek{a}_j) = (\vek{a}_i, \lambda_j \vek{a}_j)$
$\Rightarrow$   $\lambda_i  (\vek{a}_i, \vek{a}_j) = \lambda_j (\vek{a}_i,  \vek{a}_j)$. Отсюда  
с учетом $\lambda _i\neq \lambda _j$ получаем $(\vek{a}_i, \vek{a}_j) = 0$, что и требовалось.
\edok



\begin{theor}[Основная теорема о самосопряженных преобразованиях]\label{t10_4_3} 
Пусть $\dim \mathcal{E}<\infty$. Для самоспоряженного преобразования $\varphi\in L(\mathcal{E}, \mathcal{E})$
существует ОНБ из собственных векторов.
\end{theor}
\dok Проведем индукцию по $\dim \mathcal{E}$. База $\dim \mathcal{E}=1$ очевидна.

Пусть $\dim \mathcal{E}=n$ и предположим, для размерности $n-1$ утверждение теоремы верно.
Рассмотрим собственный вектор и нормируем его, получим $\vek{e}_n$ такой, что $|\vek{e}_n|=1$ и $\varphi(\vek{e}_n)=\lambda_n \vek{e}_n$.
Подпространство $\lin{\vek{e}_n}$ размерности 1 инвариантно, значит $U=\lin{\vek{e}_n}^{\bot}$ --- $(n-1)$-мерное инвариантное подпространство.
Как мы замечали, сужение $\varphi$ на $U$ --- самосопряженное, и по предположению индукции в $U$ существует ОНБ $\vek{e}_1, \ldots, \vek{e}_{n-1}$ из собственных векторов.
Но $\vek{e}_n\perp \vek{e}_i$ для всех $i=1, \ldots, n-1$, поэтому $\vek{e}_1, \ldots, \vek{e}_{n-1}, \vek{e}_{n}$ --- искомый ОНБ в $\mathcal{E}$.
\edok

Последняя теорема фактически усиливает предложение \ref{p10_4_3}:
получается, что для самосопряженного преобразования $\mathcal{E} = V_{\lambda_1}\oplus V_{\lambda_2} \oplus \ldots \oplus V_{\lambda_k}$, 
где $V_{\lambda_i}$, $i=1, \ldots, k$, --- попарно ортогональные собственные подпространства.

Геометрическая интерпретация последней теоремы: самосопряженное преобразование ---  
композиция <<обобщенных растяжений>>  (т.е. с любым вещественным коэффициентом)
вдоль ортогональных осей. 
В частности, ортогональные проектирования и отражения являются самосопряженными преобразованиями.


 %сопряженные, с-с
\section{Изометрии. Ортогональные  и унитарные преобразования}

\subsection{Изометрия (изоморфизм евклидовых (унитарных) пространств)}

Пусть даны два евклидовых (унитарных) пространства $\mathcal{E}_1$ и $\mathcal{E}_2$.

\defin{Отображение $\varphi: \mathcal{E}_1\to \mathcal{E}_2$ называется {\it изометрией}, или {\it изоморфизмом евклидовых (унитарных) пространств},
если $\varphi$ --- изоморфизм векторных пространств с дополнительным условием: 
$\forall \, \vek{a}, \vek{b} \in \mathcal{E}_1$ выполнено $(\varphi (\vek{a}), \varphi (\vek{b}))=(\vek{a},\vek{b})$.
}

Итак, изоморфизм еквлидовых пространств  --- это обычный изоморфизм векторных пространств, который вдобавок сохраняет скалярное произведение.

\defin{Два евклидовых (унитарных) пространства $\mathcal{E}_1$ и $\mathcal{E}_2$
называются {\it изоморфными}, если существует изоморфизм $\varphi: \mathcal{E}_1\to \mathcal{E}_2$.
}

Обозначение $\cong$ (как и в случае обычного изоморфизма).
Как и для обычного изоморфизма показывается, что отношение $\cong$ --- отношение эквивалентности.


\begin{predl}\label{p10_2_3} 
Пусть $\dim \mathcal{E}_1 = \dim \mathcal{E}_2 = n<\infty $ и 
$\varphi \in L(\mathcal{E}_1\to \mathcal{E}_2)$, $\bazis{e} = (\vek{e}_1, \ldots, \vek{e}_n)$ --- ОНБ в $\mathcal{E}_1$.
Тогда $\varphi$ является изоморфизмом евклидовых (унитарных) пространств $\Leftrightarrow$ $(\varphi(\vek{e}_1), \ldots, \varphi(\vek{e}_n))$ --- ОНБ в $\mathcal{E}_2$. 
\end{predl}
\dok 
\dokright Сразу следует из определения изоморфизма евклидовых (унитарных) пространств:
$\varphi(\vek{e}_i), \varphi(\vek{e}_j)) = (\vek{e}_i, \vek{e}_j)) = \delta_{ij}$.\\
\dokleft Так как $\varphi$ переводит базис в базис, то $\varphi$ --- (обычный) изоморфизм векторных пространств (см.....).
Возьмем произвольные векторы $\vek{a}$ и  $\vek{b}$ из $V$ и разложим их по базису $\bazis{e}$:
$\vek{a} = \sum\limits_{i=1}^n x_i \vek{e}_i$, $\vek{b} = \sum\limits_{i=1}^n y_i \vek{e}_i$. Тогда
$\varphi(\vek{a}) = \sum\limits_{i=1}^n x_i \varphi(\vek{e}_i)$, $\varphi(\vek{b}) = \sum\limits_{i=1}^n y_i \varphi(\vek{e}_i)$.
Тогда по следствию 4 из предложения \ref{p10_2_2}:
$(\vek{a}, \vek{b}) = \sum\limits_{i=1}^n x_i\overline{y_i} = (\varphi(\vek{a}), \varphi(\vek{b}))$.
\edok

\begin{theor}\label{t10_2_1} 
Два конечномерных евклидовых (унитарных) пространства $\mathcal{E}_1$ и $\mathcal{E}_2$
изоморфны $\Leftrightarrow$ $\dim \mathcal{E}_1 = \dim \mathcal{E}_2$.
\end{theor}
\dok
\dokright 
Если $\dim \mathcal{E}_1 \neq \dim \mathcal{E}_2$, то $\mathcal{E}_1$ и $\mathcal{E}_2$ не изоморфны даже как обычные векторные пространства.
\dokleft
Если $\dim \mathcal{E}_1 = \dim \mathcal{E}_2$, то достаточно взять ОНБ в каждом из пространств и взять линейное отображение $\varphi: \mathcal{E}_1 \to \mathcal{E}_2$, 
переводящее ОНБ в ОНБ. % (такое $\varphi$ существует по ....).
Согласно предложению \ref{p10_2_3}, $\varphi$ будет изоморфизмом.
\edok

\otstup

{\bf Упражнение.} Докажите, что в определении изометрии условие 
$(\varphi (\vek{a}), \varphi (\vek{b}))=(\vek{a},\vek{b})$
можно заменить на более слабое $(\varphi (\vek{a}), \varphi (\vek{a}))=(\vek{a},\vek{a})$.


\subsection{Ортогональные  и унитарные преобразования}


\defin{
Линейное преобразование $\varphi: \mathcal{E}\to \mathcal{E}$ называется ортогональным (унитарным), если
 $\forall$ $\vek{a}, \vek{b}\in V$ выполнено
$$(\varphi(\vek{a}), \varphi(\vek{b}))= (\vek{a}, \vek{b}).$$
}

Ортогональное преобразование сохраняется скалярное произведение. Геометрически это означает сохранение длин и (в случае евклидова пространства) углов,
т.е. если мыслить себе векторы как радиус-векторы с началом в $O$, то происходит <<движение>> с неподвижной точкой $O$.
Для конечномерных евклидовых (унитарных) пространств ортогональные (унитарные) преобразование --- это в точности изоморфизмы на себя:

\begin{predl}\label{p10_5_1} 
Пусть $\dim \mathcal{E}=n<\infty $, $\varphi \in L(\mathcal{E}\to \mathcal{E})$. 
Тогда $\varphi$ --- ортогональное (унитарное) $\Leftrightarrow$ $\varphi$ --- изоморфизм евклидовых (унитарных) пространств.
\end{predl}
\dok \dokright Достаточно доказать биективность $\varphi$. Но из определелния ортогональности преобразования следует, что 
$\varphi$ переводит ОНБ в некоторую ортонормированную систему из $n$ векторов, т.е. в ОНБ. %\label{p10_2_3}\\
\\
\dokleft Очевидно по определению изоморфизма.
\edok

\begin{sled1}
Пусть $\dim \mathcal{E}=n<\infty $, $\varphi \in L(\mathcal{E}\to \mathcal{E})$, 
$(\vek{e}_1, \ldots, \vek{e}_n)$ --- ОНБ в $\mathcal{E}$.
Тогда $\varphi$ является ортогональным  (унитарным) преобразованием $\Leftrightarrow$ $(\varphi(\vek{e}_1), \ldots, \varphi(\vek{e}_n))$ --- ОНБ в $\mathcal{E}$. 
\end{sled1}


\begin{sled2}
Пусть $\dim \mathcal{E}=n<\infty $, $\bazis{e}=(\vek{e}_1, \ldots, \vek{e}_n)$ --- ОНБ в $\mathcal{E}$.
Пусть $\varphi \in L(\mathcal{E}\to \mathcal{E})$ так, что 
$\varphi \rsootv{\bazis{e}, \bazis{e}} A$.
Тогда $\varphi$ является ортогональным  (унитарным) преобразованием $\Leftrightarrow$ $A$ --- ортогональная (унитарная) матрица.
\end{sled2}
%МАТРИЦА!!

\begin{sled3}
Пусть $\dim \mathcal{E}=n<\infty $, $\varphi \in L(\mathcal{E}\to \mathcal{E})$.
Тогда $\varphi$ является ортогональным  (унитарным) преобразованием $\Leftrightarrow$ $\varphi$ обратимо, причем $\varphi ^{*} = \varphi ^{-1}$. 
\end{sled3}
\dok
Докажем в терминах матриц (хотя можно вывести непосредственно из определения).
Пусть $\bazis{e}$ --- ОНБ в $\mathcal{E}$ так, что 
$\varphi \rsootv{\bazis{e}, \bazis{e}} A$. Тогда $\varphi^{*} \rsootv{\bazis{e}, \bazis{e}} A^{*}$, 
$\varphi^{-1} \rsootv{\bazis{e}, \bazis{e}} A^{-1}$. Поскольку $A^{*}=A^{-1}$, получаем $\varphi^{*}=\varphi^{-1}$.
\edok


\begin{predl}[Групповые свойства]\label{p10_5_2} 
Пусть $\dim \mathcal{E}=n<\infty $, $\varphi, \psi \in L(\mathcal{E}, \mathcal{E})$ --- ортогональные (унитарные) преобразования.
Тогда преобразования $\varphi \psi$ и $\varphi ^{-1}$ --- также ортогональные (унитарные).
\end{predl}
\dok Можно вывести из определения или из предложения... (ортог. матрицы).
\edok


\begin{predl}\label{p10_5_3} 
Пусть $\varphi \in L(\mathcal{E}, \mathcal{E})$ --- ортогональное (унитарное) и $\lambda_0$ --- его характеристическое число.
Тогда $|\lambda_0|=1$.
\end{predl}
\dok 1) Докажем вначале утверждение для унитарного пространства (над $\mathbb{C}$). Пусть $\vek{a}$ --- собственный вектор, 
соответствующий собственному значению $\lambda_0$.
Равенство $(\varphi(\vek{a}), \varphi(\vek{a})) = (\vek{a}, \vek{a})$
принимает вид $(\lambda_0 \vek{a}, \lambda_0 \vek{a}) = (\vek{a}, \vek{a})$, откуда 
$\lambda_0 \overline{\lambda_0}  (\vek{a}, \vek{a}) = (\vek{a},  \vek{a})$ $\Rightarrow$ $\lambda_0 \overline{\lambda_0}=1$ $\Rightarrow$ $|\lambda_0|=1$.\\
2) Покажем, как случай евклидова пространства (над $\mathbb{R}$) свести к разобранному. В п.1) доказано, что 
для каждой унитарной матрицы $A$ уравнение $|A-\lambda E|=0$ имеет лишь корни по модулю равные 1. Этого достаточно, так как  $\varphi$ в любом ОНБ имеет 
ортогональную матрицу (частный случай унитарной матрицы).
\edok

В унитарном (над $\mathbb{C}$) пространстве имеется результат, а аналогичный теореме \ref{t10_4_3}.

\begin{theor}[канонический вид унитарного преобразовния]\label{t10_5_3} 
Пусть $\mathcal{E}$ --- унитарное пространство, $\dim \mathcal{E}<\infty$. Для унитарного преобразования $\varphi\in L(\mathcal{E}, \mathcal{E})$
существует ОНБ из собственных векторов.
\end{theor}
\dok Доказательство повторяет доказательство теоремы \ref{t10_4_3} с следующим небольшим отличием
в обосновании инвариантности $U=\lin{\vek{e}_n}^{\bot}$ относительно $\varphi$:
по теореме \ref{t10_3_1} $U$ инвариантно относительно $\varphi^{*}=\varphi^{-1}$; но тогда по предложению .... главы ... ,
$U$ инвариантно и относительно $\varphi$.
\edok


{\footnotesize 
Аналогом последней теоремы для евклидова пространства (над $\mathbb{R}$) является следующее утверждение о каноническом виде ортогонального преобразования.
Пусть $\varphi\in L(\mathcal{E}, \mathcal{E})$ --- ортогональное. Тогда $\mathcal{E}$ --- 
прямая сумма попарно ортогональных инвариантных подпространств размерности 1 или 2.\\
Доказать это по той же схеме, что и теорему \ref{t10_4_3} используя, что у любого преобразования вещественного пространства
найдется одномерное либо двумерное инвариантное подпространство.
}


\subsection{Полярное разложение}


\begin{theor}\label{t10_6_105} 
Пусть $\dim \mathcal{E}=n<\infty $, $\varphi, \in L(\mathcal{E}, \mathcal{E})$.
Тогда существуют самосопряженное преобразование  $\psi$ и ортогональное (унитарное) преобразование $\theta$ такие, что
$$\varphi = \psi \theta.$$ 
\end{theor}
\dok 
\edok

Мы получили следующую информацию о геометрии произвольного преобразования евклидова пространства:
это композиция некоторого <<движения>> и <<обощенных растяжений>> к ортогональным осям.


{\bf Упражнение.}
Выведите из теоремы существование разложения
$\varphi = \theta_1 \psi _1$, где $\psi _1$ --- самосопряженное, $\theta _1$ --- ортогональное.


 %изоморфизм, ортог. преобразования, полярное разложение
\section{Билинейные формы в евклидовом пространстве}

\subsection{Связь между билинейными формами и преобразованиями}

Пусть $\mathcal{E}$ --- евклидово (унитарное) пространство. 
Каждому линейному оператору  $\varphi\in L(\mathcal{E}, \mathcal{E})$
сопоставим отображение $\beta _{\varphi} : \mathcal{E} \times \mathcal{E} \to \mathbb{R}$ по формуле 

$$\boxed{\beta _{\varphi} (\vek{a}, \vek{b}) = (\vek{a}, \varphi(\vek{b}))}. \eqno(*)$$ 


\begin{predl}\label{dd} 
$\beta _{\varphi}$, определенное $(*)$ --- билинейная (полуторалинейная) форма.
\end{predl}
\dok Непосредственно проверяется, с использованием линейности $\varphi$.
\edok

\begin{predl}\label{oper_bilin} 
Пусть $\dim \mathcal{E}<\infty$, 
$\varphi\in L(\mathcal{E}, \mathcal{E})$, $\bazis{e}$ --- некоторый ОНБ и
$\varphi \rsootv{\bazis{e}, \bazis{e}} A$. 
Тогда $\beta_{\varphi} \rsootv{\bazis{e}} \overline{A}$. 
\end{predl}
\dok Запишем в координатах (как обычно, полагая $\vek{a} = \bazis{e}X$, $\vek{b} = \bazis{e}Y$):\\
$\beta (\vek{a}, \vek{b}) = (\vek{a}, \varphi(\vek{b})) = X^T\overline{AY} = X^T\overline{A}\overline{Y}}$.
Мы видим координатную запись билинейной формы с матрицей $B=\overline{A}$. Соглаасно предложению..., все доказано.
\edok


Мы видим, что сопоставление $\varphi \to \beta_{\varphi}$  согласуется матрицами в ОНБ.
Это объясняется также и тем, что законы преобразования при переходе от ОНБ к ОНБ
 (с некоторой ортогональной (унитарной) матрицей перехода $S$)
для матрицы  оператора $A$ и 
(комплексно-сопряженной) матрицы билинейной формы $\overline{B}$
 выглядят одинаково:
$A\to S^{-1}AS$, $\overline{B} \to \overline{S^{T}}\overline{B}S$.


\begin{sled1}\label{111} 
Пусть $\dim \mathcal{E}<\infty$. 
Сопоставление $\varphi \to \beta_{\varphi}$ задает биекцию 
$L(\mathcal{E}, \mathcal{E}) \to \mathcal{B} (\mathcal{E})$. %(изоморфизм, в C не совсем)
\end{sled1}


\begin{sled2}\label{2111} 
Пусть $\dim \mathcal{E}<\infty$. 
Сопоставление $\varphi \to \beta_{\varphi}$ задает биекцию 
между множеством самосопряженных операторов и 
$\mathcal{B}_{sym} (\mathcal{E}) $ (или $\mathcal{K} (\mathcal{E})$). %(изоморфизм, в C не совсем)
\end{sled2}

%Квадратичную (эрмитову) форму, порожденную $\beta_{\varphi}$, можно обозначить
%$k_{\varphi}$.


Таким образом, сопоставление 
 $\varphi \to \beta_{\varphi}$ дает возможножность 
сводить изучение билинейных форм к изучению операторов и наоборот.
В частности, изучение квадратичных (эрмитровых) форм может быть сведено
к изучению самосопряженных операторов.

%Убедимся, что эта формула дает естественное соответствие между билинейными формами и операторами, 
%поэтому изучение билинейных форм можно сводить к изучению операторов и наоборот.
%ссылка на тензоры и естественные изоморфизмы

%в обратную сторону тоже соответствие работает (напр. у Винберга) ---
% нахождение с.в. через максимизацию значения кв. функционала на сфере.

Иногда термины {\it положительная определенность} или {\it положительность} 
мы будет применять и к самосопряженным преобразованиям (имея в виду указанную биекцию).

%задача ---- верно ли, что если $G$ пол. определена, то и $G^{-1}$ тоже?

%критерий положительной определенности в терминах корней хар. многочлена

%еще про пол. опрделенность...

\subsection{Приведение к главным осям}



\begin{theor}\label{t10_7_1} 
Пусть $k$ --- квадратичная форма в евклидовом (унитарном) пространстве $\mathcal{E}$.
%или билин. симметрическая
Тогда существует ОНБ, в котором $k$ имеет диагональный вид.
\end{theor}
\dok СХЕМА: ссылаемся на соответсвующую теорему о с/с операторах.
\edok

\otstup

{\bf Упражнение.}
%Критерий положительной определенности: 
Пусть дана квадратичная форма $k$ на конечномерном  пространстве $V$ (без евклидовой структуры).
 Пусть $k$ в некотором базисе имеет матрицу $B$. Тогда $k$ положительно определена 
$\Leftrightarrow$ все корни уравнения $|B-\lambda E|=0$ положительны. 

%Это можно было (по крайней мере в одну сторону) и раньше доказать из матричной записи 
%$X^TBX $ и рассмотрения с.в.

Задачу нахождение указанного диагонального вида и соответствующего ОНБ иногда называют
{\it приведением формы к главным осям}.

%ЧЕРЕЗ присоедин. преобразование? (пример подъема индекса?) или через матрицы?

\subsection{Приложение к классификации кривых второго порядка}


\begin{predl}\label{} 
Пусть  $\ell$ --- кривая второго порядка на плоскости. Тогда существует ПДКС $Oxy$, в котором 
$\ell$ задается уравнением $Ax^2+By^2+2Dx+2Ey+F=0$.
\end{theor}
\dok 
\edok


\begin{predl}\label{} 
Пусть  $\ell$ --- поверхность второго порядка в пространстве. Тогда существует ПДКС $Oxyz$, в котором
$\ell$ задается уравнением $Ax^2+By^2+Cz^2+2Dx+2Ey+2Fz+G=0$.
\end{theor}
\dok 
\edok




\subsection{Пара форм в  векторном пространстве $V$ (без евклидовой структуры)}

\begin{sled}[Теорема о паре форм]
Пусть в векторном  пространстве $V$ (без  евклидовой или унитарной структуры) заданы 
 две формы $\beta \in \mathcal{B}_{sym}$ и $g \in \mathcal{B}_{sym}$,
причем $g$ положительно определена. Тогда существует базис, в котором
$g$ имеет канонический вид, а $\beta$ имеет диагональный вид (в частности, обе формы 
$\beta$ и $g$ имеют диагональный вид).
\end{sled}
\dok 
Зафиксируем билинейную (полуторалинейную) симметричную форму, которая порождает $g$, как скалярное произведение.
При этом $V$ превратилось в евклидово (унитарное) пространство. Теперь достаточно воспользоваться теоремой \ref{t10_7_1}, поскольку ОНБ --- это базис, в котором $g$ имеет канонический вид (см. следствие 2 предложения \ref{p10_2_2}).
\edok

\otstup
{\bf АЛГОРИТМ} приведения пары форм к диагональному виду.

Пусть $\beta\rsootv{\bazis{e}} B$ и 
$g \rsootv{\bazis{e}} G$. Как и в доказательстве теоремы, считаем 
$V$ евклидовым, где скалярное произведение задается как $(\vek{a}, \vek{b}) = g(\vek{a}, \vek{b})$;
$G$ --- матрица Грама данного базиса $\bazis{e}$. Нам требуется 
найти базис $\bazis{e}'$, для которого $\beta \rsootv{\bazis{e}'} diag$ и 
$g \rsootv{\bazis{e}'} E$. 

Переформулировка (с учетом предложения \ref{{oper_bilin}):
требуется найти ОНБ, в котором самоспряженное преобразование $\varphi$, 
связанное с $\beta$ (так, что $\beta=\beta_{\varphi}$), имеет диагональный вид. 
(Задача сведена к известной).

Равенство $(*)$ расписывается в координатах в базисе $\bazis{e}$ 
как $X^TB\overline{Y} = X^TG (\overline{A}\overline{Y} )$, где 
$\varphi \rsootv{\bazis{e}, \bazis{e}} A$. Отсюда (в силу...)
$B= G \overline{A}$ и $A = \overline{G^{-1}B}$. Далее 
проделываются шаги алгоритма нахождения ОНБ, в котором самосопряженное преобразование имеет диагональный вид.


I. (c.з. для $\varphi$).

$|\overline{G^{-1}B}-\lambda E| =0$, или эквивалентный вид (домножив на $|G|$ и используя, что $\lambda \in \mathbb{R}$), 

$|B-\lambda G| =0$.

После нахождений $\lambda _i$ (с кратностями $s_i$) уже известен диагональный вид 
$diag$.

II. (c.подпр. для $\varphi$).

СЛУ $(\overline{G^{-1}B}-\lambda _i E)X =0$ $\Leftrightarrow$
$(\overline{B-\lambda _i G})X =0$. Поэтому:

$V_{\lambda_i} = \Sol ((\overline{B-\lambda _i G})X =0)X =0)$.

ФСР  = базис $V_{\lambda_i}$.

III. Внутри каждого $V_{\lambda_i}$ 

ортогонализуем базис,

а затем нормируем базис.

 (применяя формулы, помним, что $(\vek{a}, \vek{b}) = X^TG\overline{Y}$,
т.е. матрица Грама = $G$).



%неочевидно, что $\lambda$

\otstup

{\bf Задача.}
Приведите пример пары квадратичных форм, для которых не существует базиса,
в котором они обе имеют диагональный вид.

Указание. Можно подобрать формы, для которых инвариант 
$|F-\lambda G|$ имеет  невещественный корень.
%$\det (F-\lambda G)$ имеет комплексные корни



 %кв. формы
%\input{4_6-short}
%\input{4_7-short}
%\input{4_8-short}
%\input{4_9-short} "`Евклидовы пр-ва"' с не положительно определенным скалярным произведением

\chapter{Сопряженное пространство%(пространство линейных функций)
}\label{lin_funk}

Пусть $V$ --- векторное пространство над $\mathbb{R}$ (или $\mathbb{C}$).
%(Над $\mathbb{C}$)

%ВООБЩЕ ПОГРУЗИТЬ В ТЕОРИЮ БИЛИНЕЙНЫХ ОТОБРАЖЕНИЙ... B(V, V_1, ... )


\section{Связь с общей теорией линейных отображений}

\subsection{Определение}

\defin{
%Пусть $\widetilde{V} = 1$ (т.~е. $\widetilde{V} = \mathbb{R}$ ($ \mathbb{C}$)), тогда
Пространство $\overline{L}(V, \widetilde{V})$, где  $\widetilde{V} = \mathbb{R}$ (или $ \mathbb{C}$)  % $\dim \widetilde{V} = 1$, 
называется {\it сопряженным (или двойственным)} пространством для пространства~$V$.
}

Таким образом, сопряженное пространство --- это частный случай пространства  $\overline{L}(V, \widetilde{V})$, где   $\dim \widetilde{V} = 1$.
Элементы сопряженного пространства --- линейные функции (функционалы), поэтому сопряженное пространство также называют 
{\it пространством линейных функций}. Сопряженное пространство для пространства~$V$ обозначается $V^{*}$.

\begin{predl}\label{p8_4_111}
Если $\dim V = n< \infty$, то $\dim V^{*} = n$.
\end{predl}
\dok Это частный случай следствия из теоремы \ref{p8_3_333} главы 2.
%Если $\dim V = n$, то $\dim V^{*} = \dim V \cdot  \dim \widetilde{V} = n\cdot 1 = n$.
\edok

\otstup

Вся теория о линейных отображениях переносится на частный случай пространства $V^*$.
В частности, если зафиксировать базис $\bazis{e} = (\vek{e}_1, \vek{e}_2, \ldots , \vek{e}_n)$ 
в пространстве $V$ 
то каждая линейная функция $\ell$ получает в соответствие матрицу $1\times n$, то есть строку
$(\ell(\vek{e}_1), \ell(\vek{e}_2), \ldots , \ell(\vek{e}_n))$.  
(Здесь мы считаем, что в пространстве $\widetilde{V} = \mathbb{R}$ (или $ \mathbb{C}$)) зафиксирован базис --- число 1.) 


Если $\vek{a}\in V$ и $\ell \in V^{*}$, то наряду с записью $\ell (\vek{a})$
будем использовать запись $\langle \vek{a}, \ell \rangle$. Эта запись удобна, так как имеется 
линейность по каждому аргументу, т.е. $\forall$ $\vek{a}, \vek{a}_1, \vek{a}_2 \in V$;
$\forall$ $\ell,  \ell _1, \ell _2 \in V^{*}$; $\forall \lambda \in \mathbb{R} (\mathbb{C})$ выполнено:   
$$\langle \vek{a}_1+ \vek{a}_2, \ell \rangle = \langle \vek{a}_1, \ell \rangle + \langle \vek{a}_2, 
\ell \rangle, \,\,\,\,\,\,
\langle \lambda \vek{a}, \ell \rangle = \lambda  \langle \vek{a}, \ell \rangle;$$
$$\langle  \vek{a}, \ell_1+ \ell_2 \rangle =  \langle \vek{a}, \ell_1 \rangle + \langle \vek{a}, \ell_2 \rangle, \,\,\,\,\,\,
\langle \vek{a}, \lambda  \ell \rangle = \overline{\lambda } \langle \vek{a}, \ell \rangle.$$

Иными, словами, скобка $\langle \cdot, \cdot \rangle$ задает билинейную (полуторалинейную) функцию
$V\times V^{*} \to \mathbb{R} (\mathbb{C})$.

\subsection{Взаимный базис}
Везде при работе с базисами и координатами полагаем, что 
$\dim V = n<\infty $. 
Чтобы в дальнейшем была возможность использовать сокращенную тензорную запись суммирования, 
координаты векторов пространства $V$ будем нумеровать верхним индексом,
а для пространства $V^{*}$ --- наоборот, векторы нумеруем верхними индексами, а координаты --- нижними.

\defin{
Базис $\bazis{e}^{*} = (\vek{e}^1, \vek{e}^2, \ldots , \vek{e}^n)$ пространства
$V^{*}$
называется {\it взаимным (или биортогональным, или двойственным)} для базиса
$\bazis{e} = (\vek{e}_1, \vek{e}_2, \ldots , \vek{e}_n)$ пространства
$V$, если  $\langle \vek{e}_i, \vek{e}^j \rangle = \delta_{i}^j$ ($i=1, 2, \ldots, n$, $j=1, 2, \ldots, n$).
%\footnote{Как обычно, $\delta_{ij}=1$ при $i=j$ и $\delta_{ij}=0$ при $i\neq j$.}
}

%Нетрудно понять, как вектор из взимного базиса действует на произвольный вектор пространства $V$.

\begin{predl}\label{p8_4_00}
Пусть $\dim V = n$. Тогда для любого базиса в $V$ существует единственный взаимный базис
в $V^{*}$.
\end{predl}
\dok Для каждого конкретного $j$ вектор $\vek{e}^j\in V^*$ определяется %условиями $\langle \vek{e}_i, \vek{\ell}_j \rangle = \delta_{ij}$
однозначно как линейное отображение $V\to \mathbb{R} (\mathbb{C})$, которому в базисе $\bazis{e}$ соответствует строка $E_j=(0\, 0\, \ldots \, 1\,  0 \, \ldots \, 0)$ (единица на $j$-м месте). 
Строки $E_1, \ldots, E_n$ образуют базис в $\mathbf{M}_{1\times n}$, поэтому 
$\vek{e}^1, \vek{e}^2, \ldots , \vek{e}^n$ образуют базис пространства $V^{*}$.
\edok

\begin{sled}\label{p8_4_00}
Пусть $\dim V = n$ и $\vek{a}\in V$. Если $\forall$ $\ell \in V^{*}$ выполнено
$\langle \vek{a}, \ell \rangle = 0$, то $\vek{a} =  \vek{0}$.
\end{sled}
\dok От противного, если $\vek{a} \neq  \vek{0}$, то $\vek{a}$ можно включить в некоторый базис $\bazis{e}$, 
так что $\vek{a} =  \vek{e}_1$. Тогда, если $\ell = \vek{e}^1$, то
условие $\langle \vek{a}, \ell \rangle = 0$ нарушится.
\edok



\begin{predl}\label{p8_4_0}
Пусть $\dim V = n<\infty$, $\bazis{e}^*$ --- взаимный базис для базиса $\bazis{e}$. Пусть
векторы $\vek{a}\in V$, $\ell \in V^{*}$ разложены по этим базисам:
$\vek{a} = x^i\vek{e}_i$, $\ell = y_j\vek{e}^j$. 
Тогда 
$  \langle \vek{a}, \ell \rangle = x^i\overline{y_i}$.
\end{predl}
\dok
Достаточно раскрыть скобку по линейности и поспользоваться биортогональностью базисов:
$  \langle \vek{a}, \ell \rangle = \langle  x^i\vek{e}_i, y_j\vek{e}^j \rangle  = 
x^i \overline{y_j} \langle  \vek{e}_i, \vek{e}^j \rangle= x^i\overline{y_i}$.
\edok

\otstup

Таким образом,  значение $\langle \vek{a}, \ell \rangle$ равно {\it свертке} $x^i\overline{y_i}$.
В частности, видим, как вектор из взаимного базиса действует на произвольном векторе из $V$:

\begin{sled}\label{s8_4_0}
Пусть $\dim V = n<\infty$, $\bazis{e}^*$ --- взаимный базис для базиса $\bazis{e}$ и 
%векторы $\vek{a}\in V$, $\ell \in V^{*}$ разложены по этим базисам:
$\vek{a} = x^i\vek{e}_i$.
Тогда 
$  \langle \vek{a}, \vek{e}^i \rangle = x^i$.
\end{sled}


\begin{predl}\label{p8_4_1}
Пусть $\dim V = n$.  Пусть $\bazis{e}$ и $\bazis{e}'$ --- базисы в $V$, 
а $\bazis{e}^*$ и $\bazis{e}'^*$ --- их взаимные базисы в $V^{*}$ соответственно.
Пусть $S$ --- матрица перехода от $\bazis{e}$ к $\bazis{e}'$, 
а $C$ --- матрица перехода от $\bazis{e}^*$ к $\bazis{e}'^*$.
Тогда $$C = ({S}^{-1})^*.$$
\end{predl}
\dok  Во встречающихся матрицах перехода обозначаем верхним индексом номер строки, 
а нижним --- номер столбца. Так, если $S=(s^k_i)$ --- матрица перехода от 
$\bazis{e} = (\vek{e}_1, \ldots , \vek{e}_n)$ к 
$\bazis{e}' = (\vek{e}'_1, \ldots , \vek{e}'_n)$, то согласно определению матрицы перехода,
$\vek{e}'_i = \sum\limits_{k=1}^n s^k_i \vek{e}_k$, или используя тензорную запись суммирования, 
$$\vek{e}'_i =  s^k_i \vek{e}_k.$$
Аналогично, если $C=(c^m_j)$ --- матрица перехода от 
$\bazis{e}^* = (\vek{e}^1, \ldots , \vek{e}^n)$ к 
$\bazis{e}'^* = (\vek{e}'^1, \ldots , \vek{e}'^n)$, то 
$\vek{e}'^j = \sum\limits_{m=1}^n c^m_j \vek{e}^m$. Чтобы здесь также 
использовать тензорное суммирование, введем матрицу $R=C^T$, так что $R=(r^j_m)$ и  $c^m_j = r^j_m$.
Тогда 
$$\vek{e}'^j =  r^j_m \vek{e}^m$$
Условие биортогональности запишется как 
$\delta_i^j = \lin{\vek{e}'_i, \vek{e}'^j } =    \lin{s^k_i \vek{e}_k,  r^j_m \vek{e}^m}$.
Раскрывая по линейности и используя биортогональность  $\bazis{e}$ и $\bazis{e}^*$, 
получаем условия $\delta_i^j =  s^k_i \overline{r^j_k}$, равносильные матричному равенству $\overline{R}S = E$, откуда $R = \overline{S}^{-1} = \overline{S^{-1}}$, 
отсюда $C=R^T =({S}^{-1})^*$.
\edok

\otstup

Работа одновременно в $V$ и $V^*$ иногда дает больше, чем работа в одном из них. 
Например, следующее предложение является признаком линейной независимости (при $n=k$ 
признаком базиса в $V$).


\begin{predl}\label{p8_4_2}
Пусть система векторов $\bazis{e} = (\vek{e}_1, \vek{e}_2, \ldots , \vek{e}_k)$ пространства
$V$ и система векторов $\bazis{\ell} = (\vek{\ell}^1, \vek{\ell}^2, \ldots , \vek{\ell}^k)$ пространства
$V^{*}$ %(априори не известно, что системы линейно независимы) 
биортогональны, т.е.  таковы, что
 $\langle \vek{e}_i, \vek{\ell}^j \rangle = \delta_{i}^j$ для всех $i, j\in \{1, \ldots, k\}$.
Тогда $\bazis{e}$ и $\bazis{\ell}$ --- линейно независимые системы в $V$ и $V^*$ соответственно. \\
В частности, если $\dim V = n=k$, то $\bazis{e}$ --- базис в $V$, а $\bazis{\ell}$ --- 
взаимный базис для $\bazis{e}$.
\end{predl}
\dok Достаточно доказать линейную независимость системы векторов $\vek{e}_1, \vek{e}_2, \ldots , \vek{e}_k$
(линейную независимость системы векторов $\vek{\ell}^1, \vek{\ell}^2, \ldots , \vek{\ell}^k$ можно доказать аналогично).\\
Пусть $\lambda^i \vek{e}_i=\vek{0}$. Применив к этому равенству $\vek{\ell}^j\in V^{*}$, имеем
 $\lambda ^j=0$. Таким образом, все коэффициенты в этой линейной комбинации должны быть равны 0.
\edok


\subsection{Примеры}

\example{III.1. (интегральный функционал)
Пусть $V=C[a, b]$, и $f_0 \in V$.
Определим $\widetilde{f_0} \in V^{*}$:
$\forall$ $g\in V$ положим $\langle g, \widetilde{f_0} \rangle = \int\limits_{a}^{b} g(x)f_0(x) \, dx$.
\\
Другой пример: $\delta$-функция --- линейная функция $\delta_{\alpha} \in V^{*}$ такая, что
$\forall$~$g\in V$ выполнено:
$\langle g, \delta_{\alpha} \rangle = g(\alpha)$.
}

\example{III.2.
Пусть $V=\mathbf{P}_n$ (многочлены степени $\leq n$); $\alpha_0, \alpha_1, \ldots, \alpha_n$ --- различные числа.
Для системы линейных функнций $\vek{\ell}^i = \delta_{\alpha _i}$, $i=0, 1, \ldots, n$, биортогональной будет
система многочленов $$p_i(x) = \frac{\prod \limits_{k\neq i} (x-\alpha_k)}{\prod \limits_{k\neq i} (\alpha_i-\alpha_k)},$$
$i=0, 1, \ldots, n$. Из предложения \ref{p8_4_2} следует, что 
$(\vek{\ell}^0, \vek{\ell}^1, \ldots, \vek{\ell}^n)$ --- взаимный базис для базиса $(p_0, p_1, \ldots, p_n)$.\\
Разложение многочлена $f\in \mathbf{P}_n$ по $p_i$ имеет вид 
$$\sum\limits_{i=0}^n f(\alpha_i)\, p_i.$$ 
Мы получили так  называемый {\it интерполяционный многчлен} Лагранжа, дающий явную формулу для многочлена
степени не выше $n$, принимающего в заданных $n+1$ точках предписанные значения.
%% С кратными узлами????
}

\section{Естественные изоморфизмы}

Во многих рассуждениях у нас появлялись изоморфизмы, которые бы изменились при другом выборе базиса
(как, скажем, сопоставление вектору его координатного столбца).
В ситуациях, когда этого не происходит, т.е. если изоморфизм не меняется при замене базиса,
будем говорить, что изоморфизм естественный.

\begin{theor}\label{t8_55}
%Пусть $\dim V = n<\infty$. Тогда 
Отображение, сопоставляющее
вектору $\vek{a}\in V$ отображение $\widetilde{\vek{a}}: V^{*}\to \mathbb{R} (\mathbb{C})$ по правилу
\begin{equation}\label{vv**}
\langle  \ell, \widetilde{\vek{a}} \rangle = \overline {\langle  \vek{a}, \ell \rangle} 
\end{equation}
(для любого $\ell \in V^{*}$), является инъективным гомоморфизмом (вложением) $V\to V^{**}$.
\end{theor}
\dok Для доказательства достаточно проверить следующие утверждения. 

1) $\widetilde{\vek{a}}$, заданное правилом (\ref{vv**}), линейно, т.е. действительно $\widetilde{\vek{a}} \in V^{**}$.

2) Соответствие $\vek{a}\to \widetilde{\vek{a}}$ линейно, т.е. проверить, что 
$\widetilde{\vek{a}_1+\vek{a}_2} = \widetilde{\vek{a}_1} + \widetilde{\vek{a}_2}$ и 
$\widetilde{\lambda \vek{a}} = \lambda \widetilde{\vek{a}}$.

3) Тривиальность ядра отображения $\vek{a}\to \widetilde{\vek{a}}$. Если $\widetilde{\vek{a}} = \vek{0}$ (то есть для любого $\ell \in V^{*}$ выполнено
$\langle  \ell, \widetilde{\vek{a}} \rangle  = 0$, то $\vek{a}=\vek{0}$.
\edok


\begin{sled}\label{s8_55}
В случае $\dim V = n<\infty$ указанное соответствие $\vek{a}\to \widetilde{\vek{a}}$ является естественным изоморфизмом
$V\to V^{**}$.
\end{sled}

Соответствие $\vek{a}\to \widetilde{\vek{a}}$ позволяет отождествить далее конечномерное пространство со своим дважды
сопряженным. 

\begin{theor}\label{t8_56}
%Пусть $\dim V = n<\infty$. Тогда 
Пусть $\mathcal{E}$ --- евклидово (унитарное) пространство.
Отображение $\mathcal{E}\to \mathcal{E}^{*}$, сопоставляющее
вектору $\vek{a}\in \mathcal{E}$ отображение $\widehat{\vek{a}}: V^{*}\to \mathbb{R} (\mathbb{C})$
по правилу
\begin{equation}\label{ee*}
\langle  \vek{x}, \widehat{\vek{a}} \rangle = (\vek{x}, \vek{a})
\end{equation}
(для любого $\vek{x} \in \mathcal{E}$), является инъективным гомоморфизмом (вложением) $\mathcal{E}
\to \mathcal{E}^{*}$.
\end{theor}
\dok Для доказательства достаточно проверить следующие утверждения. 

1) $\widehat{\vek{a}}$, заданное правилом \ref{ee*}, линейно, т.е. действительно $\widehat{\vek{a}} \in \mathcal{E}^{*}$.

2) Соответствие $\vek{a}\to \widehat{\vek{a}}$ линейно, т.е. проверить, что 
$\widehat{\vek{a}_1+\vek{a}_2} = \widehat{\vek{a}_1} + \widehat{\vek{a}_2}$ и 
$\widehat{\lambda \vek{a}} = \lambda \widehat{\vek{a}}$.

3) Тривиальность ядра отображения $\vek{a}\to \widehat{\vek{a}}$. 
Если $\widehat{\vek{a}} = \vek{0}$ (то есть для любого $\vek{x} \in V$ выполнено
$\langle  \vek{x}, \widehat{\vek{a}} \rangle  = 0)$, то $\vek{a}=\vek{0}$.
\edok


\begin{sled}\label{s8_56}
В случае $\dim V = n<\infty$ указанное соответствие $\vek{a}\to \widehat{\vek{a}}$ является естественным изоморфизмом
$\mathcal{E}\to \mathcal{E}^{*}$.
\end{sled}

Соответствие $\vek{a}\to \widehat{\vek{a}}$ позволяет отождествить далее конечномерное евклидово (унитарное) 
пространство со своим сопряженным. 
В этом смысле теорию евклидовых (унитарных) пространств можно считать частным случаем 
теории двойственности.



\section{Биортогональность. Соответствие между подпространствами в $V$ и $V^*$}

\subsection{Биортогональность. }

\defin{Множества $U\subset V$ и $W\subset V^*$ называются {\it биортогональными}, если 
$\forall \, \vek{a}\in U$ и $\forall \, \ell\in W$ выполнено $\langle \vek{a}, \ell \rangle   = 0$.
}

Для биортогональности векторов и множеств из $V$ и $V^*$ сохраним обозначение $\perp$.

%Пусть дано $U\leq V$.

\begin{predl}[признак биортогональности]\label{p10_2_3} 
Пусть $U= \lin{\vek{a}_1, \ldots, \vek{a}_k}$ $\leq V$, $W= \lin{\ell_1, \ldots, \ell_l}$ $\leq V^*$.
Тогда  $U \perp W$
$\Leftrightarrow$ $\vek{a}_i\perp \ell_j$, $i=1, 2, \ldots, k$, $j=1, 2, \ldots, l$.
\end{predl}
\dok Аналогично доказательству предложения \ref{p10_2_3}.
%\dokright Очевидно из определения ортогональности подпространств.\\
%\dokleft Пусть $\vek{a}\in U_1$, $\vek{b}\in U_2$. Тогда существуют разложения
% $\vek{a} = \sum\limits_{i=1}^k \alpha _i \vek{a}_i$,  
%$\vek{b} = \sum\limits_{j=1}^l \beta _j \vek{b}_j$. Тогда 
%из линейности скалярного произведения получаем $(\vek{a}, \vek{b}) = \sum\limits_{i=1}^k \sum\limits_{j=1}^l  %\alpha _i \overline{\beta _j} (\vek{a}_i, \vek{b}_j)= 0$, %
%т.е. $\vek{a} \perp \vek{b}$.
\edok


%%%%%%%%%%%%%%%%%%%%
%%%%%%%%%%%%%%%%%%%%
%%%%%%%%%%%%%%%%%%%

\subsection{Биортогональное дополнение}


\defin{Подмножество  $W = \{\ell \in V^* \, |\, U\perp \ell\} $ пространства $V^{*}$
называется {\it аннулятором}, или биортогональным дополнением подпространства $U$.
}

Для евклидова пространства отождествление (\ref{ee*}) превращает определение аннулятора в определение
ортогонального дополнения.
Для аннулятора примем обозначение $U^{\perp}$.

\begin{predl}\label{t8_56}
Для любого $U\leq V$ подмножество $U^{\perp}$ является подпространтвом в $V^*$.
\end{predl}
\dok
\edok

\otstup

Для любого $W\leq V^*$ подмножество $W^{\perp}$ будет является подпространтвом в $V^{**}$. 
Но в конечномерном случае в силу отождествления  $V\cong V^{**}$ заданного (\ref{vv**}), мы можем отождествить
$W^{\perp}$ с подпространтвом в $V$, которое также называется {\it нуль-пространством } подпространства $W$.
Зафиксируем также определение нуль-пространства, не использующее отождествление  $V\cong V^{**}$.

\defin{Подмножество  $U = \{\vek{a} \in V \, |\, \vek{a}\perp W\} $ пространства $V$
называется {\it нуль-пространством} подпространства $W\leq V^*$.
}
 
Для нуль-пространства сохраняем обозначение $W^{\perp}$.

\begin{predl}\label{t8_56}
Для любого $W\leq V^*$ подмножество $W^{\perp}$ является подпространтвом в $V$.
\end{predl}
\dok
\edok

%%%%%%%%%%%%%%%%%%
%%%%%%%%%%%%%%%%%%
%%%%%%%%%%%%%%%%%%%


\begin{theor}
Пусть  $\dim V=n< \infty$,  $U\leq V$. Тогда \\
1) $(U^{\bot})^{\bot} = U$;
2)  $\dim U + \dim U^{\bot} = n$.
\end{theor}
\dok Пусть $\vek{e}_1, \vek{e}_2, \ldots , \vek{e}_k$ --- базиc в $U$. Дополним его до 
базиса $\bazis{e} = (\vek{e}_1,  \ldots , \vek{e}_k ,  \ldots , \vek{e}_n)$ в $V$.
Мы покажем, что для $U^{\perp}$ имеется следующее описание, из которого вытеают оба утверждения:
$U^{\perp} = \lin{\vek{e}^{k+1}, \vek{e}^{k+2},  \ldots , \vek{e}^n}$, где 
$\vek{e}^{1},  \ldots , \vek{e}^n$ --- взаимный базис для баззиса $\vek{e}_{1},  \ldots , \vek{e}_n$  пространства $V$.

\ldots
\edok


{\bf Упражнение.}
%Для подпространств конечномерного евклидова пространства: \\
Пусть $\dim V<\infty$. 
Для $U_i\leq V$, $i=1, 2$, докажите, что $(U_1+U_2)^{\bot} = U_1^{\bot} \cap U_2^{\bot}$
и аналогично, $(U_1\cap U_2)^{\bot} = U_1^{\bot} + U_2^{\bot}$.




\section{Сопряженное преобразование}

%Рассмотрим линейное преобразовние $\ell : \mathcal{E}\to \mathcal{E}$.

\defin{
Преобразование $\psi : V^*\to V^*$
называют {\it сопряженным} преобразованию $\varphi \in L(V,V)$, если $\forall$ 
$\vek{a} \in V$, и  $\forall$ 
$\ell \in V^*$ выполнено
\begin{equation}\label{soprv*}
\boxed{\langle \varphi(\vek{a}), \ell\rangle= \langle \vek{a}, \psi(\ell) \rangle}.
\end{equation}
}

Обозначение для сопряженного преобразования: $\varphi^{*}$. 
Для евклидова пространстве, в силу отождествления (\ref{ee*}), это определения согласуется 
с определением из главы 5.
Из  определения $\varphi^{*}$ однозначно определено. Проверим, что оно линейно. 

\begin{predl}\label{p8_9_0}
Пусть $\varphi \in L(V, V)$. Тогда $\varphi^* \in L(V^*, V^*)$. 
\end{predl}
\dok
\edok


\begin{predl}\label{p8_9_1}
Пусть $\dim V<\infty $, $\bazis{e}$ --- базис в $V$, и $\bazis{e}^*$ --- его 
взаимный базис в $V^*$. Пусть $\varphi \in L(V, V)$, 
 $\varphi \rsootv{\bazis{e}, \bazis{e}} A$. 
$\varphi ^* \rsootv{\bazis{e}^*, \bazis{e}^*} A^{*}$. 
Тогда
\end{predl}
\dok
Аналогично предложению \ref{p10_3_1}. 
\edok


\begin{sled1}
Пусть $\dim V <\infty $, $\varphi, \psi \in L(V, V)$. Тогда \\
1) $(\varphi ^{*})^{*} = \varphi$;\\
2) $(\varphi \psi)^{*} = \psi^{*} \varphi^{*}$;\\
3) $\rg (\varphi) = \rg (\varphi ^{*})$;\\
4) $\overline{\chi_{\varphi} (\lambda)} = \chi_{\varphi ^{*}} (\overline{\lambda})$.
\end{sled1}
\dok Введем в $V$ и $V^*$ взаимные базисы и перейдем к матрицам в этих базисах. 
Далее доказываем так же, как следствие 2 из предложения \ref{p10_3_1} главы \ref{evkl_prostr}.
\edok

\begin{theor}\label{t8_9_1} 
Пусть $U\leq V$, $\dim V <\infty$, $\varphi \in L(V, V)$. Тогда  \\
$U$ инвариантно относительно $\varphi$ 
$\Leftrightarrow$ 
$U^{\bot}$ инвариантно относительно $\varphi ^{*}$.
\end{theor}
\dok 
Аналогично теореме \ref{t10_3_1} главы \ref{evkl_prostr}.
\edok

\begin{theor}[Теорема Фредгольма]\label{t8_9_2} 
Пусть $\dim V=n <\infty$, $\varphi \in L(V, V)$. Тогда  \\
$$\boxed{\Ker \varphi ^{*} = (\Im \varphi)^{\bot }}.$$
\end{theor}
\dok  
Аналогично теореме \ref{t10_3_2} главы \ref{evkl_prostr}.
\edok


\otstup
Отметим, что все содержание этого параграфа обобщается на случай 
линейного отображения $\varphi \in L(V, \widetilde{V})$.
Для $\varphi \in L(V, \widetilde{V})$ так же формула (\ref{soprv*}) позволяет
определить {\it сопряженное отображение} $\varphi^* \in L(\widetilde{V}^*, V^*)$,
для которого выполняются аналоги всех предложений и теорем из этого параграфа.



\chapter{Тензоры
}%\label{lin_funk}

В этой главе рассматриваем векторные пространства над полем $\mathbb{F}$.
(где возникает деление (симметризация и пр.) предполагаем, что характеристика поля равна 0).

%(Над $\mathbb{C}$)

%ВООБЩЕ ПОГРУЗИТЬ В ТЕОРИЮ БИЛИНЕЙНЫХ ОТОБРАЖЕНИЙ... B(V, V_1, ... )


\section{Полилинейные отображения}

\subsection{Определение и координантая запись}

%\defin{
%Полилинейным отображением...
%}

Отображение $\beta : V_1\times V_2 \times \ldots \times V_k \to \mathbb{F}$
{\it полилинейно}, если $\beta (\vek{a}_1, \vek{a}_2, \ldots, \vek{a}_k)$ линейно по каждому из аргументов.

\otstup

Множество всех полилинейных отображений обозначим $\Hom(V_1, V_2, \ldots , V_k; \mathbb{F})$.

\otstup

Примеры. В старых обозначениях для $V$ над $\mathbb{R}$:\\
 $\Hom(V, V; \mathbb{R}) = \mathcal{B}(V)$;\\
$\Hom(V; \mathbb{R}) = V^*$;\\
$\Hom(V^*; \mathbb{R}) = V$ (с учетом отождествления $V$ и $V^{**}$).\\


Пусть $\dim V_t  = n_t$ и в каждом $V_t$ зафиксирован базис:
в $V_1$ --- базис $\bazis{e} = (\vek{e}_1, \ldots, \vek{e}_{n_1})$,
в $V_2$ --- базис $\bazis{f} = (\vek{f}_1, \ldots, \vek{f}_{n_2})$, и т.д.

Сопоставим каждому $\beta \in \Hom(V_1, V_2, \ldots , V_k; \mathbb{F})$ массив 
из $n_1\ldots n_k$ констант $b_{i_1 i_2 \ldots i_k}$, $i_t\in \{1, \ldots, n_t\}$
(далее для упрощения обозначений пишем $b_{ij\ldots }$):
\begin{equation}\label{b_{ij..}}
\beta \rsootv{\bazis{e}, \bazis{f}, \ldots } b_{ij \ldots },  
\end{equation}
где $\boxed{b_{i j \ldots } = \beta (\vek{e}_{i}, \vek{f}_{j}, \ldots )}$.

$b_{i j \ldots }$ называются {\it компонентами} полилинейного отображения в базисах $\bazis{e}, \bazis{f}, \ldots $. 

\begin{theor}[координатная запись]\label{t20_1_1}
Пусть $\beta \in \Hom(V_1, V_2, \ldots , V_k; \mathbb{F})$ и 
$\beta \rsootv{\bazis{e}, \bazis{f}, \ldots } b_{i j \ldots }$. 
Пусть векторы $\vek{a} \in V_1$, $\vek{b} \in V_2, \ldots, $ разложены по базисам:
$\vek{a} = x^i \vek{e}_i $, $\vek{b} = y^j \vek{f}_j $, \ldots
Тогда 
\begin{equation}\label{beta()}
\boxed{ \beta (\vek{a}, \vek{b}, \ldots ) = b_{i j \ldots } x^iy^j\ldots } .
\end{equation}
\end{theor}
\dok Раскроем $\beta (\vek{a}, \vek{b}, \ldots)  $, ....
\edok

 (\ref{b_{ij..}}) задает взаимно-однозначное соответствие 
 $\Hom(V_1, V_2, \ldots , V_k; \mathbb{F}) \to \mathbb{F} ^{n_1n_2\ldots n_k}$



\subsection{Замена базиса}
 

Во встречающихся матрицах перехода обозначаем верхним индексом номер строки, 
а нижним --- номер столбца, так если $S=(s^i_j)$ --- матрица перехода от базиса
$(\vek{e}_1, \ldots )$ к базису $(\vek{e}'_1, \ldots )$, то замена базиса
происходит по правилу $$\vek{e}'_j = s^i_j \vek{e}_i,$$
а, скажем, замена координат вектора --- 
по правилу $$x^i = s^i_j x'^j.$$


\begin{theor}\label{t20_1_2}
Пусть в $V_1$ выбраны базисы $\bazis{e}$ и $\bazis{e}'$, связанные матрицей перехода $s^i_j$, 
в $V_2$ выбраны базисы $\bazis{f}$ и $\bazis{f}'$, связанные матрицей перехода $t^i_j$, и т.д.
Пусть $\beta \in T(V_1, V_2, \ldots , V_k)$ так, что 
$$\beta \rsootv{\bazis{e}, \bazis{f}, \ldots } b_{i j \ldots },$$ 
$$\beta \rsootv{\bazis{e}', \bazis{f}', \ldots } b'_{i j \ldots }$$
Тогда  
\begin{equation}\label{zamena_tenz}
\boxed{b'_{i j \ldots }  = b_{k l \ldots } s^k_i t^l_j \ldots  }.
\end{equation}
\end{theor}
%\dok Пусть $\vek{a} \in V_1, \vek{b} \in V_2, \ldots $ --- произвольные векторы. 
%Пусть $\vek{a}=x^i\vek{e}_i = x'^i\vek{e}'_i$, 
%$\vek{b}=\bazis{e}Y = \bazis{e}'Y'$, и т.д.
%Тогда по теореме \ref{t9_1_1}  имеем $\beta (\vek{a}, \vek{b}) = X^T B \overline{Y}$ и $\beta (\vek{a}, \vek{b}) = X'^T B \overline{Y'}$
%Подставляя  $X=SX'$, $Y=SY'$ (см. теорему \ref{t7_3_2}, глава \ref{lin_prostr}),
% имеем $\beta (\vek{a}, \vek{b}) = (SX')^T B \overline{SY'} = X'^T S^T B \overline{S} \overline{Y'} = X'^T (S^T B \overline{S}) \overline{Y'} $. 
%В силу предложения \ref{p9_1_2} получаем требуемое: $B'=S^TB\overline{S}$.
%\edok

(Координатное определение тензора)


\subsection{Линейные операции}


$\Hom(V_1, V_2, \ldots , V_k; \mathbb{F})$ --- векторное пространство над $\mathbb{F}$. 




\section{Тензоры и основные операции над ними}

%(опуская <<совсем правильное>> определение, ...)

\subsection{Тензор как полилинейная функция}

\defin{
Тензором типа $(p,q)$ над векторным пространством $V$ называют
полилинейное отображение из множества
$\Hom( \underbrace{V^*, \ldots , V^*}_{p}, \underbrace{V, \ldots , V}_{q}; \mathbb{F})$.
}


Более заумное название тензора типа $(p,q)$: $p$ раз контравариантный и $q$ раз ковариантный.

Обозначение (помимо $\Hom( \underbrace{V^*, \ldots , V^*}_{p}, \underbrace{V, \ldots , V}_{q}; \mathbb{F})$):
$$T^p_q(V).$$ 
(или просто $T^p_q$, если ясно, о каком $V$ идет речь).

Для $\tau \in T^p_q(V)$ определено значение $\tau (\underbrace{\ell, \ldots }_{p}, \underbrace{\vek{a}, 
\vek{b}, \ldots }_{q})$ на упорядоченном наборе из $p$ линейных функционалах (ковекторах) и $q$ векторах.

$T^p_q(V)$ --- векторное пространство. \\
Что такое $T^0_0(V)$?

Выписывая компоненты в базисах, условимся во всех $q$ копиях $V$ выбирать один и тот же базис 
$\bazis{e} = (\vek{e}_1, \ldots, \vek{e}_n)$, а во всех 
$p$ копиях $V^*$ выбирать базис
$\bazis{e}^* = (\vek{e}^1, \ldots, \vek{e}^n)$, биортогональный базису $\bazis{e}$.
Компоненты пишем с $p$ верхними и $q$ нижними индексами, для возможности использовать тензорное суммирование.
 
Для $\tau \in T^p_q(V)$ соответствие (\ref{b_{ij..}}) приобретает вид:

$\tau \, \, \rsootv{\bazis{e}^*, \ldots , \bazis{e}, \bazis{e}, \ldots } \, \,  
t_{i_1i_2 \ldots i_q}^{j_1j_2\ldots j_p},  $
или (сокращая обозначения)
$$\tau \, \, \rsootv{\bazis{e}} \, \,  
t_{ik \ldots }^{j\ldots }.  $$

Массив из $n^{p+q}$ констант $t_{ik \ldots }^{j\ldots } = \tau 
(\underbrace{\vek{e}^j, \ldots  }_q, \underbrace{\vek{e}_i, \vek{e}_k, \ldots }_p, )$ --- 
компоненты тензора $\tau $  в базисе $\bazis{e}$.

Теперь формула (\ref{beta()}) приобретает вид
$$\tau (\underbrace{\ell, \ldots }_{p}, \underbrace{\vek{a}, 
\vek{b}, \ldots }_{q})  = t_{ik \ldots }^{j\ldots } x^iy^k \ldots z_j\ldots , $$
где $\vek{a} = x^i\vek{e}_i$, $\vek{b} = y^k\vek{e}_k$, \ldots , $\ell = z_j\vek{e}^j$, \ldots ---
разложения векторов ($p+q$ аргументов, к которым применяется $\tau$)  по базисам $\bazis{e}$ и 
$\bazis{e}^*$


Формула (\ref{zamena_tenz}) изменения компонент при переходе от базиса $\bazis{e}$ к 
$\bazis{e}'$ и соответственно от базиса $\bazis{e}^*$ к 
$\bazis{e}'^*$:

$$
\boxed{
{t'}_{ik \ldots }^{j\ldots }  = t_{i'k' \ldots }^{j'\ldots }  \, \,  s^{i'}_i s^{k'}_k\ldots r^j_{j'} \ldots  
},
$$
где (как, скажем, и в доказательства предложения \ref{p8_4_1} главы \ref{lin_funk}),
$S=(s^{i'}_i)$ --- матрица перехода от $\bazis{e}$ к 
$\bazis{e}'$, а $R=(r^j_{j'}) = S^{-1}$ --- транспонированная к матрице перехода от 
$\bazis{e}^*$ к $\bazis{e}'^*$ (транспонирование связано с нашей договоренностью о
верхних и нижних индексах при работе в $V^*$).




\subsection{Тензорное умножение}

\defin{
Тензорным произведением тензора  $\alpha \in T^p_q$ на 
тензор  $\beta \in T^{p'}_{q'}$
называют тензор $\gamma \in T^{p+p'}_{q+q'}$ такой, что
$$\gamma (\underbrace{\ell, \ldots }_{p}, \underbrace{m, \ldots }_{p'}, 
\underbrace{\vek{a}, \vek{b}, \ldots }_{q}, 
\underbrace{\vek{c}, \vek{d}, \ldots }_{q'}) \,  =      \, 
 \alpha (\underbrace{\ell, \ldots }_{p}, 
\underbrace{\vek{a}, \vek{b}, \ldots }_{q}) \, \cdot \, 
\beta (\underbrace{m, \ldots }_{p'}, \underbrace{\vek{c}, 
\vek{d}, \ldots }_{q'}).$$
}

Обозначение: $\alpha \otimes \beta$.

Ассоциативность, дистрибутивность относительно сложения. 

Нет коммутативности.

<<организуем>> алгебру
$$\bigoplus\limits_{q=0}^{\infty} T^0_q$$
%$$\bigoplus\limits_{p=0}^{\infty} T^p_0$$

%Тензорная %(градуированная) 
%алгебра $T(V)$%(ассоциативная 
$$\bigoplus\limits_{p, q} T^p_q .$$





В координатах --- <<независимое умножение>>. Пусть
$$\alpha \, \, \rsootv{\bazis{e}} \, \,  
a_{i_1i_2 \ldots i_q}^{j_1j_2\ldots j_p},  \,\,\, $$
$$\beta \, \, \rsootv{\bazis{e}} \, \,  
b_{i'_1i'_2 \ldots i'_{q'}}^{j'_1j'_2\ldots j'_{p'}}. $$
Тогда 
$$\gamma \, \, \rsootv{\bazis{e}} \, \,  
c_{i_1i_2 \ldots i_q\,  i'_1i'_2 \ldots i'_{q'}}^{j_1j_2\ldots j_p\, j'_1j'_2\ldots j'_{p'}} \,\,\, =
\,\,\, a_{i_1i_2 \ldots i_q}^{j_1j_2\ldots j_p}\,\, \cdot \,\, 
b_{i'_1i'_2 \ldots i'_{q'}}^{j'_1j'_2\ldots j'_{p'}}. $$

Разложимый тензор --- тензор, равный произведению тензоров валентности 1 (векторов и ковекторов).

\example{ Как действует тензор $\vek{e}^3 \otimes \vek{e}^1 \otimes \vek{e}^3 \in T^0_3$?\\
$(\vek{e}^3 \otimes \vek{e}^1 \otimes \vek{e}^3 )(\vek{e}_i, \vek{e}_j, \vek{e}_k) = 
\delta _i^3 \delta _j^1 \delta _k^3$.\\
$(\vek{e}^3 \otimes \vek{e}^1 \otimes \vek{e}^3 )(\vek{a}, \vek{b}, \vek{c}) = x^3y^1z^3$, 
где $ \vek{a} = x^i\vek{e}_i$, $ \vek{b} = y^j\vek{e}_j$, $ \vek{c} = z^k\vek{e}_k$. 
}

\otstup

Тензоры вида  $\underbrace{\vek{e}_j \otimes \ldots }_p\otimes 
\underbrace{\vek{e}^i \otimes \vek{e}^k \otimes \ldots }_q$
--- <<стандартный>> (т.е. согласованный с $\bazis{e}$) базис в $T^p_q$,
так что 
$$\tau  = t^{j\ldots}_{ik\ldots } \,\,
\underbrace{\vek{e}_j \otimes \ldots }_p\otimes 
\underbrace{\vek{e}^i \otimes \vek{e}^k \otimes \ldots }_q, $$
где 
$\tau \, \, \rsootv{\bazis{e}} \, \,  
t_{ik \ldots }^{j\ldots }.  $



\subsection{Свертка}

Зафиксируем некоторый базис и зададим {\it свертку }  (по паре первых аргументов) как линейное отображение 

$$T^p_q\to T^{p-1}_{q-1}$$

по следующему правилу. Достаточно определить значение свертки вначале на 
<<стандартном базисе>> $T^p_q(V)$, связанным с базисом $\bazis{e}$ пространства $V$
(далее продолжается по линейности):

\begin{equation}\label{svert}
\underbrace{\vek{e}_j \otimes \vek{e}_l \otimes \ldots }_p\otimes 
\underbrace{\vek{e}^i \otimes \vek{e}^k \otimes \ldots }_q
\mapsto \delta_j^i  \, \, \underbrace{\vek{e}_l \otimes \ldots }_{p-1}\otimes 
\underbrace{\vek{e}^k \otimes \ldots}_{q-1}.
\end{equation}


Из линейности легко следует, что 

$$\underbrace{ \vek{a} \otimes \vek{b} \otimes \ldots }_p \otimes 
\underbrace{\ell \otimes m \otimes \ldots }_q
\mapsto \lin{\vek{a}, \ell} \,\,
\underbrace{\vek{b} \otimes \ldots }_{p-1}\otimes \underbrace{m \otimes \ldots}_{q-1}.$$

В частности, это означает, что (\ref{svert}) работает и для
другого выбора базиса. 

В координатах переход к свертке выглядит просто: 
$$t_{ik\ldots }^{jl\ldots } \mapsto t_{ik\ldots }^{il\ldots }.$$


\subsection{Примеры.}


-- Значение  линейной формы на векторе: 
$\vek{a} \otimes \ell$ --- далее свертка (в результате $l_{i} x^i $).

\otstup

-- Значение  билинейной формы
$\vek{a} \otimes \vek{b} \otimes \beta  $ --- далее свертка (в результате $b_{ij} x^i y^j$)

\otstup

---Матрица $A=(a^i_j)$ линейного отображения $\varphi : V\to V$ --- 
тензор из $T^1_1$.\\
$\tr A  = a^i_i$ --- свертка.\\
Матрица произведения двух отображений:
$a^i_j b^j_k$.

\otstup

%Умноженивк в алгебре (свертка со структурным тензором)

--- В евклидовом пространстве $g_{ij}$ --- метрический Тензор (матрица Грама).\\
Тензорное домножение на $g_{ij}$ с последующей сверткой по одному из аргументов --- <<опускание индекса>>.\\
Например: $b_{jk} = a^i_j g_{ik}$.\\
(связь между операторами и билинейными формами в еквлидовом пространстве).



\section{Симметричные и кососимметричные тензоры. }


\subsection{Перестановка аргументов в функциях нескольких переменных.}

%Определим эти операции для $T^0_q$ (можно обобщить для $T^p_q$, выполняя операции отдельно с верхними и нижними индексами).

Если $X$ и $Y$ --- произвольные множества, то  
на множестве $\mathcal{F}$ отображений $\underbrace{X\times X\times \ldots \times X}_q \, \to \, Y$ 
естественным образом действует группа перестановок $S_q = S_{\{1, 2, \ldots, q\}}$, так что 
$\sigma \in S_q$ переставляет аргументы отображения $f\in \mathcal{F}$:
% и перестановки  определена функция 
%$\sigma f \in \mathcal{F} $ по правилу 
$$(\sigma f) (x_1, \ldots , x_q) = f (x_{\sigma (1)}, \ldots , x_{\sigma (q)}).$$

Очевидно, $f\mapsto \sigma f$ задает биекцию $\mathcal{F} \to \mathcal{F}$.
Далее считаем, что $Y$ --- векторное пространство (над некоторым полем $\mathbb{F}$), тогда $f\mapsto \sigma f$ задает изоморфизм $\mathcal{F} \to \mathcal{F}$.


\subsection{Симметрические функции. Симметризация.}

\defin{
Функция $ f \in \mathcal{F} $ называется {\it симметрической}, если 
 $\forall$ $\sigma \in S_q$ выполнено $$\sigma f =f .$$
}

Подмножество всех симметрических $f\in \mathcal{F}$ можно обозначить $\mathcal{F}^+$.

Можно определять более общее понятие --- симметричность относительно 
действия заданной подгруппы в $H\leq S_q$,
заменяя в определении симметричности условие $\forall$ $\sigma \in S_q$ на $\forall$ $\sigma \in H$.
Например, если 
$H=S_{\{1, 2, \ldots, p\}}$, получаем условие симметричности относительно
первых $p$ аргументов
(и аналогичнно для $H=S_{\{p+1, p+2, \ldots, q\}}$ --- условие симметричности относительно
последних $q-p$ аргументов).
%$H$ --- симметричность относительно четных перестановок




 
С помощью действия $S_q$ на $\mathcal{F} $ (в случае поля нулевой характеристики) 
можем определить {\it усреднение}, или {\it симметризацию} как
$$\Sym (f) = \dfrac{1}{q!} \sum\limits_{\sigma \in S_q} \sigma f , $$
или более общо, симметризацию относительно действия заданной подгруппы в $H\leq S_q$
как $$\Sym_H(f) = \dfrac{1}{|H|} \sum\limits_{\sigma \in H} \sigma f .$$

Следующее предложение показывает, что  $\Sym : \mathcal{F} \to \mathcal{F}$ является проектором на 
$\mathcal{F}^+$.

\begin{predl}
1) $\forall$ $f \in \mathcal{F} $ выполнено $\Sym (f) \in \mathcal{F}^+ $;\\
2) если $f \in \mathcal{F}^+ $, то $\Sym (f)=f$.
\end{predl}



\begin{lem}\label{premix}
Пусть $\sigma_0 \in H \leq S_q$ ($H$ --- подгруппа в  $S_q$). Тогда
$\forall$ $f \in \mathcal{F} $ выполнено
$$\Sym_H(\sigma_0f) = \Sym_H (f) .$$
\end{lem}


%лемма о повторном размешивании...
 
\begin{lem}\label{double_mix}
Пусть $H\leq G\leq S_q$ (две подгруппы $S_q$, одна вложена в другую). Тогда
$\forall$ $f \in \mathcal{F} $ выполнено
$$\Sym_G (\Sym_H(f)) = \Sym_G (f) .$$
\end{lem}



\subsection{Кососимметрические функции. Альтернирование.}

Вспомним, что каждой перестановке приписан {\it знак} $\varepsilon(\sigma)$, 
равный $\pm 1$ (в зависимости от четности $\sigma$).

\defin{
Функция $ f \in \mathcal{F} $ называется {\it кососимметрической}, если 
 $\forall$ $\sigma \in S_q$ выполнено $$\sigma f = \varepsilon(\sigma) \cdot f .$$
}

Подмножество всех кососимметрических $f\in \mathcal{F}$ можно обозначить $\mathcal{F}^-$.

{\bf Упражнение}. Докажите, что  $\mathcal{F}^+ \oplus \mathcal{F}^-$ --- прямая сумма.
Однако, равенство $\mathcal{F} = \mathcal{F}^+ \oplus \mathcal{F}^-$, верное в случае функций от $q=2$ переменных вообще говоря, перестает быть верным при $q>2$.


%Можно определять более общее понятие --- симметричность относительно 
%действия заданной подгруппы в $H\leq S_q$,
%заменяя в определении симметричности условие $\forall$ $\sigma \in S_q$ на $\forall$ $\sigma \in H$.
%Например, если 
%$H=S_{\{1, 2, \ldots, p\}}$, получаем условие симметричности относительно
%первых $p$ аргументов
%(и аналогичнно для $H=S_{\{p+1, p+2, \ldots, q\}}$ --- условие симметричности относительно
%последних $q-p$ аргументов).
%%$H$ --- симметричность относительно четных перестановок

 
Аналогично симметризации можно определить <<усреднение со знаком>>, или {\it альтернирование} как
$$\Alt (f) = \dfrac{1}{q!} \sum\limits_{\sigma \in S_q} \varepsilon(\sigma) \cdot \sigma f .$$
Более общо, %симметризацию относительно действия заданной 
для подгруппы в $H\leq S_q$
$$\Alt_H(f) = \dfrac{1}{|H|} \sum\limits_{\sigma \in H} \varepsilon(\sigma) \cdot \sigma f .$$
Отметим, что если $H$ состоит только из четных перестановок, то $\Alt_H(f) = \Sym_H(f)$.



Следующее предложение покавается,  $\Sym : \mathcal{F} \to \mathcal{F}$ является проектором на 
$\mathcal{F}^-$.

\begin{predl}
1) $\forall$ $f \in \mathcal{F} $ выполнено $\Alt (f) \in \mathcal{F}^- $;\\
2) если $f \in \mathcal{F}^- $, то $\Alt (f)=f$.
\end{predl}



\begin{lem}\label{premix_alt}
Пусть $\sigma_0 \in H \leq S_q$ ($H$ --- подгруппа в  $S_q$). Тогда
$\forall$ $f \in \mathcal{F} $ выполнено
$$\Alt_H(\sigma_0f) = \varepsilon (\sigma_0)\cdot \Alt_H (f) .$$
\end{lem}


%лемма о повторном размешивании...
 
\begin{lem}\label{double_mix_alt}
Пусть $H\leq G\leq S_q$ (две подгруппы $S_q$, одна вложена в другую). Тогда
$\forall$ $f \in \mathcal{F} $ выполнено
$$\Alt_G (\Alt_H(f)) = \Alt_G (f) .$$
\end{lem}




\subsection{Применение к тензорам.}


Так как тензоры из являются функциями 
$\underbrace{V\times V\times \ldots \times V}_q \, \to \, \mathbb{F}$,  
согласно общей излагаемой теории, на пространстве $T^0_q = T^0_q (V) $ определяется изоморфизм $T^0_q\to T^0_q$
перестановки аргументов $\tau \mapsto \sigma \tau$.
Далее определяются линейные проекторы $T^0_q\to T^0_q$ симметризация $\tau \mapsto \Sym ( \tau)$ и
 альтернирование $\tau \mapsto \Alt ( \tau)$.

Множества симметрических и кососимметрических тензоров
обозначаем (наряду c $T^0_q (V)^+ $ и $T^0_q (V)^- $) через $S^0_q (V) $ и $\Lambda^0_q (V) $ соответственно.
Ясно, что $S^0_q (V) $ и $\Lambda^0_q (V) $ ---  подпространства в $T^0_q (V) $. 



В координтанатах: пусть 
$\tau \, \, \rsootv{\bazis{e}} \, \,  
t_{ik \ldots }.   $
Тогда 
$$\sigma \tau \, \, \rsootv{\bazis{e}} \, \,  
t_{\sigma(i) \sigma(k) \ldots }.    $$
(Простой пример: транспонирование матрицы билинейной формы: $b_{ij} \mapsto b_{ji}$.)

Для обозначения компонент тензора после симметризации и альтернирования применяются иногда 
круглые и квадратные скобки. Так, 
$$\Sym(\tau ) \, \, \rsootv{\bazis{e}} \, \,  
t_{(ik \ldots )}  = \dfrac{1}{q!}\sum\limits_{\sigma \in S_q} t_{\sigma(i) \sigma(k) \ldots };   $$
$$\Alt(\tau ) \, \, \rsootv{\bazis{e}} \, \,  
t_{[ik \ldots ]}  = \dfrac{1}{q!}\sum\limits_{\sigma \in S_q} \varepsilon(\sigma) \cdot t_{\sigma(i) \sigma(k) \ldots }.    $$



Рассмотрим тензоры  $\alpha \in T^0_p$ и
$\beta \in T^{0}_{q}$.
Тогда, согласно определению,\\
$$(\alpha \otimes \beta)  (\underbrace{\vek{a}, \vek{b}, \ldots }_{p}, 
\underbrace{\vek{c}, \vek{d}, \ldots }_{q}) \,  =      \, 
\alpha (\underbrace{\vek{a}, \vek{b}, \ldots }_{p}) \, \cdot \, 
\beta (\underbrace{\vek{c}, \vek{d}, \ldots }_{q}) = 
(\beta \otimes \alpha )  (\underbrace{\vek{c}, \vek{d}, \ldots }_{q}, \underbrace{\vek{a}, \vek{b}, \ldots }_{p}).$$
Отсюда 
$$\beta \otimes \alpha  = \sigma_{q, p} (\alpha \otimes \beta), $$
где перестановка $\sigma_{q, p}$ задается как
$\begin{pmatrix}
1 & 2 & \ldots &      q & q+1 & q+2 & \ldots & p+q  \\
p+1 & p+2 & \ldots & p+q & 1 & 2 & \ldots & p
\end{pmatrix}.$

Так как, $\varepsilon (\sigma_{q, p})= (-1)^{pq}$, с учетом 
лемм \ref{premix} и \ref{premix_alt}, получаем следующее.


\begin{predl}\label{comm_sym}
$\forall$  $\alpha \in T^0_p$ и $\beta \in T^{0}_{q}$ выполнено
$$\Sym (\alpha \otimes \beta)  = \Sym (\beta \otimes \alpha ).$$
\end{predl}


\begin{predl}\label{comm_alt}
$\forall$  $\alpha \in T^0_p$ и $\beta \in T^{0}_{q}$ выполнено
$$\Alt (\alpha \otimes \beta)  = (-1)^{pq} \, \Alt (\beta \otimes \alpha ).$$
\end{predl}

\begin{predl}\label{double_mix_sym_1}
$\forall$  $\alpha \in T^0_p$ и $\beta \in T^{0}_{q}$ выполнено
$$\Sym ((\Sym (\alpha)  \otimes \beta)  =  \Sym (\alpha \otimes \Sym (\beta) ) = \Sym (\alpha \otimes \beta).$$
\end{predl}
\dok СХЕМА. 
Заметим, что $\Sym (\alpha)  \otimes \beta  = \Sym_H  (\alpha  \otimes \beta)$, где $S_p\cong H \leq S_{p+q}$ ---  подгруппа перестановок, для которых последние $q$ элементов неподвижны.
\edok

\begin{predl}\label{double_mix_alt_1}
$\forall$  $\alpha \in T^0_p$ и $\beta \in T^{0}_{q}$ выполнено
$$\Alt ((\Alt (\alpha)  \otimes \beta)  =  \Alt (\alpha \otimes \Alt (\beta) ) = \Alt (\alpha \otimes \beta).$$
\end{predl}

\subsection{Симметрические тензоры. Базис в $S^0_q$.}


Обозначим $q! \Sym (\vek{e}^{i_1} \otimes \vek{e}^{i_2} \otimes \ldots \otimes \vek{e}^{i_q})$ 
коротко $\vek{e}^{i_1}\vek{e}^{i_2}\ldots \vek{e}^{i_q}$.

\example{Например, \\
$\vek{e}^{3}\vek{e}^{1}\vek{e}^{3} = 
2(\vek{e}^{3} \otimes \vek{e}^{1} \otimes \vek{e}^{3} + \vek{e}^{1} \otimes \vek{e}^{3} \otimes \vek{e}^{3}
+\vek{e}^{3} \otimes \vek{e}^{3} \otimes \vek{e}^{1}) $.\\
$\vek{e}^{3}\vek{e}^{1}\vek{e}^{2} = 
\vek{e}^{1} \otimes \vek{e}^{2} \otimes \vek{e}^{3} + \vek{e}^{1} \otimes \vek{e}^{3} \otimes \vek{e}^{2}+
\vek{e}^{2} \otimes \vek{e}^{1} \otimes \vek{e}^{3} + \vek{e}^{2} \otimes \vek{e}^{3} \otimes \vek{e}^{1}+
\vek{e}^{3} \otimes \vek{e}^{1} \otimes \vek{e}^{2} + \vek{e}^{3} \otimes \vek{e}^{2} \otimes \vek{e}^{1}
$.
}

Из-за коммутативности $\vek{e}^{i_1}\vek{e}^{i_2}\ldots \vek{e}^{i_q}$ 
приводится к виду $(\vek{e}^1)^{k_1}\ldots (\vek{e}^n)^{k_n}$,
где $k_i$ --- целые неотрицательные числа, сумма которых равна $q$.


\begin{predl}
1) $\{(\vek{e}^1)^{k_1}\ldots (\vek{e}^n)^{k_n} \, |\, k_i\in \mathbb{Z}_{+}, \sum k_i = q\}$ ---
 базис в пространстве $S^0_q$;\\
2) $\dim S^0_q = C_{n+q-1}^q$.
\end{predl}


\subsection{%Симметрическое произведение. 
Симметрическая алгебра.}




Для $\tau \in S^0_p$, $\tau ' \in S^0_{q}$ определяется {\it симметрическое умножение} по правилу: 
$$\tau \lor \tau ' =  \dfrac{(p+q)!}{p!q!} \Sym(\tau \otimes \tau '). $$

(иногда определяется без нормирующего множителя).
Из предложений \ref{comm_sym} и 
\ref{double_mix_sym_1} следует коммутативность и ассоциативность симметрического умножения.


Иногда знак $\lor$ опускается, что согласуется с 
обозначением $\vek{e}^{i_1}\vek{e}^{i_2}\ldots \vek{e}^{i_q}$:
$$\vek{e}^{i_1}\lor \vek{e}^{i_2}\lor \ldots \lor \vek{e}^{i_q} = \vek{e}^{i_1}\vek{e}^{i_2}\ldots \vek{e}^{i_q}.$$


$$S(V ) = \bigoplus\limits_{q=0}^{\infty} S^0_q.$$

$S(V )$ --- {\it симметрическая алгебра} относительно $+, \lor$.
(ассоциативная, коммутативная)



\subsection{Кососимметрические тензоры. Базис в $\Lambda^0_q$.}


Обозначим (временно) $q! \Alt (\vek{e}^{i_1} \otimes \vek{e}^{i_2} \otimes \ldots \otimes \vek{e}^{i_q})$ 
коротко $\vek{e}^{i_1} \circ \vek{e}^{i_2}\circ  \ldots \circ  \vek{e}^{i_q}$.

\example{Например, \\
$\vek{e}^{3} \circ \vek{e}^{1} \circ \vek{e}^{3} = 0$; \\
%2(\vek{e}^{3} \otimes \vek{e}^{1} \otimes \vek{e}^{3} + \vek{e}^{1} \otimes \vek{e}^{3} \otimes \vek{e}^{3}
%+\vek{e}^{3} \otimes \vek{e}^{3} \otimes \vek{e}^{1}) $.
$\vek{e}^{1}\circ \vek{e}^{2}\circ \vek{e}^{3} = 
\vek{e}^{1} \otimes \vek{e}^{2} \otimes \vek{e}^{3} - \vek{e}^{1} \otimes \vek{e}^{3} \otimes \vek{e}^{2}+
\vek{e}^{2} \otimes \vek{e}^{3} \otimes \vek{e}^{1} - \vek{e}^{2} \otimes \vek{e}^{1} \otimes \vek{e}^{3}+
\vek{e}^{3} \otimes \vek{e}^{1} \otimes \vek{e}^{2} - \vek{e}^{3} \otimes \vek{e}^{2} \otimes \vek{e}^{1}
$.
}

В частности, в трехмерном пространстве 
$(\vek{e}^{1}\circ \vek{e}^{2}\circ \vek{e}^{3}) (\vek{a}, \vek{b}, \vek{c}) = $
детерминант из координатных столбцов векторов $\vek{a}, \vek{b}, \vek{c}$.
%$=\varepsilon_{ijk}x^iy^jz^k$.


Из-за косокоммутативности $\vek{e}^{i_1} \circ \vek{e}^{i_2} \circ \ldots \circ \vek{e}^{i_q}$ 
приводится к виду $0$ либо $\pm \vek{e}^{j_1} \circ \vek{e}^{j_2} \circ \ldots \circ \vek{e}^{j_q} $, где
$j_1<j_2<\ldots <j_q$.
% (\vek{e}^1)^{k_1}\ldots (\vek{e}^n)^{k_n}$
%где $k_i$ --- целые неотрицательные числа, сумма которых равна $q$.


\begin{predl}
1) $\{\vek{e}^{j_1} \circ \vek{e}^{j_2} \circ \ldots \circ \vek{e}^{j_q} \, |\, 
j_1<j_2<\ldots <j_q\}$ ---
 базис в пространстве $\Lambda^0_q$;\\
2) $\dim \Lambda^0_q = C_{n}^q$ при $q\leq n$ и $\Lambda^0_q = O$ при $q>n$.
\end{predl}


%Элементы $\Lambda^0_q$ называются $q$-векторами (поливекторы). здесь путаница между $V$ и $V^*$.

%Поливектор вида $ разложимый поливектор


\subsection{ 
Внешняя алгебра.}




Для $\tau \in \Lambda^0_p$, $\tau ' \in \Lambda^0_{q}$ определяется {\it внешнее умножение} по правилу: 
$$\tau \wedge \tau ' =  \dfrac{(p+q)!}{p!q!} \Alt(\tau \otimes \tau '). $$

(иногда определяется без нормирующего множителя).
Из предложений \ref{comm_alt} и 
\ref{double_mix_alt_1} следует косокоммутативность:
$$\tau \wedge \tau ' = (-1)^{pq}\tau '\wedge \tau $$
(в частности, при нечетном $p$ $\tau \wedge \tau =0$) 
и ассоциативность внешнего умножения.

%Иногда знак $\wedge$ опускается, что согласуется с 
%обозначением $\vek{e}^{i_1}\vek{e}^{i_2}\ldots \vek{e}^{i_q}$:
$$\vek{e}^{i_1}\circ \vek{e}^{i_2}\circ \ldots \circ \vek{e}^{i_q} = 
\vek{e}^{i_1}\wedge \vek{e}^{i_2}\wedge \ldots \wedge \vek{e}^{i_q}.$$


$$\Lambda(V ) = \bigoplus\limits_{q=0}^{n} \Lambda^0_q.$$

$\Lambda(V )$ --- {\it внешняя алгебра} относительно $+, \wedge$.
(ассоциативная, косокоммутативная)



%ПРИМЕР. $\det$





%

%\input{3krivye_i_poverh-2-short}
%\input{3krivye_i_poverh-3-short}
%\input{3krivye_i_poverh-4-short}
%\input{3krivye_i_poverh-5-short}

%\input{4affin_08}

%\input{4affin-1}
%\input{4affin-2}
%\input{4affin-3}
%\input{4affin-4}

%\input{matricy}
%\input{sistemy}


%\input{lin_prostr}
%\input{lin_otobr}
%\input{kvadr_formy}
%\input{evkl_prostr}

%\input{prilozh_mnozh}
%\input{prilozh_otobr}

%\input{7_sistemy_lekc_plan}
%\newpage
%\input{8_lin_prostr_lekc_plan}
%\newpage
%\input{9_lin_otobr_lekc_plan}
%\newpage
%\input{10_sob_vektory_plan}
%\newpage
%\input{11_kvadr_formy_plan}
%\newpage
%\input{12_evkl_prostr_plan}
%\newpage
%\input{13_evkl_prostr_otobr_formy_plan}
%\newpage
%\input{oboznacheniya}
%\input{library}
%\begin{theindex}
%\input{lekcii.ind}
%\end{theindex}

\end{document}
