\section{Ортогональные системы векторов. Ортогональное дополнение. Ортогонализация} %\label{ortogonal_vectors}


\subsection{Ортогональные системы векторов. ОНБ}

\defin{Система векторов $\vek{a}_1, \ldots, \vek{a}_k$ в евклидовом (унитарном) пространстве называется {\it ортогональной}, если 
$\vek{a}_i\perp \vek{a}_j$ для всех $1\leq i<j\leq k$.
}

\defin{Ортогональная система векторов $\vek{a}_1, \ldots, \vek{a}_k$ называется {\it ортонормированной}, если 
$|\vek{a}_i|=1$ для всех $1\leq i\leq k$.
}

В частности, можно говорить об ортогональных базисах и ортонормированных базисах (далее используем сокращение ОНБ).

\begin{predl}[теорема Пифагора]\label{p10_2_1}
Пусть $\vek{a}_1, \ldots, \vek{a}_k$ --- ортогональная система векторов. Тогда 
$|\vek{a}_1 + \ldots + \vek{a}_k|^2=|\vek{a}_1|^2+\ldots + |\vek{a}_k|^2$.
\end{predl}
\dok
$|\vek{a}_1 + \ldots + \vek{a}_k|^2=(\vek{a}_1 + \ldots + \vek{a}_k, \vek{a}_1 + \ldots + \vek{a}_k)$.
Раскрывая по линейности с учетом того, что $(\vek{a}_i, \vek{a}_j)=0$ при $i\neq j$, 
получаем $(\vek{a}_1, \vek{a}_1)+\ldots + (\vek{a}_k, \vek{a}_k) = |\vek{a}_1|^2+\ldots + |\vek{a}_k|^2$.
\edok

\otstup

Отметим следующий почти очевидный критерий ортогональности в терминах матрицы Грама.

\begin{predl}\label{p10_2_2} 
Система векторов $\vek{a}_1, \ldots, \vek{a}_k$ является ортогональной $\Leftrightarrow$ $\Gamma(\vek{a}_1, \ldots, \vek{a}_k)$ --- диагональная матрица. \\
Система векторов $\vek{a}_1, \ldots, \vek{a}_k$ является ортонормированной $\Leftrightarrow$ $\Gamma(\vek{a}_1, \ldots, \vek{a}_k)$ --- единичная матрица. 
\end{predl}
\dok Сразу следует из определения матрицы Грама.
\edok

\begin{sled1}
Ортогональная система ненулевых векторов линейно независима.
\end{sled1}
\dok Следует из предыдущего предложения с учетом предложения \ref{p10_1_1}.
{\footnotesize Также это ясно, например, из теоремы Пифагора.}
\edok

\begin{sled2}
Для конечномерного евклидова пространства: ортогональный базис --- это базис, в котором форма скалярного произведения имеет диагональный вид,
ОНБ --- это базис, в котором форма скалярного произведения имеет канонический вид. 
\end{sled2}

\begin{sled3}
В конечномерном евклидовом пространстве существет ОНБ.
\end{sled3}
\dok %ОНБ --- это базис, в котором форма скалярного произведения имеет канонический вид. 
Ввиду следствия 2, это частный случай следствия из теоремы \ref{t9_3_1} главы \ref{kvadr_formy}.
\edok

\begin{sled4}
Пусть $\bazis{e} = (\vek{e}_1, \vek{e}_2, \ldots, \vek{e}_n)$ --- ОНБ в $\mathcal{E}$.
Если векторы $\vek{a}, \vek{b} \in \mathcal{E}$ таковы, что
$\vek{a} = \bazis{e} X$ и $\vek{b} = \bazis{e} Y$, то $$\boxed{(\vek{a}, \vek{b}) = X^T\overline{Y}.}$$
\end{sled4}
\dok 
Это утверждение --- частный случай теоремы \ref{t10_2_1}.
\edok

\subsection{Переход от ОНБ к ОНБ. Ортогональные и унитарные матрицы}

\defin{Матрица $Q\in \mathbf{M}_{n\times n} (\mathbb{C})$ называется {\it унитарной}, если $$Q^{*}Q=E.$$
}

\defin{Матрица $Q\in \mathbf{M}_{n\times n} (\mathbb{R})$ называется {\it ортогональной}, если $$Q^{T}Q=E.$$
}

Множество всех ортогональных и унитарных матриц $n\times n$  обозначаем соответственно $O_n$ и $U_n$.
Сравнивая определения, получаем  $O_n\subset U_n$ и более того, $O_n= U_n\cap \mathbf{M}_{n\times n} (\mathbb{R})$
(т.е. ортогональные матрицы это в точной вещественные унитарные матрицы).


\begin{predl}\label{p10_2_100} 
Для матрицы $Q\in \mathbf{M}_{n\times n} (\mathbb{C}) $
следующие условия эквивалентны: \\
1) $Q\in U_n$;\\
2) $\exists \, Q^{-1} $ и $Q^{-1} = Q^{*}$;\\
3) столбцы матрицы $Q$ образуют ОНБ в унитарном пространстве столбцов $\mathbb{C}^n = \mathbf{M}_{n\times 1}$, наделенном
стандартным скалярным произведением $(X, Y)=X^T\overline{Y}$.
\end{predl}
\dok Очевидно, 1) $\Leftrightarrow$ 2).\\
Пусть $Q_1, \ldots, Q_n$ --- столбцы матрицы $Q$. Равеноство $Q^{*}Q=E$ означает, что $\overline {Q_i^{T}} Q_j = \delta_{ij}$ или 
$Q_i^{T} \overline {Q_j} = \delta_{ij}$. Значит, 1) $\Leftrightarrow$ 3).
\edok

\otstup

Видим, в частности, что обращать ортогональную матрицу легко: достаточно ее транспонировать.
Cвойство 3), возможно, наиболее просто для проверки <<вручную>>, 
является ли матрица $Q$ ортогональной (унитарной).

%Эквивалентные условия: $QQ^{*}=E$; $Q^{*}Q=E$;
%столбцы (строки) --- ортонормированный базис в $\mathbb{R}_n = \mathbf{M}_{n\times 1}$ ($\mathbb{C}_n$) 
%(со стандартным скалярным произведением $(X, Y)=X^T\overline{Y}$.

\begin{predl}\label{p10_2_101} 
Пусть  $Q\in U_n$. Тогда $Q^T \in U_n$, $\overline{Q} \in U_n$,  $Q^{*} \in U_n$.
\end{predl}
\dok 
\edok

\otstup

Видим равноправие строк и столбцов ортогональной (унитарной) матрицы, в частности, что строки ортогональной (унитарной) матрицы --- тоже ОНБ в пространстве строк со стандартным скалярным произведением.

\begin{predl}[Групповое свойство]\label{p10_2_102} 
Пусть  $Q, R\in U_n$. Тогда $QR\in U_n$, $Q^{-1} \in U_n $.
\end{predl}
\dok 
\edok

\begin{predl}\label{p10_2_103} 
Пусть  $Q\in U_n$. Тогда $|\det Q| = 1$.
\end{predl}
\dok 
Имеем $\det (Q^{*}Q)=\det E = 1$. Отсюда $\det Q^{*}\cdot \det Q= 1$ $\Leftrightarrow$
$\overline{\det Q^{T}}\cdot \det Q= 1$ $\Leftrightarrow$
$\overline{\det Q}\cdot \det Q= 1$ $\Leftrightarrow$ $|\det Q|^2 = 1$.
\edok

\otstup

Имеется следующая связь между ортогональными (унитарными) матрицами и переходом от ОНБ к ОНБ.

\begin{theor}\label{t10_2_104} 
Пусть $\dim \mathcal{E}=n<\infty $, $\bazis{e}=(\vek{e}_1, \ldots, \vek{e}_n)$ --- ОНБ в $\mathcal{E}$,
$\bazis{e}'=(\vek{e}'_1, \ldots, \vek{e}'_n)$ --- некоторый базис в $\mathcal{E}$. Пусть $S$ --- матрица перехода от  
базиса $\bazis{e}$ к базису $\bazis{e}'$. Тогда\\
Тогда $\varphi$ является ортогональным  $\Leftrightarrow$ $S$ --- ортогональная (унитарная) матрица.
\end{theor}
\dok 
Это сразу следует из определения матрицы перехода и условия 3) в предложении \ref{p10_2_100}.
\edok

\otstup

Таким образом, ортогональные (унитарные) матрицы --- в точности матриы перехода от ОНБ к ОНБ.
Иногда именно в теорминах матрицы перехода естественно интерпретировать ортогональные (унитарные) матрицы.
Например, можно дать другое доказательство предожения \ref{p10_2_102}.
%УПР.?

\subsection{Ортогональные подпространства}

\defin{Множества $U_1$ и $U_2$ евклидова (унитарного) пространства называются {\it ортогональными}, если 
$\forall \, \vek{a}_1\in U_1$ и $\forall \, \vek{a}_2\in U_2$ выполнено $\vek{a}_1\perp \vek{a}_2$.
}

Ортогональность двух векторов получается как частный случай (вектор считаем одноэлементным подмножеством).
Обозначение для ортогональности прежнее, например $\vek{a}\perp U$ --- значит вектор $\vek{a}$ ортогонален любому вектору из $U$.
Чаще нас будут интересовать ортогональные подпространства.
Почти очевидно, что ортогональные подпространства пересекаются тривиально. Более общий факт дает следующая теорема.

\begin{theor}\label{t10_2_2} 
Пусть $U_1, \ldots, U_k$ --- попарно ортогональные подпространства евклидова (унитарного) пространства $\mathcal{E}$. 
Тогда сумма $U_1 + \ldots + U_k$ является прямой суммой.
\end{theor}
\dok
Предположим противное, пусть, скажем найдется ненулевой $\vek{a}\in U_1\cap (U_2+ \ldots +U_k)$ 
(пользуемся критерием-1 прямой суммы --- см. теорему \ref{t7_4_1} главы \ref{lin_prostr}).
Тогда $\vek{a}=\vek{a}_2+\ldots + \vek{a}_k$, где $\vek{a}_i\in U_i$. Перенесем все слагаемые в левую часть и вычеркнем нулевые векторы, тогда 
получем, что сумма ненулевых попарно ортогональных векторов равна $\vek{0}$. Это противоречит следствию 1 из предложения \ref{p10_2_2}. 
\edok


\begin{predl}[признак ортогональности]\label{p10_2_3} 
Пусть $U_1= \lin{\vek{a}_1, \ldots, \vek{a}_k}$, $U_2= \lin{\vek{b}_1, \ldots, \vek{b}_l}$.
Тогда \\ $U_1 \perp U_2$
$\Leftrightarrow$ $\vek{a}_i\perp \vek{b}_j$, $i=1, 2, \ldots, k$, $j=1, 2, \ldots, l$.
\end{predl}
\dok
\dokright Очевидно из определения ортогональности подпространств.\\
\dokleft Пусть $\vek{a}\in U_1$, $\vek{b}\in U_2$. Тогда существуют разложения
 $\vek{a} = \sum\limits_{i=1}^k \alpha _i \vek{a}_i$,  
$\vek{b} = \sum\limits_{j=1}^l \beta _j \vek{b}_j$. Тогда 
из линейности скалярного произведения получаем $(\vek{a}, \vek{b}) = \sum\limits_{i=1}^k \sum\limits_{j=1}^l  \alpha _i \overline{\beta _j} (\vek{a}_i, \vek{b}_j)= 0$, 
т.е. $\vek{a} \perp \vek{b}$.
\edok



\subsection{Ортогональное дополнение. Ортогональная проекция}

\defin{
{\it Ортогональным  дополнением} подпространства $U$ (в евклидовом (унитарном) пространстве $\mathcal{E}$) называется 
множество всех векторов, ортогональных $U$.
}

 Обозначение для ортогонального дополнения: $U^{\bot}$. Итак,  $U^{\bot} = \{\vek{a} \, | \, \vek{a}\perp U\}$.
Очевидно, $U\perp U^{\bot}$, значит согласно теореме \ref{t10_2_2}, имеем  $U\bigoplus U^{\bot}$.
В случае $\dim U<\infty$ можно усилить так.

\begin{theor}\label{t10_2_3} 
Пусть $U\leq \mathcal{E}$, $\dim U=k<\infty$.
Тогда $$\boxed{U\bigoplus U^{\bot}=\mathcal{E}}.$$
\end{theor}
\dok 
Представим произвольный вектор $\vek{a}\in \mathcal{E}$ в виде суммы $\vek{a}_1+\vek{a}_2$, где $\vek{a}_1\in U$,  $\vek{a}_2\perp U$.

Возьмем в $U$ ортогональный базис $\vek{b}_1, \ldots, \vek{b}_k$ 
и представим произвольный вектор $\vek{a}\in \mathcal{E}$ 
в виде $\vek{a} = \alpha_1\vek{b}_1+\ldots + \alpha_k\vek{b}_k +\vek{c}$. Достаточно подобрать коэффициенты $\alpha_1, \ldots, \alpha_k$ так, чтобы
$\vek{c}=  \vek{a} - \alpha_1\vek{b}_1 - \ldots - \alpha_k\vek{b}_k$ был ортогонален $U$. Последнее равносильно (см. предложение \ref{p10_2_3})
тому, что $(\vek{c}, \vek{b}_i)=0$, $i=1, \ldots, k$. 
Ввиду ортогональности $\vek{b}_i\perp \vek{b}_j$, $i\neq j$, имеем $(\vek{c}, \vek{b}_i)=0$
$\Leftrightarrow$  $(\vek{a}, \vek{b}_i)-\alpha_i (\vek{b}_i, \vek{b}_i)=0$ $\Leftrightarrow$ $\alpha_i = \dfrac{(\vek{a}, \vek{b}_i)}{(\vek{b}_i, \vek{b}_i)}$.
\edok

\begin{sled1}
Если  $\dim \mathcal{E}=n< \infty$,  $U\leq \mathcal{E}$ и $\dim U =k$, то $\dim U^{\bot} = n-k$.
\end{sled1}

\begin{sled2}
Если  $\dim \mathcal{E}=n<\infty$ и $U\leq \mathcal{E}$, то $(U^{\bot})^{\bot} = U$.
\end{sled2}
\dok 
Так как $U\perp U^{\bot}$, то $U \subset (U^{\bot})^{\bot}$. Но по следствию 1, $\dim U =\dim  (U^{\bot})^{\bot} = n-\dim U^{\bot}$.
Значит (см....), $(U^{\bot})^{\bot} = U$.
\edok

{\bf Упражнение.}
%Для подпространств конечномерного евклидова пространства: \\
Для $U_i\leq \mathcal{E}$, $\dim U_i<\infty$, $i=1, 2$ докажите, что $(U_1+U_2)^{\bot} = U_1^{\bot} \cap U_2^{\bot}$.


\begin{sled3}
Пусть  $\dim \mathcal{E}=n< \infty$. Тогда данную ортогональную систему ненулевых векторов можно дополнить до ортогонального базиса.
\end{sled3}
\dok Пусть  $\vek{b}_1, \ldots, \vek{b}_k$ --- данная ортогональная система, положим $U = \lin{\vek{b}_1, \ldots, \vek{b}_k}$.
В силу \ref{p10_2_2}, $\vek{b}_1, \ldots, \vek{b}_k$ ---  ортогональный базис в $U$. Пусть 
$\vek{b}_{k+1}, \ldots, \vek{b}_n$ ---  ортогональный базис в $U^{\bot}$. 
Тогда $\vek{b}_1, \ldots, \vek{b}_k, \vek{b}_{k+1},\ldots, \vek{b}_n$ ---  ортогональный базис в $\mathcal{E}$.
\edok



Формула $U\bigoplus U^{\bot}=\mathcal{E}$  оправдывает термин <<ортогональное дополнение>>. Это частный случай прямого дополнения (см. ....).
В таком случае упростим терминологию и обозначения.

\defin{
{\it Ортогональной проекцией} вектора $\vek{a}$ на подпространство $U\leq \mathcal{E}$ называется проекция $\vek{a}$ на $U$ вдоль $U^{\bot}$.
}

Обозначение для ортогональной проекции: $\pr_U \vek{a}$

\begin{predl}[формула проекции]\label{10_2_4} 
Пусть $\vek{a}\in \mathcal{E}$, $U\leq \mathcal{E}$ 
и $\vek{b}_1, \ldots, \vek{b}_k$ --- ортогональный базис в $U$. Тогда 
$$\boxed{\pr_U \vek{a} = \sum\limits_{i=1}^k \frac{(\vek{a}, \vek{b}_i)}{(\vek{b}_i, \vek{b}_i)} \vek{b}_i.}$$
\end{predl}
\dok 
В доказательстве теоремы \ref{t10_2_3} мы уже нашли нужное выражение для $\pr_U \vek{a}$ 
(в виде $\alpha_1\vek{b}_1+\ldots + \alpha_k\vek{b}_k$, где $\alpha_i = \dfrac{(\vek{a}, \vek{b}_i)}{(\vek{b}_i, \vek{b}_i)}$.
\edok



\otstup

Проекции могут естественно возникать в задачах на экстремум:

\begin{predl}[расстояние до подпространства]\label{10_2_5} 
Пусть $U\leq \mathcal{E}$, $\dim U<\infty$. 
Тогда $\min\limits_{\vek{x}\in U} |\vek{a}-\vek{x}| = |\pr_{U^{\bot}} \vek{a}|$.
\end{predl}
\dok Представим $\vek{a}$ как $\vek{a}=\vek{a}_1+\vek{a}_2$, где $\vek{a}_1 = \pr_U \vek{a}$, $\vek{a}_2 = \pr_{U^{\bot}} \vek{a}$.
Тогда для $\vek{x}\in U$ имеем $|\vek{a}-\vek{x}|^2 = |\vek{a}_1+\vek{a}_2-\vek{x}|^2 = |(\vek{a}_1-\vek{x})+\vek{a}_2|^2 $.
Так как $(\vek{a}_1-\vek{x}) \perp \vek{a}_2$, по теореме Пифагора 
$|(\vek{a}_1-\vek{x})+\vek{a}_2|^2 = |\vek{a}_1-\vek{x}|^2+|\vek{a}_2|^2\geq |\vek{a}_2|^2$.
При этом неравенство обращается в равенство при $\vek{x}=\vek{a}_1$.
\edok

Следующее предложение показывает, что при работе в с координатами в ОНБ очень легко получать описание ортогонального дополнения.

\begin{predl}[Ортогональное дополнение в координатах]\label{10_2_5} 
Пусть $\bazis{e}$ --- ОНБ в $\mathcal{E}$,
$U = \lin{\vek{a}_1, \ldots, \vek{a}_k}$, где $\vek{a}_1, \ldots, \vek{a}_k$ --- векторы c с координатными столбцами $X_1, \ldots, X_k$ в ОНБ $\bazis{e}$,
$\Phi$ --- матрица со столбцами $X_i$: $\Phi = (X_1\, \ldots \, X_k)$.
Тогда $U^{\bot}$ задается (в коодинатах в ОНБ $\bazis{e}$) как $\Sol (\Phi^* X =O)$.
\end{predl}
\dok  Система $\Phi^* X =O$ состоит из уравнений вида $\overline{X_i^T}X = 0$. 
Эти уравнения означают, что вектор $\vek{x}$ с координатным столбцом $X$ ортогонален $\vek{a}_i$, $i=1, \ldots, k$.
В силу предложения \ref{p10_2_3}, это эквивалентно условию $\vek{b}\perp U$.
\edok

%Почему любое подпр-во задается системой в коодинатах????? НАОБОРОТ, если $U$ задано системой.


\begin{sled}[Теорема Фредгольма о совместности СЛУ]
Пусть дана СЛУ $AX=b$. %  (с матрицей коэффициентов $A$ размера $m\times n$. 
Тогда $AX=b$ совместна $\Leftrightarrow$ $\forall \, Y_0\in \Sol(A^{*}Y=O)$ выполнено <<условие ортогональности>>: $b^{*}Y_0=0$.
\end{sled}
\dok 
Рассмотрим пространство $\mathcal{E}=\mathbb{M}_{m\times 1}$ столбцов высоты $m$ со стандартным скалярным произведением.
Столбцы $a_{\bullet 1}, \ldots, a_{\bullet n}$ матрицы $A$ и столбец $b$ лежат в $\mathcal{E}$.
Условие совместности системы $AX=b$ эквивалентно тому, что $b\in U$, где $U = \lin{a_{\bullet 1}, \ldots, a_{\bullet n}}$.
Далее, $U^{\bot} = \Sol(A^{*}Y=O)$, поэтому <<условие ортогональности>>  $\forall \, Y_0\in \Sol(A^{*}Y=O)$
эквивалентно тому, что $b\perp U^{\bot}$. Но очевидно, $b\in U$ $\Leftrightarrow$ $b\perp U^{\bot}$.
\edok


\subsection {Ортогонализация}

Пусть изначально подпространство задано как линейная оболочка произвольной конечной системы векторов: $U = \lin{\vek{a}_1, \ldots, \vek{a}_k}$,
а требуется найти ортогональный (или ОНБ) базис в $U$ (например, чтобы после этого была возможность использовать формулу для $\pr_U \vek{a}$).
Фактически это та же процедура нахождения базиса, в котором форма скалярного произведения имеет диагональный 
(канонический вид).
Но ввиду важности этого частного случая, дадим несколько другое описание прцедуры (с геометрической точки зрения).
Алгоритм называется {\it ортогонализацей Грама-Шмидта}.

Положим $U_i=\lin{\vek{a}_1, \ldots, \vek{a}_i}$, так что $U_{i+1}=U_i\cup \{\vek{a}_{i+1}\}$.
Идея следующая: будем по очереди <<поправлять>> векторы $\vek{a}_{i}$, 
заменяя на $\vek{b}_{i}=\pr_{U_{i-1}^{\bot}} \vek{a}_{i}$ так, 
что $\vek{b}_{i}\perp U_{i-1}$ и $U_i=\lin{\vek{b}_1, \ldots, \vek{b}_i}$.

Получим явные формулы. 
Первый ненулевой вектор $\vek{a}_t$ из списка $\vek{a}_1, \ldots, \vek{a}_k$ является ортогональным базисом в $U_t$.
Пусть на некотором шаге мы уже имеем ортогональный базис $\vek{b}_1, \ldots, \vek{b}_{\ell}$ в $U_m$.
Заменяем $\vek{a}_{m+1}$ на вектор 
$\vek{a}_{m+1}- \pr_{U_m} \vek{a}_{m+1}$, т.е. на \\
$\vek{a}_{m+1}- \sum\limits_{i=1}^{\ell} \dfrac{(\vek{a}_{m+1}, \vek{b}_{i})}{(\vek{b}_{i}, \vek{b}_{i})} \vek{b}_{i}$.
Если последний вектор нулевой (это соответсвует случаю $\vek{a}_{m+1}\in U_m$), то пропустим его, если он ненулевой, 
то объявим $\vek{b}_{\ell+1} $ равным 
$$ \vek{a}_{m+1}- \sum\limits_{i=1}^{\ell} \dfrac{(\vek{a}_{m+1}, \vek{b}_{i})}{(\vek{b}_{i}, \vek{b}_{i})} \vek{b}_{i},$$ 
тем самым достраивая ортогональный базис в $U_{m+1}$.

\otstup

{\bf Упражнение.}
а) Пусть $\bazis{a}=(\vek{a}_1, \vek{a}_2, \ldots, \vek{a}_n)$ --- базис. 
Докажите, что существует ОНБ $\bazis{e}$, такой, что матрица перехода 
от $\bazis{e}$ к $\bazis{a}$ --- верхнетреугольная.\\
б) Докажите, что любую матрицу $A\in \mathbf{M}_{n\times n}(\mathbb{R})$ можно представить в виде
$A=QT$, где $Q\in O_n$, а $T$ --- верхнетреугольная.


\subsection{$k$-мерный объем}

Формально $k$-мерный объем определяется и изучается в теории меры. Тем не менее сделаем несколько важных замечаний на этот счет (говоря об объемах, полагаем $\mathbb{F}=\mathbb{R}$).

Ориентированным $n$-мерным  объемом $\vol_{\pm}$ параллелепипеда, построенного 
на векторах $\vek{a}_1, \vek{a}_2, \ldots, \vek{a}_n$, будем здесь считать (и конечно это согласуется с 
результатами из курса анализа) значение детерминанта матрицы $A$, составленной из координатнных столбцов 
векторов $\vek{a}_1, \vek{a}_2, \ldots, \vek{a}_n$ в заданном ОНБ $\bazis{e}$
(т.е. $A$ --- матрица линейного отображения, переводящего $\vek{e}_i$ в $\vek{a}_i$, $i=1, \ldots, n$).
{\footnote При этом модуль этого детерминанта не зависит от выбора ОНБ, 
так как $\det$ матрицы переходв от ОНБ к ОНБ равен $\pm 1$, см. выкладку для $k$-мерных объемов ниже.}
%Таким образом, если $\vek{a}_1, \vek{a}_2, \ldots, \vek{a}_n$ --- линейно зависимая система, то
%$\vol(\vek{a}_1, \vek{a}_2, \ldots, \vek{a}_n)=0$, 
Можно сразу отметить, что $\vol_{\pm}$ не меняется при стлбцовых элементарных преобразованиях III типа 
(прибавление к столбцу дрогого столбца, умноженного на константу), в частности, не меняется в процессе ортогонализации (без выполнения нормировки).


Имеем $(\vol  (\vek{a}_1, \vek{a}_2, \ldots, \vek{a}_n)) ^2  = \det A^T \cdot \det A = \det (A^T A)$.
Матрица $A^TA$, как несложно видеть, совпадает с матрицей Грама
$\Gamma (\vek{a}_1, \vek{a}_2, \ldots, \vek{a}_n)$.
Так проясняется геометрический смысл 
 определителя $\det (\Gamma (\vek{a}_1, \vek{a}_2, \ldots, \vek{a}_n))$,
который согласуется с предложением \ref{p10_1_1}.

Покажем, что $k$-мерный объем тоже равен
$\sqrt{\det (\Gamma (\vek{a}_1, \vek{a}_2, \ldots, \vek{a}_k))}$.
В частности, для одномерного объема имеем $\vol(\vek{a})=|\vek{a}|$.


Действительно, пусть $A\in \mathbb{M}_{n\times k}(\mathbb{R})$ --- матрица,
 составленная из координатнных столбцов 
векторов $\vek{a}_1, \vek{a}_2, \ldots, \vek{a}_k$ в заданном ОНБ $\bazis{e}$.
Кроме того, рассмотрим некоторое $k$-мерное подпространство $U\supset 
\lin{\vek{a}_1, \vek{a}_2, \ldots, \vek{a}_k}$, и выберем в $U$ некоторый ОНБ
 $\bazis{f} = (\vek{f}_1, \vek{f}_2, \ldots, \vek{f}_k)$,
так что $\bazis{f} =  \bazis{e}C$, где $C\in \mathbb{M}_{n\times k}(\mathbb{R})$ --- матрица,
 составленная из координатнных столбцов 
векторов $\vek{f}_1, \vek{f}_2, \ldots, \vek{f}_k$ в  ОНБ $\bazis{e}$;
в силу ортонормированности системы $\bazis{f}$ имеем $C^TC = E$.
Пусть $A_0\in \mathbb{M}_{k\times k}(\mathbb{R})$ --- матрица,
 составленная из координатнных столбцов 
векторов $\vek{a}_1, \vek{a}_2, \ldots, \vek{a}_k$ в  $\bazis{f}$,
так что $A=CA_0$.
Тогда 
$\det (A^T A) = \det (A_0^TС^T СA_0) = \det (A_0^T A_0)  = 
(\vol  (\vek{a}_1, \vek{a}_2, \ldots, \vek{a}_k)) ^2 $.


Отсюда несложно видеть, что в случае $\vek{a}_i\perp \vek{b}_j$
верно равенство (*)
$\vol  (\vek{a}_1, \vek{a}_2, \ldots, \vek{a}_k, \vek{b}_1, \vek{b}_2, \ldots, \vek{b}_l)
=  \vol  (\vek{a}_1, \vek{a}_2, \ldots, \vek{a}_k) 
\cdot \vol (\vek{b}_1, \vek{b}_2, \ldots, \vek{b}_l) $.

В ОБЩЕМ СЛУЧАЕ НЕР-ВО. Ф-ЛУ ЧЕРЕЗ УГОЛ МЕЖДУ ПОДПРОСТРАНТСВАМИ? (Посмотреть Лин. алгебра в задачах Прасолова)

Многократное применение этого равенства дает формулу вычисления объема {\it
прямоугольного параллелепипеда}, натянутого на попарно ортогональные векторы 
$\vek{a}_1, \vek{a}_2, \ldots, \vek{a}_k$:\\
$\vol  (\vek{a}_1, \vek{a}_2, \ldots, \vek{a}_k) = \prod_{i=1}^k |\vek{a}_i|.$


%ПРАВИЛО КРАМЕРА И ЕГО ГЕОМ, СМЫСЛ.ЧЕРЕЗ ОБЪЕМЫ

Можно показать, что справедлив принцип <<объем равен площадь основания на высоту>>
для вычисления $k$-мерного объема:  
$\vol  (\vek{a}_1, \vek{a}_2, \ldots, \vek{a}_k) = 
\vol  (\vek{a}_1, \vek{a}_2, \ldots, \vek{a}_{k-1})\cdot 
|\pr_{\lin{\vek{a}_1, \vek{a}_2, \ldots, \vek{a}_{k-1}}} \vek{a}_k|.$

Действительно, выберем в $U\supset 
\lin{\vek{a}_1, \vek{a}_2, \ldots, \vek{a}_k}$, ОНБ,
в котором матрица, составленная из координатных столбцов векторов 
$\vek{a}_1, \vek{a}_2, \ldots, \vek{a}_k$, является верхнетреугольной.
Ортогонализуем $\vek{a}_k$, заменяя его на $\vek{a}_k - \pr_{\lin{\vek{a}_1, \vek{a}_2, \ldots, \vek{a}_{k-1}}} \vek{a}_k$, и пользуемся (*).

%НЕРАВЕНСТВО В ОБЩЕМ СЛУЧАЕ --- В процессе ортогонализации длина не увеличивается.

С помощью этого соображения можно выразить расстояние до подпространства через 
объемы (см. расстояние до подпространства --- ниже.)


ОБОБЩЕНИЕ ЧЕРЕЗ ПРОЕКЦИЮ $l$-мерного пар-пипеда на орт. дополнение? 
ВООБЩЕ: ОЦЕНКИ ОБЪЕМА ТЕЛА (Выпуклого ) ЧЕРЕЗ площади проекции (разных размерностей).

%(все таки сводить в угловым минорам? см. Винберг стр. 196, 208.--- кажется, не обязательно.
%Прасолов, Тихомиров = посмотреть это все....