\chapter{Векторные пространства}\label{lin_prostr}

\section{Векторные пространства и подпространства}

\subsection{Аксиомы и их следствия}

{\it Бинарная операция} на множестве $Z$ --- это некоторое отображение $\varphi : Z \times Z \to Z$,
таким образом для каждой упорядоченной пары $(a, b)$ элементов множества $Z$ определен элемент $\varphi (a, b)\in Z$ --- 
{\it результат} применения операции $\varphi$ к $a$ и $b$.
Ниже будут встречаться бинарные операции в записи, более привычной для арифметических действий. Скажем, операцию можно обозначить
<<$+$>>  и тогда вместо $\varphi (a, b)$ писать $a+b$.
{\it Унарная операция} на множестве $Z$ --- это просто отбражение $Z \to Z$. Скажем, умножение на число 2 
формально можно считать
таким отображением $\psi : \mathbb{F} \to \mathbb{F}$, что $\forall x\in \mathbb{F}$ $ \psi (x) = 2x$.

Далее предполагаем, что $V$ --- произвольное множество, а $\mathbb{F}$ --- 
некоторое поле (скажем, поле $\mathbb{R}$), при этом  на $V$  определена бинарная операция, которую называем
{\it сложением} и обозначаем <<$+$>>, а также  для каждого $\lambda \in \mathbb{F}$ на 
$V$ определена унарная операция, которую называем {\it умножением} на $\lambda$. 
Результат применения умножения на $\lambda$ к  $\vek{a} \in V$ обозначаем $\lambda \cdot \vek{a}$ или $\lambda \vek{a}$.

\defin{$V$ называется  {\it векторным  пространством (или линейным пространством) 
над полем $\mathbb{F}$)}, если
выполнены следующие свойства V1---V8 (аксиомы векторного пространства):\\
V1. $(\vek a + \vek b)+\vek c = \vek a + (\vek b+\vek c)$ \,\,\,\,\,\, ($\forall$ $\vek{a}, \vek{b}, \vek{c} \in V$);\\
V2. $\exists$ $\vek{0}\in V$ $\forall$ $\vek{a} \in V$:   $\vek a + \vek{0}= \vek{0} + \vek{a} =\vek a$;\\
V3. $\forall$ $\vek{a} \in V$ $\exists$ $\vek x \in V$: $\vek x+ \vek{a}= \vek a+ \vek{x} = \vek{0}$;\\ 
V4. $\vek a + \vek b = \vek b + \vek a$ \,\,\,\,\,\, ($\forall$ $\vek{a}, \vek{b} \in V$);\\
V5. $\lambda(\vek a + \vek b)=\lambda \vek a+\lambda \vek b$ \,\,\,\,\,\, ($\forall$ $\vek{a}, \vek{b} \in V$, $\forall$ $\lambda \in \mathbb{F}$ );\\
V6. $(\lambda+\mu)\vek a=\lambda \vek a+\mu \vek a$ \,\,\,\,\,\, ($\forall$ $\vek{a} \in V$, $\forall$ $\lambda, \mu \in \mathbb{F}$ );\\
V7. $1\cdot \vek a=\vek a$  \,\,\,\,\,\, ($\forall$ $\vek{a} \in V$);\\
V8. $(\lambda \mu)\vek a = \lambda (\mu \vek a)$ \,\,\,\,\,\, ($\forall$ $\vek{a} \in V$, $\forall$ $\lambda, \mu \in \mathbb{F}$ ).
}


На протяжении главы $V$ будет обозначать векторное пространство.
Элементы векторного пространства $V$ будем называть {\it векторами} (независимо от их природы).
Элементы поля $\mathbb{F}$ будем иногда называть {\it константами} или {\it скалярами}.

Заметим, что аксиомы V1---V4 повторяют аксиомы абелевой группы.
Аксиомы V1 и V4 называются аксиомами {\it ассоциативности} и {\it коммутативности} сложения.
Вектор $\vek{0}$ из аксиомы V2 (нейтральный элемент относительно сложнения)
 называется {\it нулевым} вектором.\index{Вектор!нулевой} 
Вектор $\vek{x}$ из аксиомы V3  называется {\it противоположным} 
вектором для вектора $\vek{a}$ и обозначатся $-\vek{a}$.
Используя пртивоположный вектор, можно определить операцию {\it вычитания} равенством $\vek{a}-\vek{b}:= \vek{a}+(-\vek{b})$,
в частности, V2 принимает вид $\vek{a}-\vek{a}=\vek{0}$.

%V5 и V6 можно назвать свойством линейности, или дистрибутивности (по векторам и по константам соответственно),
%V7 --- условие нормировки, V8 --- смешанная ассоциативность умножения.\\
%В аксиомах V1---V4 встречается лишь операция сложения. Множество $V$ с операцией ''$+$'', удовлетворяющее только %аксиоме V1, называется
%{\it полугруппой}; %аксиомам V1 и V2 --- {\it полугруппой с единицей}
%удовлетворяющее аксиомам V1---V3 --- {\it группой};
%аксиомам V1---V4 --- {\it коммутативной}, или {\it абелевой} группой.


Из аксиом следуют привычные правила, которыми мы пользуемся, скажем, при работе с векторами в геометрии.
Например, аксиомы V1 и V4 обеспечивают то, что результат вычисления
суммы векторов $\sum\limits_{i=1}^{k} \vek{a}_i$ 
%линейной комбинации $\sum\limits_{i=1}^{k} \lambda_i \vek{a_i}$ 
не зависит от порядка выполнения операций.
Отметим следствия аксиом V1---V4.

\begin{predl}\label{sled_aksiom}
%1. Результат вычисления
%линейной комбинации $\sum\limits_{i=1}^{k} \lambda_i \vek{a_i}$ не зависит от порядка выполнения операций;
1). Нулевой вектор единственный;\\
2). $\forall$ $\vek{a}\in V$ противоположный ему вектор единственный;\\
3). (закон сокращения)
$\vek{a}+\vek{b}=\vek{a}+\vek{c}$
$\Leftrightarrow$ $\vek{b}=\vek{c}$ \,\,\,\,\,\, ($\forall$ $\vek{a}, \vek{b}, \vek{c} \in V$).
\end{predl}
\dok
1). Пусть $\vek{0}$ и $\vek{0'}$ --- два нулевых вектора. Тогда $\vek{0} = \vek{0} + \vek{0'}= \vek{0'}$, то есть $\vek{0}$ и $\vek{0'}$ совпадают.
\\
2). Пусть векторы $\vek{x}$ и $\vek{y}$ оба являются противоположными для вектора $\vek{a}$.  
Тогда $\vek{x} = \vek{x} + \vek{0}  = \vek{x} + (\vek{a} + \vek{y})  = (\vek{x} + \vek{a}) + \vek{y} = \vek{0}+\vek{y} \hm=\vek{y}$, 
то есть $\vek{x}$ и $\vek{y}$ совпадают.
\\
3). $\vek{a}+\vek{b}=\vek{a}+\vek{c}$ $\Rightarrow$ $-\vek{a}+(\vek{a}+\vek{b})=-\vek{a}+(\vek{a}+\vek{c})$
$\Rightarrow$ $(-\vek{a}+\vek{a})+\vek{b}=(-\vek{a}+\vek{a})+\vek{c}$
$\Rightarrow$ $\vek{0}+\vek{b}=\vek{0}+\vek{c}$
$\Rightarrow$ $\vek{b}=\vek{c}$.
Обратное следствие очевидно.
\edok

\otstup
Аксиомы V5 и V6 можно назвать свойством линейности, или дистрибутивности (по векторам и по константам соответственно),
V7 --- <<условие нормировки>>. Зафиксируем еще некоторые следствия аксиом.

\begin{predl}\label{sled_aksiom1}
1) $0\cdot \vek{a}  = \lambda \cdot \vek{0}= \vek{0}$ \,\,\,\,\,\, ($\forall$ $\vek{a}\in V$, $\forall$ $\lambda\in \mathbb{F}$);\\
2) $-(\lambda \vek{a})=(-\lambda )\vek{a} = \lambda (-\vek{a})$ \,\,\,\,\,\, ($\forall$ $\vek{a}\in V$, $\forall$ $\lambda\in \mathbb{F}$);\\
3)  $\lambda(\vek{a}-\vek{b}) = \lambda \vek{a}- \lambda \vek{b}$ \,\,\,\,\,\, ($\forall$ $\vek{a}, \vek{b}\in V$, $\forall$ $\lambda\in \mathbb{F}$);\\
4)  $(\lambda - \mu) \vek{a} = \lambda \vek{a}- \mu \vek{a}$\,\,\,\,\,\, ($\forall$ $\vek{a} \in V$, $\forall$ $\lambda , \mu \in \mathbb{F}$) .
\end{predl}
\dok
1)  $0\cdot \vek{a} = (0+0)\cdot \vek{a} = 0\cdot \vek{a} + 0\cdot \vek{a}$. 
С другой стороны, $0\cdot \vek{a} =\vek{0}+0\cdot \vek{a}$. По закону сокращения 
(см. предложение \ref{sled_aksiom}, 3) $0\cdot \vek{a}=\vek{0}$.
Аналогично 
$\lambda \cdot \vek{0} = \lambda (\vek{0}+\vek{0}) = \lambda \cdot \vek{0} + \lambda \cdot \vek{0}$, откуда $\lambda \cdot \vek{0}=\vek{0}$.
\\
2). $(-\lambda )\vek{a} + \lambda \vek{a} = (-\lambda  + \lambda) \vek{a} = 0\cdot \vek{a}  = \vek{0}$, поэтому 
$(-\lambda )\vek{a}$ является противоположным для $\lambda \vek{a}$, то есть равен $-(\lambda \vek{a})$.
Аналогично 
$\lambda (-\vek{a}) + \lambda \vek{a} = \lambda (-\vek{a}+ \vek{a}) = \lambda \cdot \vek{0}  = \vek{0}$, поэтому 
$\lambda (-\vek{a}) = -(\lambda \vek{a})$.
\\
3). Вытекает из утверждения 2) и определения вычитания векторов: 
$\lambda(\vek{a}-\vek{b}) = \lambda(\vek{a}+(-\vek{b})) = \lambda \vek{a}+ \lambda (-\vek{b}) =   \lambda \vek{a}- \lambda \vek{b}$.
\\
4). Аналогично 3).
\edok


\subsection{Подпространства}

\defin{
Непустое подмножество $U$ векторного пространства $V$ называется {\it подпространством}, если
$\forall$ $\vek{a}, \vek{b} \in U$, $\forall$ $\lambda\in \mathbb{F}$ выполнено:\\
П1. $\vek{a}+\vek{b} \in U$;\\
П2. $\lambda \vek{a} \in U$.
}

Тот факт, что $U$ является подпространством в векторном пространстве $V$, будем обозначать $U\leq V$.

Свойства П1 и П2 означают, что подпространство является подмножеством, замкнутым относительно
операций сложения и умножения на число, тем самым, подпространство само является векторным
пространством относительно операций в объемлющем векторном пространстве.
Любое векторное пространство $V$ содержит {\it тривиальные} подпространства $V$ и $O= \{ \vek{0} \}$ ({\it нулевое} подпространство).

\begin{predl}\label{nul_podprostr}
Пусть $U\leq V$, тогда $\vek{0} \in U$.
\end{predl}
\dok Пусть $\vek{a} \in V$, тогда, согласно П2, $\vek{0} =0\cdot \vek{a} \in V$.
\edok

\begin{predl}\label{minus_podprostr}
Пусть $U\leq V$ и $\vek{a}\in U$ тогда $-\vek{a} \in U$.
\end{predl}
\dok Достаточно в П2 подставить $\lambda = -1$.
\edok

%{\footnotesize Види}
\otstup
Видим, что подпространтво --- это подгруппа абелевой группы $(V, +)$, 
замкнутая относительно умножения на константы.


\begin{predl}\label{cap}
Пересечение подпространств является подпространством.
\end{predl}
\dok Пусть $U_i\leq V$ для $i\in I$, где $I$ --- некоторое множество индексов; $U=\bigcap\limits_{i\in I} U_i$.
Проверим П1 для множества $U$ (П2 проверяется аналогично).

Пусть $\vek{a}, \vek{b} \in U$, тогда $\vek{a}, \vek{b} \in U_i$ ($\forall$ $i\in I$).
Так как $U_i$ --- подпространство, то $\vek{a}+ \vek{b} \in U_i$ ($\forall$ $i\in I$), тем самым $\vek{a}+ \vek{b} \in U$, что и требовалось.
\edok

\otstup

{\bf Упражнение.}
Пусть $U_1\leq V$, $U_2\leq V$. Докажите, что объединение $U_1\cup U_2$ является подпространством
$\Leftrightarrow$ $U_1\subset U_2$ или $U_2\subset U_1$.



\subsection{Линейные комбинации}

Пусть $\vek{a}_1, \vek{a}_2, \ldots, \vek{a}_k \in V$,
$\lambda_1, \lambda_2, \ldots , \lambda_k\in \mathbb{F}$.

\defin{
Сумма $\sum\limits_{i=1}^{k} \lambda_i \vek{a}_i$
 называется
{\it линейной комбинацией}\index{Линейная!комбинация} векторов
$\vek{a}_1, \vek{a}_2, \ldots, \vek{a}_k$ с {\it коэффициентами}\index{Коэффициент!линейной!комбинации}
$\lambda_1, \lambda_2, \ldots , \lambda_k$.
}

Линейная комбинация $\sum\limits_{i=1}^{k} \lambda_i \vek{a_i}$ называется {\it тривиальной}, 
если все ее коэффициенты равны 0, то есть $\lambda_1=\ldots = \lambda_k = 0$.
В противном случае линейная комбинация называется {\it нетривиальной}.\index{Линейная!комбинация!нетривиальная}
%Если $|\lambda_1|+|\lambda_2|+ \ldots +|\lambda_k| >0$
%(то есть хотя бы один из коэффициентов не равен $0$), то говорят, что
%линейная комбинация (\ref{eq7_2_1}) {\it нетривиальная}
Ясно, что тривиальная линейная комбинация равна $\vek{0}$.

Количество слагаемых $k$ иногда называют {\it длиной} линейной комбинации. 
Можно позволить себе работать и с пустой линейной комбинацией ($k=0$), полагая ее значение равным $\vek{0}$.

Линейную комбинацию условимся записывать также в следующем компактном виде:
$\sum\limits_{i=1}^{k} \lambda_i \vek{a}_i =
\bazis{a} \lambda$, где $\bazis{a} = (\vek{a}_1\,  \vek{a}_2\,  \ldots \, \vek{a}_k)$ --- строка векторов, а
$\lambda  = \stolbec{\lambda_1 \\ \lambda_2 \\ \vdots \\ \lambda_k}$ --- столбец коэффициентов.
Это согласуется с правилом умножения матриц (умножаем строку на столбец).

%{\it footnotesize Другой вариант сокращенной записи --- тензорное суммирование. }

В том случае, когда $\vek{b}\in V$ равен некоторой линейной комбинации векторов
$\vek{a}_1, \vek{a}_2, \ldots, \vek{a}_k$ говорят, что
$\vek b$ {\it раскладывается} по векторам $\vek{a}_1, \vek{a}_2, \ldots, \vek{a}_k$
или {\it линейно выражается} через векторы $\vek{a}_1, \vek{a}_2, \ldots, \vek{a}_k$.



\begin{predl}\label{podpr_lin_komb}
Пусть $U\leq V$. Тогда  $\forall$ $\vek{a}_1, \vek{a}_2, \ldots, \vek{a}_k \in U$ и
$\forall$  $\lambda_1, \lambda_2, \ldots , \lambda_k\in \mathbb{F}$:
 $\sum\limits_{i=1}^{k} \lambda_i \vek{a}_i \in U$.
\end{predl}
\dok
Достаточно многократно применить П1 и П2
\edok

\otstup

Таким образом, условие замкнутости относительно взятия любой линейной комбинации %из последнего предложения 
эквивалентно определению подпространства (в П1 и П2 мы видим частные случаи линейных комбинаций: $\vek{a}+\vek{b}$ и $\lambda \vek{a}$).


\subsection{Линейная оболочка}

Важное понятие линейной оболочки, которое определим ниже, 
 позволяет, в частности, конструировать подпространства.

\defin{
{\it Линейной оболочкой} системы векторов $\vek{a}_1, \ldots, \vek{a}_k$ 
называется множество всех векторов, которые линейно выражаются через $\vek{a}_1, \ldots, \vek{a}_k$.
}

Линейная оболочка векторов $\vek{a}_1, \ldots, \vek{a}_k$ обозначается  $\lin{\vek{a}_1, \ldots, \vek{a}_k}$.
Формальная запись опеделения: \\
$\lin{\vek{a}_1, \ldots, \vek{a}_k} : = \left\{ \sum\limits_{i=1}^{k} \lambda_i \vek{a}_i \, | \,
 \lambda_1, \lambda_2, \ldots , \lambda_k \in \mathbb{F} \right\}$.

Несложно распространить определение на произвольные (в том числе бесконечные) системы векторов.

\defin{
{\it Линейной оболочкой} системы векторов $\mathcal{A}$ 
называется множество всех векторов, каждый из которых линейно выражается через несколько векторов из $\mathcal{A}$.\\
}

%Линейная оболочка множества $\mathcal{A}$ обозначается $\lin{\mathcal{A}}$.
Обозначение линейной оболочки: $\lin{\mathcal{A}}$. Очевидно, для любого подмножества
$\mathcal{A}\subset V$ выполнено $\mathcal{A}\subset \lin{\mathcal{A}}$.
Условимся считать, что $\lin{\varnothing} = O$.

Отметим, что везде рассматриваются линейные комбинации из конечного 
количества слагаемых, хотя длина линейной комбинации не ограничивается.
Формально, \\
$\lin{\mathcal{A}} = \left\{ \sum\limits_{i=1}^{k} \lambda_i \vek{a}_i \, | \,
k\in \mathbb{Z}^+, \, \vek{a}_1, \vek{a}_2, \ldots , \vek{a}_k \in \mathcal{A},
\, \lambda_1, \lambda_2, \ldots , \lambda_k \in \mathbb{F}\right\}$.


\begin{predl}\label{lin_ob}
$\lin{\mathcal{A}}$ является подпространством
для любой системы векторов $\mathcal{A}$.
\end{predl}
\dok Проверим для $\lin{\mathcal{A}}$ свойство П1 (П2 проверяется аналогично). 
Пусть $\vek{a}, \vek{b} \in \lin{\mathcal{A}}$. Это означает, что 
$\vek{a} = \sum\limits_{i=1}^{k} \alpha_i \vek{a}_i$, $\vek{b} = \sum\limits_{j=1}^{m} \beta_j \vek{b}_j$, 
где $\vek{a}_i, \vek{b}_j \in \mathcal{A}$, $\alpha_i, \beta_j \in \mathbb{F}$. Тогда 
$\vek{a} + \vek{b}= \sum\limits_{i=1}^{k} \alpha_i \vek{a}_i+ \sum\limits_{j=1}^{m} \beta_j \vek{b}_j$.
В правой части линейная комбинация длины $k+m$ векторов из $\mathcal{A}$, значит 
$\vek{a} + \vek{b}\in \lin{\mathcal{A}}$.
\edok

\otstup


Пусть подмножество $\mathcal{A}\subset V$ и подпространство $U\leq V$ таковы, что $\mathcal{A}\subset U$.
Предложение \ref{podpr_lin_komb} показывает, что тогда и $\lin{\mathcal{A}} \subset U$. 
Таким образом, предложение \ref{lin_ob} по сути означает, что %для любого подмножества $\mathcal{A}\subset V$ его линейная оболочка 
$\lin{\mathcal{A}}$ --- это минимальное (по включению) подпространство, 
содержащее $\mathcal{A}$.
\otstup

{\bf Упражнение.}
Подмножество $\mathcal{A}\subset V$ является подпространством 
$\Leftrightarrow$ $\mathcal{A} =\lin{\mathcal{A}}$.

%Ясно, что $\mathcal{A} \subset \lin{\mathcal{A}}$, причем  $\mathcal{A} = \lin{\mathcal{A}}$)
%$\Leftrightarrow$ $\mathcal{A}$ является подпространством.


%В частности, $\lin{\mathcal{A}} = \lin{\lin{\mathcal{A}}}$.
%упр. на подпространство: (аксиома замыкания...

%Упражнение.
%Отметим также, что если $\mathcal{B} \subset \lin{\mathcal{A}}$,
%то $\lin{\mathcal{B}} \subset \lin{\mathcal{A}}$.

%и вообще --- выполнены аксиомы операции замыкания...
%это потом используется в теореме о рангах... --- дважды линейно выражать...


\subsection{Примеры}

\example{I. %(Геометрия)
Пусть $O$ --- фиксированная точка геометрического пространства (начало отсчета).
%Геометрический пример векторного пространства --- р
Множество %$V_3$ 
всех радиус-векторов (с обычными операциями сложения векторов и умножения на число) --- пример векторного пространства над $\mathbb{R}$.
Радиус-вектор можно отождествить с точкой пространства --- концом этого радиус-вектора.
Тогда нетривиальные подпространства --- это прямые и плоскости, проходящие через $O$.\\
Если векторы $\vek{a}_1, \ldots, \vek{a}_k$ коллинеарны и среди них есть хотя бы один ненулевой, то $\lin{\vek{a}_1, \ldots, \vek{a}_k}$ 
--- это прямая. \\
Если же векторы $\vek{a}_1, \ldots, \vek{a}_k$ компланарны, но не коллинеарны, то $\lin{\vek{a}_1, \ldots, \vek{a}_k}$ 
--- это плоскость.
}

\example{%II. (Матрицы)\\
II.1. Множество $\mathbf{M}_{m\times n}(\mathbb{F})$ матриц $m\times n$ (заполненных константами)
--- векторное пространство относительно операций сложения матриц и умножения на число.\\
$\mathbb{F}^n$ отождествляем с множеством столбцов $\mathbf{M}_{n\times 1}(\mathbb{F})$.\\
Пусть
$\mathbf{M}_{n\times n}^{+} (\mathbb{F})= \{A\in \mathbf{M}_{n\times n}\,|\, A^T=A \}$ ---
множество симметричных матриц $n\times n$,\\
$\mathbf{M}_{n\times n}^{-} (\mathbb{F}) = \{A\in \mathbf{M}_{n\times n}\,|\, A^T=-A \}$ --- множество 
кососимметричных матриц $n\times n$.\\
Тогда $\mathbf{M}_{n\times n}^{+} (\mathbb{F}) 
\leq \mathbf{M}_{n\times n}(\mathbb{F})$, $\mathbf{M}_{n\times n}^{-} (\mathbb{F})\leq \mathbf{M}_{n\times n} (\mathbb{F})$.
}
\example{II.2.
Для матрицы $A\in \mathbf{M}_{m\times n}$ рассмотрим  однородную систему линейных уравнений (СЛУ) $AX=O$ (где $X\in \mathbf{M}_{n\times 1}$ --- столбец неизвестных)
 и множество ее решений $U= \Sol(AX=O)$. Тогда $U\leq \mathbf{M}_{n\times 1}$.
% --- подпространство в векторном пространстве $\mathbb{F}^n = M_{n\times 1}$. 
\\
Если $A_iX=O$, $i=1, 2, \ldots , k$ --- несколько однородных СЛУ относительно $X\in \mathbf{M}_{n\times 1}$, 
то СЛУ, представляющая собой объединение этих систем, имеет в качестве множества решений пересечение $\bigcap\limits_{i=1}^{k} \Sol(A_iX=O)$.
}

\example{%III. (Функции)\\
III.1. Пусть $S$ --- некоторое множество. Множество  
$\mathbf{F}(S)$ всех функций $f: S \to \mathbb{R}$ --- векторное пространство над $\mathbb{R}$
относительно операций сложения функций и умножения функции на число.\\
Пусть $\mathbf{F}^{+}(\mathbb{R}) = \{f\in \mathbf{F}(\mathbb{R}) \, |\, \forall\, x\in \mathbb{F} \,\,f(-x)=f(x)\}$ --- 
множество всех четных функций, аналогично\\
$\mathbf{F}^{-}(\mathbb{R}) = \{f\in \mathbf{F}(\mathbb{R}) \, |\, \forall\, x\in \mathbb{R} \,\, f(-x)=-f(x)\}$ --- 
множество всех нечетных функций. \\Тогда 
$\mathbf{F}^{+}(\mathbb{R})\leq \mathbf{F}(\mathbb{R})$, $\mathbf{F}^{-}(\mathbb{R})\leq \mathbf{F}(\mathbb{R})$.
}
\example{III.2.
Если $S$ --- интервал числовой прямой, имеется цепочка вложенных подпространств
$\mathbf{F}(S)\geq \mathbf{C}(S) \geq \mathbf{C}^1(S) \geq \ldots \geq \mathbf{C}^k(S) \geq \ldots \geq \mathbf{C}^{\infty}(S) \geq 
\mathbf{P}(S)$, \\где 
$\mathbf{C}(S)$ --- множество непрерывных функций $S \to \mathbb{R}$, 
$\mathbf{C}^k(S)$ --- множество функций, имеющих непрерывную $k$-ю производную,
$\mathbf{P}(S)$ --- множество полиномиальных функций (многочленов).\\ 
Множество (формальных) многочленов $\mathbf{P}$ может быть задано линейной оболочкой: 
$\mathbf{P} = \lin{1, x, x^2, x^3, \ldots}$.\\
$\mathbf{P} \geq \mathbf{P}_n=\lin{1, x, x^2, \ldots, x^n}$, где 
$\mathbf{P}_n$ --- множество многочленов $f$ степени $\deg f\leq n$.
}
\example{III.3.
В пространстве $\mathbf{F}(\mathbb{N})=\{(f_1, f_2, \ldots) \, | \, f_i\in \mathbb{R}\}$ 
всех последовательностей есть подпространство ограниченных последовательностей, а в нем --- подпространство сходящихся 
последовательностей.\\
<<Бесконечная система линейных уравнений>> $f_{i+1}=f_i+f_{i-1}$, $i=2, 3, \ldots$, задает подпространство {\it фибоначчиевых} последовательностей.
}
\example{III.4.
Рассмотрим линейное однородное дифференциальное уравнение $x^{(n)}+ a_{n-1}(t)x^{(n-1)}+\ldots + a_0(t)x=0$ относительно неизвестной функции 
$x(t)$ (где $a_i(t)$ --- данные непрерывные функции $\mathbb{R}\to \mathbb{R}$). 
Множество решений этого уравнения --- подпространство в пространстве всех  
 функций $\mathbb{R}\to \mathbb{R}$.
%Подпространствами (в соответсвующих пространствах) являются и множество решений однородной системы диффуров, УРЧПов
}

\example{IV.
Если $\mathbb{F}$ --- подполе некоторого поля $\mathbb{K}$, то операции в $\mathbb{K}$
индуцируют на $\mathbb{K}$ структуру векторного пространства над $\mathbb{F}$.
%размерность --- степень расширения. --- позже или можно не говорить
}
