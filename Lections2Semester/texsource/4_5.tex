\section{Билинейные формы в евклидовом пространстве}

\subsection{Связь между билинейными формами и преобразованиями}

Пусть $\mathcal{E}$ --- евклидово (унитарное) пространство. 
Каждому линейному оператору  $\varphi\in L(\mathcal{E}, \mathcal{E})$
сопоставим отображение $\beta _{\varphi} : \mathcal{E} \times \mathcal{E} \to \mathbb{R}$ по формуле 

$$\boxed{\beta _{\varphi} (\vek{a}, \vek{b}) = (\vek{a}, \varphi(\vek{b}))}. \eqno(*)$$ 


\begin{predl}\label{dd} 
$\beta _{\varphi}$, определенное $(*)$ --- билинейная (полуторалинейная) форма.
\end{predl}
\dok Непосредственно проверяется, с использованием линейности $\varphi$.
\edok

\begin{predl}\label{oper_bilin} 
Пусть $\dim \mathcal{E}<\infty$, 
$\varphi\in L(\mathcal{E}, \mathcal{E})$, $\bazis{e}$ --- некоторый ОНБ и
$\varphi \rsootv{\bazis{e}, \bazis{e}} A$. 
Тогда $\beta_{\varphi} \rsootv{\bazis{e}} \overline{A}$. 
\end{predl}
\dok Запишем в координатах (как обычно, полагая $\vek{a} = \bazis{e}X$, $\vek{b} = \bazis{e}Y$):\\
$\beta (\vek{a}, \vek{b}) = (\vek{a}, \varphi(\vek{b})) = X^T\overline{AY} = X^T\overline{A}\overline{Y}}$.
Мы видим координатную запись билинейной формы с матрицей $B=\overline{A}$. Соглаасно предложению..., все доказано.
\edok


Мы видим, что сопоставление $\varphi \to \beta_{\varphi}$  согласуется матрицами в ОНБ.
Это объясняется также и тем, что законы преобразования при переходе от ОНБ к ОНБ
 (с некоторой ортогональной (унитарной) матрицей перехода $S$)
для матрицы  оператора $A$ и 
(комплексно-сопряженной) матрицы билинейной формы $\overline{B}$
 выглядят одинаково:
$A\to S^{-1}AS$, $\overline{B} \to \overline{S^{T}}\overline{B}S$.


\begin{sled1}\label{111} 
Пусть $\dim \mathcal{E}<\infty$. 
Сопоставление $\varphi \to \beta_{\varphi}$ задает биекцию 
$L(\mathcal{E}, \mathcal{E}) \to \mathcal{B} (\mathcal{E})$. %(изоморфизм, в C не совсем)
\end{sled1}


\begin{sled2}\label{2111} 
Пусть $\dim \mathcal{E}<\infty$. 
Сопоставление $\varphi \to \beta_{\varphi}$ задает биекцию 
между множеством самосопряженных операторов и 
$\mathcal{B}_{sym} (\mathcal{E}) $ (или $\mathcal{K} (\mathcal{E})$). %(изоморфизм, в C не совсем)
\end{sled2}

%Квадратичную (эрмитову) форму, порожденную $\beta_{\varphi}$, можно обозначить
%$k_{\varphi}$.


Таким образом, сопоставление 
 $\varphi \to \beta_{\varphi}$ дает возможножность 
сводить изучение билинейных форм к изучению операторов и наоборот.
В частности, изучение квадратичных (эрмитровых) форм может быть сведено
к изучению самосопряженных операторов.

%Убедимся, что эта формула дает естественное соответствие между билинейными формами и операторами, 
%поэтому изучение билинейных форм можно сводить к изучению операторов и наоборот.
%ссылка на тензоры и естественные изоморфизмы

%в обратную сторону тоже соответствие работает (напр. у Винберга) ---
% нахождение с.в. через максимизацию значения кв. функционала на сфере.

Иногда термины {\it положительная определенность} или {\it положительность} 
мы будет применять и к самосопряженным преобразованиям (имея в виду указанную биекцию).

%задача ---- верно ли, что если $G$ пол. определена, то и $G^{-1}$ тоже?

%критерий положительной определенности в терминах корней хар. многочлена

%еще про пол. опрделенность...

\subsection{Приведение к главным осям}



\begin{theor}\label{t10_7_1} 
Пусть $k$ --- квадратичная форма в евклидовом (унитарном) пространстве $\mathcal{E}$.
%или билин. симметрическая
Тогда существует ОНБ, в котором $k$ имеет диагональный вид.
\end{theor}
\dok СХЕМА: ссылаемся на соответсвующую теорему о с/с операторах.
\edok

\otstup

{\bf Упражнение.}
%Критерий положительной определенности: 
Пусть дана квадратичная форма $k$ на конечномерном  пространстве $V$ (без евклидовой структуры).
 Пусть $k$ в некотором базисе имеет матрицу $B$. Тогда $k$ положительно определена 
$\Leftrightarrow$ все корни уравнения $|B-\lambda E|=0$ положительны. 

%Это можно было (по крайней мере в одну сторону) и раньше доказать из матричной записи 
%$X^TBX $ и рассмотрения с.в.

Задачу нахождение указанного диагонального вида и соответствующего ОНБ иногда называют
{\it приведением формы к главным осям}.

%ЧЕРЕЗ присоедин. преобразование? (пример подъема индекса?) или через матрицы?

\subsection{Приложение к классификации кривых второго порядка}


\begin{predl}\label{} 
Пусть  $\ell$ --- кривая второго порядка на плоскости. Тогда существует ПДКС $Oxy$, в котором 
$\ell$ задается уравнением $Ax^2+By^2+2Dx+2Ey+F=0$.
\end{theor}
\dok 
\edok


\begin{predl}\label{} 
Пусть  $\ell$ --- поверхность второго порядка в пространстве. Тогда существует ПДКС $Oxyz$, в котором
$\ell$ задается уравнением $Ax^2+By^2+Cz^2+2Dx+2Ey+2Fz+G=0$.
\end{theor}
\dok 
\edok




\subsection{Пара форм в  векторном пространстве $V$ (без евклидовой структуры)}

\begin{sled}[Теорема о паре форм]
Пусть в векторном  пространстве $V$ (без  евклидовой или унитарной структуры) заданы 
 две формы $\beta \in \mathcal{B}_{sym}$ и $g \in \mathcal{B}_{sym}$,
причем $g$ положительно определена. Тогда существует базис, в котором
$g$ имеет канонический вид, а $\beta$ имеет диагональный вид (в частности, обе формы 
$\beta$ и $g$ имеют диагональный вид).
\end{sled}
\dok 
Зафиксируем билинейную (полуторалинейную) симметричную форму, которая порождает $g$, как скалярное произведение.
При этом $V$ превратилось в евклидово (унитарное) пространство. Теперь достаточно воспользоваться теоремой \ref{t10_7_1}, поскольку ОНБ --- это базис, в котором $g$ имеет канонический вид (см. следствие 2 предложения \ref{p10_2_2}).
\edok

\otstup
{\bf АЛГОРИТМ} приведения пары форм к диагональному виду.

Пусть $\beta\rsootv{\bazis{e}} B$ и 
$g \rsootv{\bazis{e}} G$. Как и в доказательстве теоремы, считаем 
$V$ евклидовым, где скалярное произведение задается как $(\vek{a}, \vek{b}) = g(\vek{a}, \vek{b})$;
$G$ --- матрица Грама данного базиса $\bazis{e}$. Нам требуется 
найти базис $\bazis{e}'$, для которого $\beta \rsootv{\bazis{e}'} diag$ и 
$g \rsootv{\bazis{e}'} E$. 

Переформулировка (с учетом предложения \ref{{oper_bilin}):
требуется найти ОНБ, в котором самоспряженное преобразование $\varphi$, 
связанное с $\beta$ (так, что $\beta=\beta_{\varphi}$), имеет диагональный вид. 
(Задача сведена к известной).

Равенство $(*)$ расписывается в координатах в базисе $\bazis{e}$ 
как $X^TB\overline{Y} = X^TG (\overline{A}\overline{Y} )$, где 
$\varphi \rsootv{\bazis{e}, \bazis{e}} A$. Отсюда (в силу...)
$B= G \overline{A}$ и $A = \overline{G^{-1}B}$. Далее 
проделываются шаги алгоритма нахождения ОНБ, в котором самосопряженное преобразование имеет диагональный вид.


I. (c.з. для $\varphi$).

$|\overline{G^{-1}B}-\lambda E| =0$, или эквивалентный вид (домножив на $|G|$ и используя, что $\lambda \in \mathbb{R}$), 

$|B-\lambda G| =0$.

После нахождений $\lambda _i$ (с кратностями $s_i$) уже известен диагональный вид 
$diag$.

II. (c.подпр. для $\varphi$).

СЛУ $(\overline{G^{-1}B}-\lambda _i E)X =0$ $\Leftrightarrow$
$(\overline{B-\lambda _i G})X =0$. Поэтому:

$V_{\lambda_i} = \Sol ((\overline{B-\lambda _i G})X =0)X =0)$.

ФСР  = базис $V_{\lambda_i}$.

III. Внутри каждого $V_{\lambda_i}$ 

ортогонализуем базис,

а затем нормируем базис.

 (применяя формулы, помним, что $(\vek{a}, \vek{b}) = X^TG\overline{Y}$,
т.е. матрица Грама = $G$).



%неочевидно, что $\lambda$

\otstup

{\bf Задача.}
Приведите пример пары квадратичных форм, для которых не существует базиса,
в котором они обе имеют диагональный вид.

Указание. Можно подобрать формы, для которых инвариант 
$|F-\lambda G|$ имеет  невещественный корень.
%$\det (F-\lambda G)$ имеет комплексные корни



