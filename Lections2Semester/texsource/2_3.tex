\section{Образ и ядро}


\subsection{Образ}

В предложении \ref{p8_1_103} мы видели, что при линейном отображении $\varphi \in L(V, \widetilde{V})$ образ $\varphi (U)$ подпространства $U\leq V$
является подпространством в  $\widetilde{V}$.
Для образа отображения $\varphi $ наряду с обозначением $\varphi (V)$ используют обозначение $\Im \varphi$.
Очевидно, $\varphi \in L(V, \widetilde{V})$ является сюръективным $\Leftrightarrow$ $\widetilde{V} = \Im \varphi$.

%\begin{theor}\label{t8_2_1}
%Пусть $\varphi \in L(V, \widetilde{V})$, и $\bazis{e} = (\vek{e}_1, \vek{e}_2, \ldots , \vek{e}_n)$
%--- базис в  $V$. Тогда 
%$\Im \varphi = \lin{\varphi(\vek{e}_1), \varphi(\vek{e}_2), \ldots , \varphi(\vek{e}_n) }.$
%\end{theor}
%\dok  Следует из \ref{p8_1_101}.
%\edok
%%ЭТО ПОВТОР!!

\begin{theor}[координатное описание образа]\label{t8_2_111}
Пусть $\varphi \in L(V, \widetilde{V})$,  $\bazis{e}$, $\bazis{f}$ --- базисы в $V$ и $\widetilde{V}$.
Пусть $\varphi \rsootv{\bazis{e}, \bazis{f}} A$.
Для вектора $\vek{b} \in \widetilde{V}$, $\vek{b}=\bazis{f}Y$ выполнено:
$\vek{b} \in \Im \varphi$ $\Leftrightarrow$ $Y\in \lin{a_{\bullet 1}, a_{\bullet 2}, \ldots, a_{\bullet n}}$.
\end{theor}
\dok  Следует из сопоставления определения матрицы линейного отображения и предложения \ref{p8_1_103}.
\edok

Иначе говоря, теорема утверждает, что в терминах координатных столбцов $\Im \varphi$ задается 
как линейная оболочка столбцов матрицы линейного отображения.
Это, в частности, дает еще одно объяснение того факта, что 
$\rg A$ не зависит от выбора базисов в пространствах $V$ и $\widetilde{V}$.

\begin{sled}
В условиях теоремы $\boxed{\dim \Im \varphi = \rg A} $.
\end{sled}


\subsection{Ядро}

Покажем вначале, что при линейном отображении полный прообраз подпространства является подпространством.

\begin{predl}\label{proobr}
Пусть $\varphi \in L(V, \widetilde{V})$ и  $\widetilde{U}\leq \widetilde{V}$.
Тогда 
%1. $\varphi (U)\leq \widetilde{V}$; \\
$\{ \vek{a} \in V \, | \, \varphi (\vek{a}) \in \widetilde{U} \} \leq V$.
%$\varphi ^{-1}(\widetilde{U})$ 
%Полный прообраз подпространства $\widetilde{V}$
%--- подпространство в $V$.
\end{predl}
\dok Положим $U=\{ \vek{a} \in V \, | \, \varphi (\vek{a}) \in \widetilde{U} \}$ и проверим для $U$ свойства П1 и П2.\\
Пусть $ \vek{a}, \vek{b}$ --- произвольные векторы из $U$. Тогда 
$\varphi(\vek{a})\in \widetilde{U}$, $\varphi(\vek{b})\in \widetilde{U}$. Из свойства П1 для 
$\widetilde{U}$ получаем, что $\varphi(\vek{a})+\varphi(\vek{b})\in \widetilde{U}$, то есть
$\varphi(\vek{a}+\vek{b})\in \widetilde{U}$. Но последнее включение означает, что $\vek{a}+\vek{b}\in U$.\\
П2 проверяется аналогично.
\edok

\defin{{\it Ядром} линейного отображения $\varphi \in L(V, \widetilde{V})$
называется  подмножество $\{ \vek{a} \in V \, | \, \varphi (\vek{a}) = \vek{0}\}$.
}

Обозначение для ядра --- $\Ker \varphi$. %Для образа отображения $\varphi $ наряду с обозначением $\varphi (V)$ используют обозначение $\Im \varphi$.
Опеределение можно переформулировать так: $\Ker \varphi$ --- это полный прообраз нулевого подпространства,
поэтому из предложения \ref{proobr} вытекает


\begin{sled}\label{Ker}
Если $\varphi \in L(V, \widetilde{V})$, то $\Ker \varphi \leq V$. %и $\Im \varphi \leq \widetilde{V}$. % --- подпространства в $V$ и $\widetilde{V}$ соответственно.
\end{sled}




\otstup 

Для выяснения, является ли $\varphi$ инъективным, полезен следующий критерий.

\begin{predl}\label{p8_2_2}
$\varphi \in L(V, \widetilde{V})$ инъективно $\Leftrightarrow$ $\Ker \varphi = O$.
\end{predl}
\dok 
Следующее доказательство похоже на доказательство пункта 2 теоремы \ref{t8_1_1}.\\
\dokright Пусть $\vek{a}\in \Ker \varphi $, то есть $\varphi  (\vek{a})=\vek{0}$.
Но также $\varphi  (\vek{0})=\vek{0}$, и поскольку $\varphi$ инъективно, имеем $\vek{a}=\vek{0}$.\\
\dokright Пусть $\varphi  (\vek{a})=\varphi  (\vek{b})$. Тогда $\varphi  (\vek{a}-\vek{b})=\vek{0}$.
С учетом $\Ker \varphi = O$ имеем $\vek{a}-\vek{b}=\vek{0}$, то есть $\vek{a}=\vek{b}$.
Тем самым, $\varphi  (\vek{a})=\varphi  (\vek{b})$ $\Rightarrow$ $\vek{a}=\vek{b}$, и инъективность доказана.
\edok


\begin{theor}[координатное описание ядра]\label{t8_2_122}
Пусть $\varphi \in L(V, \widetilde{V})$,  $\bazis{e}$, $\bazis{f}$ --- базисы в $V$ и $\widetilde{V}$.
Пусть $\varphi \rsootv{\bazis{e}, \bazis{f}} A$.
Для вектора $\vek{a} \in V$, $\vek{a}=\bazis{e}X$ выполнено:
$\vek{a} \in \Ker \varphi$ $\Leftrightarrow$ $AX =O$. 
\end{theor}
\dok  Достаточно сопоставить определение ядра с  формулой $Y=AX$ (см. теорему \ref{t8_3_1}).
\edok

\otstup

Иначе говоря, теорема утверждает, что в терминах координатных столбцов $\Ker \varphi$ задается как 
общее решение  $\Sol (AX=O)$ однородной СЛУ с матрицей коэффициентов $A$.

\begin{sled}
В условиях теоремы $\boxed{\dim \Ker \varphi = n - \rg A}$.
\end{sled}




\subsection{Связь между размерностями ядра и образа}

Установим формулу, связывающую размерности  ядра и образа.


\begin{theor}\label{t8_2_2}
Пусть $\varphi \in L(V, \widetilde{V})$ и $\dim V =n < \infty $.
Тогда 
\begin{equation}\label{Ker_Im}
\boxed{\dim \Ker \varphi + \dim \Im \varphi = n}.
\end{equation}
\end{theor}
\dok Можно считать, что $\dim \widetilde{V}<\infty $ (иначе заменим $\widetilde{V}$ на любое конечномерное подпространство, содержащее $\Im \varphi$, 
замена  не влияет на $\Im$ и $\Ker $). 
Рассмотрим матрицу $A$ отображения $\varphi$ в некоторых базисах.
По следствию из теоремы \ref{t8_2_111} имеем $\dim \Im \varphi = \rg A$,
а по следствию из теоремы \ref{t8_2_122} имеем $\dim \Ker \varphi = n - \rg A$.
Отсюда вытекет нужная формула размерностей.
\edok

\begin{sled}\label{sootv_mezhdu_podpr}
Пусть $\varphi \in L(V, \widetilde{V})$, $\dim V<\infty $ и  $U\leq V$. Тогда
\begin{equation}\label{dimUphiU}
\dim U \leq \dim \varphi (U) +  \dim \Ker \varphi,
\end{equation}
причем равенство  достигается $\Leftrightarrow$
$U\geq \Ker \varphi$.
\end{sled}
\dok Ограничение $\varphi$ на $U$ (т.е. $\varphi  |_{U}  \in L(U, \widetilde{V})$),
обозначим $\psi$. Применив (\ref{Ker_Im}) к $\psi$,
имеем $ \dim \varphi (U) +  \dim \Ker \psi = \dim U $, при этом 
$\Ker \psi = (\Ker \varphi )\cap U \leq \Ker \varphi$,
откуда следуют нужные утверждения.
\edok

\otstup 

Отметим, что (\ref{Ker_Im}) находится в согласии с эквивалентностью условий 1), 2), 3) 
в теореме \ref{t_isom1}.

\otstup 

{\footnotesize
Мы доказали теорему \ref{t8_2_2}, перейдя c <<языка отображений>> на <<матричный язык>>. Более <<концептуально>> ---
получить формулу размерностей (\ref{Ker_Im}) как следствие 
{\it теоремы о гомоморфизме}  $V / \Ker \varphi \cong \Im \varphi$ (это стандартная теорема 
курса алгебры для многих алгебраических систем), которая вытекает из естественной биекции
$\vek{a}+ \Ker \varphi \mapsto \varphi (\vek{a})$.
Как следствие теоремы о гомоморфизме можно получить биекцию между подпространствами в $V$, содержашими $\Ker \varphi$, и подпространствами в $\Im \varphi$ (что соответсвует случаю равенства в (\ref{dimUphiU})).
}

\otstup 

В следующем упражнении намечен еще один путь доказательства формулы из теоремы \ref{t8_2_2} без привлечения матрицы линейного отображения.

\otstup

{\bf Упражнение.} Пусть $\varphi \in L(V, \widetilde{V})$ и $\dim V =n < \infty $.
Пусть $\vek{e}_{r+1}, \ldots, \vek{e}_{n}$ --- базис в $\Ker \varphi$. Дополним его до базиса 
$\vek{e}_{1}, \vek{e}_{2}, \ldots, \vek{e}_{n}$ пространства $V$. 
Докажите, что тогда $\Im \varphi = \lin{\varphi (\vek{e}_{1}), \ldots, \varphi (\vek{e}_{r})}$, причем 
$\varphi (\vek{e}_{1}), \ldots, \varphi (\vek{e}_{r})$ --- линейно независимая система.

\otstup

Предлагаем еще одно упражнение о простейшем виде матрицы линенйного отображения.

\otstup

{\bf Упражнение.} 
Пусть $\varphi \in L(V, \widetilde{V})$. Докажите, что в пространствах $V$ и $\widetilde{V}$ можно выбрать базисы $\bazis{e}$ и $\bazis{f}$
так, что %матрица отображения $\varphi$ будет иметь {\it простейший} вид
$$\varphi \rsootv{\bazis{e}, \bazis{f}} \begin{pmatrix}
E_r & O \\
O & O
\end{pmatrix}.$$
(Утверждение последнего упражнения не следует путать со случаем выбора одного и того же базиса $\bazis{e} = \bazis{f}$
для матрицы линейного преобразования, как обычно будет происходить в главе \ref{structure}.)




\subsection{Примеры}

Пусть $V = U_1 \bigoplus U_2$. Пусть  $\varphi: V\to V$ --- проектирование на $U_1$ вдоль $U_2$. 
Тогда $\Im \varphi = U_1$, $\Ker \varphi = U_2$.

%Отображение $\psi: V\to V$ такое, что $\varphi(\vek{a}) = \vek{a}_1 - \vek{a}_2$,
%называется {\it отражением}  (или {\it симметрией}) относительно $U_1$ вдоль $U_2$.

\example{III.1.
Пусть $V= \mathbf{C}^{\infty}(\mathbb{R})$. 
Общее решение $U$ линейного дифференциального уравнения
$x^{(n)}+a_{n-1}(t)x^{(n-1)}+\ldots + a_{1}(t)x' + a_0(t)x =0 $ (где $a_i(t)$ --- непрерывные функции $\mathbb{R}\to \mathbb{R}$) 
является ядром линейного дифференциального оператора $d^n+a_{n-1}d^{n-1}+\ldots +a_{1}d + a_0d^0 $.\\
Теорема существования и единственности из курса дифференциальных уравнений устанавливает изоморфизм
между решением $x\in U$ и столбцом {\it начальных условий } $\stolbec{x(t_0)\\ x'(t_0) \\ \ldots \\ x^{(n-1)}(t_0)}$,
тем самым, $\dim U = n$.
}

\example{III.2.
Пусть  $V=\mathbf{F}(\mathbb{N})=\{(a_1, a_2, \ldots) \, | \, a_i\in \mathbb{F}\}$ --- пространство числовых последовательностей.\\
Фибоначчиевы последовательности могут быть заданы как ядро линейного оператора $\varphi ^2- \varphi - \varphi ^0$, где 
$\varphi : V\to V$ --- оператор сдвига.
}



\subsection{Ядро и образ произведения отображений}

Начнем с двух почти очевидных предложений.



\begin{predl}
Пусть $\varphi \in L(V, \widetilde{V})$, $\psi \in L(\widetilde{V}, \widehat{V})$.
Тогда $$\Im (\psi \varphi ) \leq \Im \psi .$$
\end{predl}
\dok Это частный случай включения $\psi (U) \leq \psi(V)$ для $U= \varphi (V)$.
\edok


\begin{predl}\label{Ker_psi_phi}
Пусть $\varphi \in L(V, \widetilde{V})$, $\psi \in L(\widetilde{V}, \widehat{V})$.
Тогда $$\Ker (\psi \varphi ) \geq \Ker  \varphi .$$
\end{predl}
\dok Если $\vek{a}\in \Ker  \varphi$, то $\varphi (\vek{a}) = \vek{0}$
$\Leftrightarrow$  $(\psi \varphi)  (\vek{a}) = \psi (\varphi  (\vek{a})) = \vek{0}$
$\Leftrightarrow$ $\vek{a}\in \Ker (\psi \varphi )$.
\edok

\otstup

{\bf Упражнение.}
Переформулируйте и докажите теоремы об оценке ранга произведения матриц
$\rg (AB) \leq \rg A$, $\rg (AB) \leq \rg B$
в терминах линейных отображений.
%\\
%Получите доказательство теоремы о ранге произведения матриц, используя 
%линейные отображения.

\otstup
 


Зафиксируем еще несколько фактов, которые будем  использовать далее.

\begin{predl}\label{Ker_psi_phi_leq}
Пусть $\varphi \in L(V, \widetilde{V})$, $\psi \in L(\widetilde{V}, \widehat{V})$, $\dim V<\infty $.
Тогда 
\begin{equation}\label{Kerpsivarphileq}
\dim \Ker (\psi \varphi ) \leq \dim \Ker  \varphi + \dim \Ker  \psi.
\end{equation}
\end{predl}
\dok 
Пусть $U=\Ker (\psi \varphi ) $. Тогда $\varphi (U) \leq \Ker \psi$.
Согласно следствию из теоремы \ref{t8_2_2},  имеем 
$\dim U = \dim \varphi (U) +  \dim \Ker \varphi \leq  \dim \Ker  \psi + \dim \Ker  \varphi  $, 
что и требовалось.
\edok 


\otstup 
Отметим, что равенство в 
 (\ref{Kerpsivarphileq})
достигается $\Leftrightarrow$ $\varphi (U) = \Ker \psi $, или, поскольку $U$ является 
полным прообразом  $\Ker \psi $ относительно $\varphi$, эквивалентно, $ \Ker \psi \leq \Im \varphi$.

\otstup


{\bf Упражнение.}
а) Для матриц $A\in \mathbf{M}_{m\times n}$, 
$B\in \mathbf{M}_{n\times k}$ 
докажите {\it неравенство Сильвестра} $$\rg (AB) \geq \rg A +  \rg B - n.$$

б) Каков максимальный ранг матрицы $A\in \mathbf{M}_{n\times n}$, если $A^2=O$?

в)$^*$ Для матриц $A\in \mathbf{M}_{m\times n}$, 
$B\in \mathbf{M}_{n\times k}$,  $C\in \mathbf{M}_{k\times l}$ 
докажите {\it неравенство Фробениуса} (обобщающее неравенство Сильвестра) 
$$\rg (ABC) + \rg B \geq \rg (AB) +  \rg (BC).$$

\otstup


В качестве следствия  предложения \ref{Ker_psi_phi} имеем

\begin{predl}\label{sled_Ker_psi_phi}
Пусть $\varphi \in L(V, V)$, $\psi \in L(V, V)$,
причем $\varphi \psi = \psi \varphi $.
Тогда 
\begin{equation}\label{Kerpsiphigeq}
\Ker (\psi \varphi ) \geq \Ker  \varphi + \Ker  \psi .
\end{equation}
\end{predl}
\dok Так как $\Ker (\psi \varphi ) \geq \Ker  \varphi  $ и 
$\Ker (\psi \varphi )  = \Ker (\varphi \psi ) \geq \Ker  \psi  $, то \\
$\Ker (\psi \varphi ) \geq \lin{\Ker  \varphi, \Ker  \psi}  = \Ker  \varphi + \Ker  \psi$.
\edok

\begin{sled}\label{sledsled_Ker_psi_phi}
Пусть  $\varphi \in L(V, V)$, $\psi \in L(V, V)$, $\dim V <\infty$, 
причем $\varphi \psi = \psi \varphi $.
Если $\Ker  \varphi \cap  \Ker  \psi = O$, то 
$\Ker (\psi \varphi ) =\Ker  \varphi \oplus \Ker  \psi .$
\end{sled}
\dok Из (\ref{Kerpsiphigeq}), при условии $\Ker  \varphi \cap  \Ker  \psi = O$,
следует, что $\Ker (\psi \varphi ) \geq \Ker  \varphi \oplus \Ker  \psi $ и
$\dim \Ker (\psi \varphi ) \geq \dim \Ker  \varphi + \dim \Ker  \psi$.
Тогда из (\ref{Kerpsivarphileq}) вытекет требуемое.
\edok

\begin{predl}\label{Ker_polynom}
Пусть $\varphi \in L(V, V)$, а $f_1, \ldots, f_k\in \mathbb{F}[X]$ --- попарно взаимно простые многочлены.
Тогда $\sum\limits_{i=1}^k \Ker (f_i(\varphi) )$ --- прямая сумма. \\
Если, кроме того,   $\dim V<\infty$, то $\Ker (f_1(\varphi) ) \oplus \ldots \oplus 
\Ker (f_k(\varphi) ) = \Ker (f_1(\varphi)\ldots f_k(\varphi) )$.
\end{predl}
\dok Докажем в случае  $k=2$. Этого будет достаточно, так как случай произвольного $k$ 
можно разобрать,  применив утверждение для $k=2$ к многочленам
$f_1$ и $f_2\ldots f_k$, далее к $f_2$ и $f_3\ldots f_k$ и т.д.\\
Так как $f_1$ и $f_2$ взаимно просты, существуют многочлены 
$g_1, g_2\in \mathbb{F}[X]$  такие, что $g_1f_1+g_2f_2=1$.
Тогда из этого равенсва многочленов  имеем 
$g_1 (\varphi)f_1(\varphi)+g_2(\varphi)f_2(\varphi)=I_V$.\\
Применяя это операторное равенство для $\vek{a}\in \Ker (f_1(\varphi) ) \cap \Ker (f_2(\varphi) ) $, 
получаем $\vek{0}=\vek{a}$, значит 
$\Ker (f_1(\varphi) ) \cap \Ker (f_2(\varphi) ) =O$, тем самым утверждение про прямую сумму доказано.
\\
Второе утверждение сразу получаем в силу следствия из предложения \ref{sled_Ker_psi_phi}.
\edok



%Расширить  Ker произведения... Re
%не-ва для рангов.
% Ker p(f) + Ker q(f)

