
\section{Тензоры и основные операции над ними}

%(опуская <<совсем правильное>> определение, ...)

\subsection{Тензор как полилинейная функция}

\defin{
Тензором типа $(p,q)$ над векторным пространством $V$ называют
полилинейное отображение из множества
$\Hom( \underbrace{V^*, \ldots , V^*}_{p}, \underbrace{V, \ldots , V}_{q}; \mathbb{F})$.
}


Более заумное название тензора типа $(p,q)$: $p$ раз контравариантный и $q$ раз ковариантный.

Обозначение (помимо $\Hom( \underbrace{V^*, \ldots , V^*}_{p}, \underbrace{V, \ldots , V}_{q}; \mathbb{F})$):
$$T^p_q(V).$$ 
(или просто $T^p_q$, если ясно, о каком $V$ идет речь).

Для $\tau \in T^p_q(V)$ определено значение $\tau (\underbrace{\ell, \ldots }_{p}, \underbrace{\vek{a}, 
\vek{b}, \ldots }_{q})$ на упорядоченном наборе из $p$ линейных функционалах (ковекторах) и $q$ векторах.

$T^p_q(V)$ --- векторное пространство. \\
Что такое $T^0_0(V)$?

Выписывая компоненты в базисах, условимся во всех $q$ копиях $V$ выбирать один и тот же базис 
$\bazis{e} = (\vek{e}_1, \ldots, \vek{e}_n)$, а во всех 
$p$ копиях $V^*$ выбирать базис
$\bazis{e}^* = (\vek{e}^1, \ldots, \vek{e}^n)$, биортогональный базису $\bazis{e}$.
Компоненты пишем с $p$ верхними и $q$ нижними индексами, для возможности использовать тензорное суммирование.
 
Для $\tau \in T^p_q(V)$ соответствие (\ref{b_{ij..}}) приобретает вид:

$\tau \, \, \rsootv{\bazis{e}^*, \ldots , \bazis{e}, \bazis{e}, \ldots } \, \,  
t_{i_1i_2 \ldots i_q}^{j_1j_2\ldots j_p},  $
или (сокращая обозначения)
$$\tau \, \, \rsootv{\bazis{e}} \, \,  
t_{ik \ldots }^{j\ldots }.  $$

Массив из $n^{p+q}$ констант $t_{ik \ldots }^{j\ldots } = \tau 
(\underbrace{\vek{e}^j, \ldots  }_q, \underbrace{\vek{e}_i, \vek{e}_k, \ldots }_p, )$ --- 
компоненты тензора $\tau $  в базисе $\bazis{e}$.

Теперь формула (\ref{beta()}) приобретает вид
$$\tau (\underbrace{\ell, \ldots }_{p}, \underbrace{\vek{a}, 
\vek{b}, \ldots }_{q})  = t_{ik \ldots }^{j\ldots } x^iy^k \ldots z_j\ldots , $$
где $\vek{a} = x^i\vek{e}_i$, $\vek{b} = y^k\vek{e}_k$, \ldots , $\ell = z_j\vek{e}^j$, \ldots ---
разложения векторов ($p+q$ аргументов, к которым применяется $\tau$)  по базисам $\bazis{e}$ и 
$\bazis{e}^*$


Формула (\ref{zamena_tenz}) изменения компонент при переходе от базиса $\bazis{e}$ к 
$\bazis{e}'$ и соответственно от базиса $\bazis{e}^*$ к 
$\bazis{e}'^*$:

$$
\boxed{
{t'}_{ik \ldots }^{j\ldots }  = t_{i'k' \ldots }^{j'\ldots }  \, \,  s^{i'}_i s^{k'}_k\ldots r^j_{j'} \ldots  
},
$$
где (как, скажем, и в доказательства предложения \ref{p8_4_1} главы \ref{lin_funk}),
$S=(s^{i'}_i)$ --- матрица перехода от $\bazis{e}$ к 
$\bazis{e}'$, а $R=(r^j_{j'}) = S^{-1}$ --- транспонированная к матрице перехода от 
$\bazis{e}^*$ к $\bazis{e}'^*$ (транспонирование связано с нашей договоренностью о
верхних и нижних индексах при работе в $V^*$).




\subsection{Тензорное умножение}

\defin{
Тензорным произведением тензора  $\alpha \in T^p_q$ на 
тензор  $\beta \in T^{p'}_{q'}$
называют тензор $\gamma \in T^{p+p'}_{q+q'}$ такой, что
$$\gamma (\underbrace{\ell, \ldots }_{p}, \underbrace{m, \ldots }_{p'}, 
\underbrace{\vek{a}, \vek{b}, \ldots }_{q}, 
\underbrace{\vek{c}, \vek{d}, \ldots }_{q'}) \,  =      \, 
 \alpha (\underbrace{\ell, \ldots }_{p}, 
\underbrace{\vek{a}, \vek{b}, \ldots }_{q}) \, \cdot \, 
\beta (\underbrace{m, \ldots }_{p'}, \underbrace{\vek{c}, 
\vek{d}, \ldots }_{q'}).$$
}

Обозначение: $\alpha \otimes \beta$.

Ассоциативность, дистрибутивность относительно сложения. 

Нет коммутативности.

<<организуем>> алгебру
$$\bigoplus\limits_{q=0}^{\infty} T^0_q$$
%$$\bigoplus\limits_{p=0}^{\infty} T^p_0$$

%Тензорная %(градуированная) 
%алгебра $T(V)$%(ассоциативная 
$$\bigoplus\limits_{p, q} T^p_q .$$





В координатах --- <<независимое умножение>>. Пусть
$$\alpha \, \, \rsootv{\bazis{e}} \, \,  
a_{i_1i_2 \ldots i_q}^{j_1j_2\ldots j_p},  \,\,\, $$
$$\beta \, \, \rsootv{\bazis{e}} \, \,  
b_{i'_1i'_2 \ldots i'_{q'}}^{j'_1j'_2\ldots j'_{p'}}. $$
Тогда 
$$\gamma \, \, \rsootv{\bazis{e}} \, \,  
c_{i_1i_2 \ldots i_q\,  i'_1i'_2 \ldots i'_{q'}}^{j_1j_2\ldots j_p\, j'_1j'_2\ldots j'_{p'}} \,\,\, =
\,\,\, a_{i_1i_2 \ldots i_q}^{j_1j_2\ldots j_p}\,\, \cdot \,\, 
b_{i'_1i'_2 \ldots i'_{q'}}^{j'_1j'_2\ldots j'_{p'}}. $$

Разложимый тензор --- тензор, равный произведению тензоров валентности 1 (векторов и ковекторов).

\example{ Как действует тензор $\vek{e}^3 \otimes \vek{e}^1 \otimes \vek{e}^3 \in T^0_3$?\\
$(\vek{e}^3 \otimes \vek{e}^1 \otimes \vek{e}^3 )(\vek{e}_i, \vek{e}_j, \vek{e}_k) = 
\delta _i^3 \delta _j^1 \delta _k^3$.\\
$(\vek{e}^3 \otimes \vek{e}^1 \otimes \vek{e}^3 )(\vek{a}, \vek{b}, \vek{c}) = x^3y^1z^3$, 
где $ \vek{a} = x^i\vek{e}_i$, $ \vek{b} = y^j\vek{e}_j$, $ \vek{c} = z^k\vek{e}_k$. 
}

\otstup

Тензоры вида  $\underbrace{\vek{e}_j \otimes \ldots }_p\otimes 
\underbrace{\vek{e}^i \otimes \vek{e}^k \otimes \ldots }_q$
--- <<стандартный>> (т.е. согласованный с $\bazis{e}$) базис в $T^p_q$,
так что 
$$\tau  = t^{j\ldots}_{ik\ldots } \,\,
\underbrace{\vek{e}_j \otimes \ldots }_p\otimes 
\underbrace{\vek{e}^i \otimes \vek{e}^k \otimes \ldots }_q, $$
где 
$\tau \, \, \rsootv{\bazis{e}} \, \,  
t_{ik \ldots }^{j\ldots }.  $



\subsection{Свертка}

Зафиксируем некоторый базис и зададим {\it свертку }  (по паре первых аргументов) как линейное отображение 

$$T^p_q\to T^{p-1}_{q-1}$$

по следующему правилу. Достаточно определить значение свертки вначале на 
<<стандартном базисе>> $T^p_q(V)$, связанным с базисом $\bazis{e}$ пространства $V$
(далее продолжается по линейности):

\begin{equation}\label{svert}
\underbrace{\vek{e}_j \otimes \vek{e}_l \otimes \ldots }_p\otimes 
\underbrace{\vek{e}^i \otimes \vek{e}^k \otimes \ldots }_q
\mapsto \delta_j^i  \, \, \underbrace{\vek{e}_l \otimes \ldots }_{p-1}\otimes 
\underbrace{\vek{e}^k \otimes \ldots}_{q-1}.
\end{equation}


Из линейности легко следует, что 

$$\underbrace{ \vek{a} \otimes \vek{b} \otimes \ldots }_p \otimes 
\underbrace{\ell \otimes m \otimes \ldots }_q
\mapsto \lin{\vek{a}, \ell} \,\,
\underbrace{\vek{b} \otimes \ldots }_{p-1}\otimes \underbrace{m \otimes \ldots}_{q-1}.$$

В частности, это означает, что (\ref{svert}) работает и для
другого выбора базиса. 

В координатах переход к свертке выглядит просто: 
$$t_{ik\ldots }^{jl\ldots } \mapsto t_{ik\ldots }^{il\ldots }.$$


\subsection{Примеры.}


-- Значение  линейной формы на векторе: 
$\vek{a} \otimes \ell$ --- далее свертка (в результате $l_{i} x^i $).

\otstup

-- Значение  билинейной формы
$\vek{a} \otimes \vek{b} \otimes \beta  $ --- далее свертка (в результате $b_{ij} x^i y^j$)

\otstup

---Матрица $A=(a^i_j)$ линейного отображения $\varphi : V\to V$ --- 
тензор из $T^1_1$.\\
$\tr A  = a^i_i$ --- свертка.\\
Матрица произведения двух отображений:
$a^i_j b^j_k$.

\otstup

%Умноженивк в алгебре (свертка со структурным тензором)

--- В евклидовом пространстве $g_{ij}$ --- метрический Тензор (матрица Грама).\\
Тензорное домножение на $g_{ij}$ с последующей сверткой по одному из аргументов --- <<опускание индекса>>.\\
Например: $b_{jk} = a^i_j g_{ik}$.\\
(связь между операторами и билинейными формами в еквлидовом пространстве).

