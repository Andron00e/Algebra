\chapter{Линейные отображения}\label{lin_otobr}


В этой главе рассматриваются векторные пространства 
 $V$, $\widetilde{V}$ и т.д. над произвольным полем $\mathbb{F}$.
(При этом для обозначения операции сложения  в разных пространствах 
мы позволяем себе использовать один и тот же символ <<$+$>>; то же касается нулевых векторов, и т.д.)

В ситауциях, когда используется специфика поля $\mathbb{F}$, оговариваем это отдельно. 
(Например, для $\mathbb{F} = \mathbb{C}$ умножение линейного отображения на константу может быть определено
особым способом.)


\section{Определение. Операции над линейными отображениями. Изоморфизм}

\subsection{Определение и его следствия}


\defin{
Отображение $\varphi: V \to \widetilde{V}$ называется {\it линейным},
если $\forall$ $\vek{a}, \vek{b} \in  V$ и $\forall$ $\lambda\in \mathbb{F}$
выполняются равенства
\\
L1. $\varphi(\vek{a} + \vek{b}) = \varphi(\vek{a}) +  \varphi(\vek{b}) $,
\\
L2. $\varphi(\lambda \vek{a}) = \lambda  \varphi(\vek{a})$.
}


Линейное отображение также называют {\it гомоморфизмом} векторных пространств.
Множество всех линейных отображений $V \to \widetilde{V}$ обычно
обозначают $\Hom(V; \widetilde{V})$.
Мы чаще будем использовать чуть более короткое обозначение $L(V, \widetilde{V})$.

Линейное отображение $V \to V$ (т.е. в частном случае $\widetilde{V}=V$)
называют также  {\it линейным преобразованием} или {\it линейным оператором}.

Линейное отображение $V \to \mathbb{F}$
называют также  {\it линейным функционалом} или {\it линейной функцией}.
Это соответствует случаю $\dim \widetilde{V} = 1$, в этом случае 
$\widetilde{V}$ можно отождествить с полем констант~$\mathbb{F}$.


\example{
Примером линейного отображения является {\it нулевое} отображение $0: V \to \widetilde{V}$ такое, что
$\forall$ $\vek{a} \in  V$ выполнено $0(\vek{a})=\vek{0}$.
}



\begin{predl}\label{p8_1_1}
Пусть $\varphi \in L( V, \widetilde{V})$. Тогда $\forall$ $\vek{a}_i\in V$ и $\forall$ $\lambda_i\in \mathbb{F}$ выполнено
 \begin{equation}\label{eqLin}
\boxed{\varphi \left(\sum\limits_{i=1}^k \lambda _i \vek{a}_i \right) =
\sum\limits_{i=1}^k \lambda _i \varphi (\vek{a}_i)}.
\end{equation}
\end{predl}
\dok
Следует из многократного применения L1, L2.
\edok

\otstup

Отметим, что L1 и L2 представляют собой частные случаи фомулы (\ref{eqLin}).
Зафиксируем несложные, но важные следствия определения и предыдущего предложения.

\begin{predl}\label{p8_1_101}
Пусть $\varphi \in L( V, \widetilde{V})$. Тогда
\\
1). $\varphi (\vek{0}) = \vek{0}$;
\\
2). $\forall$ $\vek{a}\in V$ выполнено $\varphi(-\vek{a})=-\varphi(\vek{a})$.
\end{predl}
\dok
1). Следует из L2 для $\lambda= 0$.\\
2). Следует из L2 для $\lambda= -1$.
\edok

\begin{predl}\label{p8_1_102}
1). Если $\vek{a}_1, \vek{a}_2, \ldots , \vek{a}_k$  --- линейно зависимая система векторов, то
$\varphi(\vek{a}_1), \varphi(\vek{a}_2), \ldots , \varphi(\vek{a}_k)$
 --- тоже линейно зависимая система векторов.
%(имеется
%нетривиальная линейная комбинация, равная нулю, с тем же набором коэффициентов).
\\
2). $\forall$ $\mathcal{A}\subset V$ выполнено $\rg \varphi (\mathcal{A}) \leq \rg \mathcal{A}$.
\end{predl}
\dok
1).  Следует из предложения \ref{p8_1_1} и пункта 1) предложения \ref{p8_1_101}.\\
2). Следует из 1).
\edok

\begin{predl}[образ подпространства]\label{p8_1_103}
Пусть $\varphi \in L( V, \widetilde{V})$, $U\leq V$ и $U=\lin{\mathcal{A}}$. Тогда \\
1). $\varphi (U)\leq \widetilde{V}$, более того, $\varphi (U)=\lin{\varphi ( \mathcal{A} ) }$.
В частности, если $\vek{e}_1, \vek{e}_2, \ldots , \vek{e}_n$ --- базис в $V$, 
то (напомним, что $\varphi(V)$ также обозначается $\Im \varphi$)
$$\boxed{\Im \varphi =  \lin{\varphi(\vek{e}_1), \varphi(\vek{e}_2), \ldots , \varphi(\vek{e}_n)}}.$$
%В частности, если $U=\lin{\vek{a}_1, \vek{a}_2, \ldots , \vek{a}_k}$, то $\varphi(U)=\lin{\varphi(\vek{a}_1), \varphi(\vek{a}_2), \ldots , \varphi(\vek{a}_k)}$.
\\
2). $\dim \varphi(U) \leq \dim U$.
\end{predl}
\dok
1). Следует из предложения \ref{p8_1_1}.\\
2). Следует из 1) (и является частным случаем пункта 2 предложения \ref{p8_1_102}). 
\edok

\otstup

В следующем предложении отметим отдельно свойства линейного вложения (инъективного отображения).

\begin{predl}\label{p8_1_1000}
Пусть $\varphi \in L( V, \widetilde{V})$ и $\varphi$ инъективно. Тогда
\\
1). Если $\vek{a}_1, \vek{a}_2, \ldots , \vek{a}_k$  --- линейно независимая система векторов, то
$\varphi(\vek{a}_1), \varphi(\vek{a}_2), \ldots , \varphi(\vek{a}_k)$
 --- тоже линейно независимая система векторов.\\
2). $\forall$ $\mathcal{A}\subset V$ выполнено $\rg \varphi (\mathcal{A}) = \rg \mathcal{A}$.
В частности, для $U\leq V$ выполнено $\dim \varphi (U) = \dim U$.
\end{predl}
\dok
1). Пусть $\sum\limits_{i=1}^k \lambda _i \varphi (\vek{a}_i) = \vek{0}$, тогда
$\varphi \left(\sum\limits_{i=1}^k \lambda _i \vek{a}_i\right) = \vek{0}$. Но  $\varphi (\vek{0}) = \vek{0}$, 
поэтому в силу инъективности, $\sum\limits_{i=1}^k \lambda _i \vek{a}_i = \vek{0}$. Отсюда, поскольку 
$\vek{a}_1, \vek{a}_2, \ldots , \vek{a}_k$  --- линейно независимая система, имеем
$\lambda_1=\ldots = \lambda _k = 0$, то есть $\sum\limits_{i=1}^k \lambda _i \varphi (\vek{a}_i)$ --- тривиальная линейная комбинация.\\
2). Следует из 1).
\edok

\otstup 

%\begin{sled}\label{rgAfA}
%Пусть $\varphi \in L( V, \widetilde{V})$, $\mathcal{A}\subset V$.
%Тогда $\rg \varphi (\mathcal{A}) \leq \rg \mathcal{A}$. Если, кроме того, 
%$\varphi$ инъективно, то $\rg \mathcal{A} = \rg \varphi (\mathcal{A}) $.
%\end{sled}
%%\dok
%%\edok

Следующая теорема показывает, что
линейное отображение определено, причем единственным образом, образами базисных векторов.


\begin{theor}\label{t8_1_1}
Пусть $\bazis{e} = (\vek{e}_1, \vek{e}_2, \ldots , \vek{e}_n)$ --- базис в $V$,
и $\vek{c}_1, \vek{c}_2, \ldots , \vek{c}_n$ --- фиксированные векторы из~$\widetilde{V}$. Тогда \\
1). существует единственное линейное отображение $\varphi :V\to \widetilde{V}$ такое, что
$\varphi(\vek{e}_i) = \vek{c}_i$, $i=1, 2, \ldots , n$. \\
2). При этом $\varphi$ инъективно $\Leftrightarrow$ система $\vek{c}_1, \vek{c}_2, \ldots , \vek{c}_n$ линейно независима.
\end{theor}
\dok 1). {\it Единственность}. Пусть $\vek{a}\in V$ разложен по базису $\bazis{e}$: $\vek{a}= \sum\limits_{i=1}^n x_i\vek{e}_i$.
Тогда, в силу (\ref{eqLin}), однозначно получаем 
\begin{equation}\label{eqLin1}
\varphi(\vek{a})= \sum\limits_{i=1}^n x_i\varphi (\vek{e}_i) = \sum\limits_{i=1}^n x_i \vek{c}_i.
\end{equation}
{\it Существование}. Очевидно, что формула (\ref{eqLin1}) удовлетворяет условию $\varphi(\vek{e}_i) = \vek{c}_i$, $i=1, 2, \ldots , n$.
Достаточно доказать, что она действительно определяет линейное отображение.
Для произвольных векторов $\vek{a}, \vek{b} \in V$, таких, что $\vek{a}= \sum\limits_{i=1}^n x_i\vek{e}_i$, $\vek{b}= \sum\limits_{i=1}^n y_i\vek{e}_i$,
по определению отображения $\varphi$ (с учетом предложения \ref{p7_3_2} главы \ref{lin_prostr}) имеем: 
$\varphi(\vek{a})= \sum\limits_{i=1}^n x_i \vek{c}_i$, 
$\varphi(\vek{b})=  \sum\limits_{i=1}^n y_i \vek{c}_i$,
$\varphi(\vek{a}+\vek{b})= \sum\limits_{i=1}^n (x_i+y_i) \vek{c}_i$, 
$\varphi(\lambda \vek{a})= \sum\limits_{i=1}^n (\lambda x_i) \vek{c}_i$.
Теперь легко видеть, что  L1 и L2 выполнены.\\
2). \dokright Следует из предложения \ref{p8_1_1000}.\\
\dokleft Пусть $\varphi (\vek{a}) = \varphi (\vek{b})$ для некоторых векторов $\vek{a}\neq \vek{b}$, тогда  
$\varphi (\vek{a}- \vek{b}) = \vek{0}$. Разложим ненулевой вектор $\vek{a}- \vek{b}$ по базису $\bazis{e}$:
$\vek{a}- \vek{b} = \sum\limits_{i=1}^n \lambda_i \vek{e}_i$.
Тогда $\varphi (\vek{a}- \vek{b}) = \sum\limits_{i=1}^n \lambda_i \vek{c}_i = \vek{0}$. Получаем, что нетривиальная линейная
комбинация векторов $\vek{c}_1, \vek{c}_2, \ldots , \vek{c}_n$ равна $\vek{0}$. Противоречие.
\edok

\otstup

Утверждение теоремы \ref{t8_1_1} дает понимание, насколько мы свободны в определении линенйного отображения,
оно бывает полезно при конструировании линейных отображений с заданными свойствами.

\otstup

{\bf Упражнение.} Может ли для некоторых $\varphi \in L(V, V)$ и $\vek{a}\in V$ выполняться
одновременно условия $\varphi (\vek{a})\neq \vek{0}$, $\varphi (\varphi (\vek{a}))= \vek{0}$?


\subsection{Изоморфизм. Композиция (произведение). Обратное отображение. }

\defin{
Отображение $\varphi: V\to \widetilde{V}$ называется {\it изоморфизмом}, если оно
линейно и биективно.
}

\defin{
Векторные пространства $V$ и $\widetilde{V}$ называются
{\it изоморфными}, если существует изоморфизм $\varphi: V\to \widetilde{V}$.
}

Тот факт, что $V$ и $\widetilde{V}$ изоморфны, обозначаем $V\cong \widetilde{V}$.



\begin{predl}\label{p8_1_2}
1). Пусть $\varphi \in L(V, \widetilde{V})$, $\psi \in L(\widetilde{V}, \widehat{V})$. Тогда  
%$\varphi : V\to \widetilde{V}$ и $\psi : \widetilde{V}\to \widehat{V}$
%--- линейные отображения, то
$\psi  \varphi \in L(V, \widehat{V})$.\\
2). Если кроме того $\varphi$ и $\psi$ --- изоморфизмы, то
$\psi  \varphi$ --- также изоморфизм.
\end{predl}
\dok 1). Проверим L1 для отображения $\psi  \varphi$. Из определения композиции и условия L1 для отображений $\psi $ и $\varphi$ 
имеем: $\forall$ $\vek{a}, \vek{b}\in V$ выполнено \\ $(\psi  \varphi)(\vek{a} + \vek{b}) = 
\psi  (\varphi(\vek{a} + \vek{b})) = \psi  (\varphi(\vek{a}) + \varphi (\vek{b}))  = 
\psi  (\varphi(\vek{a})) + \psi  (\varphi (\vek{b}))  = 
(\psi  \varphi)(\vek{a}) + (\psi  \varphi) (\vek{b}). $ \\
Аналогично проверяется L2.\\
2). Следует из 1) и того, что композиция биективных отображений биективна.
\edok

\otstup

Говорят, что два преобразования $\varphi ,\psi \in L(V, V)$  {\it перестановочные} (или {\it коммутируют}), если $\varphi \psi= \psi \varphi $.
Отметим, что композиция линейных преобразований вообще говоря не подчиняется тождеству $\varphi \psi= \psi \varphi $.
Нетрудно привести соответствующие примеры (например, используя замечание после теоремы \ref{t8_1_1}).
%примеры двух преобразований $\varphi ,\psi \in L(V, V)$, которые не являются {\it перестановочными} (или {\it не коммутируют}),
%т.е. для которых $\varphi \psi\neq \psi \varphi $.

\begin{predl}\label{p8_1_3}
Если отображение $\varphi : V\to \widetilde{V}$  --- изоморфизм, то
$ \varphi ^{-1}$ --- также изоморфизм.
\end{predl}
\dok
Для данных векторов $\vek{a}, \vek{b} \in \widetilde{V}$ однозначно определены векторы
$\vek{c}= \varphi ^{-1} (\vek{a})$ и $\vek{d}= \varphi ^{-1} (\vek{b})$.

Так как $\varphi(\vek{c} + \vek{d}) = \varphi(\vek{c}) +  \varphi(\vek{d}) =
\vek{a}+\vek{b} $, то $\vek{c} + \vek{d} = \varphi ^{-1}(\vek{a}+\vek{b})$,
то есть $\varphi ^{-1}(\vek{a}+\vek{b}) = \varphi ^{-1}(\vek{a})+\varphi ^{-1}(\vek{b})$.

Далее, $\varphi(\lambda \vek{c}) = \lambda \varphi(\vek{c}) = \lambda \vek{a}$, откуда
$\varphi ^{-1}(\lambda \vek{a}) = \lambda \vek{c} = \lambda \varphi ^{-1}(\vek{a})$.
\edok

\otstup

Для целого неотрицательного $k$ определим $k$-ую степень преобразования $\varphi: V\to V$
как
$\varphi ^k= {\underbrace
{\varphi  \varphi \ldots \varphi}_{k \, \, \mbox{\scriptsize букв} \, \, \varphi}}$ при $k>0$
и как тождественное преобразование $I_V$ при $k=0$.
При этом справедливы равенства $\varphi ^{m+k}= \varphi ^m \varphi^k$
и $\varphi^{mk}= (\varphi ^m)^k$.
Если кроме того $\varphi$ --- изоморфизм, то
можно определить $k$-ую степень и для отрицательных $k$ как
$\varphi^{k}= (\varphi ^{-1})^{-k}$.
Нетрудно проверить, что в этом случае равенства $\varphi ^{m+k}= \varphi ^m \varphi ^k$
и $\varphi ^{mk}= (\varphi ^m)^k$ остаются в силе для всех $m, k\in \mathbb{Z}$.

\otstup

Очевидно $V\cong V$ (пример изоморфизма --- тождественное преобразование $I_V\, : \, V\to V$).
Предложения \ref{p8_1_2} и \ref{p8_1_3} показывают, что отношение <<быть изоморфными>>
симметрично и транзитивно, т.е. 
$V\cong \widetilde{V}$ $\Rightarrow$ $\widetilde{V} \cong V$ и
$V\cong \widetilde{V}$, $\widetilde{V} \cong \widehat{V}$ $\Rightarrow$ $V\cong \widehat{V}$.
Тем самым, все векторные пространства можно мыслить разбитыми на классы эквивалентности (так что 
два пространства из одного класса эквивалентности изоморфны, а из разных --- не изоморфны). 
Следующая теорема дает классификацию конечномерных векторных пространств.

\begin{theor}\label{t_isom}
Пусть $\dim V<\infty $, $\dim  \widetilde{V} <\infty $. 
Тогда $$V\cong \widetilde{V} \,\,\, \Leftrightarrow \,\,\,  \dim V = \dim \widetilde{V}.$$
\end{theor}
\dok \dokright 
Если $\varphi \,:\, V\to \widetilde{V}$ --- изоморфизм, то $\varphi$ инъективно и сюръективно.
Значит, согласно предложению \ref{p8_1_1000},
имеем $\dim V = \dim (\varphi (V)) = \dim \widetilde{V}$.

\dokleft Пусть $\dim V = \dim \widetilde{V} =  n$.
В предложении \ref{p7_3_2} главы \ref{lin_prostr} мы фактически сталкивались с изоморфизмом
 --- сопоставлением вектору его координатного столбца (в фиксированном базисе). Значит, у нас есть
<<эталонное>> $n$-мерное пространство $\mathbb{F}^n = \mathbf{M}_{n\times 1}$, так что 
$V\cong \mathbf{M}_{n\times 1}$, $\widetilde{V} \cong \mathbf{M}_{n\times 1}$, и следовательно,
$V\cong \widetilde{V}$.
\edok

\otstup

В следующей теореме соберем условия, эквивалентные изоморфности данного линейного отображения
в конечномерном случае.

\begin{theor}\label{t_isom1}
Пусть $\dim V = \dim  \widetilde{V}=n <\infty $, $\bazis{e} = (\vek{e}_1, \vek{e}_2, \ldots, \vek{e}_n )$ --- базис в $V$.
Пусть $\varphi \in L(V, \widetilde{V})$. Тогда следующие условия эквивалентны:\\
1) $\varphi$ --- изоморфизм;\\
2) $\varphi$ --- инъекция;\\
3) $\varphi$ --- сюръеция;\\
4)  $\varphi (\vek{e}_1), \varphi (\vek{e}_2), \ldots, \varphi ( \vek{e}_n )$ --- базис в $\widetilde{V}$.
\end{theor}
\dok 
1) $\Rightarrow$ 2) Очевидно.

2) $\Rightarrow$ 3) Согласно предложению \ref{p8_1_1000}, $\dim (\varphi (V)) = \dim V = n$. Значит, по следствию 2 из теоремы
\ref{t5_2_1}, имеем $\varphi (V)=\widetilde{V}$, то есть $\varphi$ сюръективно.

3) $\Rightarrow$ 4) Как мы знаем (см. предложение \ref{p8_1_103}), $\varphi (V) = \lin{\varphi (\vek{e}_1), \varphi (\vek{e}_2), \ldots, \varphi ( \vek{e}_n )}$.
Из того, что $\dim \varphi (V) =n$, следует, что $\rg (\varphi (\vek{e}_1), \varphi (\vek{e}_2), \ldots, \varphi ( \vek{e}_n ) )  =n$,
значит, система векторов $\varphi (\vek{e}_1), \varphi (\vek{e}_2), \ldots, \varphi ( \vek{e}_n )$ линейно независима, т.е. является базисом в $V$.

4) $\Rightarrow$ 1)  Согласно теореме \ref{t8_1_1}, $\varphi$ инъективно.
А поскольку $\varphi (V) = \lin{\varphi (\vek{e}_1), \varphi (\vek{e}_2), \ldots, \varphi ( \vek{e}_n )}$, 
$\varphi$ сюръективно.
\edok




\subsection{Линейные операции на $\Hom(V; \widetilde{V})$.}

\defin{
{\it Суммой}  отображений $\varphi, \psi \in L(V, \widetilde{V})$ называется такое
отображение $\eta: V\to \widetilde{V}$, что $\forall \vek{a} \in V$ выполнено
$$\eta (\vek{a}) = \varphi (\vek{a}) + \psi  (\vek{a}).$$
}

\defin{
{\it Произведением} отображения $\varphi \in L(V, \widetilde{V})$ на константу $\lambda\in \mathbb{F}$
называется такое отображение $\psi: V\to \widetilde{V}$, что $\forall \vek{a} \in V$ выполнено 
$$\psi (\vek{a}) = \lambda \varphi (\vek{a}).$$ 
}


Результат операции сложения и умножения на число называется, как обычно, суммой и произведением на
число,  и обозначаются обычным образом:
$\varphi  + \psi $ и $\lambda \varphi$. Таким образом, определения означают выполнение 
следующих равенств, которые выглядят вполне естественно:
$$(\varphi  + \psi) (\vek{a}) = \varphi (\vek{a}) + \psi  (\vek{a});$$
$$ (\lambda \varphi) (\vek{a}) = \lambda \varphi (\vek{a}).$$ 


При $\mathbb{F} = \mathbb{C}$ умножение линейного отображения на константу может быть определено
особым способом (и использованием комплексного сопряжения). {\it Второй вариант} определения: 
$$(\lambda \varphi) (\vek{a}) = \overline{\lambda} \varphi (\vek{a}).$$

%Так, для пространства над $\mathbb{C}$ имеются два варианта определения $\lambda \varphi$.
%Когда мы желаем указать, что в множестве $L(V, \widetilde{V})$ выбран второй вариант определения $\lambda \varphi$,
%пишем $\overline{L}(V, \widetilde{V})$.

\begin{predl}\label{p8_1_4}
Если $\varphi, \psi \in L(V, \widetilde{V})$, то
$\varphi + \psi \in L(V, \widetilde{V})$ и
$\lambda \varphi \in L(V, \widetilde{V})$.
\end{predl}
\dok Проверяется непосредственно. Например, проверим L2 для $\lambda \varphi$ со вторым вариантом определения.
Имеем $(\lambda \varphi)(\mu \vek{a}) = \overline{\lambda} \varphi(\mu \vek{a}) =  \overline{\lambda} \mu  \varphi(\vek{a}) =
\mu (\overline{\lambda}   \varphi(\vek{a})) = \mu (\lambda   \varphi)(\vek{a}).$ Тем самым, $\lambda \varphi$ удовлетворяет L2.
\edok

\begin{predl}\label{p8_1_5}
Множество $L(V, \widetilde{V})$ является векторным пространством относительно
введенных выше операций сложения и умножения на число. 
\end{predl}
\dok Проверяется непосредственно. 
\edok

\otstup

При $\mathbb{F} = \mathbb{C}$ множество $L(V, \widetilde{V})$ наделяется структурой
векторного пространства двумя способами, в соответствии с двумя способами определения 
умножения на константу. В случае рассмотрения второго варианта 
%определения пробозначаем векторное пространство
%линейных отображений $V\to \widetilde{V}$ с первым вариантом определения умножения на число,
вместо $L(V, \widetilde{V})$ пишем $\overline{L}(V, \widetilde{V})$.
% --- то же пространство со вторым вариантом
%определения умножения на число.

%Имеется следующая
%связь между композиции с линейными операциями:
% сложением и умножением на число:

\begin{predl}\label{p8_1_6}
Пусть $\varphi, \psi \in L(V, \widetilde{V})$,
$\widetilde{\varphi}, \widetilde{\psi} \in L(\widetilde{V}, \widehat{V})$.
Тогда\\
1. $\widetilde{\varphi} (\varphi + \psi) = \widetilde{\varphi} \varphi + \widetilde{\varphi}
\psi$;\\
2. $(\widetilde{\varphi} +\widetilde{\psi}) \varphi  =
\widetilde{\varphi} \varphi +\widetilde{\psi} \varphi $;\\
3. $\lambda (\widetilde{\varphi} \varphi) = (\lambda \widetilde{\varphi}) \varphi =
\widetilde{\varphi} (\lambda \varphi)$.
\end{predl}
\dok  Проверяется непосредственно.
\edok

%\otstup




\subsection{Функции от операторов.}

На множестве линейных операторов $L(V, V)$ определены операции 
сложения, умножения на константы и умножения. Предложения \ref{p8_1_5} и \ref{p8_1_6}
вметсте с условием ассоциативности умножения (которая выполнена для композиции произвольных отображений)
означают, что $L(V, V)$ имеет структуру {\it ассоциативной алгебры}.
Поэтому $L(V, V)$ можем называть алгеброй линейных операторов на пространстве $V$.

Если $\varphi \in L(V, V)$, а $f\in \mathbb{F}[X]$ --- некоторый многочлен,
так что $f(x) = \sum \limits_{i=0}^{m}a_ix^i$,
то положим $f(\varphi) = \sum \limits_{i=0}^{m}a_i \varphi^i$ (напомним, что $\varphi ^0 = I_V$ --- тождественное
преобразование).
Преобразование $f(\varphi)$ перестановочно с $\varphi$.
Более общо, если $f$ и $g$ --- два многочлена, то $f(\varphi)$ и $g(\varphi)$ ---
перестановочные преобразования.

{\footnotesize Формально, при фиксированном $\varphi \in L(V, V)$ подстановка  $f\mapsto f(\varphi)$ задает гомоморфизм алгебр $\mathbb{F}[X] \to L(V, V)$. Образ --- множество многочленов от $\varphi$ --- является
коммутативной подалгеброй в $L(V, V)$. Ниже мы говорим о ядре этого гомоморфизма.}

\otstup

Говорят, что многочлен $f\in \mathbb{F}[X]$ является {\it аннулирующим} для оператора $\varphi \in L(V, V)$
(или $f$ аннулирует $\varphi$), если $f(\varphi) = 0$ (нулевой оператор).
При $\dim V<n$ для данного $\varphi \in L(V, V)$ имеются ненулевые аннулирующие многочлены.
Отложив доказательство этого факта, установим структуру множества многочленов, аннулирующих $\varphi$.


\defin{
Ненулевой многочлен $\mu \in \mathbb{F}[X]$ называется {\it минимальным} многочленом оператора $\varphi \in L(V, V)$, если
$\mu$ аннулирует $\varphi$ и имеет минимальную степень среди всех аннулирующих $\varphi$ многочленов.
}

\begin{predl}\label{min_mn}
Пусть  $\mu \in \mathbb{F}[X]$ 
 --- минимальный многочлен оператора $\varphi \in L(V, V)$ и $f\in \mathbb{F}[X]$. 
Тогда $f$ аннулирует $\varphi$ $\Leftrightarrow$  $f\, \vdots \, \mu$ 
(делится в кольце $\mathbb{F}[X]$).
\end{predl}
\dok  Разделим $f$ на $\mu$  с остатком: $f = q\mu + r$, где $q,r \in \mathbb{F}[X]$, $\deg r < \deg \mu$.
Тогда равенство остается верным после подстановки $\varphi$: $f (\varphi) = q(\varphi)\mu (\varphi)+ r(\varphi)$.
Поскольку $\mu (\varphi) = 0$, имеем $f (\varphi) =  r (\varphi)$. \\
Значит,  $f (\varphi)= 0$ $\Leftrightarrow$  $r (\varphi)=0$.
Условие $r (\varphi)=0$ означает, что $r$ аннулирует $\varphi$ и его степень меньше степени минимального многочлена, 
т.е. $r=0$ (нулевой многочлен). Таким образом, 
$f (\varphi)= 0$ $\Leftrightarrow$  $r=0$ $\Leftrightarrow$ $f\, \vdots \, \mu$. 
\edok

\begin{sled}\label{min_mn_ed}
Для оператора $\varphi \in L(V, V)$ минимальный многочлен единственный, 
с точностью до умножения на ненулевую константу.
\end{sled}
\dok  Если  $\mu$ и $\mu '$ --- оба минимальные многочлены для $\varphi$, 
то, согласно предложению \ref{min_mn}, $\mu' \, \vdots \, \mu$ и $\mu \, \vdots \, \mu'$. 
\edok


%%%%%%%%%%%%%
%%ВОПРОС --- ПОЧЕМУ МИНИМАЛЬНЫЙ МНОГОЧЛЕН НЕ МОЖЕТ ИЗМЕНИТЬСЯ ПРИ РАСШИРЕНИИ ПОЛЯ
%%%%%%%%%%%%%



\otstup

Многочлен являются частным случаем степенного ряда, или более общо, ряда Лорана. 
Можно определять функции от оператора $\varphi$ через значение соответствующего операторного 
ряда $\sum a_i \varphi ^i$, правда  
во мноих случаях для обеспечения сходимости такого ряда 
требуются дополнительные условия на $\varphi$
(кроме того, сходимость  можно понимать 
в разных  смыслах). Так, для линейных операторов $\varphi : V\to V$, где $V$ --- 
пространство над $\mathbb{R}$ или $\mathbb{C}$, можно говорить об
$\exp (\varphi)$ и т.д. 

%При определенных условиях можно ввести функции от преобразования, например $\exp (\varphi)$

%(О перестановочности (вспомнить, когда будет разговор о матрицах) преобразований

\subsection{Примеры}


Пусть $V = U_1 \bigoplus U_2$. Тогда для каждого вектора $\vek{a}$ имеется единственное
разложение $\vek{a} = \vek{a}_1 + \vek{a}_2$, где
$\vek{a}_1$ --- проекция $\vek{a}$ на $U_1$ вдоль $U_2$,
$\vek{a}_2$ --- проекция $\vek{a}$ на $U_2$ вдоль $U_1$.

\example{
Отображение $\varphi: V\to V$ такое, что $\varphi(\vek{a}) = \vek{a}_1$,
называется {\it проектированием} на $U_1$ вдоль (или параллельно) $U_2$. 
}
\example{
Отображение $\psi: V\to V$ такое, что $\varphi(\vek{a}) = \vek{a}_1 - \vek{a}_2$,
называется {\it отражением}  (или {\it симметрией}) относительно $U_1$ вдоль (или параллельно) $U_2$.
}

Легко проверить, что отражение  и проектирование --- 
линейные преобразования, причем отражение является изоморфизмом.


\example{I.
Пусть  $\varphi: \mathbf{R}^3\to \mathbf{R}^3$ --- 
движение, для которого начало координат $O$ --- неподвижная точка (отождествляем точки с радиус-векторами),
например, $\varphi$ --- поворот вокруг оси (прямой, проходящей через $O$).\\
Линейные биективные преобразования плоскости $\varphi: \mathbf{R}^2\to \mathbf{R}^2$ ---
 %$\mathbf{R}^2 \to \mathbf{R}^2$
аффинные преобразовани, для которых $O$ --- неподвижная точка (см. ниже параграф \ref{aff}).
%Тогда $\varphi$ --- линейное и биективное преобразование.
}

\example{II.1.
Пусть $A\in \mathbf{M}_{m\times n}$ --- фиксированная матрица.
Определим $\varphi \in L(\mathbf{M}_{n\times p}, \mathbf{M}_{m\times p})$
правилом $\varphi (X) = AX$. Нетрудно проверить, что $\varphi$ линейно (ниже увидим, что в некотором смысле к этому примеру можно 
свести любое линейное отображение конечномерных пространств). \\
Аналогично можно определить отображение домножение справа на фиксированную матрицу.
}

\example{II.2.
Транспонирование матриц $m\times n$ --- пример изоморфизма
$\mathbf{M}_{m\times n}\to \mathbf{M}_{n\times m}$.
}

\example{III.1.
Пусть $\mathbf{C}[a,b]$ --- пространство всех непрерывных функций, определенных на отрезке $[a,b]$. 
Для различных отрезков $[a,b]$ пространства $\mathbf{C}[a,b]$ изоморфны. Например, изоморфизм между 
пространствами $\mathbf{C}[-2,0]$, $\mathbf{C}[0,1]$ определяется %$\varphi \in L(\mathbf{C}[-2,0], \mathbf{C}[0,1])$, определенное 
правилом $\varphi (f(x)) \hm= f(2x-2)$. % является изоморфизмом.
}

\example{III.2.
Пусть $V= \mathbf{C}^{1}(\mathbb{R})$. 
{\it Дифференцирование}  %$\dfrac{d}{dx}: Vbf{P} \to \mathbf{P}$ задается правилом $\varphi (f(x)) = f'(x)$.
$d: V \to V$ задается правилом $(d (f))(x) = f'(x)$. Нетрудно видеть, что $d$ --- линейное преобразование.
Можно определить {\it линейный дифференциальный оператор} ---  многочлен от $d$.\\
В случае функций многих переменных (для $V= \mathbf{C}^{1}(\mathbb{R}^n)$) можно рассмотреть линейные операторы взятия частной производной $\dfrac{\partial}{\partial x_i}$ (эти операторы --- не перестановочные, 
но становятся таковыми при оганичении на подпространство $\mathbf{C}^{2}(\mathbb{R}^n)$).\\
Линейное отображение $\mathbf{C} (\mathbb{R}) \to \mathbf{C} (\mathbb{R})$ {\it интегрирование} с переменным верхним пределом %(на множестве многочленов степени не выше $n$)
%$\psi: \mathbf{P}_n \to \mathbf{P}_{n+1} $ 
задается правилом $\varphi (f(x)) = \int\limits_{0}^{x} f(t)\, dt$.
%СУЖЕНИЕ....
%ИНТЕГРИРОВАНИЕ С ПОСТОЯННЫМ ПРЕДЕЛОМ....
%Урматы=Кер
%(Здесь $P$ и $P_n$ --- пространства многочленов и многочленов степени, не превосходящей $n$, соответственно.)
}

\example{III.3.
Пусть  $V=\mathbf{F}(\mathbb{N})=\{(a_1, a_2, \ldots) \, | \, a_i\in \mathbb{F}\}$ --- пространство числовых последовательностей.
{\it Оператор сдвига} $\varphi : V\to V$ определяется равенством $\varphi (a_1, a_2, \ldots) = (a_2, a_3, \ldots)$.
Определим {\it оператор первой разности} $\Delta = \varphi - I_V$, так что
$\Delta (a_1, a_2, \ldots) = (a_2-a_1, a_3-a_2, \ldots)$.\\
Для $k\in \mathbb{N}$ степень $\Delta ^k$ называют {\it оператором $k$-ой разности}.
}

\example{IV.
Пусть  $\mathbb{F}$ --- подполе некоторого поля $\mathbb{K}$, а $\varphi : \mathbb{K}\to \mathbb{K}$
--- изоморфизм поля $\mathbb{K}$ на себя (автоморфизм), при котором все элементы 
поля $\mathbb{F}$ неподвижны. Тогда $\varphi $ является линейным оператором $\mathbb{K}\to \mathbb{K}$, 
где $\mathbb{K}$ рассматривается как линейное пространство над $\mathbb{F}$.
}


