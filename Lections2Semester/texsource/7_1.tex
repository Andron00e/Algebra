\chapter{Тензоры
}%\label{lin_funk}

В этой главе рассматриваем векторные пространства над полем $\mathbb{F}$.
(где возникает деление (симметризация и пр.) предполагаем, что характеристика поля равна 0).

%(Над $\mathbb{C}$)

%ВООБЩЕ ПОГРУЗИТЬ В ТЕОРИЮ БИЛИНЕЙНЫХ ОТОБРАЖЕНИЙ... B(V, V_1, ... )


\section{Полилинейные отображения}

\subsection{Определение и координантая запись}

%\defin{
%Полилинейным отображением...
%}

Отображение $\beta : V_1\times V_2 \times \ldots \times V_k \to \mathbb{F}$
{\it полилинейно}, если $\beta (\vek{a}_1, \vek{a}_2, \ldots, \vek{a}_k)$ линейно по каждому из аргументов.

\otstup

Множество всех полилинейных отображений обозначим $\Hom(V_1, V_2, \ldots , V_k; \mathbb{F})$.

\otstup

Примеры. В старых обозначениях для $V$ над $\mathbb{R}$:\\
 $\Hom(V, V; \mathbb{R}) = \mathcal{B}(V)$;\\
$\Hom(V; \mathbb{R}) = V^*$;\\
$\Hom(V^*; \mathbb{R}) = V$ (с учетом отождествления $V$ и $V^{**}$).\\


Пусть $\dim V_t  = n_t$ и в каждом $V_t$ зафиксирован базис:
в $V_1$ --- базис $\bazis{e} = (\vek{e}_1, \ldots, \vek{e}_{n_1})$,
в $V_2$ --- базис $\bazis{f} = (\vek{f}_1, \ldots, \vek{f}_{n_2})$, и т.д.

Сопоставим каждому $\beta \in \Hom(V_1, V_2, \ldots , V_k; \mathbb{F})$ массив 
из $n_1\ldots n_k$ констант $b_{i_1 i_2 \ldots i_k}$, $i_t\in \{1, \ldots, n_t\}$
(далее для упрощения обозначений пишем $b_{ij\ldots }$):
\begin{equation}\label{b_{ij..}}
\beta \rsootv{\bazis{e}, \bazis{f}, \ldots } b_{ij \ldots },  
\end{equation}
где $\boxed{b_{i j \ldots } = \beta (\vek{e}_{i}, \vek{f}_{j}, \ldots )}$.

$b_{i j \ldots }$ называются {\it компонентами} полилинейного отображения в базисах $\bazis{e}, \bazis{f}, \ldots $. 

\begin{theor}[координатная запись]\label{t20_1_1}
Пусть $\beta \in \Hom(V_1, V_2, \ldots , V_k; \mathbb{F})$ и 
$\beta \rsootv{\bazis{e}, \bazis{f}, \ldots } b_{i j \ldots }$. 
Пусть векторы $\vek{a} \in V_1$, $\vek{b} \in V_2, \ldots, $ разложены по базисам:
$\vek{a} = x^i \vek{e}_i $, $\vek{b} = y^j \vek{f}_j $, \ldots
Тогда 
\begin{equation}\label{beta()}
\boxed{ \beta (\vek{a}, \vek{b}, \ldots ) = b_{i j \ldots } x^iy^j\ldots } .
\end{equation}
\end{theor}
\dok Раскроем $\beta (\vek{a}, \vek{b}, \ldots)  $, ....
\edok

 (\ref{b_{ij..}}) задает взаимно-однозначное соответствие 
 $\Hom(V_1, V_2, \ldots , V_k; \mathbb{F}) \to \mathbb{F} ^{n_1n_2\ldots n_k}$



\subsection{Замена базиса}
 

Во встречающихся матрицах перехода обозначаем верхним индексом номер строки, 
а нижним --- номер столбца, так если $S=(s^i_j)$ --- матрица перехода от базиса
$(\vek{e}_1, \ldots )$ к базису $(\vek{e}'_1, \ldots )$, то замена базиса
происходит по правилу $$\vek{e}'_j = s^i_j \vek{e}_i,$$
а, скажем, замена координат вектора --- 
по правилу $$x^i = s^i_j x'^j.$$


\begin{theor}\label{t20_1_2}
Пусть в $V_1$ выбраны базисы $\bazis{e}$ и $\bazis{e}'$, связанные матрицей перехода $s^i_j$, 
в $V_2$ выбраны базисы $\bazis{f}$ и $\bazis{f}'$, связанные матрицей перехода $t^i_j$, и т.д.
Пусть $\beta \in T(V_1, V_2, \ldots , V_k)$ так, что 
$$\beta \rsootv{\bazis{e}, \bazis{f}, \ldots } b_{i j \ldots },$$ 
$$\beta \rsootv{\bazis{e}', \bazis{f}', \ldots } b'_{i j \ldots }$$
Тогда  
\begin{equation}\label{zamena_tenz}
\boxed{b'_{i j \ldots }  = b_{k l \ldots } s^k_i t^l_j \ldots  }.
\end{equation}
\end{theor}
%\dok Пусть $\vek{a} \in V_1, \vek{b} \in V_2, \ldots $ --- произвольные векторы. 
%Пусть $\vek{a}=x^i\vek{e}_i = x'^i\vek{e}'_i$, 
%$\vek{b}=\bazis{e}Y = \bazis{e}'Y'$, и т.д.
%Тогда по теореме \ref{t9_1_1}  имеем $\beta (\vek{a}, \vek{b}) = X^T B \overline{Y}$ и $\beta (\vek{a}, \vek{b}) = X'^T B \overline{Y'}$
%Подставляя  $X=SX'$, $Y=SY'$ (см. теорему \ref{t7_3_2}, глава \ref{lin_prostr}),
% имеем $\beta (\vek{a}, \vek{b}) = (SX')^T B \overline{SY'} = X'^T S^T B \overline{S} \overline{Y'} = X'^T (S^T B \overline{S}) \overline{Y'} $. 
%В силу предложения \ref{p9_1_2} получаем требуемое: $B'=S^TB\overline{S}$.
%\edok

(Координатное определение тензора)


\subsection{Линейные операции}


$\Hom(V_1, V_2, \ldots , V_k; \mathbb{F})$ --- векторное пространство над $\mathbb{F}$. 


