

\section{Кососимметричные формы}

В этом параграфе $V$ --- векторное пространство над $\mathbb{R}$.


\defin{
Билинейная (полуторалинейная) форма $\beta$ на пространстве $V$ называется {\it кососимметричной},
если $\forall$ $\vek{a}, \vek{b} \in V$
$$\beta (\vek{a}, \vek{b}) = -\beta (\vek{b}, \vek{a}).$$
}


Множество всех симметричных (эрмитовых) билинейных (полуторалинейных) форм на пространстве $V$ обозначаем 
$\mathcal{B}_{Alt}(V)$.

\otstup

Для $\beta \in \mathcal{B}_{Alt}(V)$, очевидно,
 $\beta (\vek{a}, \vek{a}) = 0$. 


\begin{theor}\label{t9_2_1111}
Пусть $\dim V<\infty$, $\bazis{e}$ --- базис в $V$. Пусть $\beta \in  \mathcal{B}(V)$, $\beta \rsootv{\bazis{e}} B$.
Тогда $\beta\in \mathcal{B}_{Alt}(V)$ $\Leftrightarrow$ $B^{T}=-B$.% (где, как обычно,  $B^{*} = \overline{B^T}$).
\end{theor}
\dok Аналогично доказательству теоремы \ref{t9_2_1}.
\edok

%\dok 
%\dokright Надо доказать, что $\overline{b_{ji}}=b_{ij}$ или что $\overline{\beta(\vek{e}_j, \vek{e}_i)}=\beta(\vek{e}_i, \vek{e}_j)$ для всевозможных пар индексов.
%Но это сразу следует из определения.

%\dokleft Пусть $\vek{a}, \vek{b} \in V$ --- произвольные векторы, $\vek{a}=\bazis{e}X$, $\vek{b}=\bazis{e}Y$.
%Тогда $\beta (\vek{a}, \vek{b}) = X^T B \overline{Y}$.
%Также $\beta (\vek{b}, \vek{a}) = Y^T B \overline{X}$ или (транспонируем матрицу $1\times 1$)
%$\beta (\vek{b}, \vek{a}) = \overline{X}^T B^T Y = \overline{ X^T B^{*} \overline{Y} } = \overline{X^T B \overline{Y}}$. 
%Отсюда $\overline{\beta (\vek{b}, \vek{a})}=\beta (\vek{a}, \vek{b})$, что и требовалось.
%\edok



\begin{theor}[канонический вид]\label{t9_3_1111}
Пусть $\dim V=n<\infty$,  $\beta\in \mathcal{B}_{Alt}(V)$. 
Тогда существует базис, в котором $\beta$ имеет блочно-диагональный вид, в котором по диагонали
встречаются блоки вида  \\$(0)$ или 
$\begin{pmatrix} 0 & -1 \\ 
1& 0
\end{pmatrix}.$
\end{theor}
\dok CХЕМА: ДВОЙНЫМИ ЭЛЕМЕНТАРНЫМИ ПРЕОБРАЗОВАНИЯМИ.
\edok
