\chapter{Билинейные и квадратичные формы}\label{kvadr_formy}

На протяжении всей этой главы $V$ обозначает данное векторное пространство над полем $\mathbb{R}$ или $\mathbb{C}$.
Будем одновременно развивать теорию {\it билинейных} форм в случае пространства над $\mathbb{R}$ и 
{\it полуторалинейных} форм в случае пространства над $\mathbb{C}$. 
В большинстве случаев определения, формулировки и доказательства аналогичны.
Знак комплексного сопряжения при работе в пространстве над $\mathbb{R}$ можно игнорировать.
Терминология для пространства над $\mathbb{C}$ приводится в скобках.

\section{Билинейные формы. Матрица билинейной формы}\label{matr_bilin_formy}


\subsection{Определение}

\defin{
Отображение $\beta: V \times V \to \mathbb{R}$ ($\beta: V \times V \to \mathbb{C}$) называется {\it билинейным} ({\it полуторалинейным}),
или {\it билинейной (полуторалинейной) функцией}, или {\it билинейной (полуторалинейной) формой} на пространстве $V$,
если $\forall$~$\vek{a}, \vek{a}_1, \vek{a}_2, \vek{b}, \vek{b}_1, \vek{b}_2 \in  V$ и $\forall$ $\lambda\in \mathbb{R}$ ($\mathbb{C}$)
выполняются равенства
\\
B1.1. $\beta(\vek{a}_1 + \vek{a}_2, \vek{b}) = \beta(\vek{a}_1, \vek{b}) + \beta(\vek{a}_2, \vek{b}) $,
\\
B1.2. $\beta(\lambda \vek{a}, \vek{b}) = \lambda \beta(\vek{a}, \vek{b}) $,
\\
B2.1. $\beta(\vek{a}, \vek{b}_1 + \vek{b}_2) = \beta(\vek{a}, \vek{b}_1) + \beta(\vek{a}, \vek{b}_2) $,
\\
B2.2. $\beta(\vek{a}, \lambda \vek{b}) = \overline{\lambda} \beta(\vek{a},  \vek{b}) $.
\\
}

Множество всех билинейных форм на пространстве $V$ над полем $\mathbb{F}$
обозначают  $\Hom (V, V; \mathbb{F})$.
Мы будем пользоваться более коротким обозначением $\mathcal{B} (V)$
для множества всех билинейных (полуторалинейных) форм на пространстве $V$. 



\subsection{Матрица и билинейной формы. Координатная запись}

Если $\dim V=n<\infty$ и в $V$ зафиксирован некоторый базис $\bazis{e}=(\vek{e}_1, \vek{e}_2, \ldots , \vek{e}_n)$,
то билинейной (полуторалинейной) форме можно сопоставить матрицу $n\times n$ следующим образом.

\defin{{\it Матрицей билинейной (полуторалинейной) формы} $\beta \in \mathcal{B} (V)$
в базисе $\bazis{e}$ 
называется матрица $B= (b_{ij}) \in \mathbf{M}_{n\times n}$ такая, что $(b_{ij}) = \beta (\vek{e}_i, \vek{e}_j )$
для всех $i=1, \ldots, n$, $j=1, \ldots, n$.
}

Тот факт, что $B$ --- матрица билинейной (полуторалинейной) формы $\beta$
в базисе $\bazis{e}$ будем обозначать $\beta \rsootv{\bazis{e}} B$.
Посмотрев на определение, матрицу $B$ можно неформально назвать таблицей билинейного умножения.



\begin{theor}[координатная запись]\label{t9_1_1}
Пусть $\beta \in \mathcal{B} (V)$ и $\beta \rsootv{\bazis{e}} B$. 
Пусть $\vek{a}= \bazis{e} X$, $\vek{b}= \bazis{e} Y$.
Тогда $$\boxed{ \beta (\vek{a}, \vek{b}) = X^T B \overline{Y}} .$$
\end{theor}
\dok Раскроем $\beta (\vek{a}, \vek{b}) = \beta (\sum \limits_{i=1}^n  x_i \vek{e}_i, \sum \limits_{i=j}^n  y_j \vek{e}_j) $, пользуясь линейностью по первому и (полулинейностью)
по второму аргументам: \\
$\beta (\vek{a}, \vek{b}) = \sum \limits_{i=1}^n \sum \limits_{j=1}^n x_i \overline{y_j} \beta (\vek{e}_i, \vek{e}_j) 
= \sum \limits_{i=1}^n \sum \limits_{j=1}^n x_i b_{ij} \overline{y_j} $. Полученная двойная сумма и есть единственный элемент матрицы  $X^T B \overline{Y}$ 
(эту матрицу размера $1\times 1$ мы отождествляем с числом, записанным в единственной ее ячейке).
\edok

\otstup

Формула из предыдущей теоремы фактически эквивалентна определению матрицы билинейной формы.
Более, точно, справедливо следующее %предложение.
%

\begin{predl}\label{p9_1_2}
%\begin{zamech}
Пусть дано отображение $\beta :V\times V \to \mathbb{R}$ ($\beta :V\times V \to \mathbb{C}$)
(априори не известно, что билинейное) 
и матрица $B \in M_{n\times n}$.
Пусть $\forall$ $\vek{a}, \vek{b} \in V$, имеющих координатные столбцы $X$ и $Y$  в базисе $\bazis{e}$, 
выполнено $\beta (\vek{a}, \vek{b}) = X^T B \overline{Y}$. Тогда $\beta \in \mathcal{B} (V)$, причем
$\beta \rsootv{\bazis{e}} B$.
%\end{zamech}
\end{predl}
\dok Непосредственно проверяется, что отображение $\beta$, заданное как  $\beta (\vek{a}, \vek{b}) = X^T B \overline{Y}$,
удовлетворяет равенствам B1.1---B2.2 из опеределения, поэтому $\beta \in \mathcal{B} (V)$.

Кроме того, $\vek{e}_i = \bazis{e} E_{\bullet i}$ (где $E_{\bullet i}$ --- $i$-й столбец единичной матрицы, и из правил перемножения матриц
$\beta (\vek{e}_i, \vek{e}_j) = E_{\bullet i} ^T B \overline{E_{\bullet j}} =b_{ij}$.
Поэтому в самом деле $\beta \rsootv{\bazis{e}} B$.
\edok

\otstup

В зависимости от ситуации удобно пользоваться как определением матрицы билинейной формы, так и координатной записью $X^T B \overline{Y}$.

%%%%%%%!!!
%Идеология та же, что и для линейных отображениях --- матрица несет в себе информацию об образах на парах базисных векторов.
%?? Ср. с линейными отобр.


Как следствие предложения \ref{p9_1_2} получаем, что соответствие $\beta \rsootv{\bazis{e}} B$ (зависящее от выбора базиса $\bazis{e}$)
является взаимно-однозначным соответствием между $\mathcal{B} (V)$ и $\mathbf{M}_{n\times n}$.

{\footnotesize Отметим, что после введения на множестве $\mathcal{B} (V)$ естественных операций сложения и умножения на константу, 
$\mathcal{B} (V)$ превращается в векторное пространство, при этом соответствие $\beta \rsootv{\bazis{e}} B$ является изоморфизмом. %!! над $C$????
}

\subsection{Изменение матрицы при замене базиса}

\begin{theor}\label{t9_1_2}
Пусть в $V$ выбраны базисы $\bazis{e}$ и $\bazis{e}'$, связанные матрицей перехода $S$:
$\bazis{e}'= \bazis{e} S$.
Пусть $\beta \in \mathcal{B} (V)$ таково, что 
$\beta \rsootv{\bazis{e}} B$ и $\beta \rsootv{\bazis{e}'} B'$.
%имеет матрицу $A$ в базисах $\bazis{e}$, $\bazis{f}$, и матрицу
%$A'$ в базисах $\bazis{e}'$, $\bazis{f}'$.
Тогда  $$\boxed{B'=S^TB\overline{S}}.$$
\end{theor}
\dok Пусть $\vek{a}, \vek{b} \in V$ --- произвольные векторы. Пусть $\vek{a}=\bazis{e}X = \bazis{e}'X'$, 
$\vek{b}=\bazis{e}Y = \bazis{e}'Y'$.
Тогда по теореме \ref{t9_1_1}  имеем $\beta (\vek{a}, \vek{b}) = X^T B \overline{Y}$ и $\beta (\vek{a}, \vek{b}) = X'^T B \overline{Y'}$
Подставляя  $X=SX'$, $Y=SY'$ (см. теорему \ref{t7_3_2}, глава \ref{lin_prostr}),
 имеем $\beta (\vek{a}, \vek{b}) = (SX')^T B \overline{SY'} = X'^T S^T B \overline{S} \overline{Y'} = X'^T (S^T B \overline{S}) \overline{Y'} $. 
В силу предложения \ref{p9_1_2} получаем требуемое: $B'=S^TB\overline{S}$.
\edok

\begin{sled1}
В обозначениях теоремы $\rg B = \rg B'$, т.е. ранг матрицы билинейной формы не зависит от выбора базиса
\end{sled1}
\dok 
Достаточно заметить, что матрица перехода невырожденная, а умножение на невырожденную матрицу не меняет ранг.
\edok

\begin{sled2}
В обозначениях теоремы определители $|B|$ и $|B'|$ отличаются на положительный вещественный множитель.
\end{sled2}
\dok 
По правилу произведения определителей: 
 $|B'|=|S^T|\cdot |B| \cdot |\overline{S}|  = |S|\cdot |B| \cdot \overline{|S|} = |z|^2 \cdot |B|$, где $z=|S|$.
\edok

Следствие 1 позволяет корректно ввести ранг $\rg \beta$ билинейной формы.
%Следствие 2 будет использоваться в такой ситуации: если.....

В случае, если $S$ --- {\it элементарная матрица}, преобразование $B\to S^TB\overline{S}$ соответствует 
следующим {\it двойным элементарным преобразовниям}: выполнеяется элементарное преобразование строк (ему соответствует домножение слева на матрицу $S^T$),
а затем {\it соответствующее} элементарное преобразование столбцов строк (ему соответствует домножение справа на матрицу $\overline{S}$).
Виды двойных элементарных преобразований: прибавим к $i$-й строке $j$-ю, умноженную на $\lambda$, 
 а затем прибавим к $i$-му столбцу $j$-й, умноженный на $\overline{\lambda}$;
поменяем местами $i$-ю и $j$-ю строки, а затем поменяем местами $i$-й и $j$-й столбцы; 
$i$-ю строку умножим на $\lambda$, а затем $i$-й столбец умножим на $\overline{\lambda}$.

% вещественно и
%положительно (отрицательно), то $\det A'$ вещественно и положительно (отрицательно)
%в любом базисе.


\subsection{Сужение}

Если $U\leq V$ и $\beta \in \mathcal{B} (V)$, то можно рассмотреть {\it сужение} 
$\beta \mid_{U\times U}: U\times U \to \mathbb{R}$. %($f\mid_{U\times U}: U\times U \to \mathbb{C}$).
Очевидно, сужение билинейной формы на подпространство $U$ является билинейной формой на подпространстве $U$.

Для матрицы $B$ обозначим через $B_k$ левую верхнюю угловую подматрицу $k\times k$, так что $B_k = (b_{ij})$
для всех $i=1, \ldots, k$, $j=1, \ldots, k$. В частности, $B_1=(b_{11})$, $B_n=B$.

Если в пространстве $V$ выбран базис 
$\bazis{e}=(\vek{e}_1, \vek{e}_2, \ldots , \vek{e}_n)$, то обозначим через $\bazis{e}^{(k)}$ упорядоченную систему $(\vek{e}_1, \ldots , \vek{e}_k)$ --- это базис пространства
$U_k = \lin{\vek{e}_1, \ldots , \vek{e}_k}$. (При $k=n$ имеем $U_k=V$.)

\begin{predl}\label{p9_1_1}
Пусть $\beta \in \mathcal{B} (V)$ и $\beta \rsootv{\bazis{e}} B$.
Тогда $\beta \mid_{U\times U} \rsootv{\bazis{e}^{(k)}} B_k $.
\end{predl}
\dok Сразу следует из определения.
\edok

\otstup
Примеры (скалярное произведение, на функциях с плотностью), пр-во минковского. 
