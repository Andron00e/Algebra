\chapter{Сопряженное пространство%(пространство линейных функций)
}\label{lin_funk}

Пусть $V$ --- векторное пространство над $\mathbb{R}$ (или $\mathbb{C}$).
%(Над $\mathbb{C}$)

%ВООБЩЕ ПОГРУЗИТЬ В ТЕОРИЮ БИЛИНЕЙНЫХ ОТОБРАЖЕНИЙ... B(V, V_1, ... )


\section{Связь с общей теорией линейных отображений}

\subsection{Определение}

\defin{
%Пусть $\widetilde{V} = 1$ (т.~е. $\widetilde{V} = \mathbb{R}$ ($ \mathbb{C}$)), тогда
Пространство $\overline{L}(V, \widetilde{V})$, где  $\widetilde{V} = \mathbb{R}$ (или $ \mathbb{C}$)  % $\dim \widetilde{V} = 1$, 
называется {\it сопряженным (или двойственным)} пространством для пространства~$V$.
}

Таким образом, сопряженное пространство --- это частный случай пространства  $\overline{L}(V, \widetilde{V})$, где   $\dim \widetilde{V} = 1$.
Элементы сопряженного пространства --- линейные функции (функционалы), поэтому сопряженное пространство также называют 
{\it пространством линейных функций}. Сопряженное пространство для пространства~$V$ обозначается $V^{*}$.

\begin{predl}\label{p8_4_111}
Если $\dim V = n< \infty$, то $\dim V^{*} = n$.
\end{predl}
\dok Это частный случай следствия из теоремы \ref{p8_3_333} главы 2.
%Если $\dim V = n$, то $\dim V^{*} = \dim V \cdot  \dim \widetilde{V} = n\cdot 1 = n$.
\edok

\otstup

Вся теория о линейных отображениях переносится на частный случай пространства $V^*$.
В частности, если зафиксировать базис $\bazis{e} = (\vek{e}_1, \vek{e}_2, \ldots , \vek{e}_n)$ 
в пространстве $V$ 
то каждая линейная функция $\ell$ получает в соответствие матрицу $1\times n$, то есть строку
$(\ell(\vek{e}_1), \ell(\vek{e}_2), \ldots , \ell(\vek{e}_n))$.  
(Здесь мы считаем, что в пространстве $\widetilde{V} = \mathbb{R}$ (или $ \mathbb{C}$)) зафиксирован базис --- число 1.) 


Если $\vek{a}\in V$ и $\ell \in V^{*}$, то наряду с записью $\ell (\vek{a})$
будем использовать запись $\langle \vek{a}, \ell \rangle$. Эта запись удобна, так как имеется 
линейность по каждому аргументу, т.е. $\forall$ $\vek{a}, \vek{a}_1, \vek{a}_2 \in V$;
$\forall$ $\ell,  \ell _1, \ell _2 \in V^{*}$; $\forall \lambda \in \mathbb{R} (\mathbb{C})$ выполнено:   
$$\langle \vek{a}_1+ \vek{a}_2, \ell \rangle = \langle \vek{a}_1, \ell \rangle + \langle \vek{a}_2, 
\ell \rangle, \,\,\,\,\,\,
\langle \lambda \vek{a}, \ell \rangle = \lambda  \langle \vek{a}, \ell \rangle;$$
$$\langle  \vek{a}, \ell_1+ \ell_2 \rangle =  \langle \vek{a}, \ell_1 \rangle + \langle \vek{a}, \ell_2 \rangle, \,\,\,\,\,\,
\langle \vek{a}, \lambda  \ell \rangle = \overline{\lambda } \langle \vek{a}, \ell \rangle.$$

Иными, словами, скобка $\langle \cdot, \cdot \rangle$ задает билинейную (полуторалинейную) функцию
$V\times V^{*} \to \mathbb{R} (\mathbb{C})$.

\subsection{Взаимный базис}
Везде при работе с базисами и координатами полагаем, что 
$\dim V = n<\infty $. 
Чтобы в дальнейшем была возможность использовать сокращенную тензорную запись суммирования, 
координаты векторов пространства $V$ будем нумеровать верхним индексом,
а для пространства $V^{*}$ --- наоборот, векторы нумеруем верхними индексами, а координаты --- нижними.

\defin{
Базис $\bazis{e}^{*} = (\vek{e}^1, \vek{e}^2, \ldots , \vek{e}^n)$ пространства
$V^{*}$
называется {\it взаимным (или биортогональным, или двойственным)} для базиса
$\bazis{e} = (\vek{e}_1, \vek{e}_2, \ldots , \vek{e}_n)$ пространства
$V$, если  $\langle \vek{e}_i, \vek{e}^j \rangle = \delta_{i}^j$ ($i=1, 2, \ldots, n$, $j=1, 2, \ldots, n$).
%\footnote{Как обычно, $\delta_{ij}=1$ при $i=j$ и $\delta_{ij}=0$ при $i\neq j$.}
}

%Нетрудно понять, как вектор из взимного базиса действует на произвольный вектор пространства $V$.

\begin{predl}\label{p8_4_00}
Пусть $\dim V = n$. Тогда для любого базиса в $V$ существует единственный взаимный базис
в $V^{*}$.
\end{predl}
\dok Для каждого конкретного $j$ вектор $\vek{e}^j\in V^*$ определяется %условиями $\langle \vek{e}_i, \vek{\ell}_j \rangle = \delta_{ij}$
однозначно как линейное отображение $V\to \mathbb{R} (\mathbb{C})$, которому в базисе $\bazis{e}$ соответствует строка $E_j=(0\, 0\, \ldots \, 1\,  0 \, \ldots \, 0)$ (единица на $j$-м месте). 
Строки $E_1, \ldots, E_n$ образуют базис в $\mathbf{M}_{1\times n}$, поэтому 
$\vek{e}^1, \vek{e}^2, \ldots , \vek{e}^n$ образуют базис пространства $V^{*}$.
\edok

\begin{sled}\label{p8_4_00}
Пусть $\dim V = n$ и $\vek{a}\in V$. Если $\forall$ $\ell \in V^{*}$ выполнено
$\langle \vek{a}, \ell \rangle = 0$, то $\vek{a} =  \vek{0}$.
\end{sled}
\dok От противного, если $\vek{a} \neq  \vek{0}$, то $\vek{a}$ можно включить в некоторый базис $\bazis{e}$, 
так что $\vek{a} =  \vek{e}_1$. Тогда, если $\ell = \vek{e}^1$, то
условие $\langle \vek{a}, \ell \rangle = 0$ нарушится.
\edok



\begin{predl}\label{p8_4_0}
Пусть $\dim V = n<\infty$, $\bazis{e}^*$ --- взаимный базис для базиса $\bazis{e}$. Пусть
векторы $\vek{a}\in V$, $\ell \in V^{*}$ разложены по этим базисам:
$\vek{a} = x^i\vek{e}_i$, $\ell = y_j\vek{e}^j$. 
Тогда 
$  \langle \vek{a}, \ell \rangle = x^i\overline{y_i}$.
\end{predl}
\dok
Достаточно раскрыть скобку по линейности и поспользоваться биортогональностью базисов:
$  \langle \vek{a}, \ell \rangle = \langle  x^i\vek{e}_i, y_j\vek{e}^j \rangle  = 
x^i \overline{y_j} \langle  \vek{e}_i, \vek{e}^j \rangle= x^i\overline{y_i}$.
\edok

\otstup

Таким образом,  значение $\langle \vek{a}, \ell \rangle$ равно {\it свертке} $x^i\overline{y_i}$.
В частности, видим, как вектор из взаимного базиса действует на произвольном векторе из $V$:

\begin{sled}\label{s8_4_0}
Пусть $\dim V = n<\infty$, $\bazis{e}^*$ --- взаимный базис для базиса $\bazis{e}$ и 
%векторы $\vek{a}\in V$, $\ell \in V^{*}$ разложены по этим базисам:
$\vek{a} = x^i\vek{e}_i$.
Тогда 
$  \langle \vek{a}, \vek{e}^i \rangle = x^i$.
\end{sled}


\begin{predl}\label{p8_4_1}
Пусть $\dim V = n$.  Пусть $\bazis{e}$ и $\bazis{e}'$ --- базисы в $V$, 
а $\bazis{e}^*$ и $\bazis{e}'^*$ --- их взаимные базисы в $V^{*}$ соответственно.
Пусть $S$ --- матрица перехода от $\bazis{e}$ к $\bazis{e}'$, 
а $C$ --- матрица перехода от $\bazis{e}^*$ к $\bazis{e}'^*$.
Тогда $$C = ({S}^{-1})^*.$$
\end{predl}
\dok  Во встречающихся матрицах перехода обозначаем верхним индексом номер строки, 
а нижним --- номер столбца. Так, если $S=(s^k_i)$ --- матрица перехода от 
$\bazis{e} = (\vek{e}_1, \ldots , \vek{e}_n)$ к 
$\bazis{e}' = (\vek{e}'_1, \ldots , \vek{e}'_n)$, то согласно определению матрицы перехода,
$\vek{e}'_i = \sum\limits_{k=1}^n s^k_i \vek{e}_k$, или используя тензорную запись суммирования, 
$$\vek{e}'_i =  s^k_i \vek{e}_k.$$
Аналогично, если $C=(c^m_j)$ --- матрица перехода от 
$\bazis{e}^* = (\vek{e}^1, \ldots , \vek{e}^n)$ к 
$\bazis{e}'^* = (\vek{e}'^1, \ldots , \vek{e}'^n)$, то 
$\vek{e}'^j = \sum\limits_{m=1}^n c^m_j \vek{e}^m$. Чтобы здесь также 
использовать тензорное суммирование, введем матрицу $R=C^T$, так что $R=(r^j_m)$ и  $c^m_j = r^j_m$.
Тогда 
$$\vek{e}'^j =  r^j_m \vek{e}^m$$
Условие биортогональности запишется как 
$\delta_i^j = \lin{\vek{e}'_i, \vek{e}'^j } =    \lin{s^k_i \vek{e}_k,  r^j_m \vek{e}^m}$.
Раскрывая по линейности и используя биортогональность  $\bazis{e}$ и $\bazis{e}^*$, 
получаем условия $\delta_i^j =  s^k_i \overline{r^j_k}$, равносильные матричному равенству $\overline{R}S = E$, откуда $R = \overline{S}^{-1} = \overline{S^{-1}}$, 
отсюда $C=R^T =({S}^{-1})^*$.
\edok

\otstup

Работа одновременно в $V$ и $V^*$ иногда дает больше, чем работа в одном из них. 
Например, следующее предложение является признаком линейной независимости (при $n=k$ 
признаком базиса в $V$).


\begin{predl}\label{p8_4_2}
Пусть система векторов $\bazis{e} = (\vek{e}_1, \vek{e}_2, \ldots , \vek{e}_k)$ пространства
$V$ и система векторов $\bazis{\ell} = (\vek{\ell}^1, \vek{\ell}^2, \ldots , \vek{\ell}^k)$ пространства
$V^{*}$ %(априори не известно, что системы линейно независимы) 
биортогональны, т.е.  таковы, что
 $\langle \vek{e}_i, \vek{\ell}^j \rangle = \delta_{i}^j$ для всех $i, j\in \{1, \ldots, k\}$.
Тогда $\bazis{e}$ и $\bazis{\ell}$ --- линейно независимые системы в $V$ и $V^*$ соответственно. \\
В частности, если $\dim V = n=k$, то $\bazis{e}$ --- базис в $V$, а $\bazis{\ell}$ --- 
взаимный базис для $\bazis{e}$.
\end{predl}
\dok Достаточно доказать линейную независимость системы векторов $\vek{e}_1, \vek{e}_2, \ldots , \vek{e}_k$
(линейную независимость системы векторов $\vek{\ell}^1, \vek{\ell}^2, \ldots , \vek{\ell}^k$ можно доказать аналогично).\\
Пусть $\lambda^i \vek{e}_i=\vek{0}$. Применив к этому равенству $\vek{\ell}^j\in V^{*}$, имеем
 $\lambda ^j=0$. Таким образом, все коэффициенты в этой линейной комбинации должны быть равны 0.
\edok


\subsection{Примеры}

\example{III.1. (интегральный функционал)
Пусть $V=C[a, b]$, и $f_0 \in V$.
Определим $\widetilde{f_0} \in V^{*}$:
$\forall$ $g\in V$ положим $\langle g, \widetilde{f_0} \rangle = \int\limits_{a}^{b} g(x)f_0(x) \, dx$.
\\
Другой пример: $\delta$-функция --- линейная функция $\delta_{\alpha} \in V^{*}$ такая, что
$\forall$~$g\in V$ выполнено:
$\langle g, \delta_{\alpha} \rangle = g(\alpha)$.
}

\example{III.2.
Пусть $V=\mathbf{P}_n$ (многочлены степени $\leq n$); $\alpha_0, \alpha_1, \ldots, \alpha_n$ --- различные числа.
Для системы линейных функнций $\vek{\ell}^i = \delta_{\alpha _i}$, $i=0, 1, \ldots, n$, биортогональной будет
система многочленов $$p_i(x) = \frac{\prod \limits_{k\neq i} (x-\alpha_k)}{\prod \limits_{k\neq i} (\alpha_i-\alpha_k)},$$
$i=0, 1, \ldots, n$. Из предложения \ref{p8_4_2} следует, что 
$(\vek{\ell}^0, \vek{\ell}^1, \ldots, \vek{\ell}^n)$ --- взаимный базис для базиса $(p_0, p_1, \ldots, p_n)$.\\
Разложение многочлена $f\in \mathbf{P}_n$ по $p_i$ имеет вид 
$$\sum\limits_{i=0}^n f(\alpha_i)\, p_i.$$ 
Мы получили так  называемый {\it интерполяционный многчлен} Лагранжа, дающий явную формулу для многочлена
степени не выше $n$, принимающего в заданных $n+1$ точках предписанные значения.
%% С кратными узлами????
}

\section{Естественные изоморфизмы}

Во многих рассуждениях у нас появлялись изоморфизмы, которые бы изменились при другом выборе базиса
(как, скажем, сопоставление вектору его координатного столбца).
В ситуациях, когда этого не происходит, т.е. если изоморфизм не меняется при замене базиса,
будем говорить, что изоморфизм естественный.

\begin{theor}\label{t8_55}
%Пусть $\dim V = n<\infty$. Тогда 
Отображение, сопоставляющее
вектору $\vek{a}\in V$ отображение $\widetilde{\vek{a}}: V^{*}\to \mathbb{R} (\mathbb{C})$ по правилу
\begin{equation}\label{vv**}
\langle  \ell, \widetilde{\vek{a}} \rangle = \overline {\langle  \vek{a}, \ell \rangle} 
\end{equation}
(для любого $\ell \in V^{*}$), является инъективным гомоморфизмом (вложением) $V\to V^{**}$.
\end{theor}
\dok Для доказательства достаточно проверить следующие утверждения. 

1) $\widetilde{\vek{a}}$, заданное правилом (\ref{vv**}), линейно, т.е. действительно $\widetilde{\vek{a}} \in V^{**}$.

2) Соответствие $\vek{a}\to \widetilde{\vek{a}}$ линейно, т.е. проверить, что 
$\widetilde{\vek{a}_1+\vek{a}_2} = \widetilde{\vek{a}_1} + \widetilde{\vek{a}_2}$ и 
$\widetilde{\lambda \vek{a}} = \lambda \widetilde{\vek{a}}$.

3) Тривиальность ядра отображения $\vek{a}\to \widetilde{\vek{a}}$. Если $\widetilde{\vek{a}} = \vek{0}$ (то есть для любого $\ell \in V^{*}$ выполнено
$\langle  \ell, \widetilde{\vek{a}} \rangle  = 0$, то $\vek{a}=\vek{0}$.
\edok


\begin{sled}\label{s8_55}
В случае $\dim V = n<\infty$ указанное соответствие $\vek{a}\to \widetilde{\vek{a}}$ является естественным изоморфизмом
$V\to V^{**}$.
\end{sled}

Соответствие $\vek{a}\to \widetilde{\vek{a}}$ позволяет отождествить далее конечномерное пространство со своим дважды
сопряженным. 

\begin{theor}\label{t8_56}
%Пусть $\dim V = n<\infty$. Тогда 
Пусть $\mathcal{E}$ --- евклидово (унитарное) пространство.
Отображение $\mathcal{E}\to \mathcal{E}^{*}$, сопоставляющее
вектору $\vek{a}\in \mathcal{E}$ отображение $\widehat{\vek{a}}: V^{*}\to \mathbb{R} (\mathbb{C})$
по правилу
\begin{equation}\label{ee*}
\langle  \vek{x}, \widehat{\vek{a}} \rangle = (\vek{x}, \vek{a})
\end{equation}
(для любого $\vek{x} \in \mathcal{E}$), является инъективным гомоморфизмом (вложением) $\mathcal{E}
\to \mathcal{E}^{*}$.
\end{theor}
\dok Для доказательства достаточно проверить следующие утверждения. 

1) $\widehat{\vek{a}}$, заданное правилом \ref{ee*}, линейно, т.е. действительно $\widehat{\vek{a}} \in \mathcal{E}^{*}$.

2) Соответствие $\vek{a}\to \widehat{\vek{a}}$ линейно, т.е. проверить, что 
$\widehat{\vek{a}_1+\vek{a}_2} = \widehat{\vek{a}_1} + \widehat{\vek{a}_2}$ и 
$\widehat{\lambda \vek{a}} = \lambda \widehat{\vek{a}}$.

3) Тривиальность ядра отображения $\vek{a}\to \widehat{\vek{a}}$. 
Если $\widehat{\vek{a}} = \vek{0}$ (то есть для любого $\vek{x} \in V$ выполнено
$\langle  \vek{x}, \widehat{\vek{a}} \rangle  = 0)$, то $\vek{a}=\vek{0}$.
\edok


\begin{sled}\label{s8_56}
В случае $\dim V = n<\infty$ указанное соответствие $\vek{a}\to \widehat{\vek{a}}$ является естественным изоморфизмом
$\mathcal{E}\to \mathcal{E}^{*}$.
\end{sled}

Соответствие $\vek{a}\to \widehat{\vek{a}}$ позволяет отождествить далее конечномерное евклидово (унитарное) 
пространство со своим сопряженным. 
В этом смысле теорию евклидовых (унитарных) пространств можно считать частным случаем 
теории двойственности.



\section{Биортогональность. Соответствие между подпространствами в $V$ и $V^*$}

\subsection{Биортогональность. }

\defin{Множества $U\subset V$ и $W\subset V^*$ называются {\it биортогональными}, если 
$\forall \, \vek{a}\in U$ и $\forall \, \ell\in W$ выполнено $\langle \vek{a}, \ell \rangle   = 0$.
}

Для биортогональности векторов и множеств из $V$ и $V^*$ сохраним обозначение $\perp$.

%Пусть дано $U\leq V$.

\begin{predl}[признак биортогональности]\label{p10_2_3} 
Пусть $U= \lin{\vek{a}_1, \ldots, \vek{a}_k}$ $\leq V$, $W= \lin{\ell_1, \ldots, \ell_l}$ $\leq V^*$.
Тогда  $U \perp W$
$\Leftrightarrow$ $\vek{a}_i\perp \ell_j$, $i=1, 2, \ldots, k$, $j=1, 2, \ldots, l$.
\end{predl}
\dok Аналогично доказательству предложения \ref{p10_2_3}.
%\dokright Очевидно из определения ортогональности подпространств.\\
%\dokleft Пусть $\vek{a}\in U_1$, $\vek{b}\in U_2$. Тогда существуют разложения
% $\vek{a} = \sum\limits_{i=1}^k \alpha _i \vek{a}_i$,  
%$\vek{b} = \sum\limits_{j=1}^l \beta _j \vek{b}_j$. Тогда 
%из линейности скалярного произведения получаем $(\vek{a}, \vek{b}) = \sum\limits_{i=1}^k \sum\limits_{j=1}^l  %\alpha _i \overline{\beta _j} (\vek{a}_i, \vek{b}_j)= 0$, %
%т.е. $\vek{a} \perp \vek{b}$.
\edok


%%%%%%%%%%%%%%%%%%%%
%%%%%%%%%%%%%%%%%%%%
%%%%%%%%%%%%%%%%%%%

\subsection{Биортогональное дополнение}


\defin{Подмножество  $W = \{\ell \in V^* \, |\, U\perp \ell\} $ пространства $V^{*}$
называется {\it аннулятором}, или биортогональным дополнением подпространства $U$.
}

Для евклидова пространства отождествление (\ref{ee*}) превращает определение аннулятора в определение
ортогонального дополнения.
Для аннулятора примем обозначение $U^{\perp}$.

\begin{predl}\label{t8_56}
Для любого $U\leq V$ подмножество $U^{\perp}$ является подпространтвом в $V^*$.
\end{predl}
\dok
\edok

\otstup

Для любого $W\leq V^*$ подмножество $W^{\perp}$ будет является подпространтвом в $V^{**}$. 
Но в конечномерном случае в силу отождествления  $V\cong V^{**}$ заданного (\ref{vv**}), мы можем отождествить
$W^{\perp}$ с подпространтвом в $V$, которое также называется {\it нуль-пространством } подпространства $W$.
Зафиксируем также определение нуль-пространства, не использующее отождествление  $V\cong V^{**}$.

\defin{Подмножество  $U = \{\vek{a} \in V \, |\, \vek{a}\perp W\} $ пространства $V$
называется {\it нуль-пространством} подпространства $W\leq V^*$.
}
 
Для нуль-пространства сохраняем обозначение $W^{\perp}$.

\begin{predl}\label{t8_56}
Для любого $W\leq V^*$ подмножество $W^{\perp}$ является подпространтвом в $V$.
\end{predl}
\dok
\edok

%%%%%%%%%%%%%%%%%%
%%%%%%%%%%%%%%%%%%
%%%%%%%%%%%%%%%%%%%


\begin{theor}
Пусть  $\dim V=n< \infty$,  $U\leq V$. Тогда \\
1) $(U^{\bot})^{\bot} = U$;
2)  $\dim U + \dim U^{\bot} = n$.
\end{theor}
\dok Пусть $\vek{e}_1, \vek{e}_2, \ldots , \vek{e}_k$ --- базиc в $U$. Дополним его до 
базиса $\bazis{e} = (\vek{e}_1,  \ldots , \vek{e}_k ,  \ldots , \vek{e}_n)$ в $V$.
Мы покажем, что для $U^{\perp}$ имеется следующее описание, из которого вытеают оба утверждения:
$U^{\perp} = \lin{\vek{e}^{k+1}, \vek{e}^{k+2},  \ldots , \vek{e}^n}$, где 
$\vek{e}^{1},  \ldots , \vek{e}^n$ --- взаимный базис для баззиса $\vek{e}_{1},  \ldots , \vek{e}_n$  пространства $V$.

\ldots
\edok


{\bf Упражнение.}
%Для подпространств конечномерного евклидова пространства: \\
Пусть $\dim V<\infty$. 
Для $U_i\leq V$, $i=1, 2$, докажите, что $(U_1+U_2)^{\bot} = U_1^{\bot} \cap U_2^{\bot}$
и аналогично, $(U_1\cap U_2)^{\bot} = U_1^{\bot} + U_2^{\bot}$.




\section{Сопряженное преобразование}

%Рассмотрим линейное преобразовние $\ell : \mathcal{E}\to \mathcal{E}$.

\defin{
Преобразование $\psi : V^*\to V^*$
называют {\it сопряженным} преобразованию $\varphi \in L(V,V)$, если $\forall$ 
$\vek{a} \in V$, и  $\forall$ 
$\ell \in V^*$ выполнено
\begin{equation}\label{soprv*}
\boxed{\langle \varphi(\vek{a}), \ell\rangle= \langle \vek{a}, \psi(\ell) \rangle}.
\end{equation}
}

Обозначение для сопряженного преобразования: $\varphi^{*}$. 
Для евклидова пространстве, в силу отождествления (\ref{ee*}), это определения согласуется 
с определением из главы 5.
Из  определения $\varphi^{*}$ однозначно определено. Проверим, что оно линейно. 

\begin{predl}\label{p8_9_0}
Пусть $\varphi \in L(V, V)$. Тогда $\varphi^* \in L(V^*, V^*)$. 
\end{predl}
\dok
\edok


\begin{predl}\label{p8_9_1}
Пусть $\dim V<\infty $, $\bazis{e}$ --- базис в $V$, и $\bazis{e}^*$ --- его 
взаимный базис в $V^*$. Пусть $\varphi \in L(V, V)$, 
 $\varphi \rsootv{\bazis{e}, \bazis{e}} A$. 
$\varphi ^* \rsootv{\bazis{e}^*, \bazis{e}^*} A^{*}$. 
Тогда
\end{predl}
\dok
Аналогично предложению \ref{p10_3_1}. 
\edok


\begin{sled1}
Пусть $\dim V <\infty $, $\varphi, \psi \in L(V, V)$. Тогда \\
1) $(\varphi ^{*})^{*} = \varphi$;\\
2) $(\varphi \psi)^{*} = \psi^{*} \varphi^{*}$;\\
3) $\rg (\varphi) = \rg (\varphi ^{*})$;\\
4) $\overline{\chi_{\varphi} (\lambda)} = \chi_{\varphi ^{*}} (\overline{\lambda})$.
\end{sled1}
\dok Введем в $V$ и $V^*$ взаимные базисы и перейдем к матрицам в этих базисах. 
Далее доказываем так же, как следствие 2 из предложения \ref{p10_3_1} главы \ref{evkl_prostr}.
\edok

\begin{theor}\label{t8_9_1} 
Пусть $U\leq V$, $\dim V <\infty$, $\varphi \in L(V, V)$. Тогда  \\
$U$ инвариантно относительно $\varphi$ 
$\Leftrightarrow$ 
$U^{\bot}$ инвариантно относительно $\varphi ^{*}$.
\end{theor}
\dok 
Аналогично теореме \ref{t10_3_1} главы \ref{evkl_prostr}.
\edok

\begin{theor}[Теорема Фредгольма]\label{t8_9_2} 
Пусть $\dim V=n <\infty$, $\varphi \in L(V, V)$. Тогда  \\
$$\boxed{\Ker \varphi ^{*} = (\Im \varphi)^{\bot }}.$$
\end{theor}
\dok  
Аналогично теореме \ref{t10_3_2} главы \ref{evkl_prostr}.
\edok


\otstup
Отметим, что все содержание этого параграфа обобщается на случай 
линейного отображения $\varphi \in L(V, \widetilde{V})$.
Для $\varphi \in L(V, \widetilde{V})$ так же формула (\ref{soprv*}) позволяет
определить {\it сопряженное отображение} $\varphi^* \in L(\widetilde{V}^*, V^*)$,
для которого выполняются аналоги всех предложений и теорем из этого параграфа.
