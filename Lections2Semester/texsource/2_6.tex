
\section{Структура недиагонализируемых операторов.}


В этом параграфе $V$ --- векторное пространство над полем $\mathbb{F}$,
$\dim V = n<\infty$, а $\varphi$ --- фиксированное линейное преобразование $V\to V$.
Предполагается (если не оговаривается противное), что все характеристические числа оператора $\varphi$
принадлежат $\mathbb{F}$ (для $\mathbb{F} = \mathbb{C}$ это выполнено всегда).

Характеристические числа обозначаем $\lambda_1, \lambda_2, \ldots , \lambda_k$,
а 
$s_1, s_2, \ldots , s_k$ --- их кратности соответственно.

\subsection{Треугольный вид}


\begin{lemm}\label{trtr}
Существует базис, в котором матрица $\varphi$ верхнетреугольная, с заданным порядком 
расстановки характеристических чисел по диагонали.
\end{lemm}
\dok СХЕМА. Требуется найти флаг инвариантных подпространств $\lin{\vek{e}_1, \ldots, \vek{e}_k}$, $k=1, \ldots, n$.

Индукция. Найдем $(n-1)$-мерное инвариантное подпространство $U\geq \Im(\varphi - \lambda _i)$ 
(см. предложение \ref{p8_5_3}). Возьмем любой $\vek{e}_n\notin U$. Тогда $\varphi (\vek{e}_n) = \lambda _i \vek{e}_n + \vek{b}$, где  $\vek{b}\in U$. 
%фактор-оператор - умножение на $\lambda _i$
Тогда для ограничения $\varphi|_U : U\to U$ характеристический многочлен равен 
$\dfrac{\chi_{\varphi}(\lambda)} {  \lambda _i -\lambda }$.
Применяя предположение индукции к $U$, находим в $U$ нужные базисные векторы
$\vek{e}_1, \ldots, \vek{e}_{n-1}$.
\edok

%В вещественном случае предыдущая теорема верна тогда и только тогда, когда
%у $\varphi$  все характеристические числа вещественные.


\otstup

{\bf Упражнение.}
Пусть $f\in \mathbb{F}[X]$. Найдите характеристические числа (и их кратности)
оператора $f(\varphi)$ (зная характеристические числа (и их кратности)
оператора $\varphi$.


\subsection{Корневые подпространства. Теорема Гамильтона-Кэли.}

Определяемые ниже {\it корневые подпространства} можно мыслить как обобщение собственных 
подпространств $V_{\lambda_i} = \Ker (\varphi - \lambda_i)$.

Для каждого характеристического числа $\lambda_i$
рассмотрим систему вложенных подпространств:
\begin{equation}\label{Ker^}
O\leq  \Ker (\varphi - \lambda_i)\leq \Ker (\varphi - \lambda_i)^2 \leq 
\Ker (\varphi - \lambda_i)^3\leq \ldots  
\end{equation}
Так как $\dim V<\infty$, в этой цепочке начиная с некоторого шага наступит стабилизация.
Обозначим $m_i$ номер этого шага, так что 
$\Ker (\varphi - \lambda_i)^{m_{i}-1} %\stackrel{<}{\neq} 
\neq \Ker (\varphi - \lambda_i)^{m_{i}} = \Ker (\varphi - \lambda_i)^{m_{i}+1} = \ldots$.

Подпространство $\Ker (\varphi - \lambda_i)^{m_{i}} $ назовем
{\it корневым}, отвечающим собственному значению $\lambda_i$.
Обозначаем это корневое подпространство 
$V^{\lambda_i}$.
Равество $V^{\lambda_i} = \Ker (\varphi - \lambda_i)^{m} $ будет верно для 
всех $m\geq m_i$, и наоборот, $m_i$ --- минимальное натуральное $m$, 
для которого верно это равенство.



Следующие утверждения о $\Ker (\varphi - \lambda_i)^{t_i}$ (в частности о $V^{\lambda _i}$) 
можно сравнить с теоремами \ref{p8_5_8}, \ref{t8_5_2} и \ref{t8_5_3}
о $V_{\lambda _i}$.

\begin{lemm}\label{p8_5_88888}
Для любых  $t_i\in \mathbb{Z}_{+}$ сумма 
$\sum\limits_{i=1}^{k} \Ker (\varphi - \lambda_i)^{t_i} $ --- прямая сумма.\\
В частности, $\sum\limits_{i=1}^{k} V^{\lambda _i} $ --- прямая сумма.
\end{lemm}
\dok Это предложение сразу следует из предложения \ref{Ker_polynom} главы
\ref{lin_otobr} --- достаточно рассмотреть многочлены $f_i(x) = (x- \lambda _i)^{t_i}$.
\edok

\begin{lemm}\label{Kers_i}
$\dim \Ker (\varphi - \lambda_i)^{s_i} \geq s_i.$
\end{lemm}
\dok
СХЕМА. 
Согласно лемме \ref{trtr}, существует базис
$\bazis{e} = (\vek{e}_1, \ldots, \vek{e}_n)$, в котором матрица оператора $\varphi$ верхнетреугольная, 
причем константны, равные $\lambda_i$, расположены в первых 
$s_i$ диагональных клетках. Это будет означать что для $j=1, \ldots, s_i$ 
выполнено $\varphi (\vek{e}_j)=  \lambda_i \vek{e}_j + \sum\limits_{x=1}^{j-1} c_x \vek{e}_x$, 
или $(\varphi - \lambda_i)(\vek{e}_j ) \in \lin{\vek{e}_1,\ldots , \vek{e}_{j-1} }$;
отcюда $(\varphi - \lambda_i)^j(\vek{e}_j )=0$, в частности, 
$\vek{e}_j \in \Ker (\varphi - \lambda_i)^{s_i}$ для $j=1, \ldots, s_i$.
Требуемое доказано, поскольку в $\Ker (\varphi - \lambda_i)^{s_i} $ нашлась линейно независимая система 
из $s_i$ векторов.
\edok

\begin{sled}
$\dim V^{\lambda _i} \geq s_i.$
\end{sled}


%%"'ВЫПУКЛОСТЬ РАЗМЕРНОСТЕЙ ЯДЕР?? из жнф следует..


\begin{theor} Справедливы следующие утверждения:
\begin{equation}\label{oplusV^}
1).\,\,\,\,\,\,\,\, \boxed{V\, \, = \, \, \bigoplus\limits_{i=1}^k \,\, V^{\lambda_i} };
\end{equation}
$$2).\,\,\,\,\,\,\,\, \boxed{\dim V^{\lambda_i} =  s_i};$$
3) $m_i\leq s_i$  (т.е. стабилизация в (\ref{Ker^}) наступает на $s_i$-м шаге или ранее), $i=1, \ldots, k$
\end{theor}
\dok
1), 2). Из леммы \ref{p8_5_88888}, с учетом последнего следствия $\dim V^{\lambda _i} \geq s_i$,
размерность прямой суммы $\bigoplus\limits_{i=1}^k \dim V^{\lambda_i}$  не меньше $\sum\limits_{i=1}^k s_i=n = \dim V$, значит эта сумма обязана совпадать с $V$, а во все неравенства обязаны обращаться в равенство.
\\
3). Следует из 2) с учетом леммы \ref{Kers_i}.
\edok

\begin{predl}\label{minmn} 
Минимальным многочленом для оператора $\varphi$ является многочлен 
$$\mu(\lambda) = \prod_{i=1}^k (\lambda - \lambda _i)^{m_i}.$$
%(где $m_i$
\end{predl}
\dok Из предложения \ref{Ker_polynom} главы \ref{lin_otobr} следует, что 
\begin{equation}\label{keroplus}
 \Ker (\prod \limits_{i=1}^{k} (\varphi - \lambda_i)^{t_i}) \,\, = \,\,  \bigoplus\limits_{i=1}^{k} 
\Ker (\varphi - \lambda_i)^{t_i}. 
\end{equation}
Если $t_i=m_i$, $i=1, \ldots, k$, то в правой части (\ref{keroplus}) прямая сумма корневых подпространств, т.е. $V$,
значит $\prod \limits_{i=1}^{k} (\varphi - \lambda_i)^{m_i}$ --- нулевой оператор, т.е. 
$\mu(\lambda)$ действительно аннулирует $\varphi$.

Согласно предложению \ref{min_mn} главы \ref{lin_otobr}, минимальный многочлен для $\varphi$ 
является делителем $\mu$, а значит имеет вид $\prod_{i=1}^k (\lambda - \lambda _i)^{t_i}$,
где  $t_i\leq m_i$, $i=1, \ldots, k$. 

Пусть хотя бы одно из этих неравенств строгое, например, 
$t_1< m_1$. Тогда $\dim \Ker (\varphi - \lambda_i)^{t_1} <s_1$ и
$\dim \Ker (\varphi - \lambda_i)^{t_i} \leq s_i$, $i=2, \ldots, k$. Отсюда следует, что 
размерность прямой суммы в правой части (\ref{keroplus}) строго меньше $\sum\limits_{i=1}^k s_i = n$, 
а значит $\Ker (\prod \limits_{i=1}^{k} (\varphi - \lambda_i)^{t_i}) \neq V$. Следовательно, при  
$t_1< m_1$ многочлен $\prod_{i=1}^k (\lambda - \lambda _i)^{t_i}$ не аннулирует $\varphi$.
\edok 

\begin{predl}[критерий-2 диагонализируемости]\label{kr_diag2}
Следующие условия на $\varphi$ эквивалентны:\\
1) $\varphi$ диагонализируем;\\
2) $V_{\lambda_i} = V^{\lambda_i}$ для всех $i=1, 2, \ldots, k$; \\ % $\varphi$;
3) $\mu_{\varphi}$ раскладывается (в $\mathbb{F}[X]$) на различные линейные множители.
%(не имеет кратных корней. %если кто-то аннулирует без кратных корней (распад. на лин. множители, %то диагоналихируемо
\end{predl}
\dok
1) $\Leftrightarrow$ 2). Достаточно составить условие 4) в критерии-1 диагонализируемости (теорема \ref{t8_5_3}), 
(\ref{oplusV^}) и очевидные включения $V_{\lambda_i} \leq V^{\lambda_i}$.\\
2) $\Leftrightarrow$ 3). Условие 2) означает, что $m_i=1$ для всех $i=1, 2, \ldots, k$.
Остается воспользоваться предыдущим предложением \ref{minmn}.
\edok

\otstup

{\bf Упражнение.} Оператор $\varphi\in L(\mathbb{R}^n, \mathbb{R}^n)$ 
удовлетворяет равенству $\varphi ^3 = \varphi$. Докажите, что $\varphi$
диагонализируем.

\otstup


{\bf Упражнение.}
Пусть $\varphi$ диагонализируем, а $U\leq V$ --- инвариантное подпространство. 
Докажите, что оператор $\varphi |_U$ ограничения на $U$ тоже диагонализируем,
{\footnotesize как и фактор-оператор}.


\begin{theor}[Теорема Гамильтона-Кэли]\label{HK} 
Для всякого $\varphi \in L(V, V)$ выполнено $$\chi_{\varphi}(\varphi) = 0. $$
\end{theor}
\dok Из описания минимального многочлена $\mu_{\varphi}$ (предложение \ref{minmn}) следует, что
$\chi_{\varphi}$ делится на $\mu_{\varphi}$, и значит, многочлен $\chi_{\varphi}$ аннулирует ${\varphi}$.
\edok

\otstup

Итак, в случае $\mathbb{F}=\mathbb{C}$ теорема Гамильтона-Кэли доказана для всех операторов, 
а в случае $\mathbb{F}=\mathbb{R}$ --- верна также для всех операторов, но доказана для 
операторов, удовлетворяющих условию, сформулировнному в начале параграфа: все характеристические числа оператора $\varphi$ принадлежат $\mathbb{R}$. Устранить этот недостаток можно, посмотрев на теорему Гамильтона-Кэли как
на формальное тождество для матриц $A\in \mathbf{M}_{n\times n}(\mathbb{F})$.
Так как это тождество верно для всех матриц из  $\mathbf{M}_{n\times n}(\mathbb{C})$, то оно верно 
и для всех матриц из  $\mathbf{M}_{n\times n}(\mathbb{R}) \subset \mathbf{M}_{n\times n}(\mathbb{C})$.

\otstup


{\bf Упражнение.} %$^*$
Матрица $A\in \mathbf{M}_{n\times n}$ такова, что $A^k=O$ при некотором $k\in \mathbb{N}$.  
Докажите, что $A^n=O$.

\otstup


{\bf Упражнение.}$^*$ %позже????
Пусть $\mu \in \mathbb{R}[X]$ --- минимальный многочлен матрицы 
$A\in \mathbf{M}_{n\times n}(\mathbb{R})$. Докажите, что $\mu$ является минимальным 
многочленом для $A$, при рассмотрении $A$ как $A\in \mathbf{M}_{n\times n}(\mathbb{C})$
(и соответственно минимальный среди многочленов над $\mathbb{C}$).\\
Попробуйте обобщить это утверждение для произвольного поля и его расширения 
$\mathbb{F}\subset \mathbb{K}$.
%Лучше додумать бы.... автоморфизм поля сделать и использовать единственность \mu



\subsection{ЖНФ: определения и формулировки.}

%Жордановы клетки и жордановы цепочки.


{\it Жордановой клеткой}, отвечающей константе $\lambda_0$, называют квадратную  матрицу вида
%{\footnotesize
%Пусть $J_t(\lambda_0)$ --- матрица $t\times t$, у которой по главной диагонали --- одинаковые числа,
%на следующей диагонали над главной --- единицы, а остальные элементы --- нули:
$$\begin{pmatrix} \lambda_0 & 1 & 0 &  0 & \ldots & 0 \\ 
0& \lambda_0 & 1 & 0  & \ldots & 0 \\
0& 0& \lambda_0 & 1 &  \ldots & 0 \\
& & &\ldots  & &\\
& & &\ldots  & &\\
0& 0& \ldots & 0 & 0 &\lambda_0 
\end{pmatrix}.$$
%Жорданову клетку $t\times t$ с константами $\lambda_0$ обозначим 
%$J_t(\lambda_0)$.

Блочно-диагональная матрица, у которой каждый диагональный блок является жордановой клеткой, называется {\it жордановой матрицей}, или матрицей, имеющей {\it жорданову нормальную форму} (жнф).
Очевидно, жордановы матрицы являются верхнетреугольными.

Базис, в котором $\varphi$ имеет жнф,  называют {\it жорданов базис}.


<<Прочитаем>> жнф (по определению матрицы линейного преобразования) и поймем, что значит жорданов базис.
Каждая жорданова клетка $t\times t$ соответствует так называемой {\it жордановой цепочке} длины $t$: 
$$ \vek{0} \leftarrow \vek{e}_1 \leftarrow \vek{e}_2 \leftarrow \ldots \leftarrow \vek{e}_t. $$\\
Здесь стрелочкой обозначено применение оператора $\varphi - \lambda_i$.
Каждая жорданова цепочка начинается с собственного вектора, следующие векторы называется
{\it присоединенными} ($\vek{e}_{i+1}$ --- присоединенный для $\vek{e}_i$).

Основными утверждениями здесь являются следующие.

\begin{theor}[существование ЖНФ]
В $V$ существует базис, в котором матрица $\varphi$ --- жорданова.
\end{theor}

Теорему существования можно переформулировать так: существует базис, являющийся объединением жордановых цепочек.

\begin{theor}[единственность  ЖНФ]
Для данного $\varphi$ единственна ЖНФ  (с точностью до перестановки жордановых клеток по диагонали).
\end{theor}

Теорему единственности можно переформулировать так: 
 для каждого $\lambda _i$ количество жордановых цепочек данной длины 
для жорданова базиса определено однозначно.

%Условие единственности жнф нив каком виде не означает единственности, что жорданов базис е

Единственность жнф выполнена несмотря на то, что, 
как увидим в доказательстве сущесвтвования, 
при конструировании жорданова базиса может быть достаточно степеней свободы.



\subsection{Доказательство существования ЖНФ}

Достаточно в каждом корневом подпространстве $V^{\lambda_i}$ построить базис,
являющийся объединением жордановых цепочек. Далее в доказательстве существования
рассматриваем нильпотентный оператор $\psi : V^{\lambda_i} \to V^{\lambda_i}$, 
являющийся ограничением оператора $\varphi - \lambda_i$ на $V^{\lambda_i}$.

\otstup

СХЕМА: построение жордановых цепочкек <<сверху вниз>> в каждом корневом подпространстве,
исходя из <<башни>>  \ref{Ker^}, которую обозначим используя $\psi$:
\begin{equation} %\label{Ker^}
O\leq  \Ker \psi \leq \Ker \psi ^2 \leq 
\Ker \psi  ^3 \leq \ldots  \leq \Ker \psi  ^{m_i} = V^{\lambda_i}.
\end{equation}


Индуктивно выберем подпространства $W_{m_i+1}=O$, далее $W_{m_i}$,  $W_{m_i-1}$ и т.д.
такие, что 
\begin{equation} \label{Woplus}
W_{t+1} \oplus \Ker \psi ^t = \Ker \psi ^{t+1}.
\end{equation}
По $W_{t+1}$ определим $W_t$ ($t\geq 1$) так. 

Если для $W_{t+1}$ верно (\ref{Woplus}), то для $\psi (W_{t+1})$ выполнено:

i)  $\psi (W_{t+1}) \leq \Ker \psi ^t $;\\
(это следует только из того, что $W_{t+1} \leq \Ker \psi ^{t+1}$.)

ii) $\dim (\psi (W_{t+1})) =  \dim W_{t+1}$;\\
(это верно, так как $W_{t+1}\cap \Ker \psi = O$.)

iii) $\psi (W_{t+1}) \cap \Ker \psi ^{t-1} = O $.
(proof)

Из iii) следует, что можно определить $W_t\geq \psi (W_{t+1}) $ так, чтобы выполнялось 
(\ref{Woplus}), т.е. $W_{t} \oplus \Ker \psi ^{t-1} = \Ker \psi ^{t}.$

\otstup

Пользуясь определенными подпространствами $W_t$, несложно построить жордановы цепочки <<сверху вниз>>.
На $t$-м <<слое>>, имея базис в $\psi (W_{t+1})$, дополним его до базиса в $W_t$,
применяя $\psi$, <<спускаем>> на этаж ниже, получая базис в $\psi (W_{t})$.


\otstup

{\bf Упражнение.}
a) Докажите, что если для некотрого $t$ выполнено $\Ker (\varphi - \lambda_i)^t = 
\Ker (\varphi - \lambda_i)^{t+1}$, то 
$\Ker (\varphi - \lambda_i)^t =  V^{\lambda_i}$.

б) Докажите, что последовательность $\dim (\Ker \varphi ^t)  - \dim (\Ker \varphi ^{t-1})$, $t=1, 2, \ldots$, 
невозрастающая.


\subsection{Доказательство единственности жнф}

Пусть дан жорданов базис $B$. 
Сперва поймем, что множество $B_{\lambda _i}$ векторов из $B$, принадлежащих цепочкам, отвечающим данному $\lambda _i$, 
имеет мощность $s_i$.
(Иначе говоря, сумма длин жордановых цепочек, отвечающих $\lambda _i$,  равна $s_i$.)

Пусть $|B_{\lambda _i}| = s_i'$. Так как $B_{\lambda _i} \subset V^{\lambda _i}$,
то $s_i ' \leq \dim V^{\lambda _i} = s_i$.  С другой стороны, всего векторов в жордановом базисе $n$, 
т.е. $\sum\limits_{i=1}^k s_i' = n = \sum\limits_{i=1}^k s_i$. Значит, каждое нерваенство
$s_i ' \leq  s_i$ обращается в равенство.

Кроме того, мы видим, что
$\lin{B_{\lambda _i} } = V^{\lambda _i} $.

\otstup

Далее, как в доказательстве существования
рассматриваем $\psi : V^{\lambda_i} \to V^{\lambda_i}$, 
являющийся ограничением оператора $\varphi - \lambda_i$ на $V^{\lambda_i}$.
$\psi$ действует на жордановых цепочках как <<спуск на один этаж>>.

Поэтому $\psi (V^{\lambda_i}) $ равно линейной оболочке  всех векторов из $B_{\lambda _i} $, 
не являющихся последними векторами своих жордановых цепочек.
Аналогично, $\psi ^2(V^{\lambda_i}) $ равно линейной оболочке  всех векторов из $B_{\lambda _i} $, 
не являющихся последними и предпоследними векторами своих жордановых цепочек, и т.д.

Обозначим  $c_d$ количество жордановых цепочек длины $d$, отвечающих $\lambda _i$. Их наблюдений выше имеем равенства:
$$ \dim V^{\lambda _i} - \dim (\Im \psi ) = c_1+c_2+c_3 \ldots + c_{m_i},   $$
$$ \dim (\Im \psi ) - \dim (\Im \psi ^2) = c_2+c_3 \ldots + c_{m_i},   $$
$$ \dim (\Im \psi ^2) - \dim (\Im \psi ^3) = c_3 \ldots + c_{m_i},   $$
и т.д.
Отсюда можно выразить $c_i$ через (инвариантные) харакетристики $\varphi$.


\otstup

%{Минимальный многочлен в терминах жнф.}

Отметим, что для величины $m_i$ (шаг, на котором происходит стабилизация ядер в (\ref{Ker^}))
теперь есть еще два эквивалетных описания: %максимальная высота векторов из $V^{\lambda _i}$
максимальная длина жордановой цепочки, отвечающей $\lambda _i$, в жордановом базисе,
или максимальный размер жордановой клетки, отвечающей $\lambda _i$, в жнф.
В этих терминах можно теперь формулировать, например, 
описание минимального многочлена (см. предложение \ref{kr_diag2}).



\subsection{О связи с теорией линейных дифференциальных уравнений}

Пусть $V_0= \mathbf{C}^{\infty}$ и $d: V_0 \to V_0$ --- оператор дифференцирования. 

Множество $V\leq V_0$ решений однородного дифф.ур. 
$$y^{(n)}+a_{n-1}y^{(n-1)}+\ldots +a_1y'+a_0y=0$$
--- это $V = \Ker p(d)$, где $p(x) = x^{n}+a_{n-1}x^{n-1}+\ldots +a_1x+a_0$.

\otstup

$V$ инвариантно относительно $d$, поэтому рассмотрим $d$ как сужение $d: V\to V$.\\
Тогда $p$ --- аннулирующий многочлен для $d$. \\

\otstup

Для каждого корня $\lambda_i$ (кратности $s_i$) многочлена $p$ попробуем найти собственный вектор $f$ оператора $d$:\\
$d(f) = \lambda_i f$ --- ответ единственный (с точностью до пропорциональности): $f(x) = e^{\lambda_i x} $.

\otstup

Ясно, что $f\in V$, и так как 
$\Ker (d-\lambda_i)^{s_i} \leq  \Ker p(d)=V$, \\
для корня $\lambda_i$ вся жорданова цепочка длины $s_i$ лежит в $V$:

$e^{\lambda_i x} \leftarrow (1+x) e^{\lambda_i x} \leftarrow (1+x+\dfrac{x^2}{2}) e^{\lambda_i x}
\leftarrow (1+x+\dfrac{x^2}{2}+\dfrac{x^3}{3!}) e^{\lambda_i x} \leftarrow \ldots$.

\otstup

Из теории ДУ известно, что $\dim V = n$, 
поэтому найденный нами жорданов базис для оператора~$d$ ---это базис в пространстве решений $V$.


\subsection{Связь с теорией линейных рекуррент}


Пусть $V_0$ --- пространство всех последовательностей комплексных чисел $f=(f_0, f_1, f_2, \ldots)$ \\
 $\delta: V_0 \to V_0$ --- оператор сдвига: $\delta (f) = g$ так что $g_i=f_{i+1}$. 

\otstup

Рассматриваем подпространство $V\leq V_0$ \\решений однородной рекурренты степени $n$:
$$f_{k+n}+a_{k-1}f_{k+n-1}+\ldots +a_1f_{k+1}+a_0f_k=0.$$
$\dim V = n$ и это $V = \Ker p(\delta)$, \\где $p(x) = x^{n}+a_{n-1}x^{n-1}+\ldots +a_1x+a_0$.

\otstup

$V$ инвариантно относительно $\delta$, поэтому рассмотрим $\delta$ как сужение $\delta: V\to V$.\\
Тогда $p$ --- аннулирующий многочлен для $\delta$. \\

\otstup

Для каждого корня $\lambda_i$ (кратности $s_i$) многочлена $p$ попробуем найти собственный вектор 
$f$ оператора $\delta$:\\
$\delta(f) = \lambda_i f$ --- ответ единственный (с точностью до пропорциональности): \\
$f = (1, \lambda_i, \lambda_i^2, \ldots) $ или $f_k = \lambda_i^k$ --- геометрическая прогрессия.

\otstup

Ясно, что $f\in V$, и так как 
$\Ker (\delta-\lambda_i)^{s_i} \leq  \Ker p(\delta)=V$, \\
 для корня $\lambda_i$ имеется одна жорданова цепочка
и одна <<строго растущая>> башня $\Ker (\delta-\lambda_i) \leq \ldots \leq \Ker (\delta-\lambda_i)^{s_i}$.\\

\otstup

Для предъявления базиса (на этот раз не будем брать жорданов базис)
в корневом подпространстве достаточно взять по вектору
<<с каждого этажа>>.

\otstup

Возьмем $g\in V_0$ вида  $g_k = q_{t-1}(k)\lambda_i^k$, где $q_r$ --- многочлен степени $r$.\\
такое $g\in \Ker (\delta-\lambda_i)^{t} \setminus \Ker (\delta-\lambda_i)^{t-1}$.


 
