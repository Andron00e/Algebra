\section{Аффинные преобразования (на плоскости).}\label{aff}

\subsection{Определение и основные свойства}


%\newpage

$S$ --- аффинное пространство.

далее рассматриваем двумерный случай: $S=\mathcal{P}$ (плоскость) 

$V$ --- двумерное векторное пространство (радиус-векторы на плоскости).


%Пусть зафиксировано $\varphi \in L(V, V)$.



%\section{Линейные и аффинные преобразования плоскости.}
\defin{
Преобразование плоскости $f: \mathcal{P}\to \mathcal{P}$ называется 
линейным (линейно-аффинным) преобразованием, если \\$\exists$ $\varphi\in L(V,V)$ такое, что 
$\forall$ $ M, N \in \mathcal{P}$ выполнено
$$\overrightarrow{f(M)f(N)} = \varphi (\overrightarrow{MN}).$$
} 



Соответствующее линейное отображение $\varphi\in L(V,V)$ обозначим
$\widetilde{f}$; иногда $\widetilde{f}$ называют {\it дифференциалом} отображения $f$.


\defin{
Линейное преобразование $f: \mathcal{P}\to \mathcal{P}$ называется {\it аффинным}, если
оно биективно.
}


\begin{predl}\label{p4_2_3}.\\
Пусть $f: \mathcal{P}\to \mathcal{P}$ --- линейное преобразование.
Тогда \\ $f$ аффинно $\Leftrightarrow$ $\widetilde{f}$ биективно.
\end{predl}


\begin{predl}\label{p4_2_4}.\\
Если $f: \mathcal{P} \to \mathcal{P}$ --- аффинное преобразование, то
$f^{-1}$  также аффинное преобразование, причем $\widetilde{f^{-1}} = \widetilde{f}^{-1}$.
\end{predl}
%\dok
%\edok


\begin{predl}\label{p4_2_5}.\\
1) Если $f: \mathcal{P} \to \mathcal{P}$ и $g: \mathcal{P} \to \mathcal{P}$
--- линейные преобразования, то $gf$ также линейное, причем
$\widetilde{gf} = \widetilde{g}\widetilde{f}$.\\
2) Если $f: \mathcal{P} \to \mathcal{P}$ и $g: \mathcal{P} \to \mathcal{P}$
--- аффинные преобразования, то $gf$ также аффинное.
\end{predl}

\otstup

Как видим,  аффинные  преобразования образуют группу относительно композиции.




\subsection{Координатная запись линейного преобразования.}

Зафиксируем ДСК $(O, \bazis{e})$.

Пусть в определении линейного преобразования $f : \mathcal{P} \to \mathcal{P}$
дифференциал $\widetilde{f} : V\to V$ имеет матрицу
$A$, a $f(O)$ имеет координатный столбец
$C$. 

%Расписывая в
%координатах и в матричном виде, имеем следующее соотношение между координатными столбцами


\begin{predl}\label{p4_2_555}
Пусть $X$ и $Y$ --- координатные столбцы точки $M$  ее образа $f(M)$. Тогда
\begin{equation}
\boxed{Y=AX+C}.
\end{equation}
\end{predl}

Видим, что $C$ --- это координатный столбец для $f(O)$.
Координатные записи для $f$ ($Y=AX+C$) и $\widetilde{f}$ 
($Y=AX$) отличаются <<сдвигом на $C$>>.
Иначе, $f$ может быть представлено в виде композиции 
преобразования с неподвижной точкой $O$ и сдвига (параллельного переноса).


%Геом. смысл столбцов матрицы $A$ и столбца $C$ (?)

%
%Комментарий:
%Обозначение в  задачнике: $X^*$ вместо $Y$.

%Как видим, любое линейное преобразование может быть разложено в произведение $gh$,
%где $h$ --- преобразование

%$$
%\begin{cases}
%x^{*}=a_1 x+b_1y\cr
%y^{*}=a_2 x+b_2y,
%\end{cases}
%$$
%для которого $O$ --- неподвижная точка, а $g$ --- параллельный перенос на вектор
%$\vek{c}$.


\begin{predl}\label{p4_2_6}(критерий аффинности)
Пусть 
дифференциал $\widetilde{f}$ линейного преобразования $f: \mathcal{P} \to \mathcal{P}$ 
имеет в базисе $\bazis{e}$ матрицу $A$. 
Тогда \\ $f$ аффинно (то есть биективно) $\Leftrightarrow$ $|A|\neq 0$.
\end{predl}
%\dok Следует из предложения \ref{p4_2_3}.
%\edok
%\footnote{
%Можно показать, что если $\varphi$ не биективно, то образ $\Im \varphi$ --- либо прямая (множество
%векторов, коллинеарных данному), либо точка (нулевой вектор).
%}


%\otstup

%Последняя теорема находится в согласии с предыдущим предложением: если рассмотреть
%линейное преобразование, для которого $|A| = 0$, то параллелограмм переходит в
%отрезок или точку, то есть в "вырожденный параллелограмм" площади 0.
%Итак, мы увидели, что модуль $|A|$ является коэффициентом изменения площадей, а знак
%$|A|$ говорит об изменении или сохранении ориентации (любого) базиса.
%%Из доказанной теоремы ясно, что величина
%$\delta = \begin{vmatrix}
%a_1 & b_1\\ a_2 & b_2
%\end{vmatrix}$
%не зависит от выбора декартовой системы координат, в которой $f$ задается формулами (*).




\begin{theor}\label{t4_2_2}
%Существование и единственность линейного преобразования, заданного
%образами трех точек, не лежащих на одной прямой.
%(Преобразование аффинно $\Leftrightarrow$ три образа не лежат на одной прямой.)
Пусть даны точки $A, B, C, K, L, M \in \mathcal{P}$, причем точки $A$, $B$ и $C$ не лежат на одной
прямой.\\
Тогда существует единственное линейное преобразование $f:\mathcal{P} \to \mathcal{P}$ такое, что
$f(A)=K$, $f(B)=L$, $f(C)=M$; при этом $f$ аффинно $\Leftrightarrow$ $K$, $L$, $M$ не лежат на
одной прямой.
\end{theor}
%\dok
%\edok

%Даже нужнее для дальнейшего:
%существование и единственность линейного преобразования, заданного
%образом точки и двух неколлинеарных векторов.

%\zad Чем может являться $\Im f$ для линейного преобразования $f:\mathcal{P} \to \mathcal{P}$?

\begin{theor}[связь аффинного преобразования с заменой координат]
Пусть $f:\mathcal{P} \to \mathcal{P}$ --- аффинное преобразование.
Тогда если $M$ имеет координатный столбец
$X$ в ДСК
$(O, \vek{e_1}, \vek{e_2})$, то
$f(M)$ имеет тот же координатный столбец
$X$
в ДСК
$(f(O), \widetilde{f}(\vek{e_1}), \widetilde{f}(\vek{e_2}))$.
\end{theor}
%\footnote{Заметим, что здесь $(f(O), \widetilde{f}(\vek{e_1}), \widetilde{f}(\vek{e_2}))$ --- действительно
%ДСК, так как
%векторы $\widetilde{f}(\vek{e_1}) \nparallel$ не коллинеарны.
%}
%\dok
%\edok

%%%%%%%%%%%%%%%%%%
%%%%%%%%%%%%%%%%%%

\subsection{Примеры}


\example{
Пусть $l$ --- прямая. Выберем ПДСК ($O, \vek{e_1}, \vek{e_2}$)
так, что $O\in l$, $\vek{e_1}\parallel l$. 
%Тогда преобразование $f:\mathcal{P} \to \mathcal{P}$,
%заданное в рассматриваемой системе координат формулами
$$
\begin{cases}
y_1=x_1\cr
y_2=\lambda x_2.
\end{cases}
$$
%что это?
--- при $\lambda = 0$ (ортогональным) проектированием на прямую $l$;\\
%(это линейное, но не аффинное преобразование);\\
%при $\lambda \neq 0 $ --- аффинными преобразованиями:\\
--- при $\lambda = 1$ --- тождественное преобразование;\\
--- при $\lambda = -1$ --- {\it осевая симметрия} (или {\it отражение}) относительно прямой $l$;\\
--- при $\lambda> 0$ --- {\it сжатие} к прямой $l$ с коэффициентом
$\lambda$ (слово "сжатие" употребляется и в случае $\lambda >1$, а иногда оно заменяется на слово
"растяжение");\\
--- при $\lambda< 0$ --- композиция симметрии относительно $l$ и сжатия к $l$.\\
}


\example{
 Пусть $M$ --- некоторая точка. Выберем ДСК
($O, \vek{e_1}, \vek{e_2}$)
так, что $O=M$. 
%Тогда преобразование $f:\mathcal{P} \to \mathcal{P}$,
%заданное в рассматриваемой системе координат формулами
$$
\begin{cases}
y_1=\lambda x_1\cr
y_2=\lambda x_2.
\end{cases}
$$
--- при $\lambda\neq 0$ ---  гомотетия с коэффициентом $\lambda$.
}


\example{
Пусть $\vek{c}$ --- некоторый вектор, имеющий координаты
$\begin{pmatrix}
c_1 \\c_2
\end{pmatrix}$. в некоторой системе координат ($O, \vek{e_1}, \vek{e_2}$).
$$
\begin{cases}
y_1=x_1+c_1\cr
y_2= x_2+c_2
\end{cases}
$$
--- параллельный перенос.
}

\example{
Рассмотрим преобразование $f:\mathcal{P} \to \mathcal{P}$ поворота на угол $\varphi$ (против часовой стрелки
вокруг точки $M$). Выберем прямоугольную систему координат ($O, \vek{e_1}, \vek{e_2}$)
так, что $O=M$. %Пусть точка
$$
\begin{cases}
y_1=\cos \varphi \, x_1 - \sin \varphi \, x_2\cr
y_2=\sin \varphi \, x_1 + \cos \varphi \, x_2.
\end{cases}
$$
%и поэтому является аффинным преобразованием.
%(вспомним матрицу поворота)
}
%\footnote{Композицией осевой симметрии, параллельного переноса и поворота можно получить любое движение
%(см. $\S...$), а композицией движения и гомотетии --- любое преобразование подобия.
%Таким образом, в группе аффинных преобразований имеются вложенные подгруппы: всех преобразований подобия
%и всех движений.}

Из указанных примеров композицией можно получать новые (на самом деле все) аффинные преобразования.
Известно, что всякое афинное преобразование может быть разложено 
в виде произведения движения (=композиция поворота или отражения и сдвига) и сжатий к перпендикулярным осям.


\subsection{Геометрические свойства}



\begin{predl}\label{p4_3_1}
%Следствия (геометрические) для аффинных преобразований:
Пусть $f: \mathcal{P}\to \mathcal{P}$ --- аффинное преобразование. Тогда\\
1) образ прямой --- прямая, образ отрезка --- отрезок;\\
2) параллельные прямые переходят в параллельные прямые;\\
3) отношение длин параллельных отрезков сохраняется;\\
4) образ центрально-симметричной фигуры --- центрально-симметричная фигура.
\end{predl}
%\dok
%\edok


%Следующее теорема в некотором смысле обобщает предыдущее предложение.

\begin{theor}\label{t4_2_1}(изменение площадей)
Пусть 
дифференциал $\widetilde{f}$ линейного преобразования $f: \mathcal{P} \to \mathcal{P}$ 
имеет в базисе $\bazis{e}$ матрицу $A$. 
Тогда \\ 
$$\boxed{S(f(\Pi )) = |\det A| \cdot S(\Pi ) }.$$
\end{theor}
%\footnote{Из доказанной теоремы можно вывести, что
%при аффинном преобразовании площадь (то есть мера Жордана) любой (измеримой по Жордану) фигуры
%изменяется в $|\det A|$ раз.
%Например, исходя из формулы площади круга можно получить формулу площади эллипса %%$\dfrac{x^2}{a^2}+\dfrac{y^2}{b^2}=1$:
%$S=\pi ab$, где $a$ и $b$ --- длины его полуосей (см. далее о связи аффинных преобразований и кривых
%второго порядка).
%}
%\dok
%Следует из теоремы \ref{t4_1_1}.
%\edok


Таким образом, модуль $\det A$ отвечает за изменение площадей, а 
знак $\det A$ --- за сохранение или изменение ориентации.
Это согласуется с предложением о критерии аффинности (случай $\det A = 0$ соответсвует 
вырождению образа парлеллограмма).

%\subsection{Аффинные преобразования и кривые второго порядка.}


\begin{theor}
Образом алгебраической кривой порядка $n$ при аффинном преобразовании является
алгебраическая кривая порядка $n$.
\end{theor}
%\dok
%\edok



%%%%%%%%%%%%%%%%%%%
%%%%%%%%%%%%%%%%%%%%
%%%%%%%%%%%%%%%%%%