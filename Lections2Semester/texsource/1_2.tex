\section{Линейная зависимость. Размерность и ранг. Базис}

\subsection{Линейная зависимость}

%%%%%%%%%%%%%%%%%
%%% сразу символическую матричную запись линейной комбинации???
%%%%%%%%%%%%%%%%%

Продолжаем работать в векторном пространстве $V$ над полем $\mathbb{F}$.


\defin{
Система векторов  $\vek{a}_1, \vek{a}_2, \ldots, \vek{a}_k $
называется {\it линейно зависимой}\index{Линейная!зависимость},
если некоторая их нетривиальная линейная комбинация равна
$\vek{0}$, и {\it линейно независимой} в противном случае.
}

Полагают, что пустая система векторов линейно независима
(формально это согласуется с определением).


\begin{predl}\label{p7_2_1}
Система векторов $\vek{a}_1, \vek{a}_2, \ldots, \vek{a}_k \in V$ ($k\geq 2$)
линейно зависима $\Leftrightarrow$ 
среди векторов $\vek{a}_1, \vek{a}_2, \ldots, \vek{a}_k$ {\bf найдется} вектор, который
%$\exists $ $i\in \{1, 2, \ldots, k\}$: вектор
линейно выражается через остальные $k-1$ векторов этой системы.
\end{predl}
\dok \dokright Пусть $\lambda_1 \vek{a}_1 +\lambda_2 \vek{a}_2+ \ldots + \lambda_k \vek{a}_k
=\vek{0}$, и не все коэффициенты равны $0$,
скажем $\lambda_k \neq 0$. Тогда поделим равенство на $-\lambda_k$
и перенесем $\vek{a}_k$ в другую часть; получим
$\vek{a}_k = \sum\limits_{i=1}^{k-1}\mu_i \vek{a}_i$, где
$\mu_i=-\dfrac{\lambda_i}{\lambda_k}$, $i=1, 2, \ldots , k-1$.

\dokleft Пусть, скажем, вектор $\vek{a}_k$ раскладывается по векторам
$\vek{a}_1, \vek{a}_2, \ldots, \vek{a}_{k-1} $:
$\vek{a}_k = \sum\limits_{i=1}^{k-1}\mu_i \vek{a}_i$. Тогда
$\mu_1 \vek{a}_1 +\mu_2 \vek{a}_2+ \ldots + \mu_{k-1} \vek{a}_{k-1}-\vek{a}_k$ --- нетривиальная
линейная комбинация, равная $\vek{0}$.
\edok

\begin{predl}\label{p7_2_2}
1) Если в конечной системе векторов
имеется некоторая линейно зависимая подсистема, то и вся система линейно зависима.

2) Любая подсистема линейно независимой системы линейно независима.
\end{predl}
\dok 1) Пусть, скажем, для системы векторов $\vek{a}_1, \vek{a}_2, \ldots, \vek{a}_k$
ее подсистема $\vek{a}_1, \vek{a}_2, \ldots, \vek{a}_m$ ($m\leq k$)
линейно зависима. Тогда существует нетривиальная линейная комбинация
$\sum\limits_{i=1}^{m}\mu_i \vek{a}_i$, равная $\vek{0}$.
Значит, 
$\mu_1 \vek{a}_1 +\mu_2 \vek{a}_2+ \ldots + \mu_{m} \vek{a}_{m} +
0\cdot \vek{a}_{m+1} + \ldots + 0\cdot \vek{a}_k$ --- нетривиальная линейная комбинация, равная $\vek{0}$.

2) Это переформулировка утверждения 1).
\edok

\begin{sled}
Любая система векторов, содержащая $\vek{0}$, является линейно зависимой.
\end{sled}

\begin{predl}\label{p7_2_3}
Пусть векторы $\vek{a}_1, \vek{a}_2, \ldots, \vek{a}_k $ и $\vek{b}$ таковы, что 
$\vek{b}\in \lin{\vek{a}_1, \vek{a}_2, \ldots, \vek{a}_k }$.
Тогда коэффициенты $\lambda_1, \lambda_2, \ldots , \lambda_k$
в разложении
$\vek{b}= \sum\limits_{i=1}^{k} \lambda_i \vek{a}_i$
определяются однозначно $\Leftrightarrow$ 
система $\vek{a}_1, \vek{a}_2, \ldots, \vek{a}_k $ линейно независима.
\end{predl}
\dok
\dokright
Предположим, что напротив, 
система $\vek{a}_1, \vek{a}_2, \ldots, \vek{a}_k $ линейно зависима и существует нетривиальная линейная комбинация
$\sum\limits_{i=1}^{k} \alpha_i \vek{a_i}$, равная $\vek{0}$. 
Тогда можно прибавить эту линейную комбинацию к имеющейся линейной комбинации, равной $\vek{b}$, 
и получить новое линейное выражение: 
$\vek{b}= \sum\limits_{i=1}^{k} (\lambda_i+\alpha_i) \vek{a_i}$ (оно действительно хотя бы в одном
коэффициенте отличается от разложения
$\vek{b}= \sum\limits_{i=1}^{k} \lambda_i \vek{a_i}$).

\dokleft
Пусть система $\vek{a}_1, \vek{a}_2, \ldots, \vek{a}_k $ линейно независима, и предположим, что наряду с разложением
$\vek{b}= \sum\limits_{i=1}^{k} \lambda_i \vek{a}_i$
имеется разложение
$\vek{b}= \sum\limits_{i=1}^{k} \mu_i \vek{a}_i.$
Вычитая из первого равенства второе,  получаем
$\sum\limits_{i=1}^{k} (\lambda_i -\mu_i) \vek{a}_i=\vek{0}$.
Так как $\vek{a}_1, \vek{a}_2, \ldots, \vek{a}_k$ --- линейно независимая система, то левая часть
полученного равенства --- тривиальная линейная комбинация, откуда
$\lambda_i =\mu_i$, $i=1, 2, \ldots , k$.
\edok

\otstup

Пусть $\bazis{a} = (\vek{a}_1, \vek{a}_2, \ldots , \vek{a}_k)$ --- строка векторов, 
где $\vek{a}_1, \vek{a}_2, \ldots , \vek{a}_k$ --- линейно независимая система.
Тогда предыдущее предложение можно интерпретировать как закон сокращения:
 $\bazis{a} \lambda = \bazis{a} \mu$ $\Rightarrow$ $\lambda = \mu$.
Этот закон выполнен как для столбцов $\lambda , \mu \in \mathbf{M}_{k\times 1}$, так и для матриц $\lambda , \mu \in \mathbf{M}_{k\times m}$.
Обратим внимание на то, что если отказаться от условия линейной независимости системы
$\vek{a}_1, \vek{a}_2, \ldots , \vek{a}_k$, то следствие $\bazis{a} \lambda = \bazis{a} \mu$ $\Rightarrow$ $\lambda = \mu$,
вообще говоря, неверно.

\subsection{Ранг и размерность}

\defin{
Целое неотрицательное число $r$ называется {\it рангом}\index{Ранг}
непустой системы (возможно, бесконечной) $\mathcal{A}$ векторов из $V$
если в системе $\mathcal{A}$ найдется линейно
независимая подсистема из $r$  векторов, а любая подсистема из $r+1$ векторов является
линейно зависимой.  
}
\defin{
Будем говорить, что $\mathcal{A}$ имеет бесконечный ранг, если для любого $r\in \mathbb{N}$
в $\mathcal{A}$ найдется линейно независимая подсистема из $r$ векторов.
}

Обозначение для ранга: $\rg \mathcal{A}$. 
 В частности,
 $\rg(\vek{a}_1, \vek{a}_2, \ldots, \vek{a}_k)$ --- ранг конечной системы
векторов $\vek{a}_1, \vek{a}_2, \ldots, \vek{a}_k$.

В том случае, когда $\mathcal{A}$ является подпространством в $V$, более употребительное
название для ранга ---
{\it размерность}\index{Размерность}. Обозначение для размерности --- $\dim \mathcal{A}$.
Итак, если $U\leq V$, то $\boxed{\dim U = \rg U}$.
Пространство размерности $k$ называют $k$-{\it мерным}.
Если $\dim V < \infty$, то пространство $V$ называют {\it конечномерным}, иначе --- {\it бесконечномерным}.

Очевидно, система из одного нулевого вектора имеет ранг 0,
а ранг любой конечной системы из $k$ векторов не превосходит $k$.

\begin{predl}\label{p7_2_4}
Конечная система векторов $\vek{a}_1, \vek{a}_2, \ldots, \vek{a}_k $ линейно независима $\Leftrightarrow$
$\rg (\vek{a}_1, \vek{a}_2, \ldots, \vek{a}_k) = k$.
\end{predl}
\dok Сразу следует из определения.
\edok

\otstup

\begin{predl}\label{p7_2_5} Пусть $\mathcal{A}_1, \mathcal{A}_2$ --- две системы векторов из
$V$, причем
$\rg \mathcal{A}_1 = r_1$, $\rg \mathcal{A}_2 = r_2$. Тогда
$\rg(\mathcal{A}_1 \bigcup \mathcal{A}_2) \leq r_1+r_2$.
\end{predl}
\dok 
Пусть это не так, и в объединении наборов
$\mathcal{A}_1$ и $\mathcal{A}_2$ нашлась
линейно независимая подсистема из $r_1+r_2+1$ векторов.
Но (по определению ранга и предложению \ref{p7_2_2})
среди этих векторов не более $r_1$ векторов из $\mathcal{A}_1$
и не более $r_2$ векторов из $\mathcal{A}_2$. Противоречие.
\edok

\begin{sled}\label{sled7_2_5} Пусть $\mathcal{A}_i$, $i=1, 2, \ldots, k$ --- системы векторов из
$V$, причем
$\rg \mathcal{A}_i = r_i$. Тогда
$\rg(\mathcal{A}_1 \bigcup \ldots \bigcup \mathcal{A}_k) \leq \sum\limits_{i=1}^k r_i$.
\end{sled}



\begin{predl}\label{p5_2_5} 
Пусть $\mathcal{A}, \mathcal{B}$ --- две системы векторов из $V$, причем $\mathcal{A} \subset \mathcal{B}$ и
$\rg \mathcal{B} = r$. Тогда $\rg \mathcal{A} \leq r$.
\end{predl}
\dok Сразу следует из определения.
\edok

\otstup

Предыдущее предложение почти очевидно: если к системе векторов $\mathcal{A}$ добавить
некоторые векторы (расширить $\mathcal{A}$ до системы $\mathcal{B}$), то ранг не уменьшится.
Оказывается, ранг не изменится, если к $\mathcal{A}$ добавлять линейные комбинации векторов из
$\mathcal{A}$ (и наоборот, ранг не изменится, если из системы векторов удалить вектор,
который раскладывается по оставшимся векторам).
%
%Может ли ранг при этом остаться неизменным, и если да, то при каком условии?
%Оказывается, при добавлении к системе $\mathcal{A}$ некоторых столбцов ранг не изменится
%тогда и только тогда,
%когда каждый из добавляемых столбцов линейно выражается через столбцы системы
%$\mathcal{A}$.
Обобщение этого факта составляет содержание следующей основной теоремы о рангах.
%Теперь мы докажем основную теорему о рангах.
Доказательству теоремы предпошлем две леммы (которые являются частными
случаями теоремы).

\begin{lem1} Пусть $\mathcal{A}$ --- система векторов из $V$, 
$\rg \mathcal{A} = r<\infty $ и $\vek{a}_1, \ldots , \vek{a}_r$ ---  
линейно независимая подсистема векторов из $\mathcal{A}$.
Тогда любой вектор из $\mathcal{A}$ раскладывается по $\vek{a}_1, \ldots , \vek{a}_r$.
\end{lem1}
\dok Пусть $\vek{a}$ --- произвольный вектор из $\mathcal{A}$.
%Достаточно доказать, что $A$ линейно выражается через столбцы $A_1, $ $A_2, \hm\ldots , $ $A_r$.
%
%Это утверждение очевидно, если $A$ совпадает с одним из столбцов
%$A_1, A_2, \hm\ldots , A_r$.
Из определения ранга следует, что система из $r+1$ векторов  $\vek{a}, $ $\vek{a}_1, \ldots , \vek{a}_r$
линейно зависима. Тогда найдутся
числа $\mu, $ $\lambda_1, \hm\ldots , $ $\lambda_r\in \mathbb{R}$, не все равные нулю и такие, что
%$|\mu|+ \sum\limits_{i=1}^{r}|\lambda_i|>0$ и
$\mu \vek{a}+ \sum\limits_{i=1}^{r} \lambda_i \vek{a}_i = \vek{0}$. При этом $\mu \neq 0$, иначе система
$\vek{a}_1, \ldots , \vek{a}_r$ была бы линейно зависимой. Отсюда
$\vek{a} = - \sum\limits_{i=1}^{r} \dfrac{\lambda_i}{\mu} \vek{a}_i$.
\edok

\otstup

\begin{lem2} Пусть $\vek{a}_1, \ldots , \vek{a}_k$, $\vek{b}$ --- такие векторы из $V$, что 
$\vek{b}\in \lin{\vek{a}_1, \hm\ldots , \vek{a}_k}$.
Тогда
$\rg (\vek{a}_1, \hm\ldots , \vek{a}_k) \hm= \rg (\vek{a}_1, \hm\ldots , \vek{a}_k, \vek{b})$.
\end{lem2}
\dok %Утверждение очевидно, если $B$ совпадает с одним из столбцов
%$A_1, A_2, \hm\ldots , A_k$.
Положим $r=\rg (\vek{a}_1, \hm\ldots , \vek{a}_k)$ ($r\leq k$).
Предположим, что утверждение неверно, и в системе $\vek{a}_1, \hm\ldots , \vek{a}_k, \vek{b}$ нашлась
линейно независимая подсистема из $r+1$ векторов. Тогда один из этих
$r+1$ векторов --- это $\vek{b}$ (так как в системе $\vek{a}_1, \hm\ldots , \vek{a}_k$ нет линейно независимой подсистемы из
$r+1$ векторов).
Итак, пусть для определенности $\vek{b}, \vek{a}_1, \hm\ldots , \vek{a}_r$ --- линейно независимая система.
Из предложения \ref{p7_2_2} следует, что система $\vek{a}_1, \hm\ldots , \vek{a}_r$  линейно независима,
значит, по лемме 1 каждый из векторов $\vek{a}_1, \hm\ldots , \vek{a}_k$
лежит в $\lin{\vek{a}_1, \hm\ldots , \vek{a}_r}$.
%равен линейной комбинации векторов-столбцов $A_1, A_2, \hm\ldots , A_r$.
По условию $\vek{b}$ равен линейной комбинации векторов $\vek{a}_1, \hm\ldots , \vek{a}_k$:
$\vek{b} = \sum\limits_{i=1}^{k}\lambda _i \vek{a}_i$.
Подставив в это выражение %$B$ по векторам-столбцам $A_1, A_2, \hm\ldots , A_k$
вместо векторов  $\vek{a}_1, \hm\ldots , \vek{a}_k$ их разложения по векторам $\vek{a}_1, \hm\ldots , \vek{a}_r$, получим, что
$\vek{b}$ раскладывается по векторам $\vek{a}_1, \hm\ldots , \vek{a}_r$.
Но это противоречит линейной независимости
системы $\vek{b}, \vek{a}_1, \hm\ldots , \vek{a}_r$  (см. предложение \ref{p7_2_1}).
\edok

\otstup

\begin{theor}[основная теорема о рангах]\label{t5_2_1}\index{Теорема!основная о рангах}
Пусть $\mathcal{A}$ и $\mathcal{B}$ --- две такие системы векторов из $V$, что $\mathcal{A} \subset \mathcal{B}$.
Пусть $\rg \mathcal{A} = r< \infty$. Тогда $\rg \mathcal{B} = r$ $\Leftrightarrow$
любой вектор из $\mathcal{B}$ принадлежит $\lin{\mathcal{A}}$.
%линейно выражается через несколько %(конечное число)
%векторов-столбцов из $\mathcal{A}$.
\end{theor}
\dok
\dokright В системе $\mathcal{A} $ зафиксируем некоторую линейно независимую подсистему
$\vek{a}_1, \ldots, \vek{a}_r$ из $r$ векторов.
Так как $\vek{a}_1, \ldots, \vek{a}_r\hm\in \mathcal{B}$ и $\rg \mathcal{B} = r$,
то по лемме 1 (примененной к системе $\mathcal{B}$) любой вектор из $\mathcal{B}$ лежит в
$\lin{\vek{a}_1, \ldots, \vek{a}_r}$ и, следовательно, в $\lin{\mathcal{A}}$.

\dokleft Предположим, что утверждение неверно, и в системе $\mathcal{B}$ нашлась
линейно независимая подсистема из $r+1$  векторов $\vek{b}_1, \vek{b}_2, \hm\ldots, \vek{b}_{r+1}$.
Каждый из них линейно выражается через несколько (конечное число) векторов из
$\mathcal{A}$, поэтому   можно выбрать конечную
систему векторов $\vek{a}_1, \vek{a}_2, \hm\ldots , \vek{a}_l$ из $\mathcal{A}$,
через которые линейно выражается каждый из векторов
$\vek{b}_1, \vek{b}_2, \hm\ldots, \vek{b}_{r+1}$.
Имеем $\rg (\vek{a}_1, \vek{a}_2, \hm\ldots , \vek{a}_l, \vek{b}_1, \vek{b}_2, \hm\ldots, \vek{b}_{r+1}) \hm\geq
\rg (\vek{b}_1, \vek{b}_2, \hm\ldots, \vek{b}_{r+1}) \hm= r\hm+ 1$.
С другой стороны, применяя многократно лемму 2, имеем:
$\rg (\vek{a}_1, \vek{a}_2, \hm\ldots , \vek{a}_l, \vek{b}_1, \vek{b}_2, \hm\ldots, \vek{b}_{r+1}) \hm=
\rg (\vek{a}_1, \vek{a}_2, \hm\ldots , \vek{a}_l, \vek{b}_1, \vek{b}_2, \hm\ldots, \vek{b}_{r}) \hm=
\rg (\vek{a}_1, \vek{a}_2, \hm\ldots , \vek{a}_l, \vek{b}_1, \vek{b}_2, \hm\ldots, \vek{b}_{r-1}) \hm= \hm\ldots \hm=
\rg (\vek{a}_1, \vek{a}_2, \hm\ldots , \vek{a}_l) \hm\leq \rg \mathcal{A} \hm= r$. Противоречие.
\edok

%\otstup

%В ПРИМЕРАХ БУДЕТ!!!!!
%\begin{sled2}
%Для любой системы  $\mathcal{A}$  векторов-столбцов из $\mathbf{M}_{m\times 1}$
%ее ранг определен, причем $\rg \mathcal{A}\hm\leq m$. %$\rg \mathbf{M}_{1\times m} = m$.
%\end{sled2}
%\dok
%Очевидно, $\rg \mathcal{A}\hm\leq \rg \mathbf{M}_{m\times 1}$.
%Согласно предложению \ref{p5_2_3'}, в множестве $\mathbf{M}_{m\times 1}$ всех столбцов
%высоты $m$ имеется базисная подсистема из $m$ столбцов.
%Отсюда $\rg \mathbf{M}_{m\times 1} = m$.
%%Значит,
%%нельзя выбрать линейно независимую систему из более чем $m$ столбцов.
%\edok

%\otstup

\begin{sled1}
Для любой системы векторов $\mathcal{A} $ из $V$
выполнено $\rg \mathcal{A} = \dim \lin{\mathcal{A}}$.
\end{sled1}

\begin{sled2}
Пусть даны подпространства $U\leq W\leq V$ такие, что $\dim U = \dim W <\infty $. Тогда $U=W$.
\end{sled2}


%\otstup

\defin{
%Упорядоченную конечную 
Подсистему $\vek{a}_1, \ldots , \vek{a}_k$
системы векторов $\mathcal{A}$ будем называть {\it базисной}, %для подмножества $\mathcal{A}\subset V$,
если\\
1. $\vek{a}_1, \ldots , \vek{a}_k$ --- линейно независима,\\
2. любой вектор из $\mathcal{A}$ принадлежит линейной оболочке $\lin{\vek{a}_1, \ldots , \vek{a}_k}$.
}


%\begin{sled1}[описание базисных подсистем]
%Пусть $\mathcal{A}$ --- система векторов из $V$,
%и $\rg \mathcal{A} = r<\infty $.
%Тогда базисными для $\mathcal{A}$ являются
%в точности линейно независимые подсистемы из $r$ векторов.
%%Подсистема $A_1, $ $A_2, $ $\hm\ldots ,$ $k$ в $\mathcal{A}$ является базисной
%%$\Leftrightarrow$
%%она линейно независима и $k=r$.
%\end{sled1}


\begin{theor}\label{t7_2_2}
Пусть $\rg \mathcal{A} = r<\infty $. Тогда подсистема $\vek{a}_1, \vek{a}_2, \ldots , \vek{a}_k$ системы $\mathcal{A}$ является базисной для $\mathcal{A}$
$\Leftrightarrow$ $\vek{a}_1, \vek{a}_2, \ldots , \vek{a}_k$ линейно независима и  $k=r$.
\end{theor}
\dok Сразу следует из основной теоремы \ref{t5_2_1}.
\edok

\begin{predl}\label{p7_2_6}
Пусть $\rg \mathcal{A} = r<\infty $. Пусть $\vek{a}_1, \vek{a}_2, \ldots , \vek{a}_k$ --- линейно независимая подсистема 
системы $\mathcal{A}$. 
%Пусть $\rg \mathcal{A} = r< \infty$.
Тогда систему $\vek{a}_1, \vek{a}_2, \ldots , \vek{a}_k$ можно дополнить $r-k$ векторами до
базисной подсистемы. 
%линейно независимой системы $\vek{a}_1, \vek{a}_2, \ldots , \vek{a}_k, \vek{a}_{k+1}, \ldots , \vek{a}_r$
%векторов из $\mathcal{A}$.
\end{predl}
\dok
Если $k<r$, то $k= \rg (\vek{a}_1, \vek{a}_2, \ldots , \vek{a}_k ) < \rg \mathcal{A}$,
и по основной теореме \ref{t5_2_1} в $\mathcal{A}$ найдется вектор $\vek{a}_{k+1}$,
который не выражается линейно через $\vek{a}_1, \vek{a}_2, \ldots , \vek{a}_k$.
Тогда $\rg (\vek{a}_1, \vek{a}_2, \ldots , \vek{a}_{k+1} ) > k$, следовательно %(см. предложение \ref{p7_2_4})
$\vek{a}_1, \vek{a}_2, \ldots , \vek{a}_{k+1}$ --- линейно независимая подсистема.
Продолжая процесс добавления векторов, в конце концов придем к базисной подсистеме.
\edok

\otstup

{\bf Упражнение.}$^*$ Теорию ранга и размерности можно развить, начиная с другого (эквивалентного)
определения ранга: рангом подмножества $\mathcal{A}\subset V$ можем называть 
минимальное $r$ такое, что существует подмножество $\mathcal{B}$ такое, что $|\mathcal{B}|=r$ 
и $\mathcal{A} \subset \lin{\mathcal{B}}$.

\subsection{Базис}

%Будем рассматривать лишь конечные базисы.

\defin{ Упорядоченный конечный набор векторов $\vek{e}_1, \vek{e}_2, \ldots , \vek{e}_n$
называется {\it базисом} векторного пространства $V$, если \\
B1. $\vek{e}_1, \vek{e}_2, \ldots , \vek{e}_n$ --- линейно независимая система;\\
B2. $V= \lin{\vek{e}_1, \vek{e}_2, \ldots , \vek{e}_n}$.
}

Видим, что определение базиса согласуется с определением базисной подсистемы.

Базисы часто будем для краткости обозначать одной буквой, имея в виду упорядоченный набор векторов или 
строку из векторов, например: $\bazis{e} = (\vek{e}_1, \vek{e}_2, \ldots, \vek{e}_n)$.

\begin{theor}[Описание базисов]\label{t7_3_1}
Пусть $\dim V = n< \infty$. Тогда 
упорядоченный набор векторов  $\vek{e}_1, \ldots , \vek{e}_k$ является базисом пространства $V$
$\Leftrightarrow$ система $\vek{e}_1, \ldots , \vek{e}_k$ линейно независима и  $k=n$.
%1. Любой упорядоченный набор из $n$ линейно независимых векторов является базисом; \\
%2. Любой базис содержит ровно $n$ векторов.
\end{theor}
\dok Следует из теоремы \ref{t7_2_2}.
%определения базиса и основной теоремы о рангах.
\edok
\otstup

Существование базисов в векторных пространствах конечной размерности
следует из предыдущей теоремы. Если $\dim V = n<\infty$, то каждый базис пространства $V$
содержит ровно $n$ векторов.
Если $\dim V = \infty$, то конечного базиса в $V$ не существует
(в этом курсе мы не будем заниматься понятием бесконечного базиса).


\begin{predl}\label{p7_3_1}
Пусть $\dim V = n < \infty$, и
$\vek{e}_1, \vek{e}_2, \ldots , \vek{e}_k$ --- линейно независимая система векторов.
Тогда эту систему можно дополнить до
базиса $\vek{e}_1, \vek{e}_2, \ldots , \vek{e}_k, \vek{e}_{k+1}, \ldots , \vek{e}_n$
пространства~$V$.
\end{predl}
\dok
Следует из предложения \ref{p7_2_6}.
\edok

\otstup

\subsection{Координаты}

\defin{
Пусть в векторном пространстве $V$ зафиксирован базис
$\bazis{e} = (\vek{e}_1, \vek{e}_2, \ldots, \vek{e}_n)$.
Коэффициенты $x_1, x_2, \ldots , x_n$ в разложении
$\vek{a}= x_1\vek{e}_1+x_2 \vek{e}_2+ \ldots+x_n \vek{e}_n$
вектора $\vek{a}\in V$ по этому базису называются
{\it координатами}\index{Координаты!вектора} вектора $\vek{a}$ в базисе $\bazis{e}$.
}

Из предложения \ref{p7_2_3} следует, что упорядоченный набор координат
данного вектора $\vek{a}$ в данном базисе однозначно определен.
Упорядоченный набор координат
$(x_1, x_2, \ldots, x_n)$ удобно записывать в виде столбца:
$X = \stolbec{x_1\\ x_2\\ \vdots \\ x_n}$. Этот столбец называется {\it координатным столбцом}\index{Столбец!координатный}
вектора $\vek{a}$ в базисе~$\bazis{e}$.
Для любого упорядоченного
набора координат имеется вектор из $V$ именно с таким набором координат.
Таким образом, если в векторном
пространстве $V$ зафиксирован базис $\bazis{e}$, %$ = (\vek{e}_1, \vek{e}_2, \ldots, \vek{e}_n)$,
то имеется взаимно-однозначное соответствие между множеством $V$ и 
множеством $\mathbf{M}_{n\times 1}$ (вектору сопоставлятся координатный столбец).
Запись $\vek{a} = \bazis{e} X$
будет означать, что вектор $\vek{a}$ имеет координатный столбец  $X$  в базисе
$\bazis{e}$. 

{\footnotesize Также  разложения вида
$\sum\limits_{i=1}^nx_i\vek{e}_i$
можно записывать в виде так называемого {\it тензорного суммирования} --- следующей договоренности, предложенной Эйнштейном: 
наличие одинаковых верхнего и нижнего индекса означает суммирования по этому индексу (от 1 до $n=\dim V$).
Так, если в координатах $x_i$ вместо нижних индексов использовать верхние, то запись $\sum\limits_{i=1}^nx^i\vek{e}_i$
 можно сократить до $x^i\vek{e}_i$.
}
%(эта запись согласуется с символическим умножением матриц:
%$(\vek{e}_1 \, \vek{e}_2 \, \ldots \, \vek{e}_n) \stolbec{
%x_1\\ x_2\\ \vdots \\ x_n}$ ).

Следующее предложение показывает, что записи вида
$\vek{a} = \bazis{e}X $ согласуются с дистрибутивностью матричного умножения.



\begin{predl}[линейность сопоставления координат]\label{p7_3_2}
Пусть в $V$ зафиксирован базис $\bazis{e} = (\vek{e}_1, \vek{e}_2, \ldots, \vek{e}_n)$.
Тогда при сложении векторов соответствующие координаты складываются,
а при умножении вектора на число $\lambda\in \mathbb{R}$ соответствующие
координаты умножаются на $\lambda$.
\end{predl}
\dok По условию $\vek{a}=\bazis{e}X = \sum\limits_{i=1}^nx_i\vek{e}_i$,
$\vek{b}=\bazis{e}Y = \sum\limits_{i=1}^ny_i\vek{e}_i$.\\
Сложив равенства, имеем
$\vek{a}+\vek{b}=\sum\limits_{i=1}^n(x_i+y_i)\vek{e}_i = \bazis{e} (X+Y) $.\\
Умножив первое равенство на $\lambda$, имеем
$\lambda \vek{a}=\sum\limits_{i=1}^n(\lambda x_i )\vek{e}_i = \bazis{e} (\lambda X)$.
\edok

\otstup

Итак, при фиксации базиса $\bazis{e}$ равенство $\vek{a} = \bazis{e} X$ возникает важное линейное взаимно-однозначное соответствие (ниже для таких соответствий вводится термин <<изоморфизм>>) между $V$ и $\mathbf{M}_{n\times 1}$. % --- сопоставление вектору его координатного столбца.
Эта <<координатизация>> дает возможность вопросы о линейных операциях в $V$ перевести на матричный язык
(вместо абстрактных элементов $V$ можно заниматься столбцами). В частности, отметим такое следствие
предложения \ref{p7_3_2} (оно согласуется с общими свойствами изоморфизма --- см. главу \ref{lin_otobr}).

\begin{sled}\label{sled7_3_2}
Пусть в $V$ зафиксирован базис $\bazis{e}$, и в этом базисе векторы $\vek{a}_1, \ldots, \vek{a}_k$
имеют координатные столбцы $X_1,\ldots, X_k$ соответственно: $\vek{a}_i = \bazis{e} X_i$, $i=1, \ldots, k$.
Тогда система векторов $\vek{a}_1, \ldots, \vek{a}_k$ линейно зависима $\Leftrightarrow$
система столбцов $X_1, \ldots, X_k$ линейно зависима.
\end{sled}
\dok 
Согласно предложению \ref{p7_3_2}, 
$\sum\limits_{i=1}^k \lambda_i\vek{a}_i = \vek{0}$ $\Leftrightarrow$ 
$\sum\limits_{i=1}^k \lambda_iX_i = O$.
\edok


\subsection{Замена базиса и координат}

Пусть $\bazis{e} = (\vek{e}_1, \vek{e}_2, \ldots , \vek{e}_n)$ и
$\bazis{e}' = (\vek{e}_1', \vek{e}_2', \ldots , \vek{e}_n')$ --- два базиса
в векторном пространстве $V$ ($n=\dim V$).
%(Условно назовем их {\it старый} и {\it новый}.)

\defin{Матрица $S$ размера $n\times n$, $j$-ый столбец которой равен координатному столбцу
вектора $\vek{e}_j'$ в базисе $\bazis{e} $ ($j=1, 2, \ldots, n$), называется {\it матрицей перехода } от базиса
$\bazis{e}$ к базису~$\bazis{e}'$.
}

Определение означает, что столбцы $s_{\bullet 1}, \ldots , s_{\bullet n}$ матрицы перехода от 
$\bazis{e}$ к $\bazis{e}'$ удовлетворяют равенствам $\vek{e}_1' = \bazis{e}s_{\bullet 1}$, \ldots, 
$\vek{e}_n' = \bazis{e}s_{\bullet n}$.
Эти равенства можно записать в виде  $\boxed{\bazis{e}' = \bazis{e} S}$.
По сути это компактная запись определения матрицы перехода. 

\begin{predl}\label{p7_3_222}
Пусть в пространстве $V$ выбран базис $\bazis{e}$. Матрица $S \in \mathbf{M}_{n\times n}$
является матрицей перехода от $\bazis{e}$ к некоторому базису $\bazis{e}'$ тогда и только тогда, 
когда $S$ невырожденная.
\end{predl}
\dok Согласно следствию из предложения \ref{p7_3_2}, $S$ является 
матрицей перехода к некоторому базису $\bazis{e}'$ тогда и только тогда, 
когда столбцы $S$ образуют базис в $\mathbf{M}_{n\times 1}$.
\edok

\begin{predl}\label{p7_3_223}
Пусть в пространстве $V$ выбран базис $\bazis{e}$. Матрица 
перехода от $\bazis{e}$ к $\bazis{e}$ равна единичной матрице $E_n$.
\end{predl}
\dok Следует из определения матрицы перехода.
\edok

\begin{theor}\label{t7_3_2}
Пусть $\dim V=n<\infty $ и $\vek{a}\in V$ имеет в базисах
$\bazis{e}$ % = (\vek{e}_1, \vek{e}_2, \ldots , \vek{e}_n)$ 
и
$\bazis{e}'$ % = (\vek{e}_1', \vek{e}_2', \ldots , \vek{e}_n')$ 
координатные столбцы
$X$ и $X'$. Тогда $$\boxed{X = SX '},$$ где $S$ --- матрица перехода от
базиса $\bazis{e}$ к базису $\bazis{e}'$.
\end{theor}
\dok
По условию $\vek{a} = \bazis{e}X = \bazis{e}'X'$.
Так как $\bazis{e}' = \bazis{e} S$, имеем
$\bazis{e}X = (\bazis{e}S) X' = \bazis{e} (S X')$. В последнем равенстве используется 
ассоциативность умножения матриц, она верна и в случае <<необычной>> матрицы $\bazis{e}$,
элементы которой --- векторы.
Далее из закона сокращения (здесь используем, что $\bazis{e}$ --- линейно независимая система)
получаем $X=SX'$.
\edok

\begin{predl}\label{p7_3_3}
 Пусть $\bazis{e}$, $\bazis{e}'$, $\bazis{e}''$ --- три базиса в $V$.
Пусть $S$ --- матрица перехода от $\bazis{e}$ к $\bazis{e}'$,
а $R$ --- матрица перехода от $\bazis{e}'$ к $\bazis{e}''$. Тогда
матрица перехода от $\bazis{e}$ к $\bazis{e}''$ равна $SR$.
\end{predl}
\dok
$\bazis{e}''=\bazis{e}'R = (\bazis{e}S)R = \bazis{e}(SR)$.
\edok

\otstup

\begin{sled} Если матрица перехода от базиса $\bazis{e}$ к базису $\bazis{e}'$ равна $S$, то 
 матрица перехода от базиса $\bazis{e}'$ к базису $\bazis{e}$ равна  $S^{-1}$.
\end{sled}
\dok Положив в предложении \ref{p7_3_3} $\bazis{e}''=\bazis{e}$, с учетом предложения \ref{p7_3_223}
получим $SR=E$.
\edok

\subsection{Примеры}

\example{I.
Понятие базиса на плоскости и в 
 пространстве согласуется с определением из курса геометрии:
базисы на плоскости --- упорядоченные пары неколлинеарных векторов
(т.е. плоскость является двумерным пространством);
базисы в пространстве --- упорядоченные тройки некомпланарных векторов
(т.е. геометрическое пространство является трехмерным).
}

\example{II.1.
{\it Стандартный базис} в $\mathbf{M}_{m\times n}$ образуют матрицы, элементы которых ---
все нули, кроме одной единицы. Тогда числа, записанные в ячейках произвольной 
матрицы $A\in \mathbf{M}_{m\times n}$, --- это коэффициенты в разложении $A$ по стандартному базису.
Отсюда $\dim \mathbf{M}_{m\times n} = mn$. 
}

\otstup

{\bf Упражнение.} Найдите $\dim \mathbf{M}^+_{n\times n}(\mathbb{R})$ и $\dim \mathbf{M}^-_{n\times n}(\mathbb{R})$.
Укажите некоторые базисы этих подпространств.

\otstup

\example{II.2.
Пространство столбцов $V=\mathbf{M}_{n\times 1} (\mathbb{F})$ можно назвать
{\it стандартным} $n$-мерным пространством над полем $\mathbb{F}$ 
(в согласии с замечаниями о <<координатизации>>). \\
Базисы в $V$ --- во взаимно-однозначном соответствии 
с множеством $GL_n (\mathbb{F})$ невырожденных матриц:
каждая невырожденная матрица $A = (a_{\bullet 1}\,\, \ldots \,\, a_{\bullet n})$ определяет базис
$(a_{\bullet 1}, \ldots  , a_{\bullet n})$.
В частности, 
верхнетреугольная матрица с ненулевыми константами на главной диагонали определяет так называемый
{\it треугольный базис} простанства $V$. \\
Та же матрица $A$ совпадает с матрицей перехода от стандартного базиса к базису 
$(a_{\bullet 1}, \ldots ,a_{\bullet n})$.\\
%Так группа невырожденных матриц $GL_n (\mathbb{F})$  естественно действует на множестве базисов.\\
}

\example{II.3.
Подпространство $U\leq \mathbb{R}^n = \mathbf{M}_{n\times 1}(\mathbb{R})$ 
может быть задано
как линейная оболочка нескольких столбцов: $U=\lin{a_{\bullet 1}, \ldots, a_{\bullet k}}$.
Базис $U$ в этом случае можно найти, выполнив элементарные преобразования столбцов (они не изменяют их линейную оболочку) до ступенчатого вида.
Имеем $\dim U =\rg A$, где $A$ --- матрица,  $(a_{\bullet 1}\,\, \ldots \,\, a_{\bullet k})$.
}

\example{II.4.
Подпространство $U\leq \mathbb{R}^n$ может быть задано
как множество решений некоторой однородной СЛУ (системы линейных уравнений) $AX=O$, т.е. $U=\Sol(AX=O)$.
При этом  если $\rg A = r$, то $\dim U = n-r$. Определение базиса в $U = \Sol(AX=O)$ 
совпадает с определением ФСР (фундаментальной системы решений) СЛУ $AX=O$.\\
В указанном виде может быть задано любое подпространство $U\leq \mathbb{R}^n$ 
(имеется алгоритм получения СЛУ, для которой данная линейная оболочка столбцов явлется
множеством решений; см. также предложение \ref{10_2_5} \, главы \ref{evkl_prostr}).
}

%\example{II.3.
%Ранг матрицы $A\in \mathbf{M}_{m\times n}$ --- это ранг системы ее столбцов или системы ее строк 
%(теорема о ранге матрицы обеспечивает равенство столбцового и строчного ранга).
%Ранг можно найти, приводя матрицу элементарными преобразованиями строк к ступенчатому виду.
%Ранг матрицы равен количеству ненулевых строк в ступенчатом виде, а ведущие элементы строк в 
%ступенчатом виде помечают номера столбцов, которые образуют в исходной матрице базисную подсистему столбцов 
%(конечно, базисная подсистема столбцов не обязательно единственная). 
%}

\example{III.1.
Набор степеней $1, x, x^2, \ldots, x^n$ --- базис в пространстве $\mathbf{P}_n$ 
(формальных) многочленов степени не выше $n$. Так,  $\dim \mathbf{P}_n = n+1$.
(Пространство $\mathbf{P}$ всех многочленов уже не является конечномерным.)
\\ 
Также набор функций $1, x, x^2, \ldots, x^n$ является базисом в пространстве полиномиальных 
функций $\mathbb{R} \to \mathbb{R}$ степени не выше $n$.
Действительно, если линейная 
комбинация  $\sum\limits_{i=0}^n \lambda_ix^i$ равна нулевой функции, то 
все коэффициенты равны 0, иначе многочлен степени $k$ будет иметь больше чем $k$ корней.\\
}
\example{III.2.
Рассмотрим  пространство фибоначчиевых последовательностей $V = \{(x^1, x^2, \ldots) \, | \, x^{i+1}=x^i+x^{i-1}, \, i=2, 3, \ldots \}$.
Каждая последовательность из $V$ однозначно задается первыми двумя членами $x^1, x^2$. 
Рассмотрим две последовательности $e_1 = (1, 0, 1, 1, 2, \ldots )$ и $e_2 = (0, 1, 1, 2, 3, \ldots )$. 
Они  образуют {\it стандартный базис} в $V$: каждая последовательность $(x^1, x^2, \ldots) \in V$ совпадает с линейной комбинацией $x^1e_1+x^2e_2$
(поскольку совпадают первые два члена). В частности, $\dim V = 2$.\\
В $V$ можно обнаружить известные последовательности --- геометрические прогрессии.
Действительно $ (1, \lambda, \lambda ^2, \ldots) \in V$ $\Leftrightarrow$ $\lambda^{i+1}=\lambda^{i}+\lambda^{i-1}$, $i=2, 3, \ldots$.
Последнее множество равенств эквивалентно единственному равенству $\lambda^2=\lambda+1$, откуда $\lambda_{1,2} = \frac{1\pm \sqrt{5}}{2}$.
Найденные прогрессии $g_i = (1, \lambda_i, \lambda_i ^2, \ldots)$, $i=1, 2$, образуют базис в $V$.\\
Линейное выражение (обычной) последовательности Фибоначчи $f=(1, 1, 2, 3, 5, 8, \ldots)$ через $g_i$ 
позволит найти явную формулу для чисел Фибоначчи.\\
Равенство $f=c_1g_1+c_2g_2$ эквивалентно системе равенств для первых двух членов: \\
$1=c_1+c_2$, $1=c_1\lambda_1+c_2\lambda_2$. Отсюда $c_1=\frac{\sqrt{5}+1}{2\sqrt{5}}=\frac{\lambda_1}{\sqrt{5}}$, 
$c_2=\frac{\sqrt{5}-1}{2\sqrt{5}}=-\frac{\lambda_2}{\sqrt{5}}$. И окончательно, $n$-е число Фибоначчи
равно $c_1 \lambda_1^{n-1} + c_2 \lambda_2^{n-1}= \frac{\lambda_1^{n} - \lambda_2^{n}}{\sqrt{5}}$.
}

\example{IV.
Пусть $\mathbb{F}$ --- конечное поле из $q$ элементов (например, $\mathbb{F} = \mathbb{Z}_p$ для
некоторого простого $p$), а $V$ --- векторное пространство над $\mathbb{F}$, $\dim V = n$.\\
Фиксация любого базиса определяет биекцию $V \to \mathbf{M}_{n\times 1}(\mathbb{F}) = \mathbb{F} ^n$,
откуда $|V| = q^n$.\\
Далее, выбор (упорядоченной) линейно независимой системы $\vek{a}_1, \ldots, \vek{a}_k$ 
в $V$ можно осуществить, последовательно выбирая 
$\vek{a}_1\neq \vek{0}$, $\vek{a}_2\notin \lin{ \vek{a}_1}$, $\vek{a}_3\notin \lin{ \vek{a}_1, \vek{a}_2}$, и т.д.
Для выбора $\vek{a}_{i+1}$ имеется $(q^n-q^{i})$ возможностей (все векторы из $V$, исключая лежащие в 
$i$-мерном подпространстве $\lin{ \vek{a}_1, \ldots, \vek{a}_{i}}$). Отсюда количество таких линейно независимых систем 
равно $\prod\limits_{i=1}^k (q^n-q^{i})$.\\
В частности, количество базисов в $V$ (равное также $|GL_n(\mathbb{F})|$) равно
$\prod\limits_{i=1}^n (q^n-q^{i})$.\\
Линейно независимая система $\vek{a}_1, \ldots, \vek{a}_k$ определяет подпространство
$\lin{\vek{a}_1, \ldots, \vek{a}_k}$, и каждое $k$-мерное подпространство определяется таким образом
столькими способами, сколько в нем базисов. Отсюда находим количество $k$-мерных подпространств в $V$ как
$\dfrac{\prod\limits_{i=1}^k (q^n-q^{i})}{\prod\limits_{i=1}^k (q^k-q^{i})}$.
}




%О РАЗРЕШЕНИИ ЛИНЕЙНЫХ РЕККУРЕНТ Жорданов базис и пр. собственные функции. Оператор сдвига
%то же про диффуры.

%\example{III.3.
%Каждому решению линейного однородного ДУ $x^{(n)}+ a_{n-1}(t)x^{(n-1)}+\ldots + a_0(t)x=0$,
%сопоставим {\it столбец начальных условий} $\stolbec{x(0)\\x'(0) \\ \ldots \\ x^{(n-1)}(0)$.
%В том случае, если $a_i(t)$ --- непрерывные функции $\mathbb{R}\to \mathbb{R}$,
%теорема существования и единственности решения задачи Коши из курса дифференциальных уравнений
%доказывает, что это сопоставление биективно, поэтому является изоморфизмом пространства решений данного ДУ и $\mathbb{R}^n$.
%В частности, пространство решений $n$-мерно.
%}

