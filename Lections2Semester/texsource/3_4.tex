\section{Знакоопределенные формы. Индексы инерции}

\defin{
(Эрмитова) квадратичная форма $k$ на пространстве $V$ называется {\it положительно определенной}, если
$\forall \vek{a} $ $\in V$, $\vek{a}\neq \vek{0}$,  выполнено $k (\vek{a})>0$.
}

\defin{
(Эрмитова) квадратичная форма $k$ на пространстве $V$ называется {\it положительно полуопределенной}, если
$\forall \vek{a} $ $\in V$  выполнено $k (\vek{a})\geq 0$.
}

Аналогично определяются {\it отрицательно определенные} и {\it отрицательно полуопределенные} формы.
Конечно, существуют квадратичные формы, 
не принадлежащие ни к одному из определенных типов, например $x_1^2-x_2^2$.


\otstup

{\bf Упражнение.} Форма $k$ положительно определена и имеет в некотором базисе матрицу $B$. 
Может ли диагональный элемент матрицы $B$ быть неположительным?


\otstup

Следующая несложная теорема показывает, как выяснить по диагональному виду формы, является ли она положительно определенной, положительно полуопределенной, и т.д.


\begin{theor}\label{t9_4_1}
Пусть $k\in \mathcal{K}(V)$, $\bazis{e}$ --- базис в $V$ и $k \rsootv{\bazis{e}} \diag (d_1, d_2, \ldots, d_n)$. Тогда\\
$k$ положительно определена $\Leftrightarrow$  $d_i>0$ для всех $i=1, \ldots, n$;\\
$k$ положительно полуопределена $\Leftrightarrow$  $d_i\geq 0$ для всех $i=1, \ldots, n$;\\
$k$ отрицательно определена $\Leftrightarrow$  $d_i<0$ для всех $i=1, \ldots, n$;\\
$k$ отрицательно полуопределена $\Leftrightarrow$  $d_i\leq 0$ для всех $i=1, \ldots, n$.
\end{theor}
\dok Докажем для положительно определенных форм (в остальных случаях доказательство аналогично).

\dokright
Имеем  $d_i = k(\vek{e}_i)>0$ (из опрделения положительной определенности).

\dokleft Пусть все $d_i>0$ и $\vek{a}\neq \vek{0}$ --- произвольный вектор, $X = \stolbec{x_1\\ x_2\\ \vdots \\ x_n}$ --- его координатный столбец.
Тогда $k(\vek{a}) = \sum\limits_{i=1}^n d_i |x_i|^2 >0$, поскольку хотя бы одна координата $x_i$ ненулевая. 
\edok

\begin{sled}
Определитель матрицы положительно определенной формы положителен.
\end{sled}
\dok 
Определитель матрицы положительно определенной формы в базисе, где она имеет диагональный вид, положителен.
А значит, по следствию 2 из теоремы \ref{t9_1_2}, определитель положиетелен и для произвольного базиса.
\edok



\defin{
{\it Положительным  индексом инерции} квадратичной формы $k$
называется наибольшее целое число $p$, для которого
существует такое подпространство  $U\leq V$, $\dim U = p$, что сужение
$k \mid_U$ является положительно определенной формой.
}

Аналогично определяется отрицательный индекс инерции. Положительный и отрицательный индекс формы $k$ обозначаем $p$ (или $p(k)$) и $q$ (или $q(k)$) соответственно.
Очевидно, форма $k$ положительно определена $\Leftrightarrow$ $p(k) = n$ (где $n=\dim V$), форма $k$ положительно полуопределена 
$\Leftrightarrow$ $q(k) = 0$.


\begin{theor}[об индексах инерции]\label{t9_4_2}
Пусть $k\in \mathcal{K}(V)$, $\bazis{e}$ --- базис в $V$ и $k \rsootv{\bazis{e}} \diag (d_1, d_2, \ldots, d_n)$. Тогда
$p(k)$ равно количеству положительных чисел среди $d_1, d_2, \ldots, d_n$, а $q(k)$ --- количеству отрицательных чисел среди $d_1, d_2, \ldots, d_n$.
\end{theor}
\dok Пусть $p'$ --- количество положительных среди чисел $d_1, \ldots, d_n$. Докажем, что $p=p'$, где $p=p(k)$. Для отрицательного индекса инерции рассуждения будут аналогичны.

Можно считать, что $p'$ первых $d_i$ положительны (этого можно добиться перестановкой векторов в базисе $\bazis{e}$),
т.е. $d_1>0, \ldots, d_{p'}>0$, $d_{p'+1}\leq 0, \ldots,  d_n\leq 0$.
Положим $U_{p'} = \lin{\vek{e}_1, \ldots, \vek{e}_{p'}}$, $W_{p'} = \lin{\vek{e}_{p'+1}, \ldots, \vek{e}_{n}}$.
Заметим, что  $k \mid_{U_{p'}}$  положительно определена, а $k \mid_{W_{p'}}$ отрицательно полуопределена.
Так как $\dim U_{p'}=p'$, имеем $p\geq p'$. 

Предположим, что $p>p'$, тогда рассмотрим подпространство $U\leq V$ такое, что $\dim U=p$ и $k \mid_{U}$  положительно определена.
Имеем $\dim U+\dim W_{p'} = p+(n-p') >n $. Но $\dim (U+W_{p'}) \leq \dim V = n$, значит по теореме \ref{t7_4_3} главы \ref{lin_prostr}
получаем $\dim (U\cap W_{p'})>0$, то есть найдется ненулевой вектор $\vek{a}\in U\cap W_{p'}$.
Поскольку $\vek{a}\in U$, имеем $k(\vek{a})>0$, а так как $\vek{a}\in W_{p'}$, имеем $k(\vek{a})\leq 0$. Противоречие.
\edok

\begin{sled}
Сумма индексов инерции $p(k)+q(k)$ равна рангу $\rg k$.
\end{sled}

Для выяснения, является ли данная форма положительно определенной, без приведения ее к диагональному виду, иногда применяется следующий критерий.


\begin{theor}[критерий Сильвестра]\label{t9_4_3}
Пусть $k\in \mathcal{K}(V)$, $\bazis{e}$ --- базис в $V$ и $k \rsootv{\bazis{e}} B$. Тогда
$k$ положительно определена $\Leftrightarrow$ $|B_i|>0$ для всех $i=1, 2, \ldots, n$.
\end{theor}
\dok 
\dokright
Если $k$ положительно определена, то, очевидно $k\mid_U$ тоже положительно определена для любого подпространства $U\leq V$.
Поскольку $B_i$ --- матрица сужения $k$ на подпространство $U_i = \lin{\vek{e}_1, \ldots, \vek{e}_i}$ (см. предложение \ref{p9_1_1}), 
достаточно применить следствие из теоремы \ref{t9_4_1}.

\dokleft
 Предположим противное и найдем минимальное $m$, для которого форма  $\tilde{k} = k\mid_{U_m}$ не является положительно определенной
(где $U_m = \lin{\vek{e}_1, \ldots , \vek{e}_m}$).
По выбору $m$ получаем, что форма $k\mid_{U_{m-1}} = \tilde{k} \mid_{U_{m-1}}$ положительно определена, значит $p(\tilde{k})\geq m-1$. Так
как $p(\tilde{k}) < m$ (иначе $\tilde{k}$ была бы положительно определенной), имеем $p(\tilde{k}) =  m-1$.
%, т.е. положительный индекс инерции $p(\tilde{k})\leq m-1$.
Пусть $D=\diag (d_1, \ldots, d_m)$ %$\sum\limits_{i=1}^{m} d_i|x_i|^2$ 
--- диагональный вид (в некотором базисе) формы $\tilde{k}$. Тогда среди чисел  $d_1, \ldots, d_m$ ровно $m-1$ положительных --- все кроме одного, 
значит $|D| = d_1d_2\ldots d_m\leq 0$. Но это противоречит условию $|B_m|>0$, поскольку $D$ и $B_m$ --- матрицы одной и той же формы $\tilde{k}$ в разных базисах
(см. следствие 2 из теоремы \ref{t9_1_2}).
\edok

\otstup

Чтобы выяснить, является ли данная форма $k$ отрицательно определенной, можно проверить на положительную определенность форму $(-k)$. 

\otstup

%Следствие: критерий для отрицательно определенных форм \\
%(Вытекает из того, что $k$ отрицательно определена $\Leftrightarrow$ $-k$ положительно определена.)

{\bf Упражнение.}
Пусть $k\in \mathcal{K}(V)$, $\bazis{e}$ --- базис в $V$ и $k \rsootv{\bazis{e}} B$. Тогда
$k$ положительно полуопределена $\Rightarrow$ $|B_i|\geq 0$ для всех $i=1, 2, \ldots, n$.
Обратное утверждение неверно.


{\bf Упражнение.}
Пусть $k\in \mathcal{K}(V)$,  положительно определена. Тогда сумма всех элементов ее матрицы
больше 0.
%(значение на векторе (1,1, ..., 1)  (ясно из координатного расписывания в сумму)

{\bf Упражнение.}
%Задачи на оценивание положительного индекса инерции (большое изотропное подпространство напр. дано)
Пусть положительный индекс инерции формы $k$ не равен 0. Докажите, что 
существует базис, в котором в матрице формы $k$ все диагональные элементы положительные
(а внедиагональные --- какие угодно).
