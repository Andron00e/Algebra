\section{Диагональный и канонический вид квадратичной формы}


\defin{
Говорят, что (эрмитова) квадратичная форма $k$ имеет {\it диагональный вид} в базисе $\bazis{e}$ конечномерного пространства $V$,
если матрица формы $k$ в базисе $\bazis{e}$ диагональна.
}

\defin{
Диагональный вид (эрмитовой) квадратичной формы называется {\it каноническим}, если 
каждый диагональный элемент матрицы равен одному из чисел $1, 0, -1$.
}


Диагональный вид формы $k$ в базисе $\bazis{e}$ означает, что $k(\vek{a}) = \sum\limits_{i=1}^n d_i x_i\overline{x_i}$ или 
$k(\vek{a}) = \sum\limits_{i=1}^n d_i |x_i|^2$, где $d_i\in \mathbb{R}$.
Заметим, что от диагонального вида легко перейти к каноническому, выполнив простую замену координат: $x_i = x_i'/\sqrt{|d_i|}$ для всех $i$ с условием $d_i\neq 0$.

Докажем следующую основную теорему.


\begin{theor}\label{t9_3_1}
Пусть $\dim V=n<\infty$,  $k \in \mathcal{K} (V)$. Тогда существует базис, в котором $k$ имеет диагональный вид. 
\end{theor}
\dok 
%Доказательство проведем индукцией по $n$. База $n=1$ очевидна. Предположим теорема верна для размерностей меньших $n$. Докажем утверждение для формы $k$ на 
%$n$-мерном пространстве $V$.
%Пусть $\bazis{e}$ --- некоторый данный базис и $k \rsootv{\bazis{e}} B$.
%Достаточно научиться приходить к базису $\bazis{e}'}$, в котором у матрицы $B'$ формы $k$ 
%все элементы первой строки и первого столбца, за исключением возможно $b'_{11}$ равны нулю, т.е. $\beta().
%Далее мы можем зафиксировать базисный вектор $\vek{e}'_1$ и продолжить аналогичныете же действия с подматрицей $(n-1)\times (n-1)$, полученной отбрасыванием первой строки и первого столбца.
Достаточно научиться выполнять двойные элементарные преобразования, приводящие к матрице $B'$, у которой 
все элементы первой строки и первого столбца, за исключением возможно $b'_{11}$ равны нулю. 
Далее с матрицей $B'$ можно продолжить аналогичные двойные элементарные преобразования,
не затрагивающие первые строку и столбец, и т.д.

1) Пусть $b_{11}\neq 0$. Последовательно проведем для всех $m=2, 3, \ldots, n$, для которых $b_{m1}\neq 0$,  следующие двойные элементарные преобразовния:
вычтем из $m$-й строки 1-ю строку, умноженную на $b_{m1}/b_{11}$, а затем вычтем из $m$-го столбца 1-й столбец, умноженный на $b_{1m}/b_{11} = \overline{b_{m1}}/b_{11}$.
Двойное элементарное преобразование соответствует следующей операции с матрицами: $B\to C^TB\overline{C} $, где $C$ --- элементарная матрица, т.е. соответствует замене базиса.
После указанной серии двойных элементарных преобразований все элементы перой строки и первого столбца, за исключением $b_{11}$ станут равными нулю.

2) Пусть $b_{11}=0$, но для некоторого $m$ верно $b_{m1}\neq 0$. Тогда сведем ситуацию к случаю 1), предварительно выполнив следующее двойное элементарное преобразование:
прибавим к $1$-й строке $m$-ю строку, умноженную на $\overline{b_{m1}}$, а затем прибавим к $1$-му столбцу $m$-й столбец, умноженный на $b_{m1} = \overline{b_{1m}}$.
После этого в левом верхнем углу окажется число $2|b_{m1}|^2\neq 0$.
\edok

\begin{sled}
Пусть $\dim V=n<\infty$,  $k \in \mathcal{K} (V)$. Тогда существует базис, в котором $k$ имеет канонический вид.
\end{sled}

Алгоритм, описанный в доказательстве теоремы \ref{t9_1_2}, позволяет также вести <<протокол>>, т.е. отслеживать преобразования базиса.
Координатные столбцы исходного базиса образуют единичную матрицу (это исходная матрица перехода). Далее при каждом двойном преобразовании с текущей матрицей перехода
$S$ проделываем только столбцовое преобразование.