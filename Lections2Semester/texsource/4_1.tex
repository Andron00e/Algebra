\chapter{Евклидовы и унитарные пространства}\label{evkl_prostr}



\section{Скалярное произведение. Матрица Грама}%\label{matr_grama}

\subsection{Определения}

Билинейную (эрмитово) симметричную положительно
определенную функцию называют также {\it скалярным произведением}.

\defin{{\it Евклидовым пространством} называется векторное пространство
над $\mathbb{R}$, в котором зафиксировано скалярное произведение.
}

\defin{{\it Унитарным пространством} называется векторное пространство
над $\mathbb{C}$, в котором зафиксировано скалярное произведение.
}

Будем развивать теорию евклидовых и унитарных пространств параллельно (все случаи, когда
имеются отличия, будем отмечать). В этой главе, если не оговорено противное, 
предполагается, что мы работаем в евклидовы (унитарные) пространстве~$\mathcal{E}$.
% как правило будем обозначать


Для обозначения скалярного произведения векторов $\vek{a}$ и $\vek{b}$
будем использовать $(\vek{a},\vek{b})$ (опуская $\beta$ в записи $\beta(\vek{a}, \vek{b})$).


\defin{{\it Длиной}, или {\it нормой} вектора
$\vek{a}$ называют величину $\sqrt{ (\vek{a},\vek{a})}$.
}

Длину обозначают $|\vek{a}|$ или $||\vek{a}||$.
Из положительной определенности скалярного произведения
вытекает, что $\forall \, \vek{a} \in \mathcal{E}$ выполнено
$|\vek{a}|\geq 0$, причем $|\vek{a}|= 0$
$\Leftrightarrow$ $\vek{a} = \vek{0}$.

Операцию деления вектора на его длину (т.е. переходу к единичному вектору того же направления)
иногда называют нормированием (<<отнормируем вектор>>).


\defin{Говорят, что векторы $\vek{a}$ и $\vek{b}$ ортогональны, если $(\vek{a}, \vek{b})=0$.
}

Обозначение для ортогональности векторов: $\vek{a}\perp \vek{b}$. Нетрудно заметить, что существует единственный вектор, 
ортогональный любому вектору --- это $\vek{0}$.

\otstup

Заметим, что любое подпространство $U$ евклидова (унитарного) пространства
также естественным образом
становится евклидовым (унитарным) (в качестве скалярного произведения на $U$
выступает сужение скалярного произведения на объемлющем пространстве).

\subsection{Матрица Грама}

\defin{{\it Матрицей Грама} системы векторов
$\vek{a}_1, \vek{a}_2, \ldots, \vek{a}_k$ называется матрица
$(\gamma _{i j}) \in \mathbf{M}_{k\times k}$ такая, что
$\gamma _{i j} = (\vek{a}_i, \vek{a}_j)$.
}

Таким образом, матрица Грама представляет собой  <<таблицу умножения>> базисных векторов.
Обозначение для матрицы Грама: $\Gamma (\vek{a}_1, \vek{a}_2, \ldots, \vek{a}_k)$.

Отметим, что понятие матрицы Грама не является абсолютно  новым.
Скажем, если $\bazis{e} = (\vek{e}_1, \vek{e}_2, \ldots, \vek{e}_n)$ --- базис, то
матрица $\Gamma (\vek{e}_1, \vek{e}_2, \ldots, \vek{e}_n)$ совпадает
с матрицей билинейной формы скалярного произведения в базисе $\bazis{e}$ (см. определение из $\S$ 
\ref{matr_bilin_formy} главы 
\ref{kvadr_formy}).
Более общо, если $\vek{a}_1, \vek{a}_2, \ldots, \vek{a}_k$ --- линейно независимая система
векторов, то $\Gamma (\vek{a}_1, \vek{a}_2, \ldots, \vek{a}_k)$
совпадает с матрицей скалярного произведения на подпространстве
$U = \lin{\vek{a}_1, \vek{a}_2, \ldots, \vek{a}_k}$ (в базисе $\vek{a}_1, \vek{a}_2, \ldots, \vek{a}_k$).

Следующая теорема говорит о том, что зная матрицу 
Грама, можно вычислять скалярное произведение, а значит, длины и (как увидим далее) другие метрические характеристики.

\begin{theor}[Cкалярное произведение]\label{t10_1_1}
Пусть $\bazis{e} = (\vek{e}_1, \vek{e}_2, \ldots, \vek{e}_n)$ --- базис
в $\mathcal{E}$, и $\Gamma = \Gamma(\vek{e}_1, \vek{e}_2, \ldots, \vek{e}_n)$ --- матрица Грама.
Если векторы $\vek{a}, \vek{b} \in \mathcal{E}$ таковы, что
$\vek{a} = \bazis{e} X$ и $\vek{b} = \bazis{e} Y$, то $$\boxed{(\vek{a}, \vek{b}) = X^T\Gamma \overline{Y}.}$$
\end{theor}
\dok Это частный случай теоремы \label{t9_1_1} из главы \ref{kvadr_formy}.
\edok

Докажем следующее свойство определителя матрицы Грама, геометрический смысл которого 
связан с объемом (обсуждается ниже).

\begin{predl}\label{p10_1_1}
Если система векторов $\vek{a}_1, \vek{a}_2, \ldots, \vek{a}_k$ линейно независима,
то $| \Gamma (\vek{a}_1, \vek{a}_2, \ldots, \vek{a}_k) | >0$; иначе
$| \Gamma (\vek{a}_1, \vek{a}_2, \ldots, \vek{a}_k) | =0$.
\end{predl}
\dok 
1) Пусть система векторов $\vek{a}_1, \vek{a}_2, \ldots, \vek{a}_k$ линейно независима. Тогда, как было отмечено выше,
$\Gamma (\vek{a}_1, \vek{a}_2, \ldots, \vek{a}_k)$ --- это матрица скалярного произведения на подпространстве
$U = \lin{\vek{a}_1, \vek{a}_2, \ldots, \vek{a}_k}$ (в базисе $\vek{a}_1, \vek{a}_2, \ldots, \vek{a}_k$).
То есть $\Gamma (\vek{a}_1, \vek{a}_2, \ldots, \vek{a}_k)$ --- это матрица положительно определенной билинейной симметричной формы, значит,
$|\Gamma (\vek{a}_1, \vek{a}_2, \ldots, \vek{a}_k)|>0$ по следствию из теоремы \ref{t9_4_1} главы \ref{kvadr_formy}.

2) Пусть система векторов $\vek{a}_1, \vek{a}_2, \ldots, \vek{a}_k$ линейно зависима, т.е. 
$\sum\limits_{i=1}^n \lambda_i \vek{a}_i = \vek{0}$ для некоторого ненулевого столбца $\lambda = \stolbec{\lambda_1\\  \vdots \\ \lambda_n}$.
Домножив это равенство скалярно на каждый из векторов $\vek{a}_j$, $j=1, \ldots, k$, и воспользовавшись линейностью, 
получим $\sum\limits_{i=1}^n \lambda_i (\vek{a}_i, \vek{a}_j) = 0$. Получается, что столбец $\lambda$ 
%Система полученных равенств записывается как матричное равенство
является нетривиальным решением системы линейных уравнений с (квадратной) 
матрицей коэффициентов $ \Gamma (\vek{a}_1, \vek{a}_2, \ldots, \vek{a}_k)^T $. Значит, эта матрица вырожденная, откуда
$| \Gamma (\vek{a}_1, \vek{a}_2, \ldots, \vek{a}_k)| = 0$.
\edok


\otstup 

\begin{sled1}[Неравенство Коши-Буняковского-Шварца (КБШ)]
$\forall$ $\vek{a}, \vek{b}\in \mathcal{E}$ выполнено $$|\vek{a}|\cdot |\vek{b}| \geq |(\vek{a}, \vek{b})|.$$
\end{sled1}
\dok Неравенство $ |\Gamma (\vek{a}, \vek{b})| \geq 0 $ имеет вид 
$(\vek{a}, \vek{a})(\vek{b}, \vek{b}) - (\vek{a}, \vek{b})(\vek{b}, \vek{a}) \geq 0$
$\Leftrightarrow$
$(\vek{a}, \vek{a})(\vek{b}, \vek{b}) - (\vek{a}, \vek{b})\overline{(\vek{a}, \vek{b})} \geq 0$
$\Leftrightarrow$
$|\vek{a}|^2 \cdot |\vek{b}|^2 - |(\vek{a}, \vek{b})|^2 \geq 0 $.
\edok

\otstup
Отметим, что неравенство КБШ позволяет корректно определить угол между векторами Евклидова пространства.

\begin{sled2}[Неравенство треугольника]
$\forall$ $\vek{a}, \vek{b}\in \mathcal{E}$ выполнено $$|\vek{a}|+|\vek{b}| \geq |\vek{a} + \vek{b}| .$$
\end{sled2}
\dok
Имеем %$(|\vek{a}| + |\vek{b}|)^2  = |\vek{a}|^2+ |\vek{b}|^2 +2 |\vek{a}|\cdot |\vek{b}|$,\\
$|\vek{a} + \vek{b}|^2 = (\vek{a} + \vek{b}, \vek{a} + \vek{b}) = |\vek{a}|^2+ |\vek{b}|^2 
+ (\vek{a}, \vek{b}) + (\vek{b}, \vek{a})=  $\\
$= |\vek{a}|^2+ |\vek{b}|^2 
+ (\vek{a}, \vek{b}) + \overline{(\vek{a}, \vek{b})} \leq |\vek{a}|^2+ |\vek{b}|^2 
+ 2 |(\vek{a}, \vek{b})|$, что по неравенству КБШ не больше
чем $ |\vek{a}|^2+ |\vek{b}|^2 + 2 |(\vek{a}|\cdot |\vek{b})|  =  (|\vek{a}| + |\vek{b}|)^2 $.
\edok

\otstup

Пользуясь неравенством треугольника, несложно показать, что 
определив расстояние $\rho(\vek{a}, \vek{b}) = |\vek{a} - \vek{b}|$, 
превращаем $\mathcal{E}$ в метрическое пространство. 


\example{
II. {\it Стандартное} скалярное произведение на пространстве столбцов высоты $n$ определяется как  
$(\vek{a}, \vek{b}) = \sum\limits_{i=1}^{n}x_i\overline{y_i} = X^T\overline{Y}$,  где $\vek{a} = \bazis{e}X$, $\vek{b} = \bazis{e}Y$.
}

\example{
III. В пространстве $C[a, b]$ непрерывных (комплекснозначных) функций, опеределенных на отрезке $[a,b]$,  можно задать скалярное произведение как  \\
$(f, g) = \int\limits_{a}^{b} f(x)\overline{g(x)} \, dx$.
варианты с весами. ПРИМЕРЫ - к БИЛИНЕЙНЫМИ ФУНКЦИЯМ
}

%%СЛУЧАЙНЫЕ ВЕЛИЧИНЫ ---- Корреляция и ковариация --- евклидово пр-во (матожидание квадрата) --- дисперсия...
%% что такое по сути метод наименьших квадратов (монте-карло) 
% перемножение матриц --- в графах
%см. Кострикин-Манин стр. 126

\otstup 

{\bf Упражнение.} а) Пусть $B\in \mathbf{M}_{n\times n} (\mathbb{R})$ --- матрица положительно полуопределенной формы $\beta \in \mathcal{B}(V)$.
Докажите, что $B$ равна матрице Грама $\Gamma (\vek{a}_1, \vek{a}_2, \ldots, \vek{a}_n)$, где 
$\vek{a}_1, \vek{a}_2, \ldots, \vek{a}_n$ --- некоторая система векторов $n$-мерного евклидова пространства
$\mathcal{E}$.\\
б) Докажите, что существует (гомоморфизм) $\varphi\in L(V, \mathcal{E})$ такое, что 
$\beta (\vek{a}, \vek{b}) = (\varphi (\vek{a}), \varphi (\vek{b}))$.

