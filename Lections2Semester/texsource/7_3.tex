
\section{Симметричные и кососимметричные тензоры. }


\subsection{Перестановка аргументов в функциях нескольких переменных.}

%Определим эти операции для $T^0_q$ (можно обобщить для $T^p_q$, выполняя операции отдельно с верхними и нижними индексами).

Если $X$ и $Y$ --- произвольные множества, то  
на множестве $\mathcal{F}$ отображений $\underbrace{X\times X\times \ldots \times X}_q \, \to \, Y$ 
естественным образом действует группа перестановок $S_q = S_{\{1, 2, \ldots, q\}}$, так что 
$\sigma \in S_q$ переставляет аргументы отображения $f\in \mathcal{F}$:
% и перестановки  определена функция 
%$\sigma f \in \mathcal{F} $ по правилу 
$$(\sigma f) (x_1, \ldots , x_q) = f (x_{\sigma (1)}, \ldots , x_{\sigma (q)}).$$

Очевидно, $f\mapsto \sigma f$ задает биекцию $\mathcal{F} \to \mathcal{F}$.
Далее считаем, что $Y$ --- векторное пространство (над некоторым полем $\mathbb{F}$), тогда $f\mapsto \sigma f$ задает изоморфизм $\mathcal{F} \to \mathcal{F}$.


\subsection{Симметрические функции. Симметризация.}

\defin{
Функция $ f \in \mathcal{F} $ называется {\it симметрической}, если 
 $\forall$ $\sigma \in S_q$ выполнено $$\sigma f =f .$$
}

Подмножество всех симметрических $f\in \mathcal{F}$ можно обозначить $\mathcal{F}^+$.

Можно определять более общее понятие --- симметричность относительно 
действия заданной подгруппы в $H\leq S_q$,
заменяя в определении симметричности условие $\forall$ $\sigma \in S_q$ на $\forall$ $\sigma \in H$.
Например, если 
$H=S_{\{1, 2, \ldots, p\}}$, получаем условие симметричности относительно
первых $p$ аргументов
(и аналогичнно для $H=S_{\{p+1, p+2, \ldots, q\}}$ --- условие симметричности относительно
последних $q-p$ аргументов).
%$H$ --- симметричность относительно четных перестановок




 
С помощью действия $S_q$ на $\mathcal{F} $ (в случае поля нулевой характеристики) 
можем определить {\it усреднение}, или {\it симметризацию} как
$$\Sym (f) = \dfrac{1}{q!} \sum\limits_{\sigma \in S_q} \sigma f , $$
или более общо, симметризацию относительно действия заданной подгруппы в $H\leq S_q$
как $$\Sym_H(f) = \dfrac{1}{|H|} \sum\limits_{\sigma \in H} \sigma f .$$

Следующее предложение показывает, что  $\Sym : \mathcal{F} \to \mathcal{F}$ является проектором на 
$\mathcal{F}^+$.

\begin{predl}
1) $\forall$ $f \in \mathcal{F} $ выполнено $\Sym (f) \in \mathcal{F}^+ $;\\
2) если $f \in \mathcal{F}^+ $, то $\Sym (f)=f$.
\end{predl}



\begin{lem}\label{premix}
Пусть $\sigma_0 \in H \leq S_q$ ($H$ --- подгруппа в  $S_q$). Тогда
$\forall$ $f \in \mathcal{F} $ выполнено
$$\Sym_H(\sigma_0f) = \Sym_H (f) .$$
\end{lem}


%лемма о повторном размешивании...
 
\begin{lem}\label{double_mix}
Пусть $H\leq G\leq S_q$ (две подгруппы $S_q$, одна вложена в другую). Тогда
$\forall$ $f \in \mathcal{F} $ выполнено
$$\Sym_G (\Sym_H(f)) = \Sym_G (f) .$$
\end{lem}



\subsection{Кососимметрические функции. Альтернирование.}

Вспомним, что каждой перестановке приписан {\it знак} $\varepsilon(\sigma)$, 
равный $\pm 1$ (в зависимости от четности $\sigma$).

\defin{
Функция $ f \in \mathcal{F} $ называется {\it кососимметрической}, если 
 $\forall$ $\sigma \in S_q$ выполнено $$\sigma f = \varepsilon(\sigma) \cdot f .$$
}

Подмножество всех кососимметрических $f\in \mathcal{F}$ можно обозначить $\mathcal{F}^-$.

{\bf Упражнение}. Докажите, что  $\mathcal{F}^+ \oplus \mathcal{F}^-$ --- прямая сумма.
Однако, равенство $\mathcal{F} = \mathcal{F}^+ \oplus \mathcal{F}^-$, верное в случае функций от $q=2$ переменных вообще говоря, перестает быть верным при $q>2$.


%Можно определять более общее понятие --- симметричность относительно 
%действия заданной подгруппы в $H\leq S_q$,
%заменяя в определении симметричности условие $\forall$ $\sigma \in S_q$ на $\forall$ $\sigma \in H$.
%Например, если 
%$H=S_{\{1, 2, \ldots, p\}}$, получаем условие симметричности относительно
%первых $p$ аргументов
%(и аналогичнно для $H=S_{\{p+1, p+2, \ldots, q\}}$ --- условие симметричности относительно
%последних $q-p$ аргументов).
%%$H$ --- симметричность относительно четных перестановок

 
Аналогично симметризации можно определить <<усреднение со знаком>>, или {\it альтернирование} как
$$\Alt (f) = \dfrac{1}{q!} \sum\limits_{\sigma \in S_q} \varepsilon(\sigma) \cdot \sigma f .$$
Более общо, %симметризацию относительно действия заданной 
для подгруппы в $H\leq S_q$
$$\Alt_H(f) = \dfrac{1}{|H|} \sum\limits_{\sigma \in H} \varepsilon(\sigma) \cdot \sigma f .$$
Отметим, что если $H$ состоит только из четных перестановок, то $\Alt_H(f) = \Sym_H(f)$.



Следующее предложение покавается,  $\Sym : \mathcal{F} \to \mathcal{F}$ является проектором на 
$\mathcal{F}^-$.

\begin{predl}
1) $\forall$ $f \in \mathcal{F} $ выполнено $\Alt (f) \in \mathcal{F}^- $;\\
2) если $f \in \mathcal{F}^- $, то $\Alt (f)=f$.
\end{predl}



\begin{lem}\label{premix_alt}
Пусть $\sigma_0 \in H \leq S_q$ ($H$ --- подгруппа в  $S_q$). Тогда
$\forall$ $f \in \mathcal{F} $ выполнено
$$\Alt_H(\sigma_0f) = \varepsilon (\sigma_0)\cdot \Alt_H (f) .$$
\end{lem}


%лемма о повторном размешивании...
 
\begin{lem}\label{double_mix_alt}
Пусть $H\leq G\leq S_q$ (две подгруппы $S_q$, одна вложена в другую). Тогда
$\forall$ $f \in \mathcal{F} $ выполнено
$$\Alt_G (\Alt_H(f)) = \Alt_G (f) .$$
\end{lem}




\subsection{Применение к тензорам.}


Так как тензоры из являются функциями 
$\underbrace{V\times V\times \ldots \times V}_q \, \to \, \mathbb{F}$,  
согласно общей излагаемой теории, на пространстве $T^0_q = T^0_q (V) $ определяется изоморфизм $T^0_q\to T^0_q$
перестановки аргументов $\tau \mapsto \sigma \tau$.
Далее определяются линейные проекторы $T^0_q\to T^0_q$ симметризация $\tau \mapsto \Sym ( \tau)$ и
 альтернирование $\tau \mapsto \Alt ( \tau)$.

Множества симметрических и кососимметрических тензоров
обозначаем (наряду c $T^0_q (V)^+ $ и $T^0_q (V)^- $) через $S^0_q (V) $ и $\Lambda^0_q (V) $ соответственно.
Ясно, что $S^0_q (V) $ и $\Lambda^0_q (V) $ ---  подпространства в $T^0_q (V) $. 



В координтанатах: пусть 
$\tau \, \, \rsootv{\bazis{e}} \, \,  
t_{ik \ldots }.   $
Тогда 
$$\sigma \tau \, \, \rsootv{\bazis{e}} \, \,  
t_{\sigma(i) \sigma(k) \ldots }.    $$
(Простой пример: транспонирование матрицы билинейной формы: $b_{ij} \mapsto b_{ji}$.)

Для обозначения компонент тензора после симметризации и альтернирования применяются иногда 
круглые и квадратные скобки. Так, 
$$\Sym(\tau ) \, \, \rsootv{\bazis{e}} \, \,  
t_{(ik \ldots )}  = \dfrac{1}{q!}\sum\limits_{\sigma \in S_q} t_{\sigma(i) \sigma(k) \ldots };   $$
$$\Alt(\tau ) \, \, \rsootv{\bazis{e}} \, \,  
t_{[ik \ldots ]}  = \dfrac{1}{q!}\sum\limits_{\sigma \in S_q} \varepsilon(\sigma) \cdot t_{\sigma(i) \sigma(k) \ldots }.    $$



Рассмотрим тензоры  $\alpha \in T^0_p$ и
$\beta \in T^{0}_{q}$.
Тогда, согласно определению,\\
$$(\alpha \otimes \beta)  (\underbrace{\vek{a}, \vek{b}, \ldots }_{p}, 
\underbrace{\vek{c}, \vek{d}, \ldots }_{q}) \,  =      \, 
\alpha (\underbrace{\vek{a}, \vek{b}, \ldots }_{p}) \, \cdot \, 
\beta (\underbrace{\vek{c}, \vek{d}, \ldots }_{q}) = 
(\beta \otimes \alpha )  (\underbrace{\vek{c}, \vek{d}, \ldots }_{q}, \underbrace{\vek{a}, \vek{b}, \ldots }_{p}).$$
Отсюда 
$$\beta \otimes \alpha  = \sigma_{q, p} (\alpha \otimes \beta), $$
где перестановка $\sigma_{q, p}$ задается как
$\begin{pmatrix}
1 & 2 & \ldots &      q & q+1 & q+2 & \ldots & p+q  \\
p+1 & p+2 & \ldots & p+q & 1 & 2 & \ldots & p
\end{pmatrix}.$

Так как, $\varepsilon (\sigma_{q, p})= (-1)^{pq}$, с учетом 
лемм \ref{premix} и \ref{premix_alt}, получаем следующее.


\begin{predl}\label{comm_sym}
$\forall$  $\alpha \in T^0_p$ и $\beta \in T^{0}_{q}$ выполнено
$$\Sym (\alpha \otimes \beta)  = \Sym (\beta \otimes \alpha ).$$
\end{predl}


\begin{predl}\label{comm_alt}
$\forall$  $\alpha \in T^0_p$ и $\beta \in T^{0}_{q}$ выполнено
$$\Alt (\alpha \otimes \beta)  = (-1)^{pq} \, \Alt (\beta \otimes \alpha ).$$
\end{predl}

\begin{predl}\label{double_mix_sym_1}
$\forall$  $\alpha \in T^0_p$ и $\beta \in T^{0}_{q}$ выполнено
$$\Sym ((\Sym (\alpha)  \otimes \beta)  =  \Sym (\alpha \otimes \Sym (\beta) ) = \Sym (\alpha \otimes \beta).$$
\end{predl}
\dok СХЕМА. 
Заметим, что $\Sym (\alpha)  \otimes \beta  = \Sym_H  (\alpha  \otimes \beta)$, где $S_p\cong H \leq S_{p+q}$ ---  подгруппа перестановок, для которых последние $q$ элементов неподвижны.
\edok

\begin{predl}\label{double_mix_alt_1}
$\forall$  $\alpha \in T^0_p$ и $\beta \in T^{0}_{q}$ выполнено
$$\Alt ((\Alt (\alpha)  \otimes \beta)  =  \Alt (\alpha \otimes \Alt (\beta) ) = \Alt (\alpha \otimes \beta).$$
\end{predl}

\subsection{Симметрические тензоры. Базис в $S^0_q$.}


Обозначим $q! \Sym (\vek{e}^{i_1} \otimes \vek{e}^{i_2} \otimes \ldots \otimes \vek{e}^{i_q})$ 
коротко $\vek{e}^{i_1}\vek{e}^{i_2}\ldots \vek{e}^{i_q}$.

\example{Например, \\
$\vek{e}^{3}\vek{e}^{1}\vek{e}^{3} = 
2(\vek{e}^{3} \otimes \vek{e}^{1} \otimes \vek{e}^{3} + \vek{e}^{1} \otimes \vek{e}^{3} \otimes \vek{e}^{3}
+\vek{e}^{3} \otimes \vek{e}^{3} \otimes \vek{e}^{1}) $.\\
$\vek{e}^{3}\vek{e}^{1}\vek{e}^{2} = 
\vek{e}^{1} \otimes \vek{e}^{2} \otimes \vek{e}^{3} + \vek{e}^{1} \otimes \vek{e}^{3} \otimes \vek{e}^{2}+
\vek{e}^{2} \otimes \vek{e}^{1} \otimes \vek{e}^{3} + \vek{e}^{2} \otimes \vek{e}^{3} \otimes \vek{e}^{1}+
\vek{e}^{3} \otimes \vek{e}^{1} \otimes \vek{e}^{2} + \vek{e}^{3} \otimes \vek{e}^{2} \otimes \vek{e}^{1}
$.
}

Из-за коммутативности $\vek{e}^{i_1}\vek{e}^{i_2}\ldots \vek{e}^{i_q}$ 
приводится к виду $(\vek{e}^1)^{k_1}\ldots (\vek{e}^n)^{k_n}$,
где $k_i$ --- целые неотрицательные числа, сумма которых равна $q$.


\begin{predl}
1) $\{(\vek{e}^1)^{k_1}\ldots (\vek{e}^n)^{k_n} \, |\, k_i\in \mathbb{Z}_{+}, \sum k_i = q\}$ ---
 базис в пространстве $S^0_q$;\\
2) $\dim S^0_q = C_{n+q-1}^q$.
\end{predl}


\subsection{%Симметрическое произведение. 
Симметрическая алгебра.}




Для $\tau \in S^0_p$, $\tau ' \in S^0_{q}$ определяется {\it симметрическое умножение} по правилу: 
$$\tau \lor \tau ' =  \dfrac{(p+q)!}{p!q!} \Sym(\tau \otimes \tau '). $$

(иногда определяется без нормирующего множителя).
Из предложений \ref{comm_sym} и 
\ref{double_mix_sym_1} следует коммутативность и ассоциативность симметрического умножения.


Иногда знак $\lor$ опускается, что согласуется с 
обозначением $\vek{e}^{i_1}\vek{e}^{i_2}\ldots \vek{e}^{i_q}$:
$$\vek{e}^{i_1}\lor \vek{e}^{i_2}\lor \ldots \lor \vek{e}^{i_q} = \vek{e}^{i_1}\vek{e}^{i_2}\ldots \vek{e}^{i_q}.$$


$$S(V ) = \bigoplus\limits_{q=0}^{\infty} S^0_q.$$

$S(V )$ --- {\it симметрическая алгебра} относительно $+, \lor$.
(ассоциативная, коммутативная)



\subsection{Кососимметрические тензоры. Базис в $\Lambda^0_q$.}


Обозначим (временно) $q! \Alt (\vek{e}^{i_1} \otimes \vek{e}^{i_2} \otimes \ldots \otimes \vek{e}^{i_q})$ 
коротко $\vek{e}^{i_1} \circ \vek{e}^{i_2}\circ  \ldots \circ  \vek{e}^{i_q}$.

\example{Например, \\
$\vek{e}^{3} \circ \vek{e}^{1} \circ \vek{e}^{3} = 0$; \\
%2(\vek{e}^{3} \otimes \vek{e}^{1} \otimes \vek{e}^{3} + \vek{e}^{1} \otimes \vek{e}^{3} \otimes \vek{e}^{3}
%+\vek{e}^{3} \otimes \vek{e}^{3} \otimes \vek{e}^{1}) $.
$\vek{e}^{1}\circ \vek{e}^{2}\circ \vek{e}^{3} = 
\vek{e}^{1} \otimes \vek{e}^{2} \otimes \vek{e}^{3} - \vek{e}^{1} \otimes \vek{e}^{3} \otimes \vek{e}^{2}+
\vek{e}^{2} \otimes \vek{e}^{3} \otimes \vek{e}^{1} - \vek{e}^{2} \otimes \vek{e}^{1} \otimes \vek{e}^{3}+
\vek{e}^{3} \otimes \vek{e}^{1} \otimes \vek{e}^{2} - \vek{e}^{3} \otimes \vek{e}^{2} \otimes \vek{e}^{1}
$.
}

В частности, в трехмерном пространстве 
$(\vek{e}^{1}\circ \vek{e}^{2}\circ \vek{e}^{3}) (\vek{a}, \vek{b}, \vek{c}) = $
детерминант из координатных столбцов векторов $\vek{a}, \vek{b}, \vek{c}$.
%$=\varepsilon_{ijk}x^iy^jz^k$.


Из-за косокоммутативности $\vek{e}^{i_1} \circ \vek{e}^{i_2} \circ \ldots \circ \vek{e}^{i_q}$ 
приводится к виду $0$ либо $\pm \vek{e}^{j_1} \circ \vek{e}^{j_2} \circ \ldots \circ \vek{e}^{j_q} $, где
$j_1<j_2<\ldots <j_q$.
% (\vek{e}^1)^{k_1}\ldots (\vek{e}^n)^{k_n}$
%где $k_i$ --- целые неотрицательные числа, сумма которых равна $q$.


\begin{predl}
1) $\{\vek{e}^{j_1} \circ \vek{e}^{j_2} \circ \ldots \circ \vek{e}^{j_q} \, |\, 
j_1<j_2<\ldots <j_q\}$ ---
 базис в пространстве $\Lambda^0_q$;\\
2) $\dim \Lambda^0_q = C_{n}^q$ при $q\leq n$ и $\Lambda^0_q = O$ при $q>n$.
\end{predl}


%Элементы $\Lambda^0_q$ называются $q$-векторами (поливекторы). здесь путаница между $V$ и $V^*$.

%Поливектор вида $ разложимый поливектор


\subsection{ 
Внешняя алгебра.}




Для $\tau \in \Lambda^0_p$, $\tau ' \in \Lambda^0_{q}$ определяется {\it внешнее умножение} по правилу: 
$$\tau \wedge \tau ' =  \dfrac{(p+q)!}{p!q!} \Alt(\tau \otimes \tau '). $$

(иногда определяется без нормирующего множителя).
Из предложений \ref{comm_alt} и 
\ref{double_mix_alt_1} следует косокоммутативность:
$$\tau \wedge \tau ' = (-1)^{pq}\tau '\wedge \tau $$
(в частности, при нечетном $p$ $\tau \wedge \tau =0$) 
и ассоциативность внешнего умножения.

%Иногда знак $\wedge$ опускается, что согласуется с 
%обозначением $\vek{e}^{i_1}\vek{e}^{i_2}\ldots \vek{e}^{i_q}$:
$$\vek{e}^{i_1}\circ \vek{e}^{i_2}\circ \ldots \circ \vek{e}^{i_q} = 
\vek{e}^{i_1}\wedge \vek{e}^{i_2}\wedge \ldots \wedge \vek{e}^{i_q}.$$


$$\Lambda(V ) = \bigoplus\limits_{q=0}^{n} \Lambda^0_q.$$

$\Lambda(V )$ --- {\it внешняя алгебра} относительно $+, \wedge$.
(ассоциативная, косокоммутативная)



%ПРИМЕР. $\det$



