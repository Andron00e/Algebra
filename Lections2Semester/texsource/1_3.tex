\section{Сумма подпространств. Прямая сумма %
%и прямое дополнение
}

\subsection{Сумма подпространств}

\defin{Пусть $\mathcal{A}_1, \mathcal{A}_2, \ldots , \mathcal{A}_k$ ---
подмножества в $V$. {\it Суммой} (по Минковскому)
подмножеств $\mathcal{A}_1, \mathcal{A}_2, \ldots , \mathcal{A}_k$
называется множество
$\{\vek{a}_1 + \vek{a}_2 + \ldots + \vek{a}_k \, | \, \vek{a}_i \in \mathcal{A}_i \}$, $i=1, \ldots, k$.
}

Обозначение для суммы подмножеств: $\mathcal{A}_1+ \mathcal{A}_2+ \ldots +\mathcal{A}_k$
или $\sum\limits_{i=1}^{k} \mathcal{A}_i$.
Из определения ясно, что $\mathcal{A}_1+ \mathcal{A}_2 = \mathcal{A}_2+ \mathcal{A}_1$,
$(\mathcal{A}_1+ \mathcal{A}_2)+ \mathcal{A}_3 \hm= \mathcal{A}_1+ (\mathcal{A}_2+ \mathcal{A}_3)$.
%Но некоторые другие свойства суммы векторов для суммы множеств не выполняются, скажем,
%из $\mathcal{A}+\mathcal{A}=\mathcal{A}$ не следует, что $\mathcal{A}=\{o\}$. 


Далее будем заниматься почти всегда только суммами подпространств.

%\begin{predl}\label{p7_4_1}
%Сумма нескольких (конечного числа) подпространств является подпространством.
%\end{predl}
%\dok
%\edok

Следующее предложение связывает понятия суммы и линейной оболочки.

\begin{predl}\label{p7_4_2}
Пусть $U_i = \lin{\mathcal{A}_i}$, $i=1, 2, \ldots , k$.
Тогда
$$\sum\limits_{i=1}^k U_i=
\lin{\mathcal{A}_1 \cup \mathcal{A}_2 \cup \ldots \cup \mathcal{A}_k}.$$
\end{predl}
\dok По определению, $\sum\limits_{i=1}^k U_i$ --- множество элементов вида $\sum\limits_{i=1}^k \vek{a}_i$, где $\vek{a}_i$ может быть значением произвольной линейной комбинации векторов из $\mathcal{A}_i$, а значит множество сумм $\sum\limits_{i=1}^k \vek{a}_i$ совпадает с множеством значений линейных комбинации векторов из 
$\mathcal{A}_1 \cup \mathcal{A}_2 \cup \ldots \cup \mathcal{A}_k$.
\edok

\otstup

\begin{sled}
Сумма нескольких (конечного числа) подпространств является подпространством.
\end{sled}

Видим, что сумма $\sum\limits_{i=1}^k U_i$ подпространств $U_1, \ldots , U_k$ --- это минимальное по включению подпространство,
содержащее каждое из подпространств $U_1, \ldots,  U_k$.


\begin{predl}\label{p7_4_3}
Пусть $U_i\leq V$, $i=1, 2, \ldots , k$.
Тогда $$\dim \left(\sum\limits_{i=1}^k U_i \right) \leq \sum\limits_{i=1}^k \dim U_i.$$
\end{predl}
\dok
По предыдущему предложению, $\sum\limits_{i=1}^k U_i =
\lin{U_1 \cup U_2 \cup \ldots \cup U_k}$.
Но по основной теореме \ref{t5_2_1} и предложению
\ref{p7_2_5} имеем $\dim \lin{U_1 \cup U_2 \cup \ldots \cup U_k} \hm=
\rg(U_1 \cup U_2 \cup \ldots \cup U_k) \leq
\sum\limits_{i=1}^k \dim U_i$.
\edok

\otstup

Для операции сложения подмножеств выполнены далеко не все свойства, которыми обладает операция сложения векторов.
%Оперируя с суммой подпространств, нужно иметь в виду следующие предостережения.
Например, вообще говоря не выполнен закон сокращения, скажем, для любого подпространства $U\leq V$
справедливо равенство $U+U=U$.
Также не работает аналогия между суммой подпространств и теоретико-множественным
объединением: скажем не всегда
$(U_1+U_2)\cap U_3 \hm= (U_1\cap U_3)+(U_2\cap U_3)$
в отличие от теоретико-множественного тождества $(U_1\cup U_2)\cap U_3 = (U_1\cap U_3)\cup (U_2\cap U_3)$.

\otstup

{\bf Упраженение.} Приведите пример подпространств $U_1, U_2, U_3$ некоторого векторного пространства $V$ таких, что
$(U_1+U_2)\cap U_3 \hm\neq (U_1\cap U_3)+(U_2\cap U_3)$. 

\subsection{Прямая сумма}

Важен следующий специальный случай суммы подпространств $U_i\leq V$.

\defin{Сумма $U$ подпространств $U_1, U_2, \ldots , U_k$ называется (внутренней) {\it прямой суммой},
если $\forall \vek{a}\in U$ имеется единственный набор
$\vek{a}_i\in U_i$ ($i=1, 2, \ldots , k$) такой, что
$\vek{a} = \sum\limits_{i=1}^{k} \vek{a}_i$.
}


Обозначение для прямой суммы: $U_1 \bigoplus U_2 \bigoplus \ldots \bigoplus U_k$
или  $\bigoplus \limits_{i=1}^{k} U_i$.
Иногда говорят, что $U$ {\it разложено в прямую сумму} подпространств $U_1, U_2, \ldots , U_k$.
Пример разложения в прямую сумму: 
$V=\lin{\vek{e}_1} \bigoplus \lin{\vek{e}_2} \bigoplus \ldots
 \bigoplus \lin{\vek{e}_n}$, где  $\vek{e}_1, \vek{e}_2, \ldots , \vek{e}_n$ --- некоторый базис в $V$.


\otstup
%%%%%%%%%%%%%%%%
%%ВНЕШЯЯ ПРЯМАЯ СУММА
%%%%%%%%%%%%%%%%

{\footnotesize Наряду с внутренней прямой суммой, которой в основном будем заниматься, 
можно рассмотреть {\it внешнюю прямую сумму} векторных пространств $U_1, U_2, \ldots , U_k$ 
как декартово произведение $U_1\times  U_2\times \ldots \times U_k$, на котором 
операции сложения и умножения на константу определены <<покомпонентно>>. Конструкции
внутренней и внешней прямой суммы по сути эквивалетны (связаны естественным изоморфизмом).}



%Если $U = \bigoplus \limits_{i=1}^{k} U_i$, ниже обозначаем $\overline{U_i}=U_1+\ldots +U_{i-1}+U_{i+1}+\ldots +U_k$.
В случае рассмотрения суммы $U = \sum\limits_{i=1}^{k} U_i$ через $\overline{U_i}$ обозначаем сумму всех рассматриваемых пространств, 
за исключением $U_i$, т.е. $\overline{U_i}=U_1+\ldots +U_{i-1}+U_{i+1}+\ldots +U_k$.

\begin{theor}[критерий-1 прямой суммы]\label{t7_4_1}
Пусть $U_i\leq V$, $i=1, \ldots, k$.
Сумма подпространств $U_1, \ldots, U_k$ --- прямая сумма 
$\Leftrightarrow$
$U_i\cap \overline{U_i} = O$, $i=1, 2, \ldots , k$.
\end{theor}
\dok \dokright Предположим противное, скажем условие $U_i\cap \overline{U_i} = O$ не выполнено для $i=1$.
Это значит, что  нашелся ненулевой вектор $\vek{a}_1 \in U_1 \cap  \overline{U_1}$. 
Имеем $\vek{a}_1 = \vek{a}_2+\ldots + \vek{a}_k$ для некоторых $\vek{a}_i\in U_i$, $i=2, \ldots, k$.
Но тогда $\vek{0} = \vek{0} +\ldots + \vek{0} = \vek{a}_1 +(- \vek{a}_2)+\ldots + (-\vek{a}_k)$ --- два различных разложения нулевого вектора в сумму векторов из $U_i$
вопреки определению прямой суммы.

\dokleft Предположим противное, сумма $U=\sum\limits_{i=1}^{k} U_i$ не является прямой суммой, 
тогда для некоторого вектора $\vek{a}\in U$ найдутся два различных разложения
$\vek{a} = \sum\limits_{i=1}^{k} \vek{a}_i = \sum\limits_{i=1}^{k} \vek{b}_i$, где $\vek{a}_i\in U_i$, $\vek{b}_i\in U_i$, $i=1, \ldots, k$.
Разложения различны, значит хотя бы для одного $i$ имеем $\vek{a}_i\neq \vek{b}_i$, скажем $\vek{a}_1\neq \vek{b}_1$.
Но тогда $\vek{a}_1 - \vek{b}_1 = \sum\limits_{i=2}^{k} (\vek{b}_i-\vek{a}_i)$. В левой части равенства --- ненулевой вектор из $U_1$, а в правой части --- вектор из
$\overline{U_1}$, поэтому $U_1\cap \overline{U_1} \neq  O$. Противоречие.
\edok 

\otstup
Указанным критерием часто удобно пользоваться в случае двух подпространств. 
Этот случай выделим отдельным следствием.

\begin{sled}
Пусть $U_1\leq V$, $U_2\leq V$. Тогда $U_1+U_2$ --- прямая сумма $\Leftrightarrow$ $U_1\cap U_2 = O$.
\end{sled}

\otstup


В отличие случая двух подпространств, 
условие тривиальности попарных пересечений
$U_1\cap U_2 = U_2\cap U_3 = U_3\cap U_1 = O$ не является достаточным для того, чтобы сумма подпространств
$U_1+U_2+U_3$ была прямой суммой. 
Контрпримером могут служить три различных одномерных пространства,
лежащие в двумерном пространстве.

\otstup


{\bf Упражнение.}
Сумма подпространств $U_1, \ldots, U_k$ --- прямая сумма 
$\Leftrightarrow$
$\forall$ $\vek{a}_i\in U_i$, $\vek{a}_i\neq \vek{0}$
система $\vek{a}_1, \ldots, \vek{a}_k$ линейно независима.

\otstup

{\bf Упражнение.} Пусть $U_i\leq V$ таковы, что 
$U_1\oplus U_2\oplus U_3$ --- прямая сумма. Тогда $U=U_2\oplus U_3$ --- прямая сумма и 
$U_1\oplus U$ --- также прямая сумма.
%Не зависит от порядка; прямую сумму можно определить последовательно через
%сумму двух.

\otstup

{\bf Упражнение.} Пусть $U_i\leq V$, $i=1, \ldots, k$, --- подпространства, для которых выполнено: 
$U_j\cap (\sum\limits_{i=1}^{j-1} U_i) = O$ для всех $j=2, 3, \ldots, k$.
Тогда $\sum\limits_{i=1}^{k} U_i$ --- прямая сумма.


\begin{theor}[критерий-2 прямой суммы]\label{t7_4_2}
Пусть $U_i\leq V$, $\dim U_i=n_i<\infty$, $\bazis{e}^{(i)}$ --- базис в~$U_i$, $i=1, \ldots, k$;
$U=\sum \limits_{i=1}^{k} U_i$. Тогда следующие условия эквивалентны:\\
1) $U= \bigoplus \limits_{i=1}^{k} U_i$;\\
2) система из $\sum \limits_{i=1}^{k}n_i$ векторов $\bigcup\limits_{i=1}^{k} \bazis{e}^{(i)}$ --- базис
в  $U$;\\
3) $\dim U = \sum \limits_{i=1}^{k} n_i$.
\end{theor}
\dok Имеем  $U_i = \lin{\bazis{e}^{(i)}}$, тогда согласно предложению 
\ref{p7_4_2}, $U = \lin{\bazis{e}}$, где $\bazis{e}=\bigcup\limits_{i=1}^{k} \bazis{e}^{(i)}$.

2) $\Leftrightarrow$ 3)
По  следствию из основной теоремы \ref{t5_2_1} имеем $\dim U=\rg \bazis{e}$.
Значит (см. предложение \ref{p7_2_4}) $\dim U = \sum \limits_{i=1}^{k} n_i$ 
$\Leftrightarrow$  $\rg \bazis{e} = \sum \limits_{i=1}^{k} n_i$
$\Leftrightarrow$  система $\bazis{e}$ линейно независима.

1) $\Rightarrow$ 2) 
Предположим противное, и система $\bazis{e}$ линейно зависима.
Запишем нетривиальную линейную  комбинацию векторов  из $\bazis{e}$, равную $\vek{0}$:
$\sum \limits_{i=1}^{k}  \vek{\ell}_i=\vek{0}$, где $\vek{\ell}_i$ --- линейная комбинация векторов из $\bazis{e}^{(i)}$.
Хотя бы одна из линейных комбинаций $\vek{\ell}_i$  нетривиальная, пусть это~$\vek{\ell}_1$, тем самым $\vek{\ell}_1\neq \vek{0}$.
Тогда  $\vek{\ell}_1= - \vek{\ell}_2-\ldots - \vek{\ell}_k$. Здесь $\vek{\ell}_i$ равно некоторому вектору из $U_i$, 
%причем $\vek{l}_1$ равно ненулевому  вектору из $U_1$, так как это значение нетривиальной линейной комбинации 
%линейно независимых векторов $\bazis{e}^{(1)}$.
значит $\vek{\ell}_1\in U_1\cap \overline{U_1}$  в противоречие с
теоремой \ref{t7_4_1}.

2) $\Rightarrow$ 1) 
Предположим противное, $U$ не являетяся прямой суммой подпространств $U_i$, и скажем 
(см. теорему \ref{t7_4_1}) $\exists$ $\vek{a}\neq \vek{0}$: $\vek{a}\in U_1\cap \overline{U_1}$.
Тогда $\vek{a}= \vek{\ell}_1= \vek{\ell}_2+\ldots +\vek{\ell}_k$, где $\vek{\ell}_i\in U_i$, т.е. $\vek{\ell}_i$ равно некоторой линейной комбинации векторов из $\bazis{e}^{(i)}$.
Перенося в левую часть, получаем $\vek{\ell}_1- \vek{\ell}_2-\ldots -\vek{\ell}_k = \vek{0}$ (в левой части нетривиальная линейная комбинация, поскольку 
уже $\vek{\ell}_1$ --- нетривиальная линейная комбинация), откуда
$\bazis{e}$ --- линейно зависимая система. Противоречие. 
\edok

\subsection{Прямое дополнение. Проекции}


\defin{
Если  $U_1\leq V$ и $U_2\leq V$ таковы, что $U_1 \bigoplus U_2 = V$, 
то подпространство $U_2$ называют {\it прямым дополнением} подпространства
$U_1$ (в векторном пространстве $V$).
}

Подпространства $U_1$ и $U_2$ входят в определение симметрично, поэтому: $U_1$ --- прямое дополнение для $U_2$ 
$\Leftrightarrow$ $U_2$ --- прямое дополнение для $U_1$.  
Отметим, что прямое дополнение ничего общего не имеет с понятием теоретико-множественного дополнения.


\begin{predl}\label{p7_4_4}
Пусть $\dim V = n< \infty$. Тогда сумма размерностей подпространства и любого его прямого
дополнения равна $n$.
\end{predl}
\dok
Следует из теоремы \ref{t7_4_2}.
\edok

\begin{predl}\label{p7_4_5}
Пусть $\dim V = n< \infty$. Для любого подпространства $U\leq V$ существует прямое дополнение.
\end{predl}
\dok
Выберем в $U$ базис $\vek{e}_1, \ldots, \vek{e}_k$. %, тогда $U=\lin{\vek{e}_1, \ldots, \vek{e}_k}$.
Согласно предложению  \ref{p7_3_1}, систему $\vek{e}_1, \ldots, \vek{e}_k$ можно дополнить до
базиса $\vek{e}_1, \ldots, \vek{e}_k, \vek{e}_{k+1}, \ldots ,\vek{e}_{n}$ пространства $V$.
Тогда из теоремы \ref{t7_4_2} вытекает, что  подпространство $W=\lin{\vek{e}_{k+1}, \ldots, \vek{e}_n}$ таково, что $U\oplus W = V$.
\edok

\otstup

Заметим, что для одного пространства может существовать много прямых дополнений
(достаточно посмотреть на геометрический пример $U\leq V$, где $\dim U=1$, $\dim V=2$). 

Если $V=U_1 \bigoplus U_2$, то, согласно определению прямой суммы, 
 любой вектор $\vek{a} \in V$ однозначно представляется в виде суммы
$\vek{a} = \vek{a}_1 + \vek{a}_2$, где $\vek{a}_i\in U_i$, $i=1, 2$.
 Вектор $\vek{a}_1$ называется 
{\it проекцией} вектора $\vek{a}$ на подпространство $U_1$ {\it вдоль} $U_2$ (или {\it параллельно} $U_2$).
%вдоль подпространства $\overline{U_i}=U_1+\ldots +U_{i-1}+U_{i+1}+\ldots +U_k$}.

\otstup

{\footnotesize
Для $U\leq V$ определяется {\it фактор-пространство} $V/U$ как
фактор-множество относительно эквивалентности $\vek{a}\sim \vek{b}$ $\Leftrightarrow$ $\vek{b}-\vek{a}\in U$;
элементы $V/U$ --- смежные классы вида $\vek{a} +U$ с естественно определенными операциями
$(\vek{a} +U)+(\vek{b} +U):=(\vek{a}+\vek{b} )+U$, $\lambda (\vek{a} +U) := \lambda \vek{a} +U$.
Имеется естественный изоморфизм $W\to V/U$,
где $W$ --- прямое дополнение подпространства $U\leq V$; он задается как $\vek{a} \mapsto \vek{a} +U$.
}

\subsection{Формула размерностей суммы и пересечения}


\begin{theor}[формула Грассмана]\label{t7_4_3}
Пусть $U_1\leq V$, $U_2\leq V$.
Тогда 
$$\boxed{\dim (U_1+U_2) + \dim (U_1\cap U_2)= \dim U_1+ \dim U_2}.$$
\end{theor}
\dok Достаточно рассмотреть случай $\dim U_i<\infty $ (иначе в обеих частях формулы --- бесконечность).

Согласно предложению \ref{p7_4_5}, можем выбрать  $W\leq U_2$ так, что 
\begin{equation}\label{eqW1}
(U_1\cap U_2)\oplus W = U_2.
\end{equation}
Тогда  $U_1+U_2 = U_1+ ((U_1\cap U_2) +  W) = (U_1+ (U_1\cap U_2)) +  W = U_1+W $.
Кроме того, поскольку $W\leq U_2$, имеем  $U_1\cap W = U_1\cap U_2 \cap W = (U_1\cap U_2) \cap W = O$
(из (\ref{eqW1}) по следствию из теоремы \ref{t7_4_1}). Получаем, что $U_1+W $ --- прямая сумма:
\begin{equation}\label{eqW2}
U_1+  U_2 = U_1 \oplus W.
\end{equation}
 Согласно %предложению \ref{p7_4_4}, 
теореме \ref{t7_4_2}, из равенств (\ref{eqW1}) и (\ref{eqW2}) следует, что \\
$\dim W = \dim U_2 - \dim (U_1\cap U_2) = \dim (U_1+ U_2) - \dim U_1$, откуда следует требуемая формула размерностей.
%утверждение теоремы.
\edok

\otstup 
{\bf Упражнение.} Пусть $\dim V=4$. Существуют ли подпространства $U_1$ и $U_2$ такие, что 
$\dim U_1 = \dim U_2 = 3$ и $\dim (U_1\cap U_2)=1$?

\subsection{Примеры}


\example{I.1. Сумма двух непараллельных отрезков --- параллелограмм. (Здесь, как обычно, точки отождествляем
с концами радиус-векторов.) %Что представляет собой сумма двух многоугольников? Многоугольника и круга?
}
\example{I.2.
Пусть $V$ ---  геометрическое трехмерное векторное пространство,
$U_1\leq V$, $\dim U_1=1$ (т.е. $U_1$ --- прямая, проходящая через $O$),
 $U_2\leq V$, $\dim U_2=2$ (т.е. $U_2$ --- плоскость, проходящая через $O$).
Если $U_1 \not \hspace{-1mm} \subset U_2$, то $V=U_1\oplus U_2$. \\
Понятно, как геометрически разложить радиус-вектор $\overrightarrow{OA}$ в сумму проекций: через $A$ проведем прямую, параллельную $U_1$; пусть
она пересекает $U_2$  в точке $A_2$. Тогда $\vek{a}=\vek{a}_1+\vek{a}_2$, где $\vek{a}_1 = \overrightarrow{A_2A}$, $\vek{a}_2 = \overrightarrow{OA_2}$. 
}

\example{II.1.
Множество решений совместной СЛУ (над $\mathbb{R}$) $AX=b$ с $n$ неизвестными имеет вид $\Sol(AX=b) = \{X_0\}+\Sol(AX=O)$.
Если  $r=\rg A$, то $\Sol(AX=b)$ --- это  $(n-r)$-мерная плоскость (или $(n-r)$-мерное {\it линейное многообразие}) 
в пространстве $\mathbb{R}^n$.
}
%
%\example{Пусть $\mathcal{A}_1 = \{\vek{b} \}$ --- множество из одного вектора,
%а $\mathcal{A}_2 = U$ --- подпространство размерности $k$. Тогда
%$\mathcal{A}_1+ \mathcal{A}_2 = \{\vek{b} \} + U$ (то есть сдвиг подпространства $U$ на вектор
%$\vek{b}$) --- $k$-мерное {\it линейное многообразие} или {\it $k$-мерная} плоскость.
%}
%
\example{II.2. Пусть $\mathbf{M}_{n\times n} = \mathbf{M}_{n\times n} (\mathbb{F})$, где 
$\mathbb{F}$ --- поле характеристики, не равной 2. Тогда\\
$\mathbf{M}_{n\times n} = \mathbf{M}_{n\times n}^{+} \bigoplus \mathbf{M}_{n\times n}^{-}$.\\
Действительно, $A=\dfrac{A+A^T}{2}+\dfrac{A-A^T}{2}$ и %очевидно, 
$\mathbf{M}_{n\times n}^{+} \cap \mathbf{M}_{n\times n}^{-} = O$.
%где $\mathbf{M}_{n\times n}^{+}= \{A\in \mathbf{M}_{n\times n}| A^T=A \}$ ---
%подпространство симметричных матриц,
%$\mathbf{M}_{n\times n}^{-} = \{A\in \mathbf{M}_{n\times n}| A^T=-A \}$ --- подпространство кососимметричных
%симметричных матриц.
}
\example{II.3.
Пусть $A\in \mathbf{M}_{n\times m}(\mathbb{R})$, $A=(a_{\bullet 1} \ldots a_{\bullet m})$.
Положим $U\hm=\lin{a_{\bullet 1} \ldots a_{\bullet m}}$, $\dim U = \rg A = r$.
Пусть $W = \Sol(A^TX=O)$. Тогда $\dim W = n-r$. Кроме того
$U\cap W = O$. Действительно, 
если $Y= \stolbec{y_1\\y_2\\ \vdots \\y_n}$ таков, что $ Y\in U\cap W$, то  есть $Y^TY = O$, 
откуда $\sum\limits_{i=1}^n y_i^2 = 0$, значит $Y=O$.\\
(Отметим, что естественное объяснение этого факта получится ниже, см. предложение
\ref{10_2_5} из главы \ref{evkl_prostr}.)
}

\example{III.1.
Несложно показать, что $\mathbf{F} = \mathbf{F}^{+} \bigoplus
\mathbf{F}^{-}$, где
$\mathbf{F}^{+}$, $\mathbf{F}^{-}$ --- подпространства
четных и нечетных функций $\mathbb{R}\to \mathbb{R}$.\\
Заметим, что $e^x=\ch x + \sh x$, при этом $f(x) =\ch x$ --- четная функция, а $g(x) =\sh x$ --- нечетная функция.
Поэтому $f(x) = \ch x$ --- это проекция функции $y(x) = e^x$ на $\mathbf{F}^{+}$ вдоль $\mathbf{F}^{-}$.
}
